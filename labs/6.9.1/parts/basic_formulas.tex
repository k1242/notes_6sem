\section*{Теоретичекий минимум}

\textbf{Закон Кюри-Вейсса}.
Намагниченностью называется магнитный момент $I$ единицы объёма, который связан с внешним магнитным полем $H$, под действием которого он возникает, соотношением:
$$
I = \varkappa H,
$$
где $\chi$ -- магнитная восприимчивость вещества. 


Закон Кюри для парамагнетиков:
\begin{equation*}
    \varkappa=\frac{N g^{2} \mu_{\mathrm{B}}^{2} S(S+1)}{3 k_{\mathrm{B}} T},
\end{equation*}
где $g$ -- фактор Ланде, $N$ -- количество непаренных электронов в единице объема.


Закон Кюри-Вейсса для ферромагнетиков:
\begin{equation*}
    \varkappa=\frac{I}{H}=N \frac{g^{2} \mu_{\mathrm{B}}^{2} S(S+1)}{3 k_{\mathrm{B}}(T-\Theta)} \propto \frac{1}{T-\Theta},
    \hspace{10 mm} 
    \Theta = N \frac{\lambda g^2 \sub{\mu}{B}^2 S(S+1)}{3 \sub{k}{B}},
\end{equation*}
где $\Theta$ -- температура Кьюри, $\lambda$ -- константа Вейса. Закон носит приближенный характер и позволяет описать параагнитную фазу.




\textbf{Связь эффективного поля Вейсса с обменным интегралом.}
В теории Гейзенберга-Френкеля энергия $U_\text{обм}$ обменного взаимодейтвия атомов $i$ и $j$ выражается соотношением
$$
U_\text{обм} = -2 J S_i S_j
$$
Энергия $U$ представляет собой разность между средними значениями кулоновской энергии для параллельных и антипараллельных спинов $S_i$ и $S_j$, а $J$ -- коэффициент пропорциональности, называемый обменным интегралом, величина которого зависит от
степени перекрытия распределённых зарядов атомов $i$ и $j$ (от степени перекрытия
волновых функций электронов).

Можем приближенно установить связь между обменным интегралом $J$ и константой Вейсса $\lambda$. Получаем
для константы Вейсса $\lambda$ следующее выражение:
\begin{equation*}
    \lambda=\frac{2 n J V}{g^{2} \mu_{\mathrm{B}}^{2}}.
\end{equation*}

Так как объём, занимаемый одним атомом, равен $V=1/N$, где $N$ -- концентрация атомов, то мы окончательно получаем:
\begin{equation*}
    J=\frac{3 k_{\text {Б}} \Theta}{2 n S(S+1)}.
\end{equation*}


\textbf{Гадолинй}. 
Для гадолиния $n=12$, $J = 7/2$, $\sub{\mu}{эфф} = 7.94 \sub{\mu}{B}$, обусловлен $4f$-оболочкой, а значит
\begin{equation*}
    J = \frac{2 \sub{k}{B} \Theta}{21 n} = \frac{\sub{k}{B} \Theta}{126}.
\end{equation*}