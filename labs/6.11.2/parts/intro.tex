\subsection*{ТеорМин}


В модели свободных электронов можем получить выражение для проводимости в металлах:
\begin{equation*}
    \sigma = \frac{n e^2 \tau}{m},
    \hspace{0.5cm} \Rightarrow \hspace{0.5cm}
    \sigma \propto \frac{1}{T}.
\end{equation*}
Для полупроводников температурная зависимость будет иметь вид
\begin{equation*}
    \sigma \propto \exp\left(-\frac{\Delta}{2 kT}\right),
    \hspace{0.5cm} \Rightarrow \hspace{0.5cm}
    \Delta = - 2 k T (\ln \sigma'_T.
\end{equation*}
Удельную проводимость сможем находить из сопротивления $R$ через
\begin{equation*}
    \sigma = \frac{l}{R S},
\end{equation*}
где $l$ -- длина образца, $S$ -- поперечное сечение образца.