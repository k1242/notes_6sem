\subsection*{ТеорМин}


 Взаимодействие магнитного диполя с внешним полем приводит к появлению поправки 
\[
E = -(\boldsymbol{\mu},\mathbf{B}).
\]
Вектор $\boldsymbol{\mu}$ ориентирован по направлению полного момента количества движения $\mathbf{M}$:
\[
\boldsymbol{\mu} = \gamma \mathbf{M},
\]
где $\gamma$ -- гиромагнитное соотношение. $g$-фактор
\[
g = \dfrac{\hbar}{\mu_\text{я}}\gamma,
\hspace{0.5cm} \Rightarrow \hspace{0.5cm}
\boldsymbol{\mu} = \dfrac{\mu_\text{я}}{\hbar}g\mathbf{M}.
\]
Квадрат вектора $\mathbf{M}$ и его проекция определяются формулами
\[
\mathbf{M}^2 = \hbar^2 I(I+1),~M_z = m\hbar,
\]
где $I$ -- спин ядра, а $m$ -- целое число, по модулю не превосходящее $I$. Тогда, проектируя $\mathbf{M}$ и $\boldsymbol{\mu}$ на направление вектора $B$, получим
\[
\mu_B = \dfrac{\mu_\text{я}}{\hbar}g M_B = \mu_\text{я} g m.
\]
Таким образом, разница между расщепившимися уровнями энергии будет
\[
\Delta E = B\Delta \mu_B = B \mu_\text{я} g.
\]
Между компонентами расщепившегося уровня могут происходить электромагнитные переходы. Переходы с нижних компонент на верхние требуют затрат энергии и происходят лишь под действием внешнего высокочастотного поля. Энергия квантов, вызывающих электромагнитные переходы, точно определена, стало быть явление носит резонансный характер. Соответствующая частота 
\begin{equation*}
\omega = \dfrac{\Delta E}{\hbar} = \dfrac{\mu_\text{я}}{\hbar}Bg.
\end{equation*}
Возбуждение переходов между компонентами расщепившегося ядерного уровня носит название ядерного магнитного резонанса.