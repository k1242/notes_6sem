% document's head

% \phantom{42} 

\begin{center}
    \LARGE \textsc{Лабораторная работа №6.10.4} \\
    \vspace{3 mm}
    \large Магнитный момент лёгких ядер
\end{center}

% \hrule

\phantom{42}

\begin{flushright}
    \begin{tabular}{rr}
    % written by:
        % \textbf{Источник}: 
        % & \href{__ссылка__}{__название__} \\
        % & \\
        % \textbf{Лектор}: 
        % & _ФИО_ \\
        % & \\
        \textbf{Автор работы}: 
        & Хоружий Кирилл \\
        & \\
    % date:
        \textbf{От}: &
        \textit{\today}\\
    \end{tabular}
\end{flushright}

\thispagestyle{empty}

\vspace{10mm}


\subsection*{Цель работы}
\begin{enumerate*}
    \item Исследовать ядерный магнитный резонанс.
    \item Пронаблюдать сигнал ЯМР от различных ядер.
    \item Определить $g$-фактор для ядер.
\end{enumerate*}


% \subsection*{Оборудование}
% Призменный монохроматор-спектрометр УМ-2 ($380$-$1000$ нм), 
% фотоэлемент, покрытый Na$_2$\,K\,Sb(Cs), 
% неоновая лампа,
% усилитель постоянного тока,
% линза, два вольтметра.




% \newpage
