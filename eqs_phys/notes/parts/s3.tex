\section{Нелинейные полевые уравнения}

В линейных уравнениях обычно ищем функцию Грина.

Рассмотрим уравнение Хопфа
\begin{equation*}
    \partial)t u + u \partial_x u = 0,
\end{equation*}
которое описывает динамику плотности частиц газа.

\textbf{Метод характеристик}. 
Рассмотрим уравнение переноса
\begin{equation*}
    \partial_t  + \vc{v} \partial_t \varphi = f(t, \vc{r}), 
    \hspace{5 mm} 
    \vc{v} = \const.
\end{equation*}
Пока считаем $f = 0$. Заметим, что
\begin{equation*}
    \frac{d \varphi}{d t} = \frac{\partial \varphi}{\partial t}  + \frac{d \vc{r}}{d t} \frac{\partial \varphi}{\partial \vc{r}} = 0.
\end{equation*}
Заметим, что $\frac{d \smallvc{t}}{d t} = \vc{v}$ даст решение:
\begin{equation*}
    \frac{d \varphi}{d t} =0,
    \hspace{0.5cm} \Rightarrow \hspace{0.5cm}   
    \varphi(t, \vc{r}(t)) = \const.
\end{equation*}
Давайте продолжать, пусть при $t=0$ есть задача Коши $\varphi(t=0,\, \vc{r}) = \varphi_0(\vc{r})$. Тогда
\begin{equation*}
    \varphi(0,\, \vc{r}(0)) = \varphi_0 (\vc{r}(0)),
    \hspace{0.5cm} \Rightarrow \hspace{0.5cm}
    \vc{r}(t) = \vc{v} t + \vc{r}_0.
\end{equation*}
Таким образом на характеристиках
\begin{equation*}
    \varphi(t,\, \vc{r}(t)) = \varphi_0 (\vc{r}_0) = \varphi_0(\vc{r}(t)-\vc{v} t).
\end{equation*}
% задачки про давление
% 


