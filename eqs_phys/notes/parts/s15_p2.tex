

\subsection*{Сингулярные уравнения}


Основой решения станет \textit{формула Сохоцкого}:
\begin{equation*}
		\text{v. p.} \int_{a}^{b} \frac{f(x)}{x-x_0} \d x = 
		\pm i \pi f(x_0) + \lim_{\varepsilon \to + 0} \int_{a}^{b} \frac{f(x)}{x-x_0 \pm i \varepsilon} \d x.
\end{equation*}
Сами уравнения могут появляться из соотношений Крамерса-Кронига
\begin{equation*}
	j(t) = \int_{-\infty}^{t} \sigma(t-t') E(t') \d t'.
\end{equation*}
Получается уравнения типа свёртки. 