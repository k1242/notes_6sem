\section{Автомодельные подстановки}

\textbf{Идея}. 
Если уравнения вида $\hat{L} u(\vc{r}, t) = \ldots$ однородно по $\vc{r}$ и $t$, и изотропно, то может помочь автомодельная подстановка. Например для уравнения теплопроводности:
\begin{equation*}
	u(t, \vc{r}) = \frac{1}{t^a} f\left(\tfrac{r}{t^b}\right)
	\hspace{5 mm} \colon \hspace{5 mm} 
	t \to \lambda t
	\hspace{0.25cm} \Rightarrow \hspace{0.25cm}
	u \to \lambda^{-a} u, \ r \to \lambda^{b} r.
\end{equation*}

Требуя, чтобы $\lambda$ сокращалась, получаем $b = \frac{1}{2}$. 
Восстановить $a$ в общем виде нельзя, но считая, что $\int_{\mathbb{R}^n} u \d V = \const$, получаем $\lambda^{bn} \lambda^{-a} = 1$, откуда $a = n/2$. 


\textbf{Асимптотика}. 
На самом деле автомобедельные решения соответствуют асимптотикам на больших временах. Знаем, что для уравнения теплопроводности $\langle r^2\rangle = 2 D t$. 


апа
