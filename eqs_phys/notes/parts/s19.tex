\section{Теория групп}


\textbf{Зоопарк}. Можем выделить циклические группы $C_n$
\begin{equation*}
	C_n = \{\mathbbm{1},\, r,\, r^2,\, \ldots,\, r^{n-1}\},
	\hspace{5 mm} 
	r^{n} = \1,
\end{equation*}
группу перестановок $S_n$
\begin{equation*}
	S_n = \left\{\sigma = \begin{pmatrix}
	    1 & \ldots & n  \\
	    \sigma(1) & \ldots & \sigma(n)  \\
	\end{pmatrix}\right\},
\end{equation*}
и группу симметрий правильного $n$-угольника $D_n$. 

\begin{to_def}
    \textit{Сопряженными} будем называть элементы $g \sim h$ такие, что $\exists r \in G,\, g = r h r^{-1}$.
\end{to_def}

Группу могли разбивать на классы сопряженности. Для $C_n$ каждый элемент становился представителем класса в силу того, что группа абелева. В $S_n$ подобными классами стали перестановки с одинаковой циклической структурой. Также в $S_n$ $\forall  \sigma$ раскладывается в циклы, а циклы в транспозиции, определенной оказывается величина четность $\sigma$. Для неё выполняется
\begin{equation*}
	\sign(\sigma_1 \cdot \sigma_2) = \sign(\sigma_1) \cdot \sign(\sigma_2),
	\hspace{5 mm} 
	\sign(a, b) = -1,
	\hspace{5 mm} 
	\sign(i_1,\, i_2,\, \ldots,\, i_d) = \left[\begin{aligned}
	    &+1, & d\not{\hspace{-1.5pt}\2}\, 2,\\
	    &-1, & d \,\2\, 2.
	\end{aligned}\right.
\end{equation*}

\begin{to_def}
    \textit{Четностью} перестановки называют количество пар $i < j$ таких, что $\sigma(i) > \sigma(j)$. 
\end{to_def}

В группе $D_n$ выделяются $r$ -- повороты на ${2\pi}/{n}$ и $s$ -- отражения относительно какой-то выбранной оси. Для $n \2 2 = 0$ получатся классы $\{r^b,\,  r^{n-b}\}$, $\{s,\,  s r^2,\, s r^4,\, \}$ и $\{s r,\, s r^3,\, s r^5,\, \ldots\}$. Для $n \hspace{-2pt}\not{\hspace{-4pt}\2}\, 2$ получится $\{r^b,\, r^{n-b}\}$ и $\{s,\, s r,\,  sr^2,\, \ldots\}$. 



\subsection{Сводка по теории представлений}

\begin{to_def}
    Пусть $|G|  < + \infty$, $V$ -- конечномерное линейное пространство над $\mathbb{C}$. Гомоморфизм $\rho\colon G \mapsto \,\text{GL}(V, \mathbb{C})$ -- \textit{представление}\footnote{
    	$\text{GL}(V, \mathbb{C})$ -- невырожденные матрицы. 
    } .	 Гомоморфизм -- операция уважающая произведение в группе: $\rho(g_1 g_2) = \rho(g_1) \cdot \rho(g_2)$ с обязательным условием $\rho(\1) = \mathbb{E}$. При этом $\dim V$ -- размерность представления. 
\end{to_def}


\begin{to_def}
    Пусть есть два представления группы $G$ $\rho_1$ и $\rho_2$. Тогда
    \textit{изоморфизм представления} называют изоморфизм $\varphi$ векторных пространств $V_1 \mapsto V_2$, коммутирующий с действием группы.
\end{to_def}

Сопоставляя $\varphi$ матрицу $A$, можем записать  условие коммутативности в виде $A \rho_1 (g) = \rho_2 (g) A$ или $\rho_1 (g) = A^{-1} \rho_2 (g) A$, то есть $\rho_1 (g)$ получается из $\rho_2 (g)$ заменой базиса:
\begin{equation*}
	\begin{tikzcd}[column sep=normal,row sep=normal]
	V_1 \arrow[r,"\varphi"] \arrow[d,swap,"\rho_1"] & V_2 \arrow[d,"\rho_2"] \\
	V_1 \arrow[r, "\varphi"] & V_2 
	\end{tikzcd}
\end{equation*}



\begin{to_def}
    Пусть для $G$ есть два представления $\rho_1[V_1]$ и $\rho_2[V_2]$. Тогда для $V = V_1 \oplus V_2$ отображение вида $\rho[V](g)  = \diag(\rho_1(g),\, \rho_2(g))$ тоже представление. 
\end{to_def}


\begin{to_def}
    Представление $\rho$ называется \textit{неприводимым}, если у него нет нетривиальных\footnote{
    	Отличных от $\{0\}$ и $V$. 
    } инвариантных подпространств.
\end{to_def}


\begin{to_thr}[теорема Машке]
    Пусть $|G| < + \infty$, $\rho$ -- представление, тогда $\rho$ разбивается в сумму неприводимых представлений.
\end{to_thr}

\begin{to_lem}
    Для конечной группы $|G| < + \infty$ представление $\rho$ унитаризуемо\footnote{
    	Существует базис в котром $\rho(g)$ унитарна $\forall  g \in G$. 
    } . 
\end{to_lem}

% A\D S A = S,  

% характеры представлений

\begin{to_def}
    Пусть $\rho$ представление $G \colon |G| < +\infty$. \textit{Характером представления} называется $\chi_V = \tr \rho(g)$. 
    Характеры изоморфных представлений совпадают\footnote{
		Также знаем, что $\chi_V(\1) = \dim V$, $\chi_{V_1 \oplus V_2} = \chi_{V_1} + \chi_{V_2}$, $\chi_V (g^{-1}) = \chi_V^* (g)$,
		$\chi_{V_1 \otimes V_2} = \chi_{V_1} \cdot \chi_{V_2}$.
    } . 
\end{to_def}



\begin{to_def}
    Пусть $U,\, V$ -- векторные пространства, тогда $V \otimes U$ называют $\{v \otimes u \}$. 
\end{to_def}


\subsection*{Характеры представлений сопряженных элементов}

Для сопряженных элементов верно, что их характеры равны! Таким образом характер представления -- функция $(g)$ постоянная на классе сопряженности. Пространство таких функций называется $\sub{\mathbb{C}}{class} (G)$. Классы сопряженности будем обозначать за $C_1,\, \ldots,\, C_k$.  Базис пространство легко построить
\begin{equation*}
	\gamma_i = \left\{\begin{aligned}
	    &1, &g \in C_i \\
	    &0, &g \notin C_i
	\end{aligned}\right.
	\hspace{10 mm} 
	i = 1,\, \ldots,\, k.
\end{equation*}
Логично ввести скалярное произведение над $\sub{\mathbb{C}}{class} (G)$:
\begin{equation*}
	\bk{\varphi}{\psi} = \frac{1}{|G|} \sum_{g \in G} \varphi(g) \psi(g)^* = \frac{1}{|G|} \sum_{i=1}^{k} |C_i| \varphi(h_i) \psi^* (h_i),
	\hspace{10 mm} 
	h_i \in C_i.
\end{equation*}



\begin{to_lem}
    Пусть у группы $G$ существуют $\rho_1(g),\, \ldots,\, \rho_m(g)$ -- неприводимые представления группы $G$, размерности $d_1,\, \ldots,\, d_m$
    и характерами\footnote{
    	Где каждый характер $\chi^{(i)} (g) \in \sub{\mathbb{C}}{class} (G)$. 
    }  $\chi^{(1)}(g),\, \ldots,\, \chi^{(m)}(g)$. Тогда $\bk{\chi^{(i)}}{\chi^{(j)}} = \delta_{ij}$. 
\end{to_lem}




\begin{center}
\begin{tabular}{c|cccc}
		&$C_1$ & $C_2$ & ... & $C_n$ \\
		\hline
		$\gamma_1$ & 1 & 0 & ... & 0 \\
		$\gamma_2$ & 0 & 1 & ... & 0 \\
		$\chi^{(i)}$ & $\#_1$ & $\#_2$ & ... & $\#_n$
	\end{tabular}    
\end{center}


% \newpage

% \subsect
% Построим табличку характеров для $S_4 \colon  |S_4| = 4! = 24$. 
% \begin{center}
% \Tsec{№3. Функционал гармонического осциллятора}

Рассмотрим действие, вида
\begin{equation*}
	S = \frac{1}{2} \int_{x'}^{x''} dt\ q(t) \hat{\Gamma} q(t) + 
	\int_{x'}^{x''}  dt\ j(t) q(t),
\end{equation*}
где верно, что $\hat{\Gamma} = \hat{\Gamma}\D$. 


Сделаем замену переменных в известном интеграле
\begin{equation*}
	Z[j] = \n \int \Dq \exp\left(\frac{i}{\hbar} S(t', t'', j)\right),
	\hspace{10 mm} 
	q(t) \to \tilde{q}(t) = q(t) - \G (t),
	\hspace{5 mm} 
	\hat{\Gamma} \G (t) = 0.
\end{equation*}
Нам поможет, что $\G(t) \hat{\Gamma} = \left(\hat{\Gamma} \G(t)\right)\D$, тогда
\begin{equation*}
	S = \frac{1}{2} \int dt\ \left(\tilde{q}(t) + \G(t)\right) \Gamma \left(\tilde{q}(t) + \G (t)\right) + \int dt\ j(t) \left(\tilde{q}(t) + \G(t)\right).
\end{equation*}
Подставляя в $Z$, находим
\begin{equation*}
	Z[j] = \n \int\Dq \exp\bigg(
		\frac{i}{\hbar} \int_{t'}^{t''} dt\ \ldots
	\bigg) = \n \exp\bigg(
		\frac{i}{\hbar} \int_{t'}^{t''} dt\ j(t) \G(t)
	\bigg) \int \mathcal{D} \tilde{q}\, 
	\exp\bigg(
		\frac{i}{\hbar} \int_{t'}^{t''} dt\ \bigg(
			\frac{1}{2} \tilde{q}(t) \hat{\Gamma} \tilde{q}(t) + j(t) \tilde{q}(t)
		\bigg)
	\bigg).
\end{equation*}
Теперь делаем замену $\tilde{q} \to \bar{q} = \tilde{q} + \hat{\Gamma}^{-1} j(t)$.  Тогда
\begin{equation*}
	frac{1}{2} \tilde{q}(t) \hat{\Gamma} \tilde{q}(t) + j(t) \tilde{q}(t)  = 
	\frac{1}{2} \bar{q} \hat{\Gamma} \bar{q} - \frac{1}{2} j(t) \hat{\Gamma}^{-1} j(t),
\end{equation*}
тогда для функционального интеграла получаем
\begin{equation*}
	Z[j] = \exp\bigg(
		\frac{i}{\hbar} \int_{t'}^{t''} dt\ \bigg(
			j(t) \G(t) - \frac{1}{2} j(t) \hat{\Gamma}^{-1} j(t)
		\bigg)
	\bigg) \times  \n \int \mathcal{D} \bar{q}\ \exp\bigg(
		\frac{i}{\hbar} \int_{t'}^{t''} dt\ \frac{1}{2} \bar{q} \hat{\Gamma} \bar{q}
	\bigg),
\end{equation*}
где последний множитель может быть представлен в виде $\exp\bigg(\frac{i}{\hbar} \mathcal{G}_0\bigg)$:
\begin{equation*}
	Z[j] =  \exp\bigg(
		\frac{i}{\hbar} \bigg[
			\mathcal{G}_0  + \ldots + \ldots
		\bigg]
	\bigg) = \exp\left(\frac{i}{\hbar} G[j]\right),
\end{equation*}
что и требовалось доказать. 
% \end{center}ion*{Примеры}






% как раскладывать приводимые представления по неприводимым

% если размер группы конечный, то представления унитаризуемы

% число неприводимых представлений в точности равно классу сопряженности

% характеры отрогональны по строкам и столбцам

% 