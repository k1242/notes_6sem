\section{Неделя V}

\subsection*{№ 8.1.5}

Найдём локализованное автомодельное решение уравнения
\begin{equation*}
	\partial_t u = \partial_x (u^{-1} \partial_x u) + \partial_x u.
\end{equation*}
Подстановкой \eqref{am}, получаем систему
\begin{equation*}
	-a -1 = -b-a = -2 b,
	\hspace{0.5cm} \Rightarrow \hspace{0.5cm}
	a = b = 1,
	\hspace{0.5cm} \Rightarrow \hspace{0.5cm}
	u(t, x) = \frac{1}{t} f\left(\tfrac{r}{t}\right).
\end{equation*}
Условие локализованности можем записать в виде
\begin{equation*}
	\int_{\mathbb{R}} u(t, x) \d x = \const, 
	\hspace{0.5cm} \Rightarrow \hspace{0.5cm}
	a = b,
\end{equation*}
так что можем говорить о существование локализованного автомодельного решения. 


Подставновка предполагаемого вида $u$ в исходное уравнение даёт выражение, вида
\begin{equation*}
	f(\xi)^2 + (\xi + 1) f(\xi) f'(\xi) + f''(\xi) = \frac{f'(\xi)^2}{f(\xi)},
\end{equation*}
где заменили $\frac{r}{t} \overset{\mathrm{def}}{=}  \xi$.

Вроде бы неплохой выглядит идея посмотреть на $f(x) = 1/g(x)$, тогда
\begin{equation*}
	g(x) g''(x)-g'(x)^2+x g'(x)+g'(x)-g(x) = 0.
\end{equation*}
Если внимательно на уравнение посмотреть, то можно предположить
\begin{equation*}
	g(x) = e^{\alpha x} + a x + b,
	\hspace{0.5cm} \Rightarrow \hspace{0.5cm}
	a = - \frac{1}{\alpha},
	\ \ 
	b = - \frac{1 + \alpha}{\alpha},
\end{equation*}
но это не выглядит как что-то осмысленное.



\subsection*{№ 9.1.2}

На интервале $(0,\, \infty)$ найдём резольвенту ядра $K(t, s) = \exp(t-2s)$. 
Пользуясь \eqref{FeqRI}, находим
\begin{equation*}
	\hat{K}^2 = \int_{0}^{\infty} K(t, p_1) K(p_1, s) \d p_1 = e^{t-2s} \int_{0}^{\infty} e^{-p_1}  = e^{t-2s},
	\hspace{0.5cm} \Rightarrow \hspace{0.5cm}
	R(t, s) = e^{t-2s} + \lambda e^{t-2s} + \ldots = \frac{e^{t-2s}}{1-\lambda},
\end{equation*}
что вполне соответствует факторизуемому ядру.


\subsection*{№ 9.1.4}

Найдём решение \eqref{Feq2} для $K = \exp\left(-t^2-s^2\right)$ и $g(t) = t^2$. 
Для начала найдём резольвенту, как
\begin{equation*}
	\hat{R} = \hat{K} + \lambda \hat{K}^2 + \ldots = 
	K(t, s) + \lambda K(t, s) \int_{-\infty}^{\infty} e^{-2 p_1^2} \d p_1  +
	\ldots = \frac{K(t, s)}{1 - \lambda \sqrt{\frac{\pi}{2}}}.
\end{equation*}
Тогда $f$ можем найти в виде
\begin{equation*}
	f(t) = g(t) + \lambda \hat{R}\, g(t) = t^2 + \lambda \int_{-\infty}^{\infty} \frac{e^{-t^2 - s^2}}{1 - \lambda \sqrt{\frac{\pi}{2}}} s^2 \d s = 
	t^2 + \frac{\lambda}{\sqrt{\frac{2}{\pi}}-\lambda} \frac{e^{-t^2}}{\sqrt{2}},
\end{equation*}
что и является искомым решением.








\subsection*{№ 9.1.7}

Найдём решение \eqref{Feq2} для $K = (1 + ts)e^{-2t -s}$ и $g(t) = e^{t/2}$. 
% Найдём решение \eqref{Feq2} для $K = \exp\left(-t^2-s^2\right)$ и $g(t) = t^2$. 
Для этого находим
\begin{equation*}
	M = \left(
\begin{array}{cc}
 \frac{1}{3} & \frac{1}{9} \\
 \frac{1}{9} & \frac{2}{27} \\
\end{array}
\right),
\hspace{5 mm} 
\Phi = \{2,4\},
\hspace{5 mm} 
(\mathbbm{1} - \lambda \hat{M})^{-1} = \frac{1}{(\lambda -33) \lambda +81}\left(
\begin{array}{cc}
81-6 \lambda  & 9 \lambda  \\
 9 \lambda  & -27 (\lambda -3) \\
\end{array}
\right),
\end{equation*}
а значит
\begin{equation*}
	\vc{C} = \frac{1}{(\lambda -33) \lambda +81}\left\{6 (4 \lambda +27),324-90 \lambda \right\},
\end{equation*}
откуда находим выражение для $\varphi(t)$:
\begin{equation*}
	\varphi(t) = e^{t/2} + 6 \lambda  \frac{4 \lambda +(54-15 \lambda ) t+27}{(\lambda -33) \lambda +81} e^{-2 t}.
\end{equation*}
% Для начала найдём резольвенту:
% \begin{equation*}
% 	\int_{0}^{\infty} K(t, p_1) K(p_1, s) \d p_1 = 
% 	\sum_i K_i(t,s) \int_{0}^{\infty} K_i (p_1, p_1) \d p_1 = 
% 	\alpha\, e^{-2t -s} + \beta ts\, e^{-2t-s},
% \end{equation*}
% где $K(t,s) = e^{-2 t}e^{-s} + (t e^{-2t}) (s e^{-s})$ -- факторизуемо, и коэффициенты 
% \begin{equation*}
% 	\alpha = \int_{0}^{\infty} e^{-3 p_1} \d p_1 = \frac{1}{3},
% 	\hspace{10 mm} 
% 	\beta = \int_{0}^{\infty} p_1^2 e^{-3 p_1} \d p_1 = \frac{2}{27}.
% \end{equation*}
% Аналогично последующие слагамые
% \begin{equation*}
% 	\hat{K}^3 = \alpha^2\, e^{-2t-s} + \beta^2 ts\, e^{-2t-s}.
% \end{equation*}
% Cчитая обратную матрицу, находим
% \begin{equation*}
% 	R(t, s) = e^{-2t-s}\left(\frac{1}{1-\frac{1}{3}\lambda} + \frac{ts}{1 - \frac{2}{27}\lambda}\right).
% \end{equation*}
% Осталось найти $f$ в виде
% \begin{equation*}
% 	f(t) = e^{t/2} + \int_{0}^{\infty} R(t,s) g(s) \d s = e^{t/2} + 6 \lambda e^{-2 t} \left(\frac{1}{3-\lambda }+\frac{18 t}{27-2 \lambda }\right).
% \end{equation*}





\subsection*{№ 9.1.6}

Найдём решение \eqref{Feq2} для $K = (1 + ts)$ и $g(t) = t^2$. 
% Для начала найдём резольвенту, аналогично двум предыдущим номерам
% \begin{equation*}
% 	R(t, s) = \frac{1}{1-\lambda} + \frac{t^2 s^2}{1 - \frac{1}{5}\lambda}.
% \end{equation*}
% Теперь можем найти $f(t)$:
% \begin{equation*}
% 	f(t) = g(t) + \lambda \int_{0}^{1} R(t,s) \d s = t^2 + \lambda \left(
% 		\frac{1}{3}\frac{1}{1- \lambda }+\frac{t^2}{5-\lambda }
% 	\right) =  \frac{t^2}{1-\frac{1}{5}\lambda} + \frac{\lambda/3}{1-\lambda},
% \end{equation*}
% где уже добавка не затухает со временем, в отличии от предыдущих задач.
Для этого находим
\begin{equation*}
	M = \left(
\begin{array}{cc}
 1 & \frac{1}{2} \\
 \frac{1}{2} & \frac{1}{3} \\
\end{array}
\right),
\hspace{5 mm} 
\Phi = \{1/3, 1/4\},
\hspace{5 mm} 
(\mathbbm{1} - \lambda \hat{M})^{-1} = 
\frac{1}{12 + \lambda (\lambda-16)}
\left(
\begin{array}{cc}
 -4 (\lambda -3) & 6 \lambda  \\
 6 \lambda  & -12 (\lambda -1) \\
\end{array}
\right)
,
\end{equation*}
а значит
\begin{equation*}
	\vc{C} = \frac{1}{12 + \lambda (\lambda-16)}\left\{\frac{\lambda }{6}+4,3-\lambda \right\},
\end{equation*}
откуда находим выражение для $\varphi(t)$:
\begin{equation*}
	\varphi(t) = t^2 + \lambda \left(\frac{\lambda +24}{6 (\lambda -16) \lambda +72}- t \frac{\lambda -3}{(\lambda -16) \lambda +12}\right).
\end{equation*}

% Так как $l = 1$ см, то рассматриваем дифрацию Брегга:
% \begin{equation*f}
% 	\sin \theta = \frac{\lambda f}{2 n v} \approx 2.95 \cdot 10^{-3} \approx \theta,
% \end{equation*}
% значит расходимость будет порядка $2 \theta \approx	6 \cdot 10^{-3}$. 

