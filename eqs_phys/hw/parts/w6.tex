\section{Неделя VI}

\subsection*{9.2.4}

Решим уравнение, вида
\begin{equation*}
	\int_{0}^{t} \varphi(t-s) s \varphi(s) = f(t) = t,
	\hspace{0.5cm} \Rightarrow \hspace{0.5cm}
	\varphi(p) (-\partial_p) \varphi(p) = f(p).
\end{equation*}
Найдём некоторое решение (хотелось бы чтобы образ затухал к 0 на $\infty$)
\begin{equation*}
	- \varphi \varphi' = - \frac{1}{2} (\varphi^2)' = \frac{1}{p^2},
	\hspace{0.5cm} \Rightarrow \hspace{0.5cm}
	\varphi^2 = - 2 \int \frac{1}{p^2} = \frac{2}{p},
	\hspace{0.5cm} \Rightarrow \hspace{0.5cm}
	\varphi(p) = \pm \sqrt{\frac{2}{p} + C},
\end{equation*}
где выбираем $C=0$. Тогда
\begin{equation*}
	\varphi(t) = \pm \sqrt{\frac{2}{\pi t}}.
\end{equation*}


\subsection*{9.3.2}

Рассмотрим сингулярное уравнение
\begin{equation*}
	\lambda \varphi(x) + \vpint_{-\infty}^{+\infty} \frac{\varphi(y) \d y}{y-x} = f(x) = \frac{1}{x^2 + 1}.
\end{equation*}
Решение уравнения знаем:
\begin{equation*}
	\varphi(x) = \frac{1}{\lambda + i \pi} f(x) - \frac{1}{\lambda^2 + \pi^2} \int_{-\infty}^{+\infty} \frac{f(y) \d y}{y-x + i \varepsilon}.
\end{equation*}
Замыкая в верхней полуплоскости дугу, находим
\begin{equation*}
	\varphi(x) = \frac{\pi x + \lambda}{(1 + x^2) (\pi^2 + \lambda^2)}.
\end{equation*}



\subsection*{9.3.3}

Рассмотрим сингулярное уравнение
\begin{equation*}
	\vpint_{-1}^{1} \frac{\varphi(y)}{y-x} \d y = f(x) = x,
\end{equation*}
для которого выполняется свойство ортогональности $\int_{-1}^{1} \frac{x}{\sqrt{1-x^2}} \d x = 0$. Тогда можем записать ответ
\begin{equation*}
	\varphi(x) = \frac{1}{\pi i} \left(
		f(x) + \frac{\sqrt{1-x^2}}{\pi i} \int_{-1}^{1} \frac{f(y) \d y}{\sqrt{1-y^2} (y-x + i \varepsilon)}
	\right) = - \frac{\sqrt{1-x^2}}{\pi},
\end{equation*}
что подходит.

