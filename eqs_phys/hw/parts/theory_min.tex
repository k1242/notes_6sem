\section*{ТеорМин №1}
\addcontentsline{toc}{section}{ТеорМин №1}

\textbf{Излучение}. Волновое уравнение с источником:
\begin{equation}
    (\partial_t^2 - c^2 \nabla^2) u = \chi,
    \label{field_eq}
\end{equation}
c законом дисперсии $\varpi = cq$. 

Функция Грина оператора $\partial_t^2 - c^2 \nabla^2$:
\begin{equation*}
    G(t, r) = \frac{\theta(t)}{4 \pi c r} \delta(r - ct),
\end{equation*}
а значит выражение для поля:
\begin{equation}
    u(t, \vc{r}) = \frac{1}{4 \pi c^2} \int \frac{d^3 r_1}{R} \ \chi(t-R/c, \vc{r}_1),
    \label{field_sol}
\end{equation}
где $R = |\vc{r} - \vc{r}_1|$. 


\textbf{Уравнение диффузии}. 
Уравнение диффузии:
\begin{equation}
    \left(\partial_t - \nabla^2\right) u = 0,
    \label{difeq}
\end{equation}
решение которого может быть найдено в виде:
\begin{equation}
    u(t, \vc{x}) = \int_{\mathbb{R}^d} \frac{d y_1 \ldots d y_d}{(4 \pi t)^{d/2}} \exp\left(- \frac{(\vc{x}-\vc{y})^2}{4 t}\right) u_0(\vc{y}).
    \label{difsol}
\end{equation}
Асимтотики могут быть найдены в виде
\begin{equation}
    u(t, \vc{x}) \approx \frac{A}{(4 \pi t)^{d/2}} \exp\left(- \frac{\vc{x}^2}{4 t}\right),
    \hspace{5 mm} 
    A = \int_{ \mathbb{R}^d} dy_1 \ldots dy_d \ u_0(\vc{y}).
    \label{difas1}
\end{equation}
При $A = 0$ асимтотика будет соответствовать
\begin{equation}
    u(t, \vc{x}) \approx  \frac{\vc{B} \cdot \vc{x}}{(4 \pi t)^{d/2+1}} \exp\left(
        - \frac{\vc{x}^2}{4 t}
    \right), \hspace{5 mm} 
    \bar{B} = 2 \pi \int_{ \mathbb{R}^d} dy_1 \ldots dy_d \ \vc{y}\, u_0(\vc{y}),
    \label{difas2}
\end{equation}
где асимтотики имеют место при $t \gg l^2$, $l$ -- масштаб на котором локализовано поле.


\textbf{Меленные переменные}. Рассмотрим прозвольное возмущение гармонического осциллятора:
\begin{equation}
    \left(\partial_t^2 + \omega_0^2\right) x(t) = \varepsilon f(t, x, \dot{x}).
    \label{sloweq}
\end{equation}
Приближенно (до $o(\varepsilon)$) можем методом Боголюбова-Крылова найти
решение в виде
\begin{equation}
    x(t) = A(t) \cos(\omega_0 t + \varphi(t)),
    \label{sloweqview}
\end{equation}
где зависимость от времени амплитуды и фазы определяестся уравнениями
\begin{align}
    \partial_t A(t) &= \frac{1}{2\pi \omega_0}\int_{\omega_0 t-\pi}^{\omega_0 t+\pi} f(\tau, x, \dot{x}) \cos\left(\omega_0 \tau + \varphi(t)\right) \d (\omega_0\tau), 
    \label{slowA}
    \\
    \partial_t \varphi(t) &= \frac{-1}{2 \pi A \omega_0} \int_{\omega_0 t-\pi}^{\omega_0 t + \pi} f(\tau, x, \dot{x}) \sin(\omega_0 \tau + \varphi(t)) \d (\omega_0 \tau).
    \label{slowphi}
\end{align}

\textbf{Огибающая}. Уравнение на огибающую может быть найдено в виде решения уравнения
\begin{equation*}
    \partial_t \psi + \vc{v} \cdot \nabla \psi = \frac{i \varepsilon}{\omega} e^{i \theta} \int_{0}^{2\pi} \frac{d \varphi}{2 \pi} e^{-i \varphi} f.
\end{equation*}