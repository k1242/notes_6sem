\section*{ТеорМин №1}
\addcontentsline{toc}{section}{ТеорМин №1}

\textbf{Излучение}. Волновое уравнение с источником:
\begin{equation}
    (\partial_t^2 - c^2 \nabla^2) u = \chi,
    \label{field_eq}
\end{equation}
c законом дисперсии $\varpi = cq$. 

Функция Грина оператора $\partial_t^2 - c^2 \nabla^2$:
\begin{equation*}
    G(t, r) = \frac{\theta(t)}{4 \pi c r} \delta(r - ct),
\end{equation*}
а значит выражение для поля:
\begin{equation}
    u(t, \vc{r}) = \frac{1}{4 \pi c^2} \int \frac{d^3 r_1}{R} \ \chi(t-R/c, \vc{r}_1),
    \label{field_sol}
\end{equation}
где $R = |\vc{r} - \vc{r}_1|$. 


\textbf{Уравнение диффузии}. 
Уравнение диффузии:
\begin{equation}
    \left(\partial_t - \nabla^2\right) u = 0,
    \label{difeq}
\end{equation}
решение которого может быть найдено в виде:
\begin{equation}
    u(t, \vc{x}) = \int_{\mathbb{R}^d} \frac{d y_1 \ldots d y_d}{(4 \pi t)^{d/2}} \exp\left(- \frac{(\vc{x}-\vc{y})^2}{4 t}\right) u_0(\vc{y}).
    \label{difsol}
\end{equation}
Асимтотики могут быть найдены в виде
\begin{equation}
    u(t, \vc{x}) \approx \frac{A}{(4 \pi t)^{d/2}} \exp\left(- \frac{\vc{x}^2}{4 t}\right),
    \hspace{5 mm} 
    A = \int_{ \mathbb{R}^d} dy_1 \ldots dy_d \ u_0(\vc{y}).
    \label{difas1}
\end{equation}
При $A = 0$ асимтотика будет соответствовать
\begin{equation}
    u(t, \vc{x}) \approx  \frac{\vc{B} \cdot \vc{x}}{(4 \pi t)^{d/2+1}} \exp\left(
        - \frac{\vc{x}^2}{4 t}
    \right), \hspace{5 mm} 
    \bar{B} = 2 \pi \int_{ \mathbb{R}^d} dy_1 \ldots dy_d \ \vc{y}\, u_0(\vc{y}),
    \label{difas2}
\end{equation}
где асимтотики имеют место при $t \gg l^2$, $l$ -- масштаб на котором локализовано поле.


\textbf{Уравнение диффузии (с накачкой)}.  При наличии правой части:
\begin{equation*}
    (\partial_t - \nabla^2) u = \varphi,
\end{equation*}
можем найти функцию Грина для оператора $\partial_t - \nabla^2$
\begin{equation*}
    u(t, \vc{x}) =  \int G(t-\tau, \vc{x}-\vc{y}) \varphi(t, \vc{y}) \d \tau \d^d \vc{y},
    \hspace{10 mm} 
    G(t, \vc{r}) = \frac{\theta(t)}{(4 \pi t)^{d/2}} \exp\left(- \frac{r^2}{4 t}\right).
\end{equation*}

\textbf{Меленные переменные}. Рассмотрим прозвольное возмущение гармонического осциллятора:
\begin{equation}
    \left(\partial_t^2 + \omega_0^2\right) x(t) = \varepsilon f(t, x, \dot{x}).
    \label{sloweq}
\end{equation}
Приближенно (до $o(\varepsilon)$) можем методом Боголюбова-Крылова найти
решение в виде
\begin{equation}
    x(t) = A(t) \cos(\omega_0 t + \varphi(t)),
    \label{sloweqview}
\end{equation}
где зависимость от времени амплитуды и фазы определяестся уравнениями
\begin{align}
    \partial_t A(t) &= \frac{1}{2\pi \omega_0}\int_{\omega_0 t-\pi}^{\omega_0 t+\pi} f(\tau, x, \dot{x}) \cos\left(\omega_0 \tau + \varphi(t)\right) \d (\omega_0\tau), 
    \label{slowA}
    \\
    \partial_t \varphi(t) &= \frac{-1}{2 \pi A \omega_0} \int_{\omega_0 t-\pi}^{\omega_0 t + \pi} f(\tau, x, \dot{x}) \sin(\omega_0 \tau + \varphi(t)) \d (\omega_0 \tau).
    \label{slowphi}
\end{align}

% \textbf{Огибающая}. Уравнение на огибающую может быть найдено в виде решения уравнения
% \begin{equation*}
%     \partial_t \psi + \vc{v} \cdot \nabla \psi = \frac{i \varepsilon}{\omega} e^{i \theta} \int_{0}^{2\pi} \frac{d \varphi}{2 \pi} e^{-i \varphi} f.
% \end{equation*}


\textbf{Уравнение Хопфа}.  В акустике естественно возникает уравнение Хопфа:
\begin{equation*}
    \partial_t u + u\, \partial_x u = 0.
\end{equation*}
Решение может быть найдено в виде
\begin{equation*}
    x(t) = x_0 + u_0 (x_0) t, \hspace{5 mm} 
    u(x(t), t) = c(x_0) = u_0 (x_0).
\end{equation*}
где сначала разрешаем уравнение $c = u_0 (x_0)$ относительно $c = c(x_0)$, а потом разрешаем уравнение на $x(t)$ относительно $c = c(x(t), t)$. 
Зная, что $u(x(t), t) = c(x(t), t)$, находим $u(x, t) = c(x, t)$. 

\textbf{Уравнение Хопфа (с накачкой)}. Добавим к уравнению накачку:
\begin{equation*}
    \partial_t u + u\, \partial_x u = f(x, t).
\end{equation*}
Система может быть сведена к
\begin{equation*}
    \left\{\begin{aligned}
        \dot{u} &= f(t, x(t)) \\
        \dot{x} &= u(t, x(t))
    \end{aligned}\right.
    \hspace{5 mm} \Leftrightarrow \hspace{5 mm} 
    \ddot{x} = f(x, t),
    \hspace{0.5cm} \Rightarrow \hspace{0.5cm}
    x(t) = x(t, x_0, \dot{x}_0),
\end{equation*}
где $\dot{x}_0 = u_0(x_0)$. Сначала разрешаем уравнение $x(t)$ относительно $x_0 = x_0(t, x)$, а потом подставляем этот $x_0$ в $u(t, x) = \dot{x}(t, x_0(t,x))$, что и является решением исходной задачи.



\textbf{Уравнение Бюргерса}. Добавим диссипацию в уравнение Хопфа:
\begin{equation*}
    \partial_t u + u\, \partial_x u = \partial_x^2 u,
\end{equation*}
так получим уравнение Бюргерса.

Заметим, что преобразование Коула-Хопфа
\begin{equation*}
    \psi = \exp\left(- \tfrac{1}{2}h\right),
    \hspace{5 mm} 
    u = \partial_x h,
    \hspace{0.5cm} \Rightarrow \hspace{0.5cm}
    (\partial_t - \partial_x^2)\psi = 0.
\end{equation*}
Имея начальные условия для $\psi_0(x)$, можем найти
\begin{equation*}
    \psi(t, x) = \int_{\mathbb{R}} \psi_0 (y) \frac{\theta(t)}{\sqrt{4 \pi t}} \exp\left(-\frac{(x-y)^2}{4t}\right) \d y,
\end{equation*}
откуда находим решение
\begin{equation*}
    u(t, x) =  -2 \partial_x \ln \psi(t, x).
\end{equation*}



