\section*{ТеорМин №3}
\addcontentsline{toc}{section}{ТеорМин №3}


\textbf{Теория групп}. \textit{Сопряженными} (из одного  \textit{класса сопряженности}) называть элементы $g \sim h$ такие, что $\exists r \in G,\, g = r h r^{-1}$. Далее классы сопряженности будем обозначать за $C_1,\, \ldots,\, C_k$, элементы в них за $h_i \in C_i$.

Циклическая группа $C_n$
\begin{equation*}
	C_n = \{\mathbbm{1},\, r,\, r^2,\, \ldots,\, r^{n-1}\},
	\hspace{5 mm} 
	r^{n} = \1.
\end{equation*}
Для $C_n$ каждый элемент становился представителем класса сопряженности в силу того, что группа абелева.

Группа перестановок $S_n$
\begin{equation*}
	S_n = \left\{\sigma = \begin{pmatrix}
	    1 & \ldots & n  \\
	    \sigma(1) & \ldots & \sigma(n)  \\
	\end{pmatrix}\right\},
\end{equation*}
разбивается на классы сопряженности с одинаковой циклической структурой:
\begin{equation*}
	S_2 \to \{\1\},\,  \{(a, b)\},
	\hspace{5 mm} 
	S_3 \to \{\1\},\,  \{(a, b)\},\,  \{(a, b, c)\},
	\hspace{5 mm} 
	S_4 \to \{\1\},\,  \{(a, b)\},\,  \{(a, b, c)\},\,  \{(a, b, c, d)\},\,  \{(a, b) (c, d)\}.
\end{equation*}
Также в $S_n$ $\forall  \sigma$ раскладывается в циклы, а циклы в транспозиции, определенной оказывается величина четность $\sigma$. Для неё выполняется
\begin{equation*}
	\sign(\sigma_1 \cdot \sigma_2) = \sign(\sigma_1) \cdot \sign(\sigma_2),
	\hspace{5 mm} 
	\sign(a, b) = -1,
	\hspace{5 mm} 
	\sign(i_1,\, i_2,\, \ldots,\, i_d) = \left[\begin{aligned}
	    &+1, & d\not{\hspace{-1.5pt}\2}\, 2,\\
	    &-1, & d \,\2\, 2.
	\end{aligned}\right.
\end{equation*}
\textit{Четностью} перестановки называют количество пар $i < j$ таких, что $\sigma(i) > \sigma(j)$. 

Группа $D_n$ симметрий правильного $n$-угольника состоит из  $r$ -- поворотов на ${2\pi}/{n}$ и $s$ -- отражений относительно какой-то выбранной оси.
Для $n \2 2 = 0$ получатся классы сопряженности  $\{r^b,\,  r^{n-b}\}$, $\{s,\,  s r^2,\, s r^4,\, \}$ и $\{s r,\, s r^3,\, s r^5,\, \ldots\}$. Для $n \not{\hspace{-1.5pt}\2}\, 2$ получится $\{r^b,\, r^{n-b}\}$ и $\{s,\, s r,\,  sr^2,\, \ldots\}$. 


\textbf{Теория представлений}. Далее работаем с конечными группами $|G| < + \infty$. Элемент группы обознаем за $g \in G$. \textit{Представление} группы определяют как гомоморфизм $\rho\colon G \mapsto \,\text{GL}(V, \mathbb{C})$ (невырожденные матрицы). 


\textit{Характером представления} $\rho$ называют $\chi[V] = \tr \rho(g)$ для $g \in G$. Характеры изморфных представлений совпадают, а также
\begin{equation*}
	\chi[V](\1) = \dim V,
	\hspace{5 mm} 
	\chi[V_1 \oplus V_2]= \chi[V_1] + \chi[V_2],
	\hspace{5 mm} 
	\chi[V] (g^{-1}) = \chi^*[V] (g),
	\hspace{5 mm} 
	\chi[V_1 \otimes V_2] = \chi[V_1] \cdot \chi[V_2].
\end{equation*}

Стараемся решить задачу о разложение приводимого представления по неприводимым. 
Представление $\rho$ называется \textit{неприводимым}, если у него нет нетривиальных (отличных от $\{0\}$ и $V$) инвариантных подпространств. По теореме Машке $\forall \rho$ конечной группы $G$ разбивается на сумму неприводимых представлений. Всякое представление \textit{унитаризуемо}. 


Для характеров определим скалярное произведение $\bk{\chi^{(i)}}{\chi^{(j)}}$:
\begin{equation*}
\bk{\varphi}{\psi} = 
	\frac{1}{|G|} \sum_{g \in G} \varphi(g) \bar{\psi}(g) = \frac{1}{|G|} \sum_{i=1}^{k} |C_i| \varphi(h_i) \bar{\psi} (h_i).
\end{equation*}
Характеры ортогональны по строкам и столбцам:
\begin{equation*}
	\bk{\chi^{(i)}}{\chi^{(j)}} = \delta_{ij},
	\hspace{10 mm} 
	\sum_{n=1}^{k} \chi^{(n)} (h_i) \bar{\chi}^{(n)} (n_j) = \delta_{ij} \frac{|G|}{|C_i|}.
\end{equation*}


Число неприводимых представлений равно числу классов сопряженности. Все неприводимые представления абелевой группы одномерны, что является следствием теоремы Бернсайда:
\begin{equation*}
	\sum_{i=1}^{n} d_i^2 = |G|,
\end{equation*}
где $d_i$ -- размерность $i$-го представления. 
\textit{Критерием неприводимости} является $\bk{\chi}{\chi} = 1$,
тогда разложение на неприводимые: $\chi = a_1 \chi^{(1)} + \ldots + a_n \chi^{(n)}$, где $a_i = \bk{\chi}{\chi^{(i)}}$.

\textbf{Таблицы неприводимых представлений}
Построение для $C_n$ тривиально в силу абелевости группы. Каждый элемент представим в виде $\sqrt[n]{1}$. Построение производим с учетом свойства $\rho(r^k) = \rho(r)^k$. 

Теперь для $D_4$ размера $|D_4| = 2 \times n = 8$, будут классы сопряженности $\{\1\}$, $\{r^2\},\ \{r,\, r^3\}$ и $\{s,\, s r^2\}$, $\{s r,\, s r^3\}$. 
\begin{center}
\begin{tabular}{ccccc}
\toprule
 $\1$| 1 &  $r,\, r^3$| 2 &  $r^2$| 1 &  $s,\, s r^2$| 2 &  $s r,\, s r^3$| 2  \\
\midrule
    1 &           1 &      1 &             1 &                1 \\
    1 &           1 &      1 &            -1 &               -1 \\
    1 &          -1 &      1 &             1 &               -1 \\
    1 &          -1 &      1 &            -1 &                1 \\
    2 &           0 &     -2 &             0 &                0 \\
\bottomrule
\end{tabular}

\end{center}
Всегда есть тривиальное представление. Также из теоремы Бернсайда находим первый столбец. 

По сохранению или смене ориентация базиса можем сопоставить $\pm 1$ соответсвующием классам.  Важно помнить, что $(s r^k)^2 = \1$ и $(r^k)^4 = \1$, откуда знаем одномерные представления $\rho(s r^k) = \pm 1$ и $\rho(r^k) = \sqrt[4]{1} = \pm i, \pm 1$, откуда достраиваем одномерные представления. 

При построение $D_7$ будет важно вспомнить про сопоставление матриц поворота двухмерным представлениям, по которым найдём элементы таблицы характеров, как след соответсвующей матрицы. 

Построим табличку характеров для $S_3 \colon  |S_3| = 3! = 6$. Также из теоремы Бернсайда находим первый столбец. Для второй строчки всегда есть \textit{знаковое} представление. 
\begin{center}
\begin{tabular}{rrr}
\toprule
 $e$| 1 &  $(a, b)$| 3 &  $(a, b, c)$| 2 \\
\midrule
      1 &            1 &               1 \\
      1 &           -1 &               1 \\
      2 &            0 &              -1 \\
\bottomrule
\end{tabular}

\end{center}

Построим табличку характеров для $S_4 \colon  |S_4| = 4! = 24$. 
\begin{center}
\Tsec{№1. Линейный эффект Штарка в атоме водорода}

\red{Переписать интегралы в сферических гармониках}.
Теперь рассмотрим возмущение, вила
\begin{equation*}
    \hat{H} = \frac{\hat{p}^2}{2m} - \frac{e^2}{r} + \hat{V},
    \hspace{5 mm} 
    \hat{V} = - e E \hat{z}
\end{equation*}
Известно, что $n=2$, тогда вырождение $n^2 = 4$. Можем явно выписать несколько функций
\begin{align*}
    \ket{200} &= \frac{2}{\sqrt{4 \pi}} \left(\frac{z}{2a}\right)^{3/2} e^{-r/2a} \left(
        1 - \frac{r}{2a}
    \right), \\
    \ket{210} &= \sqrt{\frac{3}{4\pi}} \cos \theta \left(\frac{1}{2a}\right)^{3/2} e^{-r/2a} \frac{r}{\sqrt{3} a}, 
\end{align*}
а для $\ket{211}$ и $\ket{21-1}$ важно только что есть фактор $e^{im\varphi}$.

Действительно,
\begin{equation*}
    \bk{21m}[\hat{V}]{21m'} = 0,
    \hspace{5 mm} 
    m,\, m' = \pm 1.
\end{equation*}
Осталось посчитать
\begin{equation*}
    \kappa \overset{\mathrm{def}}{=}  \bk{200}[\hat{V}]{210} = \int_{\mathbb{R}^3}
        \ldots
    d^3 \vc{r} = 3 e E {a}.
\end{equation*}
Получилось матрица  ненулевыми коэффициентами только в первом блоке 2 на 2:
\begin{equation*}
    \hat{V} = \begin{pmatrix}
        0 & \kappa  \\
        \kappa & 0  \\
    \end{pmatrix},
    \hspace{10 mm} 
    \lambda_1 = \kappa, \hspace{5 mm} 
    \lambda_2 = - \kappa,
    \hspace{5 mm}   
    \lambda_3 =  \lambda_4 = 0.
\end{equation*}
Решая секулярное уравнение, находим
\begin{equation*}
    E_2 =  - \frac{\text{Ry}}{2^2},
    \hspace{5 mm} 
    \left[\hat{H} + \hat{V} - (E_2 \pm \kappa) \mathbbm{1}\right] \ket{\psi} = 0,
    \hspace{0.5cm} \Rightarrow \hspace{0.5cm}
    \vc{c}_+ = \frac{1}{\sqrt{2}} \left(1,\, 1,\, 0,\, 0\right),
    \hspace{5 mm} 
    \vc{c}_- = \frac{1}{\sqrt{2}} \left(1,\, -1,\, 0,\, 0\right).
\end{equation*}
Энергии расщепления
\begin{equation*}
    E^+ = E_2^{(0)} + \kappa,
    \hspace{5 mm} 
    E^- = E_{2}^{(0)} -\kappa.
\end{equation*}



\end{center}
Тут важно посмотреть на отображение $e_1 + e_2 + e_3 + e_4$, построив $\chi[\mathbb{C}^4]$, значениях характеров которой можем восстановить по количеству неподвижных точек (4, 2, 1, 0, 0). Неприводимое представление можем получить в виде $\chi[\mathbb{C}^4] - \chi^{(1)}$.
Также может помочь тензорное произведение представлений. 

\textbf{Преобразование Меллина}. Для функции $g(x)$ такую, что $g(x) = O(x^{-\alpha})$ при $x \to 0$ и $g(x) = x^{-\beta}$ при $x \to + \infty$ можем определить
\textit{преобразование Меллина}
\begin{equation*}
	G(\lambda) = \int_{0}^{\infty} g(x) x^{\lambda-1} \d x,
\end{equation*}
определенного в полосе $\alpha < \Re \lambda < \beta$. Обратное преобразование может быть найдено в виде
\begin{equation*}
	g(x) = \int_{C - i \infty}^{C + i \infty} G(\lambda) x^{-\lambda} \frac{d \lambda}{2 \pi i},
\end{equation*} 
для $\alpha  <C < \beta$. 

Для вычисления инетгралов бывает удобно воспользоваться сверточным свойством преобразования Меллина
\begin{equation*}
	\int_{-\infty}^{+\infty} f(x) g(x) x^{\lambda-1} \d x = 
	\int_{C_f - i \infty}^{C_f + i \infty} F(\lambda_f) G(\lambda-\lambda_f) \frac{\d \lambda_f}{2\pi i}  = \int_{C_g - i \infty}^{C_g + i \infty}  F(\lambda-\lambda_g)  G(\lambda_g) \frac{\d \lambda_g}{2\pi i},
\end{equation*}
где $\alpha_f + \alpha_g < \Re \lambda < \beta_f + \beta_g$. 
В частности, при допустимом $\lambda = 1$, получаем
\begin{equation*}
	\int_{0}^{\infty} f(x) g(x) \d x = 
	\int_{C_f - i \infty}^{C_f + i \infty} F(\lambda_f) G(1-\lambda_f) \frac{d \lambda_f}{2 \pi i}.
\end{equation*}

Приведем некоторый зоопарк по преобразованию Меллина:
\begin{gather*}
	e^{-x} \overset{M}{\to}  \Gamma(\lambda),
	\hspace{5 mm} 
	\frac{1}{1 + a x^n} \overset{M}{\to} \frac{\pi  a^{-\frac{\lambda }{n}}}{n \sin \left(\frac{\pi  \lambda }{n}\right)},
	\hspace{5 mm} 
	\frac{1}{\sqrt[m]{1 + x^n}} \overset{M}{\to}
	\frac{\Gamma \left(\frac{\lambda }{n}\right) \Gamma \left(\frac{1}{m}-\frac{\lambda }{n}\right)}{n \Gamma \left(\frac{1}{m}\right)}, 
	\hspace{5 mm} 
	\frac{1}{1-x}\overset{M}{\to}  \pi  \cot (\pi  \lambda ),
	\\
	\frac{1}{\sqrt{a x^2+b x}}
	\overset{M}{\to}
	\frac{\Gamma (1-\lambda ) \Gamma \left(\lambda -\frac{1}{2}\right) \left(\frac{a}{b}\right)^{\frac{1}{2}-\lambda }}{\sqrt{\pi b}},
	\hspace{5 mm} 
	x^n \overset{M}{\to} 2 \pi  \delta (i (n+\lambda )), 
	\hspace{5 mm} 
	\frac{1}{1 + e^{\alpha x}} \overset{M}{\to}  \left(1-2^{1-\lambda }\right) \alpha ^{-\lambda } \Gamma (\lambda ) \zeta (\lambda ),
	\\ 
	\sin x \overset{M}{\to} \sin \left(\frac{\pi  \lambda }{2}\right) \Gamma (\lambda )
	, \hspace{5 mm}  
	\cos x \overset{M}{\to} \cos \left(\frac{\pi  \lambda }{2}\right) \Gamma (\lambda ).
\end{gather*}
% А также несколько обратных преобразований
% \begin{gather*}
% 	\frac{\Gamma(\lambda)}{\Gamma(\lambda + \beta)} = - \frac{(1-x)^{\beta-1}}{\Gamma(\beta)} (\theta(|x|-1)-1).
% \end{gather*}

\textbf{Гамма функция}. Полезно будет вспомнить, что
\begin{equation*}
	\Gamma(z) = \int_{0}^{\infty} t^{z-1} e^{-t} \d t = \int_{0}^{1} (-\ln x)^{z-1} \d x = \frac{2^{z+1}}{z} \int_{0}^{1} y (-\ln y)^{z} \d y,
	\hspace{10 mm} 
	\Gamma\left(\tfrac{1}{2} + n\right) = \frac{(2n)!}{4^n n!} \sqrt{\pi} = \frac{(2n-1)!!}{2^n} \sqrt{\pi}.
\end{equation*}
Для произведения бывает удобно
\begin{equation*}
	\Gamma(z) \Gamma\left(z + \frac{1}{n}\right) \ldots \Gamma(z + \frac{n-1}{n}) = n^{\frac{1}{2} - nz } \cdot (2 \pi)^{\frac{n-1}{2}} \Gamma(nz),
	\hspace{5 mm} 
	\Gamma(1-z) \Gamma(z) = \frac{\pi}{\sin \pi z}.
\end{equation*}

