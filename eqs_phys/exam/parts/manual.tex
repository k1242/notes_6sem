\Tsec[]{Вычеты}

Интеграл по дуге может быть найден, как
\begin{align*}
    \int_C f(z) \d z = 2 \pi i \sum_{z_j} \res_{z_j} f(z),
    \hspace{5 mm} 
    \res_{z_j} f(z) &= \lim_{\varepsilon \to 0} \varepsilon \int_0^{2\pi} \frac{\d \varphi}{2\pi} e^{i \varphi} f(z_j + \varepsilon e^{i \varphi}) \\ 
    &= \frac{1}{(m-1)!} \lim_{z \to z_j} \left(
        \frac{d^{m-1} }{d z^{m-1}} (z-z_j)^m f(z)
    \right),
\end{align*}
где $m$ -- степень полюса. 


\Tsec[]{Интегралы}

Например,
\begin{equation*}
    \int_0^{\pi/2} \sin^a \varphi \cos^b \varphi \d \varphi =  \frac{1}{2} B\left(\frac{a+1}{2},\, \frac{b+1}{2}\right) = \frac{\Gamma\left(\frac{a+1}{2}\right) \Gamma\left(\frac{b+1}{2}\right)}{2 \Gamma\left(1 + \frac{a + b}{2}\right)}.
\end{equation*}
Также для $1 + m < kn$, верно
\begin{equation*}
    \int_{0}^{\infty}  \frac{x^m}{(1+x^n)^k} \d x = \bigg/
        t = \frac{1}{1+x^n}
    \bigg/ = \frac{\Gamma \left(\frac{m+1}{n}\right) \Gamma \left(k-\frac{m+1}{n}\right)}{n \Gamma (k)}.
\end{equation*}
Может быть полезно:
\begin{equation*}
    \int_{-\infty}^{+\infty} e^{\pm i z^2} \d z = \sqrt{\pi} e^{\pm i \pi / 4},
    \hspace{10 mm} 
    \int_{-\infty}^{+\infty} \cos z^2 \d z = \int_{-\infty}^{+\infty} \sin z^2 \ z = \sqrt{\frac{\pi}{2}}.
\end{equation*}

\Tsec[]{Преобразование Фурье}

В $n$-мерии само преобразование имеет вид
\begin{equation*}
	\tilde{f}(\vc{k}) = \int_{\mathbb{R}^n} d^n \vc{r} \ \exp\left(-i \vc{k} \cdot \vc{r}\right) f(\vc{r}),
	\hspace{10 mm} 
	f(\vc{r}) = \int_{\mathbb{R}^n} \frac{d^n \vc{k}}{(2\pi)^n} \ \exp\left(i \vc{k} \cdot \vc{r}\right) \tilde{f}(\vc{k}).
\end{equation*}
Для преобразования Фурье полезно помнить
\begin{equation*}
	\mathcal{F}[f^{(n)}] = (i \omega)^n \mathcal{F}[f](\omega/a),
	\hspace{5 mm} 
	\mathcal{F}[f(x-x_0)] = e^{- i \omega x_0} \mathcal{F}[f](\omega/a),
	\hspace{5 mm} 
	\mathcal{F}[f(ax)] = |a|^{-1} \mathcal{F}[f](\omega/a),
\end{equation*}
то есть что происходит при растяжение, сдвигах и дифференцирование. 



\Tsec[]{Преобразование Лапласа}

 Выпишем несколько пар оригинал-изображение:
\begin{align*}
    &t^n e^{\lambda t}
     \overset{\mathcal L}{\to}
        \frac{n!}{(p-\lambda)^{n+1}},
    &t^\alpha e^{\lambda t}
     \overset{\mathcal L}{\to}
        \frac{\Gamma(\alpha+1)}{(p-\lambda)^{n+1}},
    &&\frac{\left(1-e^{-t}\right)}{t}
     \overset{\mathcal L}{\to}
        \ln \left(1 + \frac{1}{p}\right),
    &&\frac{\sin t}{t}
     \overset{\mathcal L}{\to}
        \arctg p
    \\
    &\sin(\nu t)
     \overset{\mathcal L}{\to}
        \frac{\nu}{p^2 + \nu^2},
    &\cos(\nu t)
     \overset{\mathcal L}{\to}
        \frac{p}{p^2 + \nu^2},
    &&t \sin(\nu t)
     \overset{\mathcal L}{\to}
        \frac{2 p \nu}{(p^2 + \nu^2)^2}, 
    &&t \cos (\nu t)
     \overset{\mathcal L}{\to}
        \frac{p^2 - \nu^2}{(p^2 + \nu^2)^2}, \\
    &\sh(\nu t)
     \overset{\mathcal L}{\to}
        \frac{\nu}{p^2 - \nu^2},
    &\ch(\nu t)
     \overset{\mathcal L}{\to}
        \frac{p}{p^2 - \nu^2},
    &&e^{\lambda t} \sin(\nu t)
     \overset{\mathcal L}{\to}
        \frac{\nu}{(p-\lambda)^2 + \nu^2}, 
    &&e^{\lambda t} \cos(\nu t)
     \overset{\mathcal L}{\to}
        \frac{p-\lambda}{(p-\lambda)^2 + \nu^2}, 
\end{align*}
Также помним, что $\mathcal L [\delta(t)] = 1$, и $L[\delta(t-a)] = e^{-ap}$, при $a > 0$. 



