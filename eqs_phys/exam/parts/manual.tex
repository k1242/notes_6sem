\Tsec[]{Вычеты}

Интеграл по дуге может быть найден, как
\begin{align*}
    \int_C f(z) \d z = 2 \pi i \sum_{z_j} \res_{z_j} f(z),
    \hspace{5 mm} 
    \res_{z_j} f(z) &= \lim_{\varepsilon \to 0} \varepsilon \int_0^{2\pi} \frac{\d \varphi}{2\pi} e^{i \varphi} f(z_j + \varepsilon e^{i \varphi}) \\ 
    &= \frac{1}{(m-1)!} \lim_{z \to z_j} \left(
        \frac{d^{m-1} }{d z^{m-1}} (z-z_j)^m f(z)
    \right),
\end{align*}
где $m$ -- степень полюса. 


\Tsec[]{Интегралы}

Например,
\begin{equation*}
    \int_0^{\pi/2} \sin^\alpha \varphi \cos^\beta \varphi \d \varphi =  \frac{1}{2} B\left(\frac{a+1}{2},\, \frac{b+1}{2}\right).
\end{equation*}



Также для $1 + m < kn$, верно
\begin{equation*}
    \int_{0}^{\infty}  \frac{x^m}{(1+x^n)^k} \d x = \bigg/
        t = \frac{1}{1+x^n}
    \bigg/ = \frac{\Gamma \left(\frac{m+1}{n}\right) \Gamma \left(k-\frac{m+1}{n}\right)}{n \Gamma (k)}.
\end{equation*}




Может быть полезно:
\begin{equation*}
    \int_{-\infty}^{+\infty} e^{\pm i z^2} \d z = \sqrt{\pi} e^{\pm i \pi / 4},
    \hspace{10 mm} 
    \int_{-\infty}^{+\infty} \cos z^2 \d z = \int_{-\infty}^{+\infty} \sin z^2 \ z = \sqrt{\frac{\pi}{2}}.
\end{equation*}

% \Tsec[]{x Преобразование Фурье}

% \Tsec[]{x Преобразование Лапласа}

