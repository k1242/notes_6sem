\Tsec[]{Вычеты}

Интеграл по дуге может быть найден, как
\begin{align*}
    \int_C f(z) \d z = 2 \pi i \sum_{z_j} \res_{z_j} f(z),
    \hspace{5 mm} 
    \res_{z_j} f(z) &= \lim_{\varepsilon \to 0} \varepsilon \int_0^{2\pi} \frac{\d \varphi}{2\pi} e^{i \varphi} f(z_j + \varepsilon e^{i \varphi}) \\ 
    &= \frac{1}{(m-1)!} \lim_{z \to z_j} \left(
        \frac{d^{m-1} }{d z^{m-1}} (z-z_j)^m f(z)
    \right),
\end{align*}
где $m$ -- степень полюса. 


\Tsec[]{Интегралы}

Например,
\begin{equation*}
    \int_0^{\pi/2} \sin^a \varphi \cos^b \varphi \d \varphi =  \frac{1}{2} B\left(\frac{a+1}{2},\, \frac{b+1}{2}\right) = \frac{\Gamma\left(\frac{a+1}{2}\right) \Gamma\left(\frac{b+1}{2}\right)}{2 \Gamma\left(1 + \frac{a + b}{2}\right)},
\end{equation*}
для чего напомним несколько значений
\begin{equation*}
    \Gamma\left(\tfrac{1}{2}\right) = \sqrt{\pi},
    \hspace{5 mm} 
    \Gamma\left(\tfrac{1}{2}+n\right) = \frac{(2n-1)!!}{2^n}\sqrt{\pi}.
\end{equation*}
Также для $1 + m < kn$, верно
\begin{equation*}
    \int_{0}^{\infty}  \frac{x^m}{(1+x^n)^k} \d x = \bigg/
        t = \frac{1}{1+x^n}
    \bigg/ = \frac{\Gamma \left(\frac{m+1}{n}\right) \Gamma \left(k-\frac{m+1}{n}\right)}{n \Gamma (k)}.
\end{equation*}
Может быть полезно:
\begin{equation*}
    \int_{-\infty}^{+\infty} e^{\pm i z^2} \d z = \sqrt{\pi} e^{\pm i \pi / 4},
    \hspace{10 mm} 
    \int_{-\infty}^{+\infty} \cos z^2 \d z = \int_{-\infty}^{+\infty} \sin z^2 \ z = \sqrt{\frac{\pi}{2}}.
\end{equation*}
Чуть удобнее бывает пользоваться сразу формулой, вида
\begin{equation*}
    \int_{-\infty}^{+\infty} e^{\pm i a x^2 + i b x} \d x = \sqrt{\pi} e^{\pm i \pi/4} e^{\pm \frac{b^2}{4 a i}}.
\end{equation*}

\Tsec[]{Преобразование Фурье}
Для преобразования Фурье полезно помнить
\begin{equation*}
	\mathcal{F}[f^{(n)}] = (i \omega)^n \mathcal{F}[f](\omega/a),
	\hspace{5 mm} 
	\mathcal{F}[f(x-x_0)] = e^{- i \omega x_0} \mathcal{F}[f](\omega/a),
	\hspace{5 mm} 
	\mathcal{F}[f(ax)] = |a|^{-1} \mathcal{F}[f](\omega/a),
\end{equation*}
то есть что происходит при растяжение, сдвигах и дифференцирование. 



\Tsec[]{Преобразование Лапласа}
 Выпишем несколько пар оригинал-изображение:
\begin{align*}
    &t^n e^{\lambda t}
     \overset{\mathcal L}{\to}
        \frac{n!}{(p-\lambda)^{n+1}},
    &t^\alpha e^{\lambda t}
     \overset{\mathcal L}{\to}
        \frac{\Gamma(\alpha+1)}{(p-\lambda)^{n+1}},
    &&\frac{\left(1-e^{-t}\right)}{t}
     \overset{\mathcal L}{\to}
        \ln \left(1 + \frac{1}{p}\right),
    &&\frac{\sin t}{t}
     \overset{\mathcal L}{\to}
        \arctg p
    \\
    &\sin(\nu t)
     \overset{\mathcal L}{\to}
        \frac{\nu}{p^2 + \nu^2},
    &\cos(\nu t)
     \overset{\mathcal L}{\to}
        \frac{p}{p^2 + \nu^2},
    &&t \sin(\nu t)
     \overset{\mathcal L}{\to}
        \frac{2 p \nu}{(p^2 + \nu^2)^2}, 
    &&t \cos (\nu t)
     \overset{\mathcal L}{\to}
        \frac{p^2 - \nu^2}{(p^2 + \nu^2)^2}, \\
    &\sh(\nu t)
     \overset{\mathcal L}{\to}
        \frac{\nu}{p^2 - \nu^2},
    &\ch(\nu t)
     \overset{\mathcal L}{\to}
        \frac{p}{p^2 - \nu^2},
    &&e^{\lambda t} \sin(\nu t)
     \overset{\mathcal L}{\to}
        \frac{\nu}{(p-\lambda)^2 + \nu^2}, 
    &&e^{\lambda t} \cos(\nu t)
     \overset{\mathcal L}{\to}
        \frac{p-\lambda}{(p-\lambda)^2 + \nu^2}, 
\end{align*}
Также помним, что $\mathcal L [\delta(t)] = 1$, и $L[\delta(t-a)] = e^{-ap}$, при $a > 0$. 




\Tsec[]{Преобразование Меллина}


Для функции $g(x)$ такую, что $g(x) = O(x^{-\alpha})$ при $x \to 0$ и $g(x) = x^{-\beta}$ при $x \to + \infty$ можем определить
\textit{преобразование Меллина}
\begin{equation*}
	G(\lambda) = \int_{0}^{\infty} g(x) x^{\lambda-1} \d x,
\end{equation*}
определенного в полосе $\alpha < \Re \lambda < \beta$. Обратное преобразование может быть найдено в виде
\begin{equation*}
	g(x) = \int_{C - i \infty}^{C + i \infty} G(\lambda) x^{-\lambda} \frac{d \lambda}{2 \pi i},
\end{equation*} 
для $\alpha  <C < \beta$. 

Для вычисления инетгралов бывает удобно воспользоваться сверточным свойством преобразования Меллина
\begin{equation*}
	\int_{-\infty}^{+\infty} f(x) g(x) x^{\lambda-1} \d x = 
	\int_{C_f - i \infty}^{C_f + i \infty} F(\lambda_f) G(\lambda-\lambda_f) \frac{\d \lambda_f}{2\pi i}  = \int_{C_g - i \infty}^{C_g + i \infty}  F(\lambda-\lambda_g)  G(\lambda_g) \frac{\d \lambda_g}{2\pi i},
\end{equation*}
где $\alpha_f + \alpha_g < \Re \lambda < \beta_f + \beta_g$. 
В частности, при допустимом $\lambda = 1$, получаем
\begin{equation*}
	\int_{0}^{\infty} f(x) g(x) \d x = 
	\int_{C_f - i \infty}^{C_f + i \infty} F(\lambda_f) G(1-\lambda_f) \frac{d \lambda_f}{2 \pi i}.
\end{equation*}

Приведем некоторый зоопарк по преобразованию Меллина:
\begin{gather*}
	e^{-x} \overset{M}{\to}  \Gamma(\lambda),
	\hspace{5 mm} 
	\frac{1}{1 + a x^n} \overset{M}{\to} \frac{\pi  a^{-\frac{\lambda }{n}}}{n \sin \left(\frac{\pi  \lambda }{n}\right)},
	\hspace{5 mm} 
	\frac{1}{\sqrt[m]{1 + x^n}} \overset{M}{\to}
	\frac{\Gamma \left(\frac{\lambda }{n}\right) \Gamma \left(\frac{1}{m}-\frac{\lambda }{n}\right)}{n \Gamma \left(\frac{1}{m}\right)}, 
	\hspace{5 mm} 
	\frac{1}{1-x}\overset{M}{\to}  \pi  \cot (\pi  \lambda ),
	\\
	\frac{1}{\sqrt{a x^2+b x}}
	\overset{M}{\to}
	\frac{\Gamma (1-\lambda ) \Gamma \left(\lambda -\frac{1}{2}\right) \left(\frac{a}{b}\right)^{\frac{1}{2}-\lambda }}{\sqrt{\pi b}},
	\hspace{5 mm} 
	x^n \overset{M}{\to} 2 \pi  \delta (i (n+\lambda )), 
	\hspace{5 mm} 
	\frac{1}{1 + e^{\alpha x}} \overset{M}{\to}  \left(1-2^{1-\lambda }\right) \alpha ^{-\lambda } \Gamma (\lambda ) \zeta (\lambda ),
	\\ 
	\sin x \overset{M}{\to} \sin \left(\frac{\pi  \lambda }{2}\right) \Gamma (\lambda )
	, \hspace{5 mm}  
	\cos x \overset{M}{\to} \cos \left(\frac{\pi  \lambda }{2}\right) \Gamma (\lambda ).
\end{gather*}


\Tsec[]{Суммирование по мацубаровским частотам}
\textbf{Общая идея}. Когда нужно посчитать какую-нибудь сумму вида $\sum_{n=-\infty}^{\infty} f_n$, то может быть удобно представить $f_n$ как вычет некотрой функции $f(z) g(z)$ с $\res_n (z) = 1$ и $f(n) = f_n$. Тогда сумма сводится к интегралу, который легко берется. Функцию $g(z)$ имеет смысл выбирать таким образом, чтобы $\lim_{z \to \pm i\infty} f(z) g(z)$ был равен нулю.

\textbf{Пример №1}.
Рассмотрим сумму, вида $S(a) = \sum_{n=-\infty}^{\infty}  \frac{1}{n^2 + a^2}.$
Будем считать, что в $n \in \mathbb{Z}$, у некоторой функции $g(z)$ случается полюс первого порядка, например у функции:
\begin{equation*}
    g(z) = \pi \ctg(\pi z), 
    \hspace{5 mm} \res_n g(z) = 1.
\end{equation*}
Тогда сумму $S(a)$ можно переписать через проивезедение $f(z) g(z)$, где 
\begin{equation*}
    f(z) = \frac{1}{n^2 + a^2},
    \hspace{0.5cm} \Rightarrow \hspace{0.5cm}
    S(a) = \int \frac{\d z}{2 \pi i} \frac{\pi}{z^2 + a^2} \ctg (\pi z) = 
    \bigg/
        \res_{\pm ia}
    \bigg/ = \frac{\pi}{a} \cth (a \pi),
\end{equation*}
где воспользовались равенством $\ctg i x = - i \cth x$.


\textbf{Пример №2}. Теперь рассмотрим сумму c $f_n = e^{inx} (n^2 - \kappa^2)^{-2}$. Здесь в качестве $g(z)$ подойдет
\begin{equation*}
    g(z) = \frac{\pi e^{-i \pi z}}{\sin \pi z},
    \hspace{10 mm} 
    \res_n g(z) = 1,
\end{equation*}
тогда для $|x| < \pi$ асимптотика будет хорошей. 
