
Уравнение Бесселя:
\begin{equation*}
    \partial_z^2 J_m + \frac{1}{z} J_m + \left(1 - \frac{m^2}{z^2}\right) J_m = 0,
\end{equation*}
где $J_m(0) \in \mathbb{R}$. Знаем, что
\begin{equation*}
    e^{i z \sin \varphi} = \sum_{m \in \mathbb{Z}} 
    J_n (z) e^{i n \varphi},
    \hspace{0.25cm} \Rightarrow \hspace{0.25cm}
    J_m (z) = \frac{1}{2\pi} \int_{-\pi}^{+\pi}e^{i z \sin \varphi} e^{- i m \varphi} \d \varphi,
    \hspace{2.5 mm} 
    \Leftrightarrow
    \hspace{2.5 mm} 
    J_m (z) = \frac{1}{\pi} \int_{0}^{\pi} \cos(z \sin \varphi - m \varphi) \d \varphi.
\end{equation*}
Умеем дифференцировать:
\begin{equation}
    \frac{d J_m}{d z} = \frac{J_{m-1} (z)}{2} - \frac{J_{m+1}(z)}{2},
    \hspace{5 mm}
    \frac{m}{z} J_m (z) = \frac{1}{2} \left(J_{m+1} (z) + J_{m-1}(z)\right),
    \hspace{5 mm} 
    \frac{d }{d z} \left(z^m J_m (z)\right) = J_{m-1} (z) z^m.
\end{equation}
Откуда сразу находим
\begin{equation}
    \frac{d }{d x} \left(\frac{J_n (x)}{x^n }\right) = - \frac{J_{_n+1}(x)}{x^n},
    \hspace{10 mm} 
    J_{-m} (z) = (-1)^m J_m (z).
\end{equation}
Умеем раскладывать в ряд и уходить на бесконечность:
\begin{equation*}
    J_m (z) = \frac{z^m}{2^m} \sum_{k=0}^{\infty} \frac{(-1)^k z^{2k}}{4^k k! (m+k)!},
    \hspace{5 mm} 
    J_m (z \to \infty) = \sqrt{\frac{2}{\pi z}} \cos\left(
        z - \frac{\pi n }{2} - \frac{\pi}{4}
    \right),
\end{equation*}
соответственно с нулями в $\frac{\pi}{2} + \pi m$. 

Преобразование Фурье от функции:
\begin{equation*}
    F[J_m (z)] (k) = \int_{-\infty}^{+\infty} J_m (z) e^{-ikz} \d z 
    % = \int_{-\infty}^{+\infty} \d z e^{i kz} \frac{1}{2\pi} \int_{-\pi}^{+\pi} e^{i z \sin \varphi} e^{-i m \varphi} \d \varphi 
    = \frac{(-1)^m e^{i m \varphi_0} + e^{- i m \varphi_0}}{\sqrt{1 - k^2}}, \hspace{5 mm} 
    \varphi_0 = \arcsin k.
\end{equation*}
В частности
\begin{equation*}
    F[J_0](k) = \frac{2}{\sqrt{1-k^2}} \theta(1-k^2), \hspace{10 mm} 
    F[J_1](k) = \frac{2 i k}{\sqrt{1 - k^2}} \theta(1-k^2).
\end{equation*}
Преобразование Лапласа:
\begin{equation*}
    \Lambda[J_m] (p)=  \int_{0}^{\infty}  e^{- p z} J_m (z) \d z = 
    \frac{1}{\sqrt{p^2 + 1} (p + \sqrt{p^2 + 1})^m}.
\end{equation*}
Например,
\begin{equation*}
    \int_{0}^{\infty} \frac{J_n (z)}{z^n} \d z = \frac{1}{(2n-1)!!}.
\end{equation*}
Помним, что $J_m(0) = \delta_{m,0}$.

