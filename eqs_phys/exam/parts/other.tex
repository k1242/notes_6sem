



\Tsec[]{Вычеты}

Интеграл по дуге может быть найден, как
\begin{align*}
    \int_C f(z) \d z = 2 \pi i \sum_{z_j} \res_{z_j} f(z),
    \hspace{5 mm} 
    \res_{z_j} f(z) &= \lim_{\varepsilon \to 0} \varepsilon \int_0^{2\pi} \frac{\d \varphi}{2\pi} e^{i \varphi} f(z_j + \varepsilon e^{i \varphi}) \\ 
    &= \frac{1}{(m-1)!} \lim_{z \to z_j} \left(
        \frac{d^{m-1} }{d z^{m-1}} (z-z_j)^m f(z)
    \right),
\end{align*}
где $m$ -- степень полюса. 





\Tsec[]{Неоднородная релаксация}

Для одномерного случая
\begin{equation*}
    \big(\partial_t + \gamma(t)\big) x(t) = \varphi(t),
    \hspace{0.5cm} \Rightarrow \hspace{0.5cm}
    x(t) = \int_{-\infty}^{+\infty}  G(t,s) \varphi(s) \d s,
    \hspace{5 mm} 
    G(t,\,  s) = \theta(t-s) \exp\left(
        - \int_{s}^{t} \gamma(\tau) \d \tau
    \right),
\end{equation*}
где всё также $G(t, s>t) = 0$ в силу стремления к принципу причинности. 



\Tsec[]{Уравнение Вольтера}



\textbf{Уравнение Вольтерра}. Интегральное уравнение Вольтерра первого рода с однородным ядром:
\begin{equation*}
    \int_{0}^{t}  K(t-s) f(s) \d s = \varphi(t).
\end{equation*}
Решение может быть найдено через обратное преобразование Лапласа
\begin{equation*}
    f(t) = \int_{c-i \infty}^{c+i \infty} \frac{d p}{2 \pi i} \exp(pt) \tilde{f}(p),
    \hspace{10 mm} 
    \tilde{f}(p) = \frac{\tilde{\varphi}(p)}{\tilde{K}(p)}.
\end{equation*}
Но есть один нюанс. При $K(t),\, \varphi(t) \overset{p \to \infty}{\to} K_0,\, \varphi_0$ получается, что $\tilde{K}(p),\, \tilde{\varphi}(p) \approx \frac{K_0}{p},\, \frac{\varphi_0}{p}$, тогда
\begin{equation*}
    f(t) = \frac{\varphi_0}{K_0} \delta(t) + \int_{c-i \infty}^{c+i \infty} \frac{d p}{2 \pi i} \exp(p t)
    \left(
        \frac{\tilde{\varphi}}{\tilde{K}} - \frac{\varphi_0}{K_0}
    \right),
\end{equation*}
при этом в отсутствие аналитичности в нуле нет ничего страшного. 




\Tsec[]{Задача Штурма-Лиувилля с периодическими граничными условиями}

Рассмотрим такой же $\hat{L}$, и граничные условия в виде
\begin{equation*}
    \hat{L} = \partial_x^2 + Q(x) \partial_x + U(x),
    \hspace{10 mm}
    \left\{\begin{aligned}
        f(a) &= f(b), \\
        f'(a) &= f'(b),
    \end{aligned}\right.
\end{equation*}
которые приводят к периодичности решения. 


Рассмотрим задачу
\begin{equation*}
    \hat{L} = \partial_x^2 + \kappa^2,
\end{equation*}
с условиями на $[-\pi, \pi]$. 

При $x < y$:
\begin{equation*}
    G(x, y) = A_1 (y) \sin \kappa(x + \pi) + B_1 (y) \cos \kappa( x + \pi),
\end{equation*}
и аналогично для $x > y$:
\begin{equation*}
    G(x, y) = A_2 \sin \kappa (x - \pi) + B_2 (y) \cos \kappa (x - \pi).
\end{equation*}
Запишем граничные условия:
\begin{align*}
    G(- \pi, y) = G(\pi, y), \hspace{0.5cm} \Rightarrow \hspace{0.5cm}
    B_1 (y) = B_2 (y) \overset{\mathrm{def}}{=} B(y) \\
    G'_x (-\pi, y) = G'_x (\pi, y),
    \hspace{0.5cm} \Rightarrow \hspace{0.5cm}
    A_1 (y) = A_2 (y) \overset{\mathrm{def}}{=}  A(y).
\end{align*}
Тогда нашли, что
\begin{equation*}
    G(x, y) = \left\{\begin{aligned}
        &A \sin \kappa (x + \pi) + B \cos \kappa (x + \pi) \\
        &A \sin \kappa (x - \pi) + B \cos \kappa (x - \pi) \\
    \end{aligned}\right.
\end{equation*}
Теперь запишем непрерывность:
\begin{equation*}
     A \sin \kappa (x + \pi) + B \cos \kappa (x + \pi) 
     = 
     A \sin \kappa (x - \pi) + B \cos \kappa (x - \pi).
\end{equation*}
А также скачок производной
\begin{equation*}
    G'_x(y + 0, y) - G'_x (y-0, y) = 1,
    \hspace{0.25cm} \Rightarrow \hspace{0.25cm}
        A \cos \kappa (x - \pi) - B \sin \kappa (x - \pi)  - 
        A \cos \kappa (x + \pi) + B \cos \kappa (x + \pi)
        = \kappa^{-1}.
\end{equation*}
Решая эту систему находим, что
\begin{equation*}
    2 \sin \pi \kappa 
    \begin{pmatrix}
        \cos \kappa y & - \sin \kappa y  \\
        \sin \xi y & \cos \kappa y  \\
    \end{pmatrix} \begin{pmatrix}
        A  \\
        B  \\
    \end{pmatrix}
    = \begin{pmatrix}
        0 \\ 1/\kappa
    \end{pmatrix},
    \hspace{0.25cm} \Rightarrow \hspace{0.25cm}
    \begin{pmatrix}
        A \\ B
    \end{pmatrix} = 
    \frac{1}{2 \sin \pi \kappa} \begin{pmatrix}
        \cos \kappa y & \sin \kappa y  \\
        \sin \kappa y & \cos \kappa y  \\
    \end{pmatrix}
    \begin{pmatrix}
        0 \\ 1/\kappa
    \end{pmatrix} = 
    \frac{1}{2 \kappa \sin \pi \kappa} \begin{pmatrix}
        \sin xy \\ \cos xy
    \end{pmatrix}.
\end{equation*}
Подставляя в $G(x, y)$, находим\footnote{
    К дз будет полезно заметить, что $G(x, y) = G(x-y)$ -- задача трансляционно инвариантна. 
} 
\begin{equation*}
    G(x, y) = \frac{1}{2 \kappa \sin \pi \kappa}
    \left\{\begin{aligned}
        &\cos \left(\kappa(x-y) + \kappa \pi\right), & x < y\\
        &\cos(\kappa (x-y) - \kappa \pi), & x > y.
    \end{aligned}\right.
\end{equation*}
Всё это было, повторимся, для уравнения:
\begin{equation*}
    \left(\partial_x^2 + \kappa^2\right) f(x) = \varphi(x),
    \hspace{0.5cm} \Rightarrow \hspace{0.5cm}   
    f(x) = 
    \int_{-\pi}^{+\pi} G(x, y) \varphi(y) \d y. 
\end{equation*}





\Tsec[]{Справочные интегралы}

Может быть полезно:
\begin{equation*}
    \int_{-\infty}^{+\infty} e^{\pm i z^2} \d z = \sqrt{\pi} e^{\pm i \pi / 4},
    \hspace{10 mm} 
    \int_{-\infty}^{+\infty} \cos z^2 \d z = \int_{-\infty}^{+\infty} \sin z^2 \ z = \sqrt{\frac{\pi}{2}}.
\end{equation*}



\Tsec[]{Автомодельные решения}

Если уравнения вида $\hat{L}\, u(\vc{r}, t) = \ldots$ -- однородно и изотропно, то может помочь автомодельная подстановка:
\begin{equation}
	u(t, \vc{r}) = \frac{1}{t^a} f\left(\tfrac{r}{t^b}\right)
	\hspace{5 mm} \colon \hspace{5 mm} 
	t \to \lambda t
	\hspace{0.25cm} \Rightarrow \hspace{0.25cm}
	u \to \lambda^{-a} u, \ r \to \lambda^{b} r.
	\label{am}
\end{equation}
Восстановить $a$ в общем виде нельзя, но требуя, например, локальности решения $\int_{\mathbb{R}^n} u \d V = \const$ можем иногда найти и $a$. 




\Tsec[]{Меленные переменные}

Рассмотрим прозвольное возмущение гармонического осциллятора:
\begin{equation}
    \left(\partial_t^2 + \omega_0^2\right) x(t) = \varepsilon f(t, x, \dot{x}).
    \label{sloweq}
\end{equation}
Приближенно (до $o(\varepsilon)$) можем методом Боголюбова-Крылова найти
решение в виде
\begin{equation}
    x(t) = A(t) \cos(\omega_0 t + \varphi(t)),
    \label{sloweqview}
\end{equation}
где зависимость от времени амплитуды и фазы определяестся уравнениями
\begin{align}
    \partial_t A(t) &= \frac{1}{2\pi \omega_0}\int_{\omega_0 t-\pi}^{\omega_0 t+\pi} f(\tau, x, \dot{x}) \cos\left(\omega_0 \tau + \varphi(t)\right) \d (\omega_0\tau), 
    \label{slowA}
    \\
    \partial_t \varphi(t) &= \frac{-1}{2 \pi A \omega_0} \int_{\omega_0 t-\pi}^{\omega_0 t + \pi} f(\tau, x, \dot{x}) \sin(\omega_0 \tau + \varphi(t)) \d (\omega_0 \tau).
    \label{slowphi}
\end{align}

% \textbf{Огибающая}. Уравнение на огибающую может быть найдено в виде решения уравнения
% \begin{equation*}
%     \partial_t \psi + \vc{v} \cdot \nabla \psi = \frac{i \varepsilon}{\omega} e^{i \theta} \int_{0}^{2\pi} \frac{d \varphi}{2 \pi} e^{-i \varphi} f.
% \end{equation*}


\Tsec[]{Уравнение Хопфа}

В акустике естественно возникает уравнение Хопфа:
\begin{equation*}
    \partial_t u + u\, \partial_x u = 0.
\end{equation*}
Решение может быть найдено в виде
\begin{equation*}
    x(t) = x_0 + u_0 (x_0) t, \hspace{5 mm} 
    u(x(t), t) = c(x_0) = u_0 (x_0).
\end{equation*}
где сначала разрешаем уравнение $c = u_0 (x_0)$ относительно $c = c(x_0)$, а потом разрешаем уравнение на $x(t)$ относительно $c = c(x(t), t)$. 
Зная, что $u(x(t), t) = c(x(t), t)$, находим $u(x, t) = c(x, t)$. 

 Добавим к уравнению накачку:
\begin{equation*}
    \partial_t u + u\, \partial_x u = f(x, t).
\end{equation*}
Система может быть сведена к
\begin{equation*}
    \left\{\begin{aligned}
        \dot{u} &= f(t, x(t)) \\
        \dot{x} &= u(t, x(t))
    \end{aligned}\right.
    \hspace{5 mm} \Leftrightarrow \hspace{5 mm} 
    \ddot{x} = f(x, t),
    \hspace{0.5cm} \Rightarrow \hspace{0.5cm}
    x(t) = x(t, x_0, \dot{x}_0),
\end{equation*}
где $\dot{x}_0 = u_0(x_0)$. Сначала разрешаем уравнение $x(t)$ относительно $x_0 = x_0(t, x)$, а потом подставляем этот $x_0$ в $u(t, x) = \dot{x}(t, x_0(t,x))$, что и является решением исходной задачи.



\Tsec[]{Уравнение Бюргерса}
Добавим диссипацию в уравнение Хопфа:
\begin{equation*}
    \partial_t u + u\, \partial_x u = \partial_x^2 u,
\end{equation*}
так получим уравнение Бюргерса.

Заметим, что преобразование Коула-Хопфа
\begin{equation*}
    \psi = \exp\left(- \tfrac{1}{2}h\right),
    \hspace{5 mm} 
    u = \partial_x h,
    \hspace{0.5cm} \Rightarrow \hspace{0.5cm}
    (\partial_t - \partial_x^2)\psi = 0.
\end{equation*}
Имея начальные условия для $\psi_0(x)$, можем найти
\begin{equation*}
    \psi(t, x) = \int_{\mathbb{R}} \psi_0 (y) \frac{\theta(t)}{\sqrt{4 \pi t}} \exp\left(-\frac{(x-y)^2}{4t}\right) \d y,
\end{equation*}
откуда находим решение
\begin{equation*}
    u(t, x) =  -2 \partial_x \ln \psi(t, x).
\end{equation*}

