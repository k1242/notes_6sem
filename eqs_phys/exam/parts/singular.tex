
\Tsec[]{Сингулярные интегральные уравнения}

\textbf{Сингулярные интегральные уравнения}. Основой решения станет \textit{формула Сохоцкого}:
\begin{equation}
		\text{v. p.} \int_{a}^{b} \frac{f(x)}{x-x_0} \d x = 
		\pm i \pi f(x_0) + \lim_{\varepsilon \to + 0} \int_{a}^{b} \frac{f(x)}{x-x_0 \pm i \varepsilon} \d x.
\end{equation}
Полезно ввести преобразование Гильберта $\hat{H}$:
\begin{align*}
	\hat{H} \varphi(x) 
	=
	%  \frac{1}{\pi} \vpint_{-\infty}^{+\infty} \frac{\varphi(y)}{y-x} \d y &=  \lim_{\varepsilon \to + 0} \int_{-\infty}^{+\infty} \frac{\varphi(y) \d y}{\pi}  \frac{y-x}{(y-x)^2 + \varepsilon^2} 
	% = \\ &= 
	\lim_{\varepsilon \to + 0} \int_{-\infty}^{+\infty} \frac{\varphi(y) \d y}{2\pi} \left(
		\frac{1}{y-x + i \varepsilon} + \frac{1}{y-x-i \varepsilon}
	\right),
\end{align*}
для которого верно, что $\hat{H}^2 = - \mathbbm{1}$.

Тогда \textit{простейшее сингулярное уравнение} вида
\begin{equation}
	\pi \hat{H} \varphi(x)  + \lambda \varphi(x) = f(x)
\end{equation}
будет иметь решение относительно $\varphi(x)$:
\begin{equation}
	 \varphi(x) = \frac{\lambda}{\lambda^2 + \pi^2} f(x) - \frac{1}{\lambda^2 + \pi^2} \hat{H}[f](x) = \frac{1}{\lambda + i \pi} f(x) - \frac{1}{\lambda^2 + \pi^2} \int_{-\infty}^{+\infty}  dy\ \frac{f(y)}{y-x + i \varepsilon}.
\end{equation}

\textbf{Сингулярные интегральные уравнения с полиномильными коэффициентами}. Рассмотрим уравнение вида
\begin{equation*}
	\vpint_{-\infty}^{\infty} \frac{\varphi(y)}{y-x} \d y 
	% = \int_{-\infty}^{+\infty} \frac{\varphi(y)}{y-z + i \varepsilon} \d y + i \pi \varphi(x) 
	= x^2 \varphi(x) + f(x).
\end{equation*}
Применяя оператор $\int_{-\infty}^{+\infty} \frac{\d x}{x - z + i \varepsilon}$, приходим к уравнению
\begin{equation*}
	-  \pi i \int_{-\infty}^{+\infty} \frac{\varphi(y)}{y-z +  i \varepsilon} \d y = \int_{-\infty}^{+\infty} \frac{y^2 \varphi(y)}{y-z + i \varepsilon} \d y +  \int_{-\infty}^{+\infty}  \frac{f(y)}{y-z + i \varepsilon} \d y.
\end{equation*}
Здесь можем провести следующие рассуждения:
\begin{align*}
	\int_{-\infty}^{+\infty} \frac{y(y - z + i \varepsilon + z - i \varepsilon)}{y - z + i \varepsilon} \varphi(y) \d y &= \int_{-\infty}^{+\infty} y \varphi(y) \d y + z \int_{-\infty}^{+\infty} \frac{y - z + z}{y-z + i \varepsilon} \varphi(y) \d y 
	= \\ &= 
	\underbrace{\int_{-\infty}^{+\infty} y \varphi(y) \d y}_{C_1} + z \underbrace{\int_{-\infty}^{+\infty} \varphi(y) \d y }_{C_2}+ z^2 \int_{-\infty}^{+\infty} \frac{\varphi(y)}{y-z + i \varepsilon} \d y,
\end{align*}
а значит исходное уравнение переписывается в виде
\begin{equation*}
	\int_{-\infty}^{+\infty} \frac{\varphi(y)}{y-z + i \varepsilon} \d y = - \frac{1}{z^2 + i \pi} \int_{-\infty}^{+\infty} \frac{f(y)}{y-z + i \varepsilon} \d y - \frac{C_1 + C_2 z}{z^2 + i \pi},
\end{equation*}
решение которого мы уже знаем:
\begin{equation*}
	\varphi(x) = - \frac{C_1 + C_2 x}{x^4 + \pi^2} - \frac{f(x)}{x^2 - i \pi} - \frac{1}{x^4 + \pi^2} \int_{-\infty}^{+\infty} \frac{f(y)}{y-z + i \varepsilon} \d y.
\end{equation*}

\textbf{Сингулярные интегральные уравнения на отрезке}. Рассмотрим  уравнение на конечном отрезке
\begin{equation*}
	\vpint_{-1}^{1} \frac{\varphi(y)}{y-x} \d y = f(x).
\end{equation*}
Решение можем найти при условии на $f\colon \int_{-1}^{1} \frac{f(x)}{\sqrt{1-x^2}} \d x = 0$, тогда
\begin{equation*}
	\varphi(x) = \frac{1}{\pi i}\left(
		f(x) + \frac{\sqrt{1-x^2}}{\pi i} \int_{-1}^{1} \frac{f(y) \d y}{\sqrt{1-y^2} (y-x+i \varepsilon)}
	\right).
\end{equation*}



% \newpage