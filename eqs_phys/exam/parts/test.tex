% 1 семестр
эволюционные уравнения (причинные функции Грина)
решения уравнений с их помощью

статические линейные поля (задача Штурма-Лиувилля с разными гран. условиями, нулевая мода)

специальные функции: 
	!гамма-функция, 
	@Бесселя, 
	@ортогональные полиномы, 



% 2 семестр
динамические линейные поля — склейка первых двух тем 
	уметь получать функцию Грина 
	+ находить решение с ее помощью
	Надо знать 
		@волновое уравнение
		@ур-ие теплопроводности
		?ур-ия Гельмгольца
		@Пуассона-Лапласа

интегральные уравнения (@вырожденное ядро, @свертка, @сингулярка)

% дополнительно
	@I. автомодельные подстановки
	@II. боголюбов-крылов
	III. Хопф
	IV. Меллин


\textbf{Вычеты}. Интеграл по замкнутому контуру $C$ может быть найден, как
\begin{align*}
    \int_C f(z) \d z = 2 \pi i \sum_{z_j} \,\text{res}_{z_j} f(z),
    \hspace{5 mm} 
    \,\text{res}_{a} f(z) &= \lim_{\varepsilon \to 0} \varepsilon \int_0^{2\pi} \frac{\d \varphi}{2\pi} e^{i \varphi} f(a + \varepsilon e^{i \varphi}) \\ 
    &= \frac{1}{(m-1)!} \lim_{z \to a} \left(
        \frac{d^{m-1} }{d z^{m-1}} (z-a)^m f(z)
    \right),
\end{align*}
где $m$ -- степень полюса. 