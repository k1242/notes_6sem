\Tsec[]{Задача Штурма-Лиувилля}

\textbf{Постановка задачи}. 
Задача Штурма-Лиувилля:
\begin{equation*}
    \hat{L} = \partial_x^2 + Q(x) \partial_x + U(x),
    \hspace{5 mm} 
    \hat{L} f(x) = \varphi(x),
    \hspace{5 mm} 
    \left\{\begin{aligned}
        \alpha_1 f(a) + \beta_1 f'(a) &= 0 \\
        \alpha_2 f(b) + \beta_2 f'(b) &= 0,
    \end{aligned}\right.
\end{equation*}
где\footnote{
    Часто можно встретить нулевые граничные условия: $f(a) = f(b) = 0$. 
}  $|\alpha_1| + |\beta_2| \neq 0$ и $|\alpha_2| + |\beta_2| \neq 0$.

\textbf{Граничные условия}.  C учетом того, что функция Грина $G$ наследует граничные условия:
\begin{align*}
    \alpha_1 G(a, y) + \beta_1 G'_x (a, y) &= 0, \\
    \alpha_2 G(b, y) + \beta_2 G'_x (b, y) &= 0.
\end{align*}
Запищем уравнение на $G(x, y)$:
\begin{equation*}
    \hat{L} G(x, y) = \delta(x-y).
\end{equation*}
Решение этого уравнение можем найти решая две системы на $u(x)$ при $x < y$ и $v(x)$ при $x > y$:
\begin{equation*}
\left\{\begin{aligned}
        \hat{L} u(x) &= 0 \\
        \alpha_1 u(a) + \beta_1 u'(a) &= 0
    \end{aligned}\right.
    \hspace{10 mm} 
    \left\{\begin{aligned}
        \hat{L} v &= 0 \\
        \alpha_2 v(b) + \beta_2 v'(b) &= 0
    \end{aligned}\right.
\end{equation*}
Теперь можем выписать ответ
\begin{equation*}
    G(x, y) = \frac{1}{W(y)} \left\{\begin{aligned}
        &v(y) u(x), &x < y, \\
        &v(x) u(y), &x > y,
    \end{aligned}\right.
    \hspace{10 mm} 
    W[u, v](y) = \frac{v' (y) u(y) - v(y) u'(y)}{u(y)}.
\end{equation*}
Который существует и единственнен для $W \neq 0$. Для $W = \const $ $G(x, y) = G(y, x)$, а значит $\hat{L}^{-1}$ -- симметричный самосопряженный оператор, и у $\hat{L}$ есть ОНБ из собственных функций. 


\textbf{Вронскиан}.  Бывает удобно найти $W(x)$, записав формулу Лиувилля-Остроградского
\begin{equation*}
    W(x) = 
    \det \begin{pmatrix}
        u & v  \\
        u' & v'  \\
    \end{pmatrix}
    =
    W(x_0) \exp\left(
        - \int_{x_0}^{x}  Q(z) \d z
    \right).
\end{equation*}


\begin{to_def}
    \textit{Специальной ФСР} называется решение уравнения $\hat{L} u = 0$ и $\alpha_1 u(a) + \beta_1 u'(a) = 0$, и аналогичного уравнения по $v(x)$ с граничным условием в $b$, 
    если $W[u, v] \neq 0$, то есть $u$ и $v$ линейной независимы. 
\end{to_def}


% \noindent
% \textbf{Пример I}. Рассмотрим уравнения
% \begin{equation*}
%     \left\{\begin{aligned}
%         \partial_x^2 f(x) = \varphi(x) \\
%         f(a) = f(b) = 0
%     \end{aligned}\right.
%     \hspace{0.5cm} \Rightarrow \hspace{0.5cm}
%     u(x) = x - a,
%     \hspace{5 mm} 
%     v(x) = x - b,
%     \hspace{0.5cm} \Rightarrow \hspace{0.5cm}
%     W = \begin{pmatrix}
%         u & v  \\
%         u' & v'  \\
%     \end{pmatrix} = b-a = \const,
% \end{equation*}
% а значит
% \begin{equation*}
%     G(x, y) = \frac{1}{b-a} \left\{\begin{aligned}
%         &(y-b)(x-a), &x < y \\
%         &(x-b)(y-a), &x > y.
%     \end{aligned}\right.
% \end{equation*}

% \noindent
% \textbf{Пример II}. Рассмотрим двумерный цилиндр, радуса $R$, вне которого $\rho(r > R) = 0$, $\rho(\vc{r}) = \rho(r)$. Рассмотрим уравнения Лапласа:
% \begin{equation*}
%     \nabla^2 \varphi = - 4 \pi \rho,
%     \hspace{0.5cm} \Rightarrow \hspace{0.5cm}
%     (\partial_r^2 + \tfrac{1}{r} \partial_r) \varphi = - 4 \pi \rho.
% \end{equation*}
% Добавим граничные условия: потенциал определен с точностью до константы, так что пусть $\varphi(R) = 0$, также хотим конечность $\varphi$ при $r=0$, так что пусть $\varphi(0) = 1$.

% Получили задачу, где при $r < r'$
% \begin{equation*}
%     \left\{\begin{aligned}
%         &\left(\partial_r^2 + \tfrac{1}{r} \partial_r\right) u(r) = 0,
%         &u(0) = 1
%     \end{aligned}\right.
%     \hspace{0.5cm} \Rightarrow \hspace{0.5cm}
%     u' = \frac{C}{r},
%     \hspace{0.5cm} \Rightarrow \hspace{0.5cm}
%     u(r) = C \ln r + D = 1.
% \end{equation*}
% Аналогично, рассмотрим $r > r'$:
% \begin{equation*}
%     \left\{\begin{aligned}
%         &\left(\partial_r^2 + \tfrac{1}{r} \partial_r\right) v(r) = 0,
%         &v(R) = 0,
%     \end{aligned}\right.
%     \hspace{0.5cm} \Rightarrow \hspace{0.5cm}  
%     v(R) = C' \ln r + D',
%     \hspace{0.5cm} \Rightarrow \hspace{0.5cm}   
%     v = \ln \left(\frac{r}{R}\right).
% \end{equation*}
% Сразу вычислим 
% \begin{equation*}
%     W[u,\,  v] = \det \begin{pmatrix}
%         1 & \ln r/R  \\
%         0 & 1/r  \\
%     \end{pmatrix} = \frac{1}{r},
%     \hspace{0.5cm} \Rightarrow \hspace{0.5cm}
%     G(r, r') = r' \left\{\begin{aligned}
%         & \ln \tfrac{r'}{R}, & r < r' \\
%         & \ln \frac{r}{R}, & r > r'
%     \end{aligned}\right.
%     \hspace{0.5cm} \Rightarrow \hspace{0.5cm}
%     \varphi(r) = \int_{0}^{R} G(r, r') \left(- 4 \pi \rho(r')\right) \d r'.
% \end{equation*}




\Tsec[]{Метод Фурье в задаче Штурма-Лиувилля}

Допустим мы в $\mathcal H$, соответственно есть $\langle x | y\rangle$. Рассмотрим некоторый достаточно хороший самосопряженный компактный оператор $\hat{L}$, у которого есть ОНБ из собственных функций: $\hat{L} e_n = \lambda_n e_n$.


\begin{to_thr}[thr Гильберта-Шмидта]
    Если $\hat{L}$ -- компактный\footnote{
        $\mathcal D (A)$ -- компакт в гильбертовом пространстве.
    }  ССО, то у $A$ есть ОНБ из собственных функций. 
\end{to_thr}

Вернемся к оператору Штурма-Лиувилля, который живет в $\mathcal H = L_2[a, b]$:
\begin{equation*}
    \hat{L} = A(x) \partial_x^2 + B(x) \partial_x + C(x),
    \hspace{5 mm} 
    \langle f|g\rangle = \int_{a}^{b} f \bar{g} \d x.
\end{equation*}
Для задачи Штурма-Лиувилля $\hat{L}$  симметричен, при $B(x) = A'(x)$. 

% Итак, для оператора Штурма-Лиувилля ищем собственные функции:
% \begin{equation*}
%     \left\{\begin{aligned}
%         &\hat{L} e_n (x) = \lambda_n e_n (x) \\
%         &\text{гран. усл.}
%     \end{aligned}\right.
%     \hspace{0.5cm} \Rightarrow \hspace{0.5cm}
%     f(x) = \sum_n f_n e_n(x),
% \end{equation*}
% домножая на $e_m$, находим, что
% \begin{equation*} 
%     f_m = \frac{\langle f | e_m\rangle}{\langle e_m | e_n\rangle},
%     \hspace{10 mm} 
%     \langle f | e_m\rangle = \int_{a}^{b}  f(x) \bar{e}_m (x) \d x.
% \end{equation*}
% Метод Фурье решения краевых задач:
% \begin{equation*}
%     L(\partial_t) u(x, t) = \hat{A}_x u(x, t) + f(x, t),
% \end{equation*}
% плюс граничные условия.  Раскладывая,
% \begin{equation*}
%     \hat{A}_x e_n = \lambda_n e_n,
%     \hspace{10 mm} 
%     u = \sum_n u_n(t) e_n (x),
%     \hspace{0.5cm} \Rightarrow \hspace{0.5cm}
%     f(x, t) = \sum f_n (t) e_n (x),
% \end{equation*}
% а значит
% \begin{equation*}
%     \sum_n \left(
%         L(\partial_t) u_n (t) - \lambda_n u_n (t) - f_n (t)
%     \right) e_n (x) = 0,
% \end{equation*}
% откуда можем находить $u_n (t)$:
% \begin{equation*}
%     u(x, t) = \sum_n u_n (t) e_n (x). 
% \end{equation*}
% \subsection*{Ещё что-то}
% Рассмотрим снова некоторую граничную задачу:
% \begin{equation*}
%     \hat{L} G(x, y) = \delta(x-y).
% \end{equation*}
% Запишем граничные условия:
% \begin{equation*}
%     \alpha_1 G(a. y) + \beta_1 G'_x(a, y) = 0,
%     \hspace{10 mm} 
%     \alpha_2 G(b, y) + \beta_2 G'_x (b, y) = 0,
% \end{equation*}
% где  $|\alpha_1| + |\beta_2| \neq 0$ и $|\alpha_2| + |\beta_2| \neq 0$.
% Можем выписать ответ:
% \begin{equation*}
%     G(x, y) = \frac{1}{W(y)} \left\{\begin{aligned}
%         &v(y) u(x), &x < y; \\
%         &v(x) u(y), &x > y,
%     \end{aligned}\right.
% \end{equation*}
% где Вронскиан можно запсиать, как
% \begin{equation*}
%     W(x) = W(x_0) \exp\left(
%         - \int_{x_0}^{x}  Q(t) \d t,
%     \right)
% \end{equation*}
% где $Q(t)$ -- из оператора Штурма-Лиувилля. 
% Также решали задачу с периодическими гран. условиями, где $G$ наследовала гран. условия. 
% Решать это всё умеем двумя способами: разделяя на $x  > y$ и $x < y$, и через метод Фурье:
% \begin{equation*}
%     \hat{L} e_n  = \lambda_n e_n,
%     \hspace{10 mm} 
%     \langle e_n | e_m \rangle = \int_{a}^{b} e_n (x) \bar{e}_m(x)  \d x.
% \end{equation*}
Тогда можем найти функцию Грина, как решение уравнения $\hat{L} G(x, y) = \delta(x-y)$
\begin{equation*}
    G(x, y) = \sum_n g_n (y) e_n (x),
    \hspace{5 mm} 
    \delta(x-y) = \sum_n \delta_n ( y) e_n (x).
\end{equation*}
Находим коэффициенты Фурье: \red{нужен пример про собственные функции}
\begin{equation*}
    \delta_n (y) = \frac{\bar{e}_n (y)}{\langle e_n | e_n\rangle},
    \hspace{0.5cm} \Rightarrow \hspace{0.5cm}
    g_n (y) = \frac{1}{\lambda_n} \frac{\bar{e}_n (y)}{\langle e_n | e_n\rangle}.
\end{equation*}
% где мы решали уравнение, вида $\hat{L} G = \delta(x-y)$. 
Проблема возникает при $\lambda_n = 0$. 



\textbf{Решение}. Наличие у оператора собственного числа $\lambda_n = 0$ называется нулевой модой. Рассмотрим оператор:
\begin{equation*}
    \hat{L} = \partial_x^2,
\end{equation*}
для которого $e_n (x) = e^{i n x}$, где $\langle e_n | e_n\rangle = 2 \pi$, где $e_0 = 1$ и $\lambda_{0} = 0$. Пусть тогда
\begin{equation*}
    \delta(x) = \sum \frac{\bar{e}_n (0) e_n (x)}{\langle e_n | e_n\rangle} = \sum \frac{e^{i n x}}{2 \pi},
    \hspace{10 mm} 
    G(x) = \sum  g_n e_n (x). 
\end{equation*}
но для $\hat{L} G = \delta(x)$ оказывается нет решений (справа $e_0$ есть, а слева нет). То есть
\begin{equation*}
    \Ker \hat{L} \neq \{0\},
    \hspace{10 mm} 
    \Ker \hat{L} + \Im \hat{L} = \mathcal H,
\end{equation*}
поэтому всегда имеем ввиду, что $\hat{L} \hat{L}^{-1} = \mathbbm{1}$, но только для $\Im \hat{L}$. 

В общем, проблему уйдёт, если рассмотрим уравнение, вида
\begin{equation*}
    \hat{L} G(x) = \delta(x) - e_0(x) = \delta(x) - \frac{1}{2 \pi},
\end{equation*}
то есть справа единичный оператор только на образе $\Im \hat{L}$. 



Если в источнике есть нулевая мода, то уравнение не имеет решений. 

