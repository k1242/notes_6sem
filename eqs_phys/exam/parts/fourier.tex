В $n$-мерии само преобразование имеет вид
\begin{equation*}
	\tilde{f}(\vc{k}) = \int_{\mathbb{R}^n} d^n \vc{r} \ \exp\left(-i \vc{k} \cdot \vc{r}\right) f(\vc{r}),
	\hspace{10 mm} 
	f(\vc{r}) = \int_{\mathbb{R}^n} \frac{d^n \vc{k}}{(2\pi)^n} \ \exp\left(i \vc{k} \cdot \vc{r}\right) \tilde{f}(\vc{k}).
\end{equation*}
Для преобразования Фурье полезно помнить
\begin{equation*}
	\mathcal{F}[f^{(n)}] = (i \omega)^n \mathcal{F}[f](\omega/a),
	\hspace{5 mm} 
	\mathcal{F}[f(x-x_0)] = e^{- i \omega x_0} \mathcal{F}[f](\omega/a),
	\hspace{5 mm} 
	\mathcal{F}[f(ax)] = |a|^{-1} \mathcal{F}[f](\omega/a),
\end{equation*}
то есть что происходит при растяжение, сдвигах и дифференцирование. 

