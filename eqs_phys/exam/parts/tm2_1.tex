
\textbf{Граничные условия}. 
Вообще граничные условия можно вводить из разложения
\begin{equation*}
    u(t\to 0,\, \vc{r}) \approx \theta(t) u(t=0, \vc{r}) + t \theta(t) \partial_t u(t=0,\, \vc{r}) + \ldots,
\end{equation*}
тогда действие оператора $\partial_t^2$ на эти члены дает сингулярные слагаемые
\begin{equation*}
    (\partial_t^2 + \nabla^2)u(\vc{t,r}) = \delta'(t) u(t=0, \vc{r}) + \delta(t) \partial_t u(t=0, \vc{r}),
\end{equation*}
которые можем воспринимать как источник.


\Tsec[]{Волновое уравнение}


\textbf{Общий подход}. Найдём функцию Грина для уравнения, вида
\begin{equation*}
    \left(\partial)t^2 + \varpi^2[-i \nabla]\right) u = \chi.
\end{equation*}
Решение запишется в виде
\begin{equation*}
    \left(
        \partial_t^2 + \varpi^2[-i \nabla]
    \right) G = \delta(t) \delta(\vc{r}),
    \hspace{5 mm} 
    u(t,\, \vc{r}) = \int dt_1\, d^3 r_1\ 
    G(t-t_1,\, \vc{r}- \vc{r}_1) \chi(t_1,\, \vc{r}_1).
\end{equation*}
Переходя к пространственному Фурье-образу, приходим у уравнению с известной функцией Грина:
\begin{equation*}
    \left(\partial_t^2 + \varpi^2[\vc{q}]\right) \tilde{G} = \delta(t),
    \hspace{0.5cm} \Rightarrow \hspace{0.5cm}   
    \tilde{G}(t) = \theta(t) \frac{\sin(\varpi t)}{\varpi}.
\end{equation*}
Тогда через обратное Фурье-преобразование находим
\begin{equation*}
    G(t,\, \vc{r}) = \theta(t) \int_{\mathbb{R}^3} \frac{d^3 q}{(2\pi)^3} \frac{\sin(\varpi t)}{\varphi} \exp\left(i \vc{q} \cdot \vc{r}\right) \overset{*}{=} 
    \frac{\theta(t)}{\pi r} \int_{0}^{\infty}  \frac{dq}{2\pi} q \frac{\sin(\varpi[q] t)}{\varpi[q]} \sin(qr)
    ,
\end{equation*}
где $\overset{*}{=}$ верно для $\varpi[\vc{q}] \equiv \varpi[q]$. 



Например, для $\varpi[\vc{q}] = qc$, волновое уравнение с источником:
\begin{equation}
    (\partial_t^2 - c^2 \nabla^2) u(t, \vc{r}) = \chi(t, \vc{r}),
    \hspace{5 mm} 
    G(t, r) = \frac{\theta(t)}{4 \pi c r} \delta(r - ct).
\end{equation}
а значит выражение для поля:
\begin{equation}
    u(t, \vc{r}) = \frac{1}{4 \pi c^2} \int d^3 r_1 \ \frac{\chi(t-|\vc{r} - \vc{r}_1|/c, \vc{r}_1)}{|\vc{r} - \vc{r}_1|},
\end{equation}
\red{перейти к сферически симметричному случаю}.


\Tsec[]{Уравнение Гельмгольца}

Рассмотрим отдельно уравнение Гельмгольца
\begin{equation*}
    (\nabla^2 + \kappa^2) f = \varphi(\vc{r}).
\end{equation*}
Как и раньше, запишем найдём решение в виде
\begin{equation*}
    (\nabla^2 + \kappa^2) G(\vc{r}) = \delta(\vc{r}),
    \hspace{0.5cm} \Rightarrow \hspace{0.5cm}
    f(\vc{r}) = f_0(\vc{r}) + \int d^3 r_1 \ G(\vc{r}-\vc{r}_1) \varphi(\vc{r}_1),
\end{equation*}
где $f_0$ -- решение однородного уравнения Гельмгольца.
Функция Грина:
\begin{equation*}
    G = - \frac{\exp(i \kappa r)}{4 \pi r},
\end{equation*}
для $\omega > 0$ и $G \to G^*$ при $\omega < 0$.




\Tsec[]{Уравнение теплопроводности}

Уравнение диффузии с известным $u(0, \vc{x}) = u_0 (\vc{x})$:
\begin{equation}
    \left(\partial_t - \nabla^2\right) u(t, \vc{x}) = 0,
    \label{difeq}
\end{equation}
решение которого может быть найдено в виде:
\begin{equation}
    u(t, \vc{x}) = \int_{\mathbb{R}^d} \frac{d y_1 \ldots d y_d}{(4 \pi t)^{d/2}} \exp\left(- \frac{(\vc{x}-\vc{y})^2}{4 t}\right) u_0(\vc{y}).
    \label{difsol}
\end{equation}
Асимтотики могут быть найдены в виде
\begin{equation}
    u(t, \vc{x}) \approx \frac{A}{(4 \pi t)^{d/2}} \exp\left(- \frac{\vc{x}^2}{4 t}\right),
    \hspace{5 mm} 
    A = \int_{ \mathbb{R}^d} dy_1 \ldots dy_d \ u_0(\vc{y}).
    \label{difas1}
\end{equation}
При $A = 0$ асимтотика будет соответствовать
\begin{equation}
    u(t, \vc{x}) \approx  \frac{\vc{B} \cdot \vc{x}}{(4 \pi t)^{d/2+1}} \exp\left(
        - \frac{\vc{x}^2}{4 t}
    \right), \hspace{5 mm} 
    \bar{B} = 2 \pi \int_{ \mathbb{R}^d} dy_1 \ldots dy_d \ \vc{y}\, u_0(\vc{y}),
    \label{difas2}
\end{equation}
где асимтотики имеют место при $t \gg l^2$, $l$ -- масштаб на котором локализовано поле.

\textbf{Накачка}. 
При наличии правой части:
\begin{equation*}
    (\partial_t - \nabla^2) u = \varphi,
\end{equation*}
можем найти функцию Грина для оператора $\partial_t - \nabla^2$
\begin{equation*}
    u(t, \vc{x}) =  \int G(t-\tau, \vc{x}-\vc{y}) \varphi(t, \vc{y}) \d \tau \d^d \vc{y},
    \hspace{10 mm} 
    G(t, \vc{r}) = \frac{\theta(t)}{(4 \pi t)^{d/2}} \exp\left(- \frac{r^2}{4 t}\right).
\end{equation*}
