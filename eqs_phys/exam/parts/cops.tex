
\textbf{Общий подход}. 
Знаем, что
\begin{equation*}
    \left\{\begin{aligned}
        (\sigma \rho f')' = - \lambda \rho f, \\
        (\sigma \rho)' =  \tau \rho,
    \end{aligned}\right.
    \hspace{5 mm} 
    \rho(x) = \frac{1}{\sigma(x)} \exp\left(
        \int_{}^{x} \frac{\tau(y)}{\sigma(y)} \d y
    \right),
    \hspace{5 mm} 
    \bk{f}{g} = \int_{a}^{b} \rho(x) f(x) g(x) \d x.
\end{equation*}
Для ортогональных полиномов обязательно $\deg \sigma \leq 2$, $\deg \tau \leq 1$, $|\sigma''|+ |\tau'| \neq 0$. Тогда
\begin{equation*}
    f_n (x) = \frac{\alpha_n}{\rho(x)} \partial_x^n \left(\sigma^n (x) \times \rho(x)\right), \hspace{5 mm}     
    \alpha_n = (-1)^n a_n n! \prod_{k=0}^{n-1} \frac{1}{\lambda_n^{(k)}},
\end{equation*}
где 
\begin{equation*}
    \lambda_n = - n \tau' - \frac{n (n-1)}{2} \sigma'',
    \hspace{5 mm} 
    \lambda_n^{(k)} = \lambda_n + k \tau' + \frac{k (k-1)}{2} \sigma'',
\end{equation*}
что гордо именуется обобщенной формулой Родрига. 

Важно помнить, что $\forall$ клоп $\deg n$ $\bot$ $x^m$ при $m < n$. Можем составить рекуррентное соотношение:
\begin{equation*}
    p_{n+1} (x) = (A_n x + B_n) p_n (x) + X_n p_{n-1} (x),
    \hspace{5 mm} 
    A_n = \frac{
        \bk{p_{n+1}}{p_{n+1}}
    }{
        \bk{p_{n+1}}{x p_n}
    },
    \hspace{5 mm} 
    B_n = -A_n \frac{\bk{x p_n}{p_n}}{\bk{p_n}{p_n}},
    \hspace{5 mm} 
    C_n = - \frac{A_n}{A_{n-1}} \frac{\|p_n\|^2}{\|p_{n-1}\|^2}.
\end{equation*}
Нормировка может быть найдена, как
\begin{equation*}
    \|p_n\|^2 = (-1)^n \alpha_n a_n n! \int_{a}^{b} \sigma^n (x) \rho(x) \d x.
\end{equation*}
Производящая функция может быть найдена, как
\begin{equation*}
    \Psi(x, z) = \sum_{n=0}^{\infty}  \frac{\tilde{p}_n (\lambda)}{n!} z^n = 
    \frac{\tau(t_0)/ \rho(x)}{1 - z \sigma' (t_0)},
    \hspace{5 mm} 
    t_0 - x -  z \sigma(t_0) = 0.
\end{equation*}
Обычно $\alpha_n$ такой, что
\begin{equation*}
    \Psi(x, z) = \sum_{n=0}^{\infty} p_n (x) z^n.
\end{equation*}


\textbf{Полиномы Лежандра}. Дифференциральное уравнение ($a,\, b = -1,\, 1$):
\begin{equation*}
    \sigma(x) = 1-  x^2, 
    \hspace{5 mm} 
    \tau(x) = -2 x = \sigma',
    \hspace{5 mm} \rho = 1,
    \hspace{10 mm} 
    (1- x^2) P_n'' - 2 x P_n' + n(n+1)P_n = 0.
\end{equation*}
Формула Родрига:
\begin{equation*}
    P_n (x) = \frac{(-1)^n}{2^n n!} \partial_x^n (1-x^2)^n, 
    \hspace{5 mm} 
    \alpha_n = \frac{(-1)^n}{2^n n!},
    \hspace{5 mm} 
    a_n = \frac{(2n)!}{2^n (n!)^2}.
\end{equation*}
Нормировка:
\begin{equation*}
    \|P_n\|^2 = \frac{2}{2n + 1}.
\end{equation*}
Рекуррентное соотноешение:
\begin{equation*}
    A_n = \frac{\|p_n+1\|^2}{\bk{p_{n+1}}{x p_n}} = \frac{a_{n+1}}{a_n} = \frac{2n + 1}{n+1},
    \hspace{5 mm} 
    B_n = 0,
    \hspace{5 mm} 
    C_n = - \frac{n}{n+1},
\end{equation*}
подставляя, приходим к
\begin{equation*}
    (n+1) P_{n+1} (x) - (2n+1) x P_n + n  P_{n-1} (x) = 0.
\end{equation*}
Производяшая функция:
\begin{equation*}
    \psi(x, z) = \sum_{n=0}^{\infty} P_n (x) z^n = \frac{1}{\sqrt{z^2 - 2 z x + 1}}.
\end{equation*}
Умеем дифференцировать
\begin{equation*}
    (x^2-1) \frac{d P_n}{d x}  = n (x P_n (x) - P_{n-1} (x)).
\end{equation*}


\textbf{Полиномы Эрмита}. Дифференциральное уравнение
\begin{equation*}
    \sigma = 1, \hspace{5 mm} 
    \tau = - 2 x,
    \hspace{5 mm} \rho(x)  = e^{-x^2},
    \hspace{5 mm} 
    H_n'' - 2 x H_n' + 2 n H_n = 0, 
    \hspace{5 mm} 
    (e^{-x^2} H_n')' = -2 n e^{-x^2} H_n.
\end{equation*}
Знаем, что формула Родрига примет вид
\begin{equation*}
    H_n (x) = (-1)^n e^{x^2} \partial_x^n e^{-x^2},
    \hspace{5 mm} 
    a_n = 2^n,
\end{equation*}
тогда
\begin{equation*}
    \|H_n\|^2 =  2^n n! \sqrt{\pi}.
\end{equation*}
Рекуррентное соотношение:
\begin{equation*}
    H_{n+1} (x) = 2 x H_n (x) - 2 n H_{n-1} (x).
\end{equation*}
Производящаяя функция:
\begin{equation*}
    \psi(x,z) = e^{-z^2+2 zx} = \sum_{n=0}^{\infty}  \frac{H_n (x)}{n!} z^n.
\end{equation*}