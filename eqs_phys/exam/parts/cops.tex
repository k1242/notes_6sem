



\textbf{Полиномы Лежандра}. Дифференциральное уравнение ($a,\, b = -1,\, 1$):
\begin{equation*}
    \sigma(x) = 1-  x^2, 
    \hspace{5 mm} 
    \tau(x) = -2 x = \sigma',
    \hspace{5 mm} \rho = 1,
    \hspace{10 mm} 
    (1- x^2) P_n'' - 2 x P_n' + n(n+1)P_n = 0.
\end{equation*}
Формула Родрига:
\begin{equation*}
    P_n (x) = \frac{(-1)^n}{2^n n!} \partial_x^n (1-x^2)^n, 
    \hspace{5 mm} 
    \alpha_n = \frac{(-1)^n}{2^n n!},
    \hspace{5 mm} 
    a_n = \frac{(2n)!}{2^n (n!)^2}.
\end{equation*}
Нормировка:
\begin{equation*}
    \|P_n\|^2 = \frac{2}{2n + 1}.
\end{equation*}
Рекуррентное соотноешение:
\begin{equation*}
    A_n = \frac{\|p_n+1\|^2}{\bk{p_{n+1}}{x p_n}} = \frac{a_{n+1}}{a_n} = \frac{2n + 1}{n+1},
    \hspace{5 mm} 
    B_n = 0,
    \hspace{5 mm} 
    C_n = - \frac{n}{n+1},
\end{equation*}
подставляя, приходим к
\begin{equation*}
    (n+1) P_{n+1} (x) - (2n+1) x P_n + n  P_{n-1} (x) = 0.
\end{equation*}
Производяшая функция:
\begin{equation*}
    \psi(x, z) = \sum_{n=0}^{\infty} P_n (x) z^n = \frac{1}{\sqrt{z^2 - 2 z x + 1}}.
\end{equation*}
Умеем дифференцировать
\begin{equation*}
    (x^2-1) \frac{d P_n}{d x}  = n (x P_n (x) - P_{n-1} (x)).
\end{equation*}


\textbf{Полиномы Эрмита}. Дифференциральное уравнение
\begin{equation*}
    \sigma = 1, \hspace{5 mm} 
    \tau = - 2 x,
    \hspace{5 mm} \rho(x)  = e^{-x^2},
    \hspace{5 mm} 
    H_n'' - 2 x H_n' + 2 n H_n = 0, 
    \hspace{5 mm} 
    (e^{-x^2} H_n')' = -2 n e^{-x^2} H_n.
\end{equation*}
Знаем, что формула Родрига примет вид
\begin{equation*}
    H_n (x) = (-1)^n e^{x^2} \partial_x^n e^{-x^2},
    \hspace{5 mm} 
    a_n = 2^n,
\end{equation*}
тогда
\begin{equation*}
    \|H_n\|^2 =  2^n n! \sqrt{\pi}.
\end{equation*}
Рекуррентное соотношение:
\begin{equation*}
    H_{n+1} (x) = 2 x H_n (x) - 2 n H_{n-1} (x).
\end{equation*}
Производящаяя функция:
\begin{equation*}
    \psi(x,z) = e^{-z^2+2 zx} = \sum_{n=0}^{\infty}  \frac{H_n (x)}{n!} z^n.
\end{equation*}


\textbf{Полиномы Лаггера}. Живут на интервале от $(0,\, +\infty)$, с дифференциальным уравнением, вида
\begin{equation*}
    \hat{L} =  x \partial_x^2 + (1 +\alpha - x) \partial_x,
    \hspace{10 mm} 
    \rho = x^{\alpha} e^{-x},
\end{equation*}
для $\alpha > -1$. Собственные числа $\lambda_n = n$, формула Родрига имеет вид
\begin{equation*}
    L_n^{(\alpha)} = \frac{1}{n!} x^{-\alpha} e^x \partial_x^n \left[x^{n+\alpha} e^{-x}\right].
\end{equation*}
