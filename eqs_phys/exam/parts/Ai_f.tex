\textbf{Метод Лапласа}. Рассмотрим дифференциральное уравнение, вида
\begin{equation*}
    (a_n z + b_n) f^{(n)} + \ldots + (a_1 z + b_1) f^{(1)} + (a_0 z + b_0) f^{(0)} = 0,
    \hspace{10 mm} 
    f(z) = \int_C \tilde{f} (p) e^{pz} \d p,
\end{equation*}
где $f(p) e^{pz} |_{\partial C}^{\forall  z} = 0$. Тогда введем полиномы $A(p)$ и $B(p)$ такие, что
\begin{equation*}
    - \partial_p \left[ A(p) f(p)\right] + B(p) f(p) = 0,
    \hspace{2.5 mm} 
    A(p) = a_n p^n +\ldots + a_1 p + a_0,
    \hspace{2.5 mm} 
    B(p) = b_n p^n + b_{n-1} p^{n-1} + \ldots + b_1 p + b_0.
\end{equation*}
Решая, находим образ Лапласа
\begin{equation*}
    \tilde{f}(p) = \frac{1}{A(p)} \exp\left(
        \int_{p_0}^{p} \frac{B(t)}{A(t)}\d t
    \right).
\end{equation*}


\textbf{Метод перевала}. Действительный метод перевала:
\begin{equation*}
    \int_{-\infty}^{+\infty} e^{f(x)} g(x) \d x = g(x_0) e^{f(x_0)} \sqrt{\frac{2\pi}{|f''(x_0)|}}.
\end{equation*}
Для стационарной фазы:
\begin{equation*}
    \int_{-\infty}^{+\infty} e^{i f(x)} g(x) \d x = g(x_0) e^{i f (x_0)} 
    \sqrt{\frac{2\pi}{|f''(x_0)}} e^{\pm i \pi/4},
\end{equation*} 
где $\pm$ согласован с $\sign f''$. 
Для комплексного метода перевала
\begin{equation*}
    I = \int_C e^{f(z)} g(z) \d z = g(z_0) e^{f(z_0)} e^{i \varphi} \sqrt{\frac{2\pi}{|f''|}},
    \hspace{5 mm} 
    \varphi = \frac{1}{2}\left(\pm \pi - \arg f''(z_0) \right).
\end{equation*}



\textbf{Функция Эйри}. Решаем уравнение, вида
\begin{equation*}
    \partial_x^2 f - x f = 0,
    \hspace{0.5cm} \Rightarrow \hspace{0.5cm}
    f(x) = \int_C e^{x t - t^3/3} \d t.
\end{equation*}
Так приходим к
\begin{equation*}
    \Ai (x) = 
    \frac{1}{2 \pi i} \int_{- i \infty}^{+i \infty} e^{xt - t^3/3} \d t= 
    \frac{1}{\pi} \int_{0}^{\infty} \cos(x u + u^3/3) \d u.
\end{equation*}
Запишем асимптотики на бесконечности:
\begin{align*}
    \Ai(z) &\approx \frac{1}{2 \sqrt{\pi}} \frac{1}{z^{1/4}} \exp\left(
        - \frac{2}{3} z^{3/2}
    \right), \hspace{10 mm}  z \to +\infty\\
    \Ai(z) &\approx \frac{1}{\sqrt{\pi}} \frac{1}{(-z)^{1/4}} \sin\left(
        \frac{2}{3} (-z)^{3/2} + \frac{\pi}{4}
    \right),
    \hspace{10 mm} z \to - \infty.
\end{align*}

В качетсве второго решения выбрается 
\begin{equation*}
    \Bi (x) = \frac{1}{\pi} \int_{0}^{\infty} \left[
        e^{xu - u^3/3}  + \sin (xu + u^3/3)
    \right]\d u,
\end{equation*}
с асиматотиками, вида
\begin{align*}
    \Bi(z) &\approx \frac{1}{\sqrt{\pi}} \frac{1}{z^{1/4}} \exp\left( \frac{2}{3} z^{3/2}
    \right), \hspace{10 mm} z \to +\infty \\
    \Bi(z) &\approx  \frac{1}{\pi} \frac{1}{(-z)^{1/4}} \cos\left(\frac{2}{3}(-z)^{3/2} + \frac{\pi}{4}\right), \hspace{10 mm} z \to -\infty.
\end{align*}
