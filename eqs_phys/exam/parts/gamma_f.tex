
\textbf{Гамма-функция}.
Найдем некоторые интересные свойства:
\begin{equation*}
    \Gamma(z+1) = \int_{0}^{\infty} t^{z} e^{-t} \d t \overset{t \to \tau x}{=} 
    x^{z+1} \int_{0}^{\infty} \tau^{z} e^{- \tau x} \d \tau,
    \hspace{10 mm} 
    \frac{1}{x^z} = \frac{1}{\Gamma(z)} \int_{0}^{\infty} \tau^{z-1} e^{- \tau x} \d \tau.
\end{equation*}
% Также знаем $\Gamma(n+1) = n!$, $\Gamma(2n+1)$, $\Gamma(1/2) = \sqrt{\pi}$ и т.д.
Существует аналитическое продолжение:
\begin{equation*}
    \Gamma(z) = \frac{1}{1-e^{2 \pi i z}} \int_C t^{z-1} e^{-t} \d t.
\end{equation*}
Видим, что у $\Gamma(z)$ есть особенности $z \in \mathbb{Z}$, где $z \in \mathbb{N}$ -- УОТ, и $z \in \mathbb{Z}\backslash \mathbb{N}$ -- П1П:
\begin{equation}
    \res_{-n} \Gamma(z) = \frac{(-1)^n}{n!},
\end{equation}
Могут пригодиться следущие выражения для $\Gamma$-функции:
\begin{equation*}
    \Gamma(z) \Gamma\left(z + \tfrac{1}{2}\right) = \frac{\sqrt{\pi}}{2^{2z-1}} \Gamma(2 z),
    \hspace{10 mm} 
    \Gamma(z) \Gamma(1-z) = \frac{\pi}{\sin \pi z}.
\end{equation*}
А также \textit{формула Стирлинга}, которую нетрудно получить методом перевада:
\begin{equation}
    \Gamma(z+1) = \int_{0}^{\infty}  t^{z} e^{-t} \d t  = \int_0^\infty 
    e^{z \ln t - t} \d t \approx e^{z \ln z - z} \sqrt{2 \pi z} = \sqrt{2 \pi z} \left(\frac{z}{e}\right)^{z}.
\end{equation}


\textbf{Дигамма-функция}. По определнию дигамма-функция $\psi(z)$:
\begin{equation*}
    \psi(z) \overset{\mathrm{def}}{=}  \left(\ln \Gamma(z)\right)' = \frac{\Gamma'(z)}{\Gamma(z)}.
\end{equation*}
Заметим, что $\psi(1) = - \gamma$, где $\gamma \approx 0.58$ -- постоянная Эйлера-Маскерони. Найдём
\begin{equation*}
    \psi(z+1)= \left(\ln z + \ln \Gamma(z)\right) = \frac{1}{z} + \psi(z),
    \hspace{0.5cm} \Rightarrow \hspace{0.5cm}
    \psi(N+1) = \frac{1}{N} + \psi(N) = \sum_{n=1}^{N} \frac{1}{n} + \psi(1).
\end{equation*}
Также бывает полезно
\begin{equation*}
    \psi(x + N + 1) = \frac{1}{x + N} + \psi(x+ N) = \frac{1}{x+N} + \ldots + \frac{1}{x+1} + \psi(x+1).
\end{equation*}
Вспомним, что $\Gamma(z) \Gamma(1-z) = \frac{\pi}{\sin \pi z}$. Тогда
\begin{equation*}
    \psi(-z) - \psi(z) = \pi \ctg \pi z.
\end{equation*}
Асимптотика для $\psi(z\to \infty)$:
\begin{equation*}
    \psi(z\to \infty) = \left(\ln \Gamma(z)\right)' = \ln z + \frac{1}{2z} + o(1) = \ln z + o(1).
\end{equation*}



\textbf{Бета-функция}. Рассмотрим $B$-функцию:
\begin{equation*}
      B(\alpha,\, \beta) = \int_{0}^{1} t^{\alpha-1} (1-t)^{\beta-1} \d t,
      \hspace{5 mm} 
      \Re \alpha, \beta > 0,
      \hspace{10 mm} 
      B(\alpha,\, \beta) = \frac{\Gamma(\alpha) \Gamma(\beta)}{\Gamma(\alpha + \beta)},
\end{equation*}  
которую бывает удобно доставать в интегралах.


% асимптотика интегралов Лапласа
% \textbf{Метод перевала}.  Представим семейство интегралов с параметром $\lambda$:
% \begin{equation*}
%     I_\lambda = \int_{-\infty}^{+\infty} g(x) e^{\lambda f(x)} \d x.
% \end{equation*}
% При этом предположим, что $f(x)$ такая, что существует единственный максимум в точке $x_0$. Тогда
% \begin{equation*}
%     I_\lambda \approx g(x_0) \int_{-\infty}^{+\infty} e^{\lambda f(x)} \d x.
% \end{equation*}
% Теперь воспользуемся аналитичностью функции $f(x)$:
% \begin{equation*}
%     f(x) = f(x_0) + \frac{f'(x_0)}{2} (x-x_0)^2 + \frac{f''(x_0)}{2} (x-x_0)^2 + o(x-x_0)^2.
% \end{equation*}
% Подставляя в интеграл, находим
% \begin{equation*}
%     I_\lambda=  g(x_0) e^{\lambda f(x_0)} \int_{-\infty}^{+\infty} e^{\lambda f'(x_0) (x-x_0)^2/2} \d x = g(x_0) e^{\lambda f(x_0)} \sqrt{\frac{2\pi}{|\lambda f'' (x_0) |}}.
% \end{equation*}

% Пусть $\lambda$ нет. Тогда достаточно потребовать $|f''(x_0)|$ большой -- максимум резкий. 
% Тогда
% \begin{equation*}
%     |f''(x_0) (x-x_0)^2| \sim 1,
%     \hspace{0.5cm} \Rightarrow \hspace{0.5cm}   
%     |x-x_0| \frac{1}{\sqrt{f'(x_0)}},
%     \hspace{0.5cm} \Rightarrow \hspace{0.5cm}   
%     |f'''(x_0) (x-x_0^3)| \ll 1,
%     \hspace{0.5cm} \Rightarrow \hspace{0.5cm}
%     (f'')^3 \gg (f''')^2.
% \end{equation*}











