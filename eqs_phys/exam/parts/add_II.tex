\Tsec[]{Метод Боголюбова-Крылова}

Рассмотрим прозвольное возмущение гармонического осциллятора:
\begin{equation}
    \left(\partial_t^2 + \omega_0^2\right) x(t) = \varepsilon f(t, x, \dot{x}).
    \label{sloweq}
\end{equation}
Приближенно (до $o(\varepsilon)$) можем методом Боголюбова-Крылова найти
решение в виде
\begin{equation}
    x(t) = A(t) \cos(\omega_0 t + \varphi(t)),
    \label{sloweqview}
\end{equation}
где зависимость от времени амплитуды и фазы определяестся уравнениями
\begin{align}
    \partial_t A(t) &= \frac{1}{2\pi \omega_0}\int_{\omega_0 t-\pi}^{\omega_0 t+\pi} f(\tau, x, \dot{x}) \cos\left(\omega_0 \tau + \varphi(t)\right) \d (\omega_0\tau), 
    \label{slowA}
    \\
    \partial_t \varphi(t) &= \frac{-1}{2 \pi A \omega_0} \int_{\omega_0 t-\pi}^{\omega_0 t + \pi} f(\tau, x, \dot{x}) \sin(\omega_0 \tau + \varphi(t)) \d (\omega_0 \tau).
    \label{slowphi}
\end{align}
