\Tsec[]{Уравнения Пуассона и Лапласа}


Научимся решать \textit{уравнение Пуассона} $\nabla^2 f  = \varphi$, которое при $\varphi = 0$ переходит в \textit{уравнению Лапласа} $\nabla^2 f = 0$. 
Функции, удовлетворяющие уравнение Лапласа называют \textit{гармоническими}, которые существуют только на некоторой ограниченной области. 
Иногда бывает проще решать \textit{уравнение Дебая} $(\nabla^2 - \kappa^2) f = \varphi$.


\textbf{Функция Грина}. Решение как обычно можем искать в виде
\begin{equation*}
    f(\vc{r}) = \int_{\mathbb{R}^3} G(\vc{r}  - \vc{r}') \varphi(\vc{r}) \d^3 r.
\end{equation*}
В зависимости от размерности пространства $n$, функция Грина $G_n$ будет равна
\begin{equation}
    G_2(r) = \frac{\ln r}{2 \pi},
    \hspace{5 mm} 
    G_3(\vc{r}) = - \frac{1}{4 \pi r},
    \hspace{5 mm} 
    G_{n > 2}(x) = \frac{1}{S_{n-1}} \frac{1}{r^{n-2}},
\end{equation}
где $S_{n-1} = 2 \pi^{n/2} / \Gamma\left(\frac{n}{2}\right)$ -- площадь $n-1$ мерной единичной сферы.

\textit{Кстати}. 
Для уравнения Дебая в $\mathbb{R}^3$ функция Грина с $\hat{L} = \nabla^2 - \kappa^2$ будет равна
\begin{equation*}
    G(\vc{r}) = - \frac{e^{- \kappa r}}{4 \pi r}.
\end{equation*}
Для сферически симметричного потенциала $\varphi(\vc{r}) \equiv \varphi(r)$, решение уравнения Пуассона упростится до 
\begin{equation*}
    f(\vc{r}) = \frac{1}{r} \int_{0}^{r} \varphi(\rho) \rho^2 \d \rho + \int_{r}^{\infty} \varphi(\rho) \rho \d \rho.
\end{equation*}
\red{(проверить)}


\Tsec[]{Двумерные гармонические функции}

Отдельно рассмотрим случай $n=2$. Часто задача формулируется в виде \textit{задачи Дирихле}:
\begin{equation*}
    \nabla^2 f = 0, 
    \hspace{5 mm} 
    f|_{\partial D} = f_0 (\vc{r}),
\end{equation*}
то есть функция задана на границе некоторой области. Будем искать решение в виде
\begin{equation*}
    f(z) = u(z) + i v(z),
    \hspace{10 mm} 
    \nabla^2 u = \nabla^2 v = 0.
\end{equation*}
Если знаем комплексную функцию $f(z)$ такую, что $\Re f |_{\partial D} = f_0$, тогда $\Re f(z)$ решает задачу Дирихле.
Далее конформным преобразованием переводим любую область $D$ в круг/полуплоскость, где задача Дирихле решается, а дальше отображаем назад. 

Рассмотрим полуплоскость $D= \{(x,y) \mid y > 0\}$, c заданным значением $f(x, 0) = f_0(x)$. Тогда решением будет
\begin{equation}
    f(x, y) = \int_{\mathbb{R}} \frac{d\xi}{\pi}\ \frac{y}{(x-\xi)^2 + y^2} f_0 (\xi).
\end{equation}
Для круга радиуса $R$ c заданным граничным условием, вида $f(R \cos t,\, R \sin t) = f_0 (t)$ решением будет
\begin{equation}
    f(r \cos \varphi,\, r \sin \varphi) = 
    \int_{0}^{2\pi} \frac{dt}{2\pi}
    \frac{R^2-r^2}{R^2 - 2 R r \cos(t-\varphi) + r^2} \, f_0 (t).
\end{equation}



