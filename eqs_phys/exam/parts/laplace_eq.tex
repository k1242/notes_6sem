\Tsec[]{Уравнение Лапласа  (черновик)}


\subsection*{Многомерие \texorpdfstring{$\mathbb{R}^3$}{R3}}

Рассмотрим $\mathbb{R}^3$:
\begin{equation*}
    \nabla^2 f = \varphi,
\end{equation*}
где все линейно, всё хорошо. Как обычно будем искать функцию, виде
\begin{equation*}
    f(\vc{r}) = \int_{\mathbb{R}^3} G(\vc{r}  - \vc{r}') \varphi(\vc{r}) \d^3 r. 
\end{equation*}
Функцию Грина найдём в виде
\begin{equation*}
    \nabla^2 G(\vc{r}) = \delta(r^3) = \delta(x) \delta(y) \delta(z),
    \hspace{10 mm}  
    \int f(\vc{r}) \delta(\vc{r}- \vc{r}') \d^3 \vc{r}' = f(\vc{r}').
\end{equation*}
Можем свести уравнение Лапласа, к уравнению Дебая:
\begin{equation*}
    (\nabla^2 - \kappa^2) G(\vc{r}) = \delta(\vc{r}),
\end{equation*} 
которое очень удобно раскладывать по Фурье:
\begin{align*}
    &\text{ПФ}: 
    &G(\vc{k}) &= \int_{\mathbb{R}^3} G(\vc{r}) e^{- i \smallvc{k} \cdot \smallvc{r}} \d \vc{r}, \\
    &\text{ОПФ}: 
    &G(\vc{r}) &= \int_{\mathbb{R}^3} G(\vc{r}) e^{i \smallvc{k} \cdot \smallvc{r}} \frac{\d \vc{k}}{(2 \pi)^3}.
\end{align*}
Также вспомним, что
\begin{equation*}
    \partial_m G(\vc{r}) e^{- i \smallvc{k} \cdot \smallvc{r}} \d \vc{r} = i k_m G(\vc{k}),
\end{equation*}
а значит
\begin{equation*}
    (-k^2 - \kappa^2) G(\vc{k}) = 1,
    \hspace{0.5cm} \Rightarrow \hspace{0.5cm}
    G(\vc{k})=- \frac{1}{k^2 + \kappa^2},
    \hspace{0.5cm} \Rightarrow \hspace{0.5cm}
    G(\vc{r}) = \int_{\mathbb{R}^3} \frac{e^{i \smallvc{k} \cdot \smallvc{r}}}{k^2 + \kappa^2} \frac{\d \vc{k}}{(2 \pi)^3}.
\end{equation*}
Переходим в сферические координаты, получаем, что
\begin{equation*}
    G(\vc{r}) = - \frac{2 \pi}{(2 \pi)^3} \int_{0}^{\infty} \frac{k^2}{k^2 + \kappa^2} \int_{0}^{\pi} 
    \sin \theta  e^{i k r \cos \theta}  \d \theta \d k = 
    - \frac{e^{- \kappa r}}{4 \pi r}
    .
\end{equation*}
Устремляя $\kappa \to 0$, находим
\begin{equation*}
     \nabla^2 G = \delta(\vc{r}),
     \hspace{0.5cm} \Rightarrow \hspace{0.5cm}
     G = - \frac{1}{4 \pi r}.
 \end{equation*} 





% \textbf{Пример}. Пусть 

\subsection*{Многомерие \texorpdfstring{$\mathbb{R}^2$}{R2}}

Для Гаусса можно найти, что
\begin{equation*}
    G^{[\dim = n]}(x) = \frac{1}{\sigma_{n-1}} \frac{1}{r^{n-2}},
\end{equation*}
где $\sigma_{n-1}$ -- площадь $n-1$ мерной сферы. 



Вообще часто задача формулируется в виде задачи Дирихле:
\begin{equation*}
    \nabla^2 f = 0, 
    \hspace{5 mm} 
    f'_{\partial D} = f_0 (\vc{r}),
\end{equation*}
то есть функция задана на границе некоторой области. Пусть
\begin{equation*}
    f(z) = u(z) = i v(z),
    \hspace{5 mm} 
    \nabla^2 u = \nabla^2 v = 0.
\end{equation*}

Пусть знаем комплексную функцию $f(z)$ такую, что $\Re f |_{\partial D} = f_0$, тогда $\Re f(z)$ решает задачу Дирихле.
Далее конформным преобразованием переводим любое $D$ в круг, в круге задача Дирихле решается, а дальше отображаем назад. 


Пусть задана функция $u_0 (x) = u(x, 0)$. Вообще можно было бы разложить по Фурье $u$, и записать
\begin{equation*}
    \nabla^2  u = 0,
    \hspace{5 mm}   
    u(x, 0) = u_0 (x).
\end{equation*}
Тогда
\begin{equation*}
    u(q, y)  = \int_{\mathbb{R}} e^{- i q x} u(x, y) \d x,
    \hspace{0.5cm} \Rightarrow \hspace{0.5cm}
    \nabla^2 u = - q^2 u(q, y) = 0.
\end{equation*}
Так приходим к
\begin{equation*}
    u(q, y) = \exp\left(- |q| y\right) u(q, 0),
    \hspace{0.5cm} \Rightarrow \hspace{0.5cm}   
    u(x, y) = \int_{\mathbb{R}} \frac{\d q}{2\pi} e^{i q x} \underbrace{e^{- |q| y}}_{h(q)} u(q, 0).
\end{equation*}
Произведение Фурье образов -- свёртка:
\begin{equation*}
    u(x, y) = \int_{\mathbb{R}} \d \xi h(x - \xi,\,  y) u_0 (\xi).
\end{equation*}
Найдём, что
\begin{equation*}
    \int_{\mathbb{R}} \frac{\d q}{2 \pi} e^{i q x} e^{-|q| y} = \frac{y}{\pi (x^2 + y^2)}.
\end{equation*}
Подставляем, и находим:
\begin{equation*}
    u(x, y) = \int_{\mathbb{R}} d \xi \, \frac{y/\pi}{(x-\xi)^2 + y^2} u_0 (\xi),
\end{equation*}
где 
\begin{equation*}
    \frac{y/\pi}{(x-\xi)^2 + y^2} = \Im \frac{-1}{x + i y - \xi},
    \hspace{0.5cm} \Rightarrow \hspace{0.5cm}   
    f(z) = - \frac{1}{\pi} \int_{\mathbb{R}} \frac{1}{z-\xi} u_0 (\xi),
\end{equation*}
что в некотором смысле привело нас к интегралу Коши, так что и $\nabla^2 f = 0$ и гран. условия удовлетворяются. 



\textbf{Пример}. Рассмотрим
\begin{equation*}
    u_0 (x) = \frac{1}{1 + x^2},
    \hspace{5 mm} 
    u(x, y) = \int_{\mathbb{R}} d \xi \frac{1}{\pi} \frac{y}{y^2 + (x-\xi)^2} \frac{1}{1+\xi^2} = \frac{1 + y}{x^2 + (1+y)^2}.
\end{equation*}