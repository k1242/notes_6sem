
\Tsec[]{Уравнения Хопфа и Бюргерса}

\textbf{Уравнение Хопфа}.
В акустике естественно возникает уравнение Хопфа:
\begin{equation*}
    \partial_t u + u\, \partial_x u = 0.
\end{equation*}
Решение может быть найдено в виде
\begin{equation*}
    x(t) = x_0 + u_0 (x_0) t, \hspace{5 mm} 
    u(x(t), t) = c(x_0) = u_0 (x_0).
\end{equation*}
где сначала разрешаем уравнение $c = u_0 (x_0)$ относительно $c = c(x_0)$, а потом разрешаем уравнение на $x(t)$ относительно $c = c(x(t), t)$. 
Зная, что $u(x(t), t) = c(x(t), t)$, находим $u(x, t) = c(x, t)$. 

 Добавим к уравнению накачку:
\begin{equation*}
    \partial_t u + u\, \partial_x u = f(x, t).
\end{equation*}
Система может быть сведена к
\begin{equation*}
    \left\{\begin{aligned}
        \dot{u} &= f(t, x(t)) \\
        \dot{x} &= u(t, x(t))
    \end{aligned}\right.
    \hspace{5 mm} \Leftrightarrow \hspace{5 mm} 
    \ddot{x} = f(x, t),
    \hspace{0.5cm} \Rightarrow \hspace{0.5cm}
    x(t) = x(t, x_0, \dot{x}_0),
\end{equation*}
где $\dot{x}_0 = u_0(x_0)$. Сначала разрешаем уравнение $x(t)$ относительно $x_0 = x_0(t, x)$, а потом подставляем этот $x_0$ в $u(t, x) = \dot{x}(t, x_0(t,x))$, что и является решением исходной задачи.



\textbf{Уравнение Бюргерса}.
Добавим диссипацию в уравнение Хопфа:
\begin{equation*}
    \partial_t u + u\, \partial_x u = \partial_x^2 u,
\end{equation*}
так получим \textit{уравнение Бюргерса}.

Заметим, что преобразование Коула-Хопфа
\begin{equation*}
    \psi = \exp\left(- \tfrac{1}{2}h\right),
    \hspace{5 mm} 
    u = \partial_x h,
    \hspace{0.5cm} \Rightarrow \hspace{0.5cm}
    (\partial_t - \partial_x^2)\psi = 0.
\end{equation*}
Имея начальные условия для $\psi_0(x)$, можем найти
\begin{equation*}
    \psi(t, x) = \int_{\mathbb{R}} \psi_0 (y) \frac{\theta(t)}{\sqrt{4 \pi t}} \exp\left(-\frac{(x-y)^2}{4t}\right) \d y,
\end{equation*}
откуда находим решение
\begin{equation*}
    u(t, x) =  -2 \partial_x \ln \psi(t, x).
\end{equation*}
