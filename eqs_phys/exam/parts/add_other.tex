
\Tsec[]{Вычеты}

Интеграл по дуге может быть найден, как
\begin{align*}
    \int_C f(z) \d z = 2 \pi i \sum_{z_j} \res_{z_j} f(z),
    \hspace{5 mm} 
    \res_{z_j} f(z) &= \lim_{\varepsilon \to 0} \varepsilon \int_0^{2\pi} \frac{\d \varphi}{2\pi} e^{i \varphi} f(z_j + \varepsilon e^{i \varphi}) \\ 
    &= \frac{1}{(m-1)!} \lim_{z \to z_j} \left(
        \frac{d^{m-1} }{d z^{m-1}} (z-z_j)^m f(z)
    \right),
\end{align*}
где $m$ -- степень полюса. 





\Tsec[]{Неоднородная релаксация}

Для одномерного случая
\begin{equation*}
    \big(\partial_t + \gamma(t)\big) x(t) = \varphi(t),
    \hspace{0.5cm} \Rightarrow \hspace{0.5cm}
    x(t) = \int_{-\infty}^{+\infty}  G(t,s) \varphi(s) \d s,
    \hspace{5 mm} 
    G(t,\,  s) = \theta(t-s) \exp\left(
        - \int_{s}^{t} \gamma(\tau) \d \tau
    \right),
\end{equation*}
где всё также $G(t, s>t) = 0$ в силу стремления к принципу причинности. 



\Tsec[]{Уравнение Вольтера}



\textbf{Уравнение Вольтерра}. Интегральное уравнение Вольтерра первого рода с однородным ядром:
\begin{equation*}
    \int_{0}^{t}  K(t-s) f(s) \d s = \varphi(t).
\end{equation*}
Решение может быть найдено через обратное преобразование Лапласа
\begin{equation*}
    f(t) = \int_{c-i \infty}^{c+i \infty} \frac{d p}{2 \pi i} \exp(pt) \tilde{f}(p),
    \hspace{10 mm} 
    \tilde{f}(p) = \frac{\tilde{\varphi}(p)}{\tilde{K}(p)}.
\end{equation*}
Но есть один нюанс. При $K(t),\, \varphi(t) \overset{p \to \infty}{\to} K_0,\, \varphi_0$ получается, что $\tilde{K}(p),\, \tilde{\varphi}(p) \approx \frac{K_0}{p},\, \frac{\varphi_0}{p}$, тогда
\begin{equation*}
    f(t) = \frac{\varphi_0}{K_0} \delta(t) + \int_{c-i \infty}^{c+i \infty} \frac{d p}{2 \pi i} \exp(p t)
    \left(
        \frac{\tilde{\varphi}}{\tilde{K}} - \frac{\varphi_0}{K_0}
    \right),
\end{equation*}
при этом в отсутствие аналитичности в нуле нет ничего страшного. 




\Tsec[]{Задача Штурма-Лиувилля с периодическими граничными условиями}

Рассмотрим такой же $\hat{L}$, и граничные условия в виде
\begin{equation*}
    \hat{L} = \partial_x^2 + Q(x) \partial_x + U(x),
    \hspace{10 mm}
    \left\{\begin{aligned}
        f(a) &= f(b), \\
        f'(a) &= f'(b),
    \end{aligned}\right.
\end{equation*}
которые приводят к периодичности решения. 


Рассмотрим задачу
\begin{equation*}
    \hat{L} = \partial_x^2 + \kappa^2,
\end{equation*}
с условиями на $[-\pi, \pi]$. 

При $x < y$:
\begin{equation*}
    G(x, y) = A_1 (y) \sin \kappa(x + \pi) + B_1 (y) \cos \kappa( x + \pi),
\end{equation*}
и аналогично для $x > y$:
\begin{equation*}
    G(x, y) = A_2 \sin \kappa (x - \pi) + B_2 (y) \cos \kappa (x - \pi).
\end{equation*}
Запишем граничные условия:
\begin{align*}
    G(- \pi, y) = G(\pi, y), \hspace{0.5cm} \Rightarrow \hspace{0.5cm}
    B_1 (y) = B_2 (y) \overset{\mathrm{def}}{=} B(y) \\
    G'_x (-\pi, y) = G'_x (\pi, y),
    \hspace{0.5cm} \Rightarrow \hspace{0.5cm}
    A_1 (y) = A_2 (y) \overset{\mathrm{def}}{=}  A(y).
\end{align*}
Тогда нашли, что
\begin{equation*}
    G(x, y) = \left\{\begin{aligned}
        &A \sin \kappa (x + \pi) + B \cos \kappa (x + \pi) \\
        &A \sin \kappa (x - \pi) + B \cos \kappa (x - \pi) \\
    \end{aligned}\right.
\end{equation*}
Теперь запишем непрерывность:
\begin{equation*}
     A \sin \kappa (x + \pi) + B \cos \kappa (x + \pi) 
     = 
     A \sin \kappa (x - \pi) + B \cos \kappa (x - \pi).
\end{equation*}
А также скачок производной
\begin{equation*}
    G'_x(y + 0, y) - G'_x (y-0, y) = 1,
    \hspace{0.25cm} \Rightarrow \hspace{0.25cm}
        A \cos \kappa (x - \pi) - B \sin \kappa (x - \pi)  - 
        A \cos \kappa (x + \pi) + B \cos \kappa (x + \pi)
        = \kappa^{-1}.
\end{equation*}
Решая эту систему находим, что
\begin{equation*}
    2 \sin \pi \kappa 
    \begin{pmatrix}
        \cos \kappa y & - \sin \kappa y  \\
        \sin \xi y & \cos \kappa y  \\
    \end{pmatrix} \begin{pmatrix}
        A  \\
        B  \\
    \end{pmatrix}
    = \begin{pmatrix}
        0 \\ 1/\kappa
    \end{pmatrix},
    \hspace{0.25cm} \Rightarrow \hspace{0.25cm}
    \begin{pmatrix}
        A \\ B
    \end{pmatrix} = 
    \frac{1}{2 \sin \pi \kappa} \begin{pmatrix}
        \cos \kappa y & \sin \kappa y  \\
        \sin \kappa y & \cos \kappa y  \\
    \end{pmatrix}
    \begin{pmatrix}
        0 \\ 1/\kappa
    \end{pmatrix} = 
    \frac{1}{2 \kappa \sin \pi \kappa} \begin{pmatrix}
        \sin xy \\ \cos xy
    \end{pmatrix}.
\end{equation*}
Подставляя в $G(x, y)$, находим\footnote{
    К дз будет полезно заметить, что $G(x, y) = G(x-y)$ -- задача трансляционно инвариантна. 
} 
\begin{equation*}
    G(x, y) = \frac{1}{2 \kappa \sin \pi \kappa}
    \left\{\begin{aligned}
        &\cos \left(\kappa(x-y) + \kappa \pi\right), & x < y\\
        &\cos(\kappa (x-y) - \kappa \pi), & x > y.
    \end{aligned}\right.
\end{equation*}
Всё это было, повторимся, для уравнения:
\begin{equation*}
    \left(\partial_x^2 + \kappa^2\right) f(x) = \varphi(x),
    \hspace{0.5cm} \Rightarrow \hspace{0.5cm}   
    f(x) = 
    \int_{-\pi}^{+\pi} G(x, y) \varphi(y) \d y. 
\end{equation*}








\Tsec[]{Общий подход к ортогональным полиномам}

\textbf{Общий подход}. 
Знаем, что
\begin{equation*}
    \left\{\begin{aligned}
        (\sigma \rho f')' = - \lambda \rho f, \\
        (\sigma \rho)' =  \tau \rho,
    \end{aligned}\right.
    \hspace{5 mm} 
    \rho(x) = \frac{1}{\sigma(x)} \exp\left(
        \int_{}^{x} \frac{\tau(y)}{\sigma(y)} \d y
    \right),
    \hspace{5 mm} 
    \bk{f}{g} = \int_{a}^{b} \rho(x) f(x) g(x) \d x.
\end{equation*}
Для ортогональных полиномов обязательно $\deg \sigma \leq 2$, $\deg \tau \leq 1$, $|\sigma''|+ |\tau'| \neq 0$. Тогда
\begin{equation*}
    f_n (x) = \frac{\alpha_n}{\rho(x)} \partial_x^n \left(\sigma^n (x) \times \rho(x)\right), \hspace{5 mm}     
    \alpha_n = (-1)^n a_n n! \prod_{k=0}^{n-1} \frac{1}{\lambda_n^{(k)}},
\end{equation*}
где 
\begin{equation*}
    \lambda_n = - n \tau' - \frac{n (n-1)}{2} \sigma'',
    \hspace{5 mm} 
    \lambda_n^{(k)} = \lambda_n + k \tau' + \frac{k (k-1)}{2} \sigma'',
\end{equation*}
что гордо именуется обобщенной формулой Родрига. 

Важно помнить, что $\forall$ клоп $\deg n$ $\bot$ $x^m$ при $m < n$. Можем составить рекуррентное соотношение:
\begin{equation*}
    p_{n+1} (x) = (A_n x + B_n) p_n (x) + X_n p_{n-1} (x),
    \hspace{5 mm} 
    A_n = \frac{
        \bk{p_{n+1}}{p_{n+1}}
    }{
        \bk{p_{n+1}}{x p_n}
    },
    \hspace{5 mm} 
    B_n = -A_n \frac{\bk{x p_n}{p_n}}{\bk{p_n}{p_n}},
    \hspace{5 mm} 
    C_n = - \frac{A_n}{A_{n-1}} \frac{\|p_n\|^2}{\|p_{n-1}\|^2}.
\end{equation*}
Нормировка может быть найдена, как
\begin{equation*}
    \|p_n\|^2 = (-1)^n \alpha_n a_n n! \int_{a}^{b} \sigma^n (x) \rho(x) \d x.
\end{equation*}
Производящая функция может быть найдена, как
\begin{equation*}
    \Psi(x, z) = \sum_{n=0}^{\infty}  \frac{\tilde{p}_n (\lambda)}{n!} z^n = 
    \frac{\tau(t_0)/ \rho(x)}{1 - z \sigma' (t_0)},
    \hspace{5 mm} 
    t_0 - x -  z \sigma(t_0) = 0.
\end{equation*}
Обычно $\alpha_n$ такой, что
\begin{equation*}
    \Psi(x, z) = \sum_{n=0}^{\infty} p_n (x) z^n.
\end{equation*}


% \textbf{Огибающая}. Уравнение на огибающую может быть найдено в виде решения уравнения
% \begin{equation*}
%     \partial_t \psi + \vc{v} \cdot \nabla \psi = \frac{i \varepsilon}{\omega} e^{i \theta} \int_{0}^{2\pi} \frac{d \varphi}{2 \pi} e^{-i \varphi} f.
% \end{equation*}





