\Tsec[]{Задача Штурма-Лиувилля}

\textbf{Алгоритм (Фурье)}. Раскладываем 
\begin{equation*}
    G(x) = \sum_{n \neq 0} g_n e_n,
    \hspace{10 mm} 
    \delta(x) = \sum \frac{e_n (x)}{2 \pi},
    \hspace{0.5cm} \Rightarrow \hspace{0.5cm}
    \hat{L} G = \delta(x) - \frac{1}{2\pi}.
\end{equation*}
Знаем, что $\lambda_n g_n = \frac{1}{2\pi}$, а значит
\begin{equation*}
    g_n(x) = \frac{1}{2\pi} \frac{1}{- n^2},
    \hspace{0.5cm} \Rightarrow \hspace{0.5cm}   
    G(x) = \sum_{n \neq 0} \frac{1}{2\pi} \frac{1}{-n^2} e^{i n x},
\end{equation*}
и рассмотрим $0 < x < \pi$, суммирая это через вычеты, записываем
\begin{equation*}
    f(z) = \frac{e^{zx}}{2 \pi z^2}, 
    \hspace{0.5cm} \Rightarrow \hspace{0.5cm}   
    G(x) = \sum \oint_{in} \frac{\d z}{2 \pi i} f(z) g(z).
\end{equation*}
Соответственно, выберем
\begin{align*}
    g(z) = \frac{\pi e^{- \pi z}}{\sh (\pi z)}
\end{align*}
тогда
\begin{equation*}
    f(z) g(z) = \frac{\pi}{z^2} \frac{e^{(x-\pi)z}}{\sh \pi z},
\end{equation*}
получаем, что интеграл по душам вправо/влево  равен $0$, и остается только вычет в $z = 0$:
\begin{equation*}
    G(z) = - \res_0 f(z) g(z) = \ldots = - \frac{x^2}{4 \pi} + \frac{x}{2} - \frac{\pi}{6}.
\end{equation*}



\textbf{Алгоритм (сшивка)}. Решим задачу
\begin{equation*}
    \partial_x^2 G(x) = \delta(x) - \frac{1}{2\pi}.
\end{equation*}
Разбиваем $x < 0$ и $x > 0$:
\begin{align*}
    &x < 0, 
    & G = -\tfrac{x^2}{4 \pi} + a x + b, \\
    &x > 0, 
    & G = -\tfrac{x^2}{4 \pi} + c x + \varpi, 
\end{align*}
учитываем граничные условия:
\begin{equation*}
    G(-0) = G(+0),
    \hspace{5 mm} 
    G'(+0) - G'(-0) = 1,
    \hspace{0.5cm} \Rightarrow \hspace{0.5cm}   
    b = \varpi.
\end{equation*}
Также получаем, что $-a = b$.

Учтём, что $e_0$ не входит в $G$:
\begin{equation*}
    \langle G | e_0\rangle = 0 = \int_{-\pi}^{+\pi}G(x) \d x = 0,
    \hspace{0.5cm} \Rightarrow \hspace{0.5cm}
    b = - \frac{\pi}{6},
\end{equation*}
так и получаем все необходиме условия на $G(x, y)$. 




\textbf{Пример}. Рассмотрим снова задачу, вида
\begin{equation*}
    (\partial_x^2 + \kappa^2) G(x, y) = \delta(x-y)
\end{equation*}
и решим методом Фурье. Получим систему, вида
\begin{equation*}
    \left\{\begin{aligned}
        \partial_x^2 e_n = \lambda_n e_n \\
        e_n (-\pi) = e_n (\pi) \\ 
        e_n'(-\pi) = e_n'(\pi) \\
    \end{aligned}\right.
    \hspace{0.5cm} \Rightarrow \hspace{0.5cm}
    e = \alpha e^{i q x} + \beta e^{- i q x}, 
    \hspace{5 mm} 
    \alpha e^{i q \pi} + \beta e^{- i q \pi} = \alpha e^{- i q \pi} + \beta e^{i q \pi}.
    \hspace{0.25cm} \Rightarrow \hspace{0.25cm}
    \alpha \sin \pi q = \beta \sin \pi q,
\end{equation*}
а значит $q = n$, $n \in \mathbb{Z}$, вот и дискретность:
\begin{equation*}
    \lambda^2 = - n^2,
    \hspace{0.5cm} \Rightarrow \hspace{0.5cm}
    e_n = e^{i n x}, \hspace{5 mm} 
    e_{-n} = e^{- i nx}.
\end{equation*} 


\Tsec[]{Функция Грина для уравнения Пуассона в \texorpdfstring{$\mathbb{R}^3$}{R3}}

Рассмотрим в $\mathbb{R}^3$ уравнение $\nabla^2 f = \varphi$, где все линейно, всё хорошо. Как обычно будем искать функцию, виде
\begin{equation*}
    f(\vc{r}) = \int_{\mathbb{R}^3} G(\vc{r}  - \vc{r}') \varphi(\vc{r}) \d^3 r. 
\end{equation*}
Функцию Грина найдём в виде
\begin{equation*}
    \nabla^2 G(\vc{r}) = \delta(x) \delta(y) \delta(z).
\end{equation*}
Можем свести уравнение Лапласа, к уравнению Дебая c $\kappa = 0$:
\begin{equation*}
    (\nabla^2 - \kappa^2) G(\vc{r}) = \delta(\vc{r}),
\end{equation*} 
которое очень удобно раскладывать по Фурье:
\begin{equation*}
    \tilde{G}(\vc{k}) = \int_{\mathbb{R}^3} G(\vc{r}) e^{- i \smallvc{k} \cdot \smallvc{r}} \d \vc{r}, 
    \hspace{10 mm} 
    G(\vc{r}) = \int_{\mathbb{R}^3} \tilde{G}(\vc{k}) e^{i \smallvc{k} \cdot \smallvc{r}} \frac{\d \vc{k}}{(2 \pi)^3}.
\end{equation*}
Подставляя $G(\vc{r})$ через Фурье образ $\tilde{G}(\vc{k})$:
\begin{equation*}
    (-k^2 - \kappa^2) G(\vc{k}) = 1,
    \hspace{0.5cm} \Rightarrow \hspace{0.5cm}
    G(\vc{k})=- \frac{1}{k^2 + \kappa^2},
    \hspace{0.5cm} \Rightarrow \hspace{0.5cm}
    G(\vc{r}) = \int_{\mathbb{R}^3} \frac{e^{i \smallvc{k} \cdot \smallvc{r}}}{k^2 + \kappa^2} \frac{\d \vc{k}}{(2 \pi)^3}.
\end{equation*}
Переходим в сферические координаты, получаем, что
\begin{equation*}
    G(\vc{r}) = - \frac{2 \pi}{(2 \pi)^3} \int_{0}^{\infty} \frac{k^2}{k^2 + \kappa^2} \int_{0}^{\pi} 
    \sin \theta  e^{i k r \cos \theta}  \d \theta \d k = 
    - \frac{e^{- \kappa r}}{4 \pi r}
    .
\end{equation*}
Устремляя $\kappa \to 0$, находим
\begin{equation*}
     \nabla^2 G = \delta(\vc{r}),
     \hspace{0.5cm} \Rightarrow \hspace{0.5cm}
     G = - \frac{1}{4 \pi r}.
 \end{equation*} 


\Tsec[]{Произведение Гамма-функций}

Докажем выражение
\begin{equation*}
    \Gamma(z) \Gamma(1-z) = \frac{\pi}{\sin \pi z}.
\end{equation*}
Действительно,
\begin{equation*}
    B(z,\, 1-z) = \int_{0}^{1}  t^{z-1} (1-t)^{-z} \d t = \int_0^1 \left(\frac{t}{1-t}\right)^z \frac{\d t}{t}.
\end{equation*}
Тут логично ввести $x = \frac{t}{1-t} = -1 + \frac{1}{1-t}$, а значит
\begin{equation*}
    t = \frac{x}{x+1}, \hspace{5 mm} 
    \d t = \frac{\d x}{(x+1)^2}.
\end{equation*}
Продолжая жонглировать переменными
\begin{equation*}
    B(z,\, 1-z) = \int_{0}^{\infty} x^z \frac{x+1}{x} \frac{\d x}{(x + 1)^2} = 
    \int_{0}^{\infty} \frac{x^{z-1}}{x+1} \d x.
\end{equation*}
Который снова удобно посчитать через разрезы. 
\begin{equation*}
    B(z,\, 1-z) = \int_{\text{up}} \frac{x^{z-1}}{1+x} \d x =
    \frac{1}{1-e^{2 \pi i z}} \int_C \frac{x^{z-1}}{1+x} \d x,
\end{equation*}
но тут уже можно замкнуть дугу на бесконечности, вклад от котрой нулевой.  Осталось найти вычет в точке $-1$, тогда
\begin{equation*}
    \int_{\text{up}} \frac{x^{z-1}}{1+x} \d x = \frac{1}{1-e^{2\pi i z}}    \res_{-1} = \frac{2 \pi i (-1) e^{\pi i z}}{1 - e^{2 \pi i z}} = \frac{2 \pi i}{e^{\pi i z} - e^{-i \pi z}} = \frac{\pi}{\sin \pi z}.
\end{equation*}
