\Tsec[]{Задача Штурма-Лиувилля}

\textbf{Алгоритм (Фурье)}. Раскладываем 
\begin{equation*}
    G(x) = \sum_{n \neq 0} g_n e_n,
    \hspace{10 mm} 
    \delta(x) = \sum \frac{e_n (x)}{2 \pi},
    \hspace{0.5cm} \Rightarrow \hspace{0.5cm}
    \hat{L} G = \delta(x) - \frac{1}{2\pi}.
\end{equation*}
Знаем, что $\lambda_n g_n = \frac{1}{2\pi}$, а значит
\begin{equation*}
    g_n(x) = \frac{1}{2\pi} \frac{1}{- n^2},
    \hspace{0.5cm} \Rightarrow \hspace{0.5cm}   
    G(x) = \sum_{n \neq 0} \frac{1}{2\pi} \frac{1}{-n^2} e^{i n x},
\end{equation*}
и рассмотрим $0 < x < \pi$, суммирая это через вычеты, записываем
\begin{equation*}
    f(z) = \frac{e^{zx}}{2 \pi z^2}, 
    \hspace{0.5cm} \Rightarrow \hspace{0.5cm}   
    G(x) = \sum \oint_{in} \frac{\d z}{2 \pi i} f(z) g(z).
\end{equation*}
Соответственно, выберем
\begin{align*}
    g(z) = \frac{\pi e^{- \pi z}}{\sh (\pi z)}
\end{align*}
тогда
\begin{equation*}
    f(z) g(z) = \frac{\pi}{z^2} \frac{e^{(x-\pi)z}}{\sh \pi z},
\end{equation*}
получаем, что интеграл по душам вправо/влево  равен $0$, и остается только вычет в $z = 0$:
\begin{equation*}
    G(z) = - \res_0 f(z) g(z) = \ldots = - \frac{x^2}{4 \pi} + \frac{x}{2} - \frac{\pi}{6}.
\end{equation*}



\textbf{Алгоритм (сшивка)}. Решим задачу
\begin{equation*}
    \partial_x^2 G(x) = \delta(x) - \frac{1}{2\pi}.
\end{equation*}
Разбиваем $x < 0$ и $x > 0$:
\begin{align*}
    &x < 0, 
    & G = -\tfrac{x^2}{4 \pi} + a x + b, \\
    &x > 0, 
    & G = -\tfrac{x^2}{4 \pi} + c x + \varpi, 
\end{align*}
учитываем граничные условия:
\begin{equation*}
    G(-0) = G(+0),
    \hspace{5 mm} 
    G'(+0) - G'(-0) = 1,
    \hspace{0.5cm} \Rightarrow \hspace{0.5cm}   
    b = \varpi.
\end{equation*}
Также получаем, что $-a = b$.

Учтём, что $e_0$ не входит в $G$:
\begin{equation*}
    \langle G | e_0\rangle = 0 = \int_{-\pi}^{+\pi}G(x) \d x = 0,
    \hspace{0.5cm} \Rightarrow \hspace{0.5cm}
    b = - \frac{\pi}{6},
\end{equation*}
так и получаем все необходиме условия на $G(x, y)$. 




\textbf{Пример}. Рассмотрим снова задачу, вида
\begin{equation*}
    (\partial_x^2 + \kappa^2) G(x, y) = \delta(x-y)
\end{equation*}
и решим методом Фурье. Получим систему, вида
\begin{equation*}
    \left\{\begin{aligned}
        \partial_x^2 e_n = \lambda_n e_n \\
        e_n (-\pi) = e_n (\pi) \\ 
        e_n'(-\pi) = e_n'(\pi) \\
    \end{aligned}\right.
    \hspace{0.5cm} \Rightarrow \hspace{0.5cm}
    e = \alpha e^{i q x} + \beta e^{- i q x}, 
    \hspace{5 mm} 
    \alpha e^{i q \pi} + \beta e^{- i q \pi} = \alpha e^{- i q \pi} + \beta e^{i q \pi}.
    \hspace{0.25cm} \Rightarrow \hspace{0.25cm}
    \alpha \sin \pi q = \beta \sin \pi q,
\end{equation*}
а значит $q = n$, $n \in \mathbb{Z}$, вот и дискретность:
\begin{equation*}
    \lambda^2 = - n^2,
    \hspace{0.5cm} \Rightarrow \hspace{0.5cm}
    e_n = e^{i n x}, \hspace{5 mm} 
    e_{-n} = e^{- i nx}.
\end{equation*} 
