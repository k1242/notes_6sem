\begin{enumerate*}

% \section*{Билеты к ГКЭ по Математическому Анализу}


\subsection*{Введение в математический анализ}

\item Фундаментальные последовательности и полнота действительных чисел.
\item Теорема о промежуточных значениях непрерывной функции.
\item Свойства функций, непрерывных на компактных подмножествах прямой.
\item Теоремы о среднем Лагранжа и Коши для дифференцируемых функций.
\item Формула Тейлора с остаточным членом в форме Пеано.
\item Необходимые и достаточные условия экстремума и достаточное условие выпуклости в терминах первых и вторых производных функции одной переменной.
\item Равномерная непрерывность отображений метрических пространств, равномерная непрерывность непрерывного на метрическом компакте отображения.
\item Независимость частных производных функции нескольких переменных от порядка дифференцирования.


\subsection*{Многомерный анализ, интегралы и ряды}

\item Равномерная сходимость функциональных последовательностей и рядов. Непрерывность и дифференцируемость суммы функционального ряда.
\item  Радиус сходимости и равномерная сходимость степенного ряда, дифференцирование степенного ряда. Ряд Тейлора функции.
\item  Теорема об ограниченной сходимости для интеграла Лебега.



\subsection*{Анализ на многообразиях}

\item  Дифференциальные формы на открытых подмножествах евклидова пространства, оператор внешнего дифференцирования d и его независимость от криволинейной замены координат.
\item Интегрирование дифференциальной формы с компактным носителем в евклидовом пространстве. Зависимость интеграла от замены координат.
\item Вложенные многообразия в евклидовом пространстве, их координатные карты и ориентация.
\item Разбиение единицы в окрестности компактного подмножества многообразия и определение интеграла дифференциальной формы с компактным носителем по ориентированному многообразию.
\item Формула Стокса для ориентированного многообразия с краем.




\subsection*{Фунциональный анализ}

\item Неравенство Бесселя и равенство Парсеваля для тригонометрической системы на отрезке.
\item Достаточные условия равномерной сходимости тригонометрического ряда Фурье.
\item Теорема Вейерштрасса о приближении непрерывных на отрезке функций многочленами.
\item Регулярные и нерегулярные распределения (обобщённые функции).


\newpage


\subsection*{Аналитическая геометрия}

\item Прямые и плоскости в пространстве. Формулы расстояния от точки до прямой и плоскости, между прямыми в пространстве. Углы между прямыми и плоскостями.
\item Кривые второго порядка, их геометрические свойства.
\item Общее решение системы линейных алгебраических уравнений. Теорема Кронекера Капелли.



\subsection*{Линейная алгебра}

\item Линейное пространство, базис и размерность. Линейное отображение конечномерных пространств, его матрица. Ядро и образ линейного отображения.
\item Собственные значения и собственные векторы линейных преобразований. Диагонализируемость линейных преобразований.
\item Самосопряженные преобразования евклидовых пространств, свойства их собственных значений и собственных векторов.
\item Приведение квадратичных форм в линейном пространстве к каноническому виду. Положительно определенные квадратичные формы. Критерий Сильвестра.
\item Группы, порядок группы. Гомоморфизм и изоморфизм групп. Прямое произведение групп. Циклические группы.



\subsection*{Дифференциальные уравнения}

\item Линейные обыкновенные дифференциальные уравнения с постоянными коэффициентами иправойчастью-квазимногочленом.
\item Системы линейных однородных дифференциальных уравнений с постоянными коэффициентами, методы их решения.
\item Линейные обыкновенные дифференциальные уравнения с переменными коэффициентами. Фундаментальная система решений. Определитель Вронского. Формула Лиувилля Остроградского.
\item Простейшая задача вариационного исчисления. Необходимые условия локального экстремума.



\subsection*{Комплексный анализ}

\item Дифференцируемость функций комплексного переменного. Условия Коши-Римана. Интегральная теорема Коши.
\item Интегральная формула Коши. Разложение функции, регулярной в окрестности точки, в ряд Тейлора.
\item Разложение функции, регулярной в кольце, в ряд Лорана. Изолированные особые точки однозначного характера.
\item Вычеты. Вычисление интегралов по замкнутому контуру при помощи вычетов.
\item Целые функции и теорема Лиувилля.
\item Мероморфные функции и теорема Миттаг-Леффлера.



\end{enumerate*}