Хочется научиться переходить от уравнений вида
\begin{equation*}
    \partial_t \vc{v} = M(t) \vc{v},
\end{equation*}
к уравнениям вида
\begin{equation*}
    \partial_t \vc{v} = \tilde{M} \vc{v}.
\end{equation*}
А именно систему
\begin{equation*}
    \begin{pmatrix}
        \dot{x} \\ \dot{y} \\ \dot{z}
    \end{pmatrix} = \begin{pmatrix}
        0 & 1 & 2 \cos(\omega t) \\
        0 & -1 & -2 \cos(\omega t) \\
        -\cos(\omega t) & \cos(\omega t) & -1/2 \\
    \end{pmatrix} \begin{pmatrix}
        x \\ y \\ z
    \end{pmatrix},
\end{equation*}
хочется представить в виде
\begin{equation*}
    \begin{pmatrix}
        \dot{x} \\ \dot{y} \\ \dot{z}
    \end{pmatrix} = \begin{pmatrix}
        0 & 1 & 2 \sigma(\omega) \\
        0 & -1 & -2 \sigma(\omega) \\
        \ldots & \ldots & \ldots \\
    \end{pmatrix} \begin{pmatrix}
        x \\ y \\ z
    \end{pmatrix}.
\end{equation*}
Нас интересует только предельные значения $\vc{v}$ при $t \to \infty$.
По идее $\sigma(\omega)$ должна получиться в виде резонансного контура Лоренца.

\begin{equation*}
    \sigma(\omega) = \frac{1}{1 + \omega^2}
\end{equation*}