\textbf{Двухуровневая система}.  В качестве тестового примера рассмотрим двухуровневую систему:
\begin{equation*}
    \gset = \{1\}, \hspace{5 mm} \eset = \{2\}.
\end{equation*}





\textbf{Трёхуровневая система}. В качестве наглядного примера рассмотрим $V$-систему ($\gset = \{1\}$, $\eset = \{2,\,  3\}$) и $\Lambda$-систему($\gset = \{1, 2\}$, $\eset = \{3\}$)
\begin{equation*}
    {\mathcal L}^V_{\beta \alpha} (\hat{\rho}) = \frac{1}{\tau}\begin{pmatrix}
        C_{21}^2 \rho_{22} + C_{31}^2 \rho_{33} & \ldots & \ldots \\[4pt]
        - \tfrac{1}{2} \rho_{21} & - \rho_{22} & \ldots \\[4pt]
        -\tfrac{1}{2} \rho_{31} & \ldots & -\rho_{33} \\
    \end{pmatrix}, \hspace{10 mm} 
    {\mathcal L}^\Lambda_{\beta \alpha} (\hat{\rho}) = \frac{1}{\tau}
    \begin{pmatrix}
        C_{31}^2 \rho_{33} & \ldots & \ldots \\[4pt]
        \ldots & C_{32}^2 \rho_{33} & \ldots \\[4pt]
        - \tfrac{1}{2} \rho_{31} & -\tfrac{1}{2} \rho_{32} & - \rho_{33} \\
    \end{pmatrix},
\end{equation*}
где через $\ldots$ обозначены компоненты $\rho$, эволюция которых нам не интересна.