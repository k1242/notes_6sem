

\textbf{Спонтанное излучение}. К уравнению \eqref{master_eq}  мы добавляем релаксационное слагаемое для феноменологического описания \cite{main_2006,allen_1975} спонтанного излучения:
\begin{equation*}
    i \hbar \partial_t \hat{\rho} = [\hat{H}, \hat{\rho}] + i \hbar {\mathcal L}^\text{relax}_{\beta \alpha} (\hat{\rho}) .
\end{equation*}
Разделим уровни в системе на $g$-уровни $\{\text{g}\}$  и $e$-уровни  $\{\text{e}\}$. Тогда $\sub{\mathcal L}{relax} (\hat{\rho})$ перепишется в виде
\begin{equation*}
    {\mathcal L}^\text{relax}_{\beta \alpha} (\hat{\rho}) = \left\{\begin{aligned}
        - &\tfrac{1}{2\tau} \rho_{\beta \alpha}, &\beta \in \{\text{e}\}, \alpha \in \{\text{g}\}, \\
        - &\tfrac{1}{\tau} \rho_{\beta \beta}, & \beta = \alpha \in \{\text{e}\}, \\
        + &\tfrac{1}{\tau} \textstyle \sum\limits_{\gamma \in \{\text{e}\}} (C_{\gamma \alpha})^2 \rho_{\gamma \gamma},
        & \beta = \alpha \in \{\text{g}\}.
    \end{aligned}\right.
\end{equation*}

