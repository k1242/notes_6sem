\section*{Лекция №2}

\textbf{Классическая электродинамика}.
Знаем, что
\begin{equation*}
    m \dot{\vc{v}} = e \vc{E} + \frac{e}{c} [\vc{v} \times  \vc{H}],
    \hspace{5 mm} 
    \vc{E} = \vc{E}_0 e^{-i \omega t},
\end{equation*}
тогда в нулевом приближении
\begin{equation*}
    H = 0,
    \hspace{5 mm} 
    \vc{p} = e \vc{r},
    \hspace{0.5cm} \Rightarrow \hspace{0.5cm}  
    \varepsilon_{(0)} = 1 - f4 \pi \frac{N e^2}{m \omega^2}.
\end{equation*}
Стоит уточнить, что под $\vc{H}$ понимается постоянное внешнее магнитное  поле, а $\vc{E}$ -- поле волны.


