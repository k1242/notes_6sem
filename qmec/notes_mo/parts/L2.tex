\section*{Лекция №2}

\textbf{Классическая электродинамика}.
Знаем, что
\begin{equation*}
    m \dot{\vc{v}} = e \vc{E} + \frac{e}{c} [\vc{v} \times  \vc{H}],
    \hspace{5 mm} 
    \vc{E} = \vc{E}_0 e^{-i \omega t},
\end{equation*}
тогда в нулевом приближении
\begin{equation*}
    H = 0, \hspace{5 mm} m \omega \vc{v}_0 = e \vc{E},
    \hspace{5 mm} 
    \vc{p} = e \vc{r},
    \hspace{0.5cm} \Rightarrow \hspace{0.5cm}  
    \varepsilon_{(0)} = 1 - 4 \pi \frac{N e^2}{m \omega^2}.
\end{equation*}
Стоит уточнить, что под $\vc{H}$ понимается постоянное внешнее магнитное  поле, а $\vc{E}$ -- поле волны.

В первом приближении
\begin{equation*}
    m \vc{v}_1 = \frac{e}{c}\left[\vc{v}_0 \times \vc{H}\right],
    \hspace{0.5cm} \Rightarrow \hspace{0.5cm}
    m \vc{v}_1 = \frac{e^2}{c} \frac{1}{m \omega} \left[\vc{E} \times  \vc{H}\right].
\end{equation*}
Интегрруя, находим
\begin{equation*}
    \vc{r}_1 = \frac{-e^2 i}{c m^2 \omega^3} \left[\vc{E} \times  \vc{H}\right],
    \hspace{0.5cm} \Rightarrow \hspace{0.5cm}
    \vc{p}_1 = - \frac{N e^3 i}{m^2 c \omega^3} \left[\vc{E} \times  \vc{H}\right].
\end{equation*}
Собирая всё вместе, находим
\begin{equation*}
    \vc{D} = \left(1 - \frac{4 \pi e^2 N}{m \omega^2}\right) \vc{E} - \frac{4 \pi N e^3 i}{m2 c \omega^3} \left[\vc{E} \times  \vc{H}\right].
\end{equation*}

Таким образом пришли к тензору на $\hat{\varepsilon}$:
\begin{equation*}
    \hat{\varepsilon} = \begin{pmatrix}
        \varepsilon_{(0)}  & i g_z & -i g_y \\
        -i g_z & \varepsilon_0 & i g_x \\
        i g_y & - i g_x & \omega_{(0)} \\
    \end{pmatrix},
\end{equation*}
иначе можем записать в виде
\begin{equation*}
    \varepsilon_{ij} = \varepsilon^{(0)} \delta_{ij} -i \epsilon_{ijk} g_k,
\end{equation*}
где $\vc{g}$ -- вектор гирации:
\begin{equation*}
    \vc{g} = f \vc{H},
    \hspace{5 mm} 
    f = \frac{4 \pi N e^3}{m^2 c \omega^3},
\end{equation*}
где $f$ указан для свободного электрона. 
В изотропном магнитном материале $\vc{M} = \chi \vc{H}$.
В общем виде $\vc{g} = \hat{f} \vc{M}$:
\begin{equation*}
    \varepsilon_{ij} = \varepsilon_{ij}^{(0)} - i \epsilon_{ijk} f_{kl} M_l.
\end{equation*}

\textbf{Дисперсия}. Вообще есть дисперсия, $\vc{D} = \vc{D}(\omega)$, работает причинность, есть некоторый нелокальный отклик, а тогда
$\varepsilon = \varepsilon(\omega, \vc{k})$. 

Можем разложить
\begin{equation*}
    \varepsilon_{ij} = 
    \varepsilon_{ij}^{(0)} + 
    \underbrace{\chi_{ijk}^{(2)} E_k}_{\text{Поккельс}} +
    \underbrace{\chi^{(3)}_{ijkl} E_k E_l}_{\text{Керр}} + 
    \underbrace{\chi^{(2m)}_{ijk} B_k}_{\text{Фарадей}} + 
    \underbrace{\chi_{ijkl}^{(3m)} B_k B_l}_{\text{Каттон-Мутон}} + \ldots
\end{equation*}








\textbf{Эффект Фарадея}. Записываем уравнения Максвелла. Получаем уравнение\footnote{
    Напомним, что $\mu \approx 1$, а значит $\vc{B} = \vc{H}$.
}
\begin{equation*}
\rot \rot \vc{E} = \frac{\omega}{c} \rot \bar{B},
\hspace{0.5cm} \Rightarrow \hspace{0.5cm}
    \grad \div \vc{E} - \Delta E = \frac{\omega^2}{c} \hat{\varepsilon} \vc{E}.
\end{equation*}
Переходя к плоским монохроматическим волнам вида $e^{- i \omega t + i \smallvc{k} \smallvc{r}}$, находим
\begin{equation*}
    n^2 \vc{E} - \vc{n} (\vc{n} \cdot \vc{E})  = \hat{\varepsilon} \vc{E},
    \hspace{5 mm} 
    \vc{n} = \frac{c}{\omega} \vc{k},
\end{equation*}
которое ещё называют \textit{уравнением Френеля}.


Упростим себе жизнь $\vc{k} \parallel \vc{M} \parallel Oz$:
\begin{equation*}
    \vc{n} = \begin{pmatrix}
        0 \\ 0 \\ n
    \end{pmatrix},
    \hspace{5 mm} 
    \hat{\varepsilon} = \begin{pmatrix}
        \varepsilon_{0} & -ig & 0 \\
        -ig & \varepsilon_0 & 0 \\
        0 & 0 & 0 \\
    \end{pmatrix}.
\end{equation*}
Так приходим к уравнениям
\begin{align*}
    n^2 E_x &= \varepsilon_0 E_x + i g E_y, \\
    n^2 E_y &= \varepsilon_0 E_y - i g E_x, \\
    n^2 E_z &= \varepsilon_0 E_z + n^2 E_z.
\end{align*}
Приравнивая определитель к нулю, находим
\begin{equation*}
    \begin{Vmatrix}
        n^2 - \varepsilon_0 & -i g  \\
        i g & n^2 - \varepsilon_0  \\
    \end{Vmatrix} = 0,
    \hspace{0.5cm} \Rightarrow \hspace{0.5cm}
    (n^2 - \varepsilon_0)^2 - g^2 = 0,
    \hspace{0.5cm} \Rightarrow \hspace{0.5cm}
    n^2 = \varepsilon_0 \pm g.
\end{equation*}
Так приходим к показателю преломления
\begin{equation*}
    n_{\pm} = \pm \sqrt{\varepsilon_0 \pm g}.
\end{equation*}

Подставляя $n$, находим волну
\begin{equation*}
    \vc{E} = \begin{pmatrix}
        1 \\ \mp i \\ 0
    \end{pmatrix} E_0 \exp{\left(- i \omega t + i k_0 z \sqrt{\varepsilon_0 + g} z\right)}.
\end{equation*}
то есть мы получили круговую поляризацию, с разной фазовой скоротью для правой и левой поляризации.

Угол Фарадея можем найти, рассмотрев плоскую поляризацию, как сумму двух круговых:
\begin{equation*}
    \vc{E} = E_0 \begin{pmatrix}
        \cos(k_0 \Delta n z) \\ \sin\left(k_0 \Delta n z\right)
    \end{pmatrix} e^{-i \omega t + i k_0 n_0 z},
    \hspace{10 mm} 
    \Delta n = \frac{n_+ - n_-}{2},
    \hspace{5 mm} 
    n_0 = \frac{n_+ + n_-}{2}.
\end{equation*}

Обычно $\varepsilon_0 \gg g$, а значит
\begin{equation*}
    \Delta n = \frac{g}{2 \sqrt{\varepsilon_0}},
    \hspace{0.5cm} \Rightarrow \hspace{0.5cm}
    \theta = k_0 \Delta n z = \frac{k_0 g}{2 n_0} z.
\end{equation*}