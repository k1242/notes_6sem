\section*{Лекция №2}

\textbf{Классическая электродинамика}.
Знаем, что
\begin{equation*}
    m \dot{\vc{v}} = e \vc{E} + \frac{e}{c} [\vc{v} \times  \vc{H}],
    \hspace{5 mm} 
    \vc{E} = \vc{E}_0 e^{-i \omega t},
\end{equation*}
тогда в нулевом приближении
\begin{equation*}
    H = 0, \hspace{5 mm} m \omega \vc{v}_0 = e \vc{E},
    \hspace{5 mm} 
    \vc{p} = e \vc{r},
    \hspace{0.5cm} \Rightarrow \hspace{0.5cm}  
    \varepsilon_{(0)} = 1 - 4 \pi \frac{N e^2}{m \omega^2}.
\end{equation*}
Стоит уточнить, что под $\vc{H}$ понимается постоянное внешнее магнитное  поле, а $\vc{E}$ -- поле волны.

В первом приближении
\begin{equation*}
    m \vc{v}_1 = \frac{e}{c}\left[\vc{v}_0 \times \vc{H}\right],
    \hspace{0.5cm} \Rightarrow \hspace{0.5cm}
    m \vc{v}_1 = \frac{e^2}{c} \frac{1}{m \omega} \left[\vc{E} \times  \vc{H}\right].
\end{equation*}
Интегрруя, находим
\begin{equation*}
    \vc{r}_1 = \frac{-e^2 i}{c m^2 \omega^3} \left[\vc{E} \times  \vc{H}\right],
    \hspace{0.5cm} \Rightarrow \hspace{0.5cm}
    \vc{p}_1 = - \frac{N e^3 i}{m^2 c \omega^3} \left[\vc{E} \times  \vc{H}\right].
\end{equation*}
Собирая всё вместе, находим
\begin{equation*}
    \vc{D} = \left(1 - \frac{4 \pi e^2 N}{m \omega^2}\right) \vc{E} - \frac{4 \pi N e^3 i}{m2 c \omega^3} \left[\vc{E} \times  \vc{H}\right].
\end{equation*}

Таким образом пришли к тензору на $\hat{\varepsilon}$:
\begin{equation*}
    \hat{\varepsilon} = \begin{pmatrix}
        \varepsilon_{(0)}  & i g_z & -i g_y \\
        -i g_z & \varepsilon_0 & i g_x \\
        i g_y & - i g_x & \omega_{(0)} \\
    \end{pmatrix},
\end{equation*}
иначе можем записать в виде
\begin{equation*}
    \varepsilon_{ij} = \varepsilon^{(0)} \delta_{ij} -i \epsilon_{ijk} g_k,
\end{equation*}
где $\vc{g}$ -- вектор гирации:
\begin{equation*}
    \vc{g} = f \vc{H},
    \hspace{5 mm} 
    f = \frac{4 \pi N e^3}{m^2 c \omega^3},
\end{equation*}
где $f$ указан для свободного электрона. 
В изотропном магнитном материале $\vc{M} = \chi \vc{H}$.
В общем виде $\vc{g} = \hat{f} \vc{M}$:
\begin{equation*}
    \varepsilon_{ij} = \varepsilon_{ij}^{(0)} - i \epsilon_{ijk} f_{kl} M_l.
\end{equation*}

\textbf{Дисперсия}. Вообще есть дисперсия, $\vc{D} = \vc{D}(\omega)$, работает причинность, есть некоторый нелокальный отклик, а тогда
$\varepsilon = \varepsilon(\omega, \vc{k})$. 

Можем разложить
\begin{equation*}
    \varepsilon_{ij} = \varepsilon_{ij}^{(0)} + \chi_{ijk}^{(2)} E_k + \chi^{(3)}_{ijkl} E_k E_l + \chi^{(2m)}_{ijk} B_k + \chi_{ijkl}^{(3m)} B_k B_l + \ldots
\end{equation*}
где $\chi_{ijk}^{(2)} E_k$ -- эффект Поккельса, $\chi^{(3)}_{ijkl} E_k E_l$ -- эффект Керра.




\noindent
\textit{ДЗ} (Эффект Каттона-Мутона). Вывести квадратичную поправку к $\hat{\varepsilon}$: $v = v_0 + v_1 + v_2$. Дедлайн -- неделя.

\textbf{Эффект Фарадея}. 