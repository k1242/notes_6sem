\section*{Лекция №3}

% \textbf{Введение в акустику}. Введем охапку величин
% \begin{equation*}
%     F = k \Delta x,
%     \hspace{5 mm} 
%     \ldots
% \end{equation*}
% Рассмотрим одномерную задачу, вдоль которой приложим силу, тогда каждая точка сместится:
% \begin{equation*}
%     \lim_{dx \to 0} u(x) - u(x+dx) = \frac{d u}{d x} = S.
% \end{equation*}


Из уравнений Максвелла можем получть волновое уравнение, перейти к монохроматическим волнам, получить уравнение Гельмгольца, рассмотреть плоские волны, а дальше потребуем наличия нетривиального решения:
\begin{equation*}
    \vc{k} \cdot [\vc{k} \times  \vc{E}] - \frac{\omega^2}{c^2} \hat{\varepsilon} \vc{E} = 0,
    \hspace{0.5cm} \Rightarrow \hspace{0.5cm}
    \det\left(
        k_i k_j + \ldots
    \right) = 0,
\end{equation*}
таким образом приходим к двулучепреломлению.

Тут полезно ввести \textit{оптическую индикатрису}:
\begin{equation*}
    \hat{\varepsilon} = \diag{\varepsilon_x,\, \varepsilon_y,\, \varepsilon_z},
    \hspace{10 mm} 
    \frac{x^2}{\varepsilon_x} + \frac{y^2}{\varepsilon_y} + \frac{z^2}{\varepsilon_z} = \frac{\omega^2}{c^2}.
\end{equation*}
Ещё удобнее ввести $\hat{B} = \hat{\varepsilon}^{-1}$, который называют тензором диэлектричекой непроницаемости.


Если падает волна с $\vc{k}_i$, то строя поскость $\bot \vc{k}_i$ переходим к эллепсоиду вращения.




