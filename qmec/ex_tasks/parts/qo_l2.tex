\section*{Лекция №2 (вторичное квантование)}



Для одиночного фотона можно построить волновые функции\footnote{
    А.И. Ахиезер, В. Б. Берестецкий, <<Квантовая электродинамика>> -- обязательно прочитать (\S 1, \S 2)!
}, -- это ЭМ поле, что является явным проявлением родства уравнения Шрёдингера и волнового уравнения оптики.


Вторичное квантование (П. Дирак, 1927) -- введение нового объекта описания: осциллятора ЭМ поля или совокупность одинаковых фотоов в объеме когерентности. Ключевыми становятся числа заполнения. 


\textbf{Осциллятор}. Пришли к новым операторам
\begin{equation*}
    \hat{a} = \frac{\omega \hat{x} + i \hat{p}}{\sqrt{i \hbar \omega}},
    \hspace{5 mm} 
    \hat{a}^\dagger = \frac{\omega \hat{x} - i \hat{p}}{\sqrt{2 \hbar \omega}}.
\end{equation*}
Оператор поля:
\begin{equation*}
    \hat{E} = \sqrt{\frac{8 \pi \hbar \omega}{V}} \frac{1}{2}\left(
        \hat{a} e^{- i \omega t + i \smallvc{k} \cdot \vc{r}} + 
        \hat{a}^\dagger e^{i \omega t - i \smallvc{k} \cdot \smallvc{r}}
    \right),
\end{equation*}
числа фотонов и Гамильтониан:
\begin{equation*}
    \hat{n} = \hat{a}^\dagger \hat{a}\D,
    \hspace{5 mm} 
    \hat{H} = \frac{\hbar \omega}{2} \left(\hat{a}^\dagger \hat{a} + \hat{a} \hat{a}^\dagger\right)= \hbar \omega \left(\hat{a}\D \hat{a} + \frac{1}{2}\right),
\end{equation*}
где $V$ -- объем когерентности и всё также верно, что $[\hat{a}, \hat{a}\D] = 1$.

Добавка $\frac{1}{2}$ -- непередаваемая часть гамильтониана, так что $\hat{n} = \hat{a}\D \hat{a}$ более чем логично.


Как обычно живём в базисе $\hat{H}$: $\ket{n}$, которые допускают интерпретацию в виде $n$-фотонных состояниях ЭМ поля, Фоковских сотояний.

Свободная эволюция ЭМ поля:
\begin{equation*}
    i \hbar \partial_t \ket{n} = \hat{H} \ket{n},
    \hspace{0.5cm} \Rightarrow \hspace{0.5cm}
    \ket{n(t)} = \exp\left(
        - \frac{it}{\hbar} \hat{H}
    \right) \ket{n(0)} = e^{- i n \omega t} \left(
        e^{- i \omega t/2} \ket{n(0)}
    \right).
\end{equation*}

\textbf{Лестничные операторы}. Как и с осциллятором, приходим
\begin{equation*}
    \hat{a} \ket{n} = \sqrt{n} \ket{n-1},
    \hspace{10 mm} 
    \hat{a}\D = \sqrt{n+1} \ket{n+1},
\end{equation*}
то есть имеем оператор рождения и уничтожения фотона.