\Tsec{№10. Аномальный эффект Зеемана}

Расщепление будет происходить на величину
\begin{equation*}
	E = \hbar \Omega g_{LSJ} M_j,
	\hspace{10 mm} 
	g = \frac{3}{2} + \frac{S(S+1) - L(L+1)}{2 J(J+1)},
	\hspace{5 mm} 
	\Omega = \frac{e H}{2 m c}.
\end{equation*}
Вспомним правила отбора $|\Delta J| \leq 1$, $|\Delta M_s | \leq 1$, $\Delta S = 0$, а также про смену четности Е1 перехода и запрет на $J=J'=0$. 

Для уровния $\term{2}{P}{3/2}$ фактор Ланде равен $4/3$. Осталось отдельно рассмотреть переходы с $\term{2}{S}{1/2}$ на $\term{2}{P}{1/2}$ и c $\term{2}{S}{1/2}$ на $\term{2}{P}{3/2}$, что проще проделать руками, чем приводить здесь.