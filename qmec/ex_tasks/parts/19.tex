\Tsec{№19. Рассеяние в борновском приближении}

Рассмотрим в борновском приближении два короткодейтсвующих потенциала. Амплитуда рассения может быть найдена в виде
\begin{equation*}
	f(\theta) = - \frac{m}{2 \pi \hbar^2} \int V(r) e^{-i \smallvc{q} \smallvc{r}} \d^3 \vc{r} = - \frac{2m}{\hbar^2 q} \int_{0}^{\infty} V(r) \sin(qr) r \d r,
	\hspace{10 mm} 
	\vc{q} \overset{\mathrm{def}}{=} \vc{k}' - \vc{k},
	\hspace{5 mm} 
	q = 2 k \sin(\theta/2).
\end{equation*}
Полное сечение рассения находим интегрируя амплитуду рассеяния:
\begin{equation*}
	\sigma = \int |f(\theta)|^2 \d \Omega = 2 \pi \int_{0}^{\pi} |f(\theta)|^2 \sin \theta \d \theta.
\end{equation*}
Условие применимости запишется в виде
\begin{equation*}
	\frac{m}{2\pi \hbar^2} \bigg| \int \frac{e^{i k r}}{r} V(r) e^{i k z} \d^3 r \bigg| \ll 1.
\end{equation*}


\textbf{Потенциал Юкавы}. Подставляя $V(r) = \frac{\alpha}{r} e^{- \kappa r}$, находим
\begin{equation*}
	f = - \frac{2 m \alpha}{\hbar^2 (\kappa^2 + q^2)},
	\hspace{0.5cm} \Rightarrow \hspace{0.5cm}
	\sigma = \left(\frac{m \alpha}{\hbar^2 \kappa}\right)^2 \frac{4\pi}{4 k^2 + \kappa^2},
	\hspace{5 mm} 
	k^2 = \frac{2 mE}{\hbar^2}.
\end{equation*}
Условие применимости для любых энергий: $\alpha m / \kappa \ll \hbar^2$. Для быстрых частиц можем ослабить условие до $\alpha \ll \hbar \times \hbar k/m$.

\red{В пределе $\kappa \to 0$ получить резерфодовское сечение на отталкивающем кулоновском центре.}

% \textbf{Прямоугольная яма}. Аналогично вычисляем
% \begin{equation*}
% 	f = \frac{2m V_0 a}{\hbar^2 q^2} \left(\cos qa - \frac{\sin q a}{qa}\right),
% 	\hspace{0.25cm} \Rightarrow \hspace{0.25cm}
% 	\sigma = \frac{2\pi}{k^2}\left(
% 		\frac{m V_0 a^2}{\hbar}
% 	\right)^2 \left(
% 		1 - \frac{1}{(2ka)^2} + \frac{\sin 4 ka}{(2 ka)^3} - \frac{\sin^2 2ka}{(2ka)^4}
% 	\right),
% \end{equation*}
% с условием применимости $\sqrt{2 m V_0} a \ll \hbar$ и для быстрых частиц $\sqrt{2 m V_0} a \ll  \hbar \sqrt{ka}$.