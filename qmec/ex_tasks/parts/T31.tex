\subsection*{Т31}

Найдём сечение рассеяния для $k a \ll 1$, а значит доминирует $s$-рассеяние и $p$-рассеяние. 
Для потенциала
\begin{equation*}
	U(r) = \left\{\begin{aligned}
	    -&U_0, &r \leq a, \\
	    &0,  &r>a.
	\end{aligned}\right.
\end{equation*}
Теперь
\begin{equation*}
	R_{k0} = \frac{1}{r} u(r),
	\hspace{5 mm} 
	u(0) = 0,
	\hspace{10 mm} 
	- \frac{\hbar^2}{2m} u'' + V u = E u.
\end{equation*}
Для $r > a$ $u'' + k^2 u = 0$, тогда
\begin{equation*}
	\sub{u}{II} = A \sin(kr + \delta_0).
\end{equation*}
Для $r\leq a$ 
\begin{equation*}
	u'' + (k^2 + \kappa^2) u = 0,
	\hspace{5 mm} 
	\tilde{k}^2 \overset{\mathrm{def}}{=} k^2 + \kappa^2,
	\hspace{5 mm} 
	U_0 = \frac{\hbar^2 \kappa^2}{2m},
	\hspace{0.5cm} \Rightarrow \hspace{0.5cm}
	\sub{U}{I} = B \sin(\tilde{k} r).
\end{equation*}
Сшиваем на границах:
\begin{equation*}
	\frac{\sub{U}{I}'}{\sub{U}{I}} = \frac{\sub{U}{II}'}{\sub{U}{II}},
	\hspace{0.5cm} \Rightarrow \hspace{0.5cm}
	\tg(ka + \delta_0) = \frac{k}{\tilde{k}} \tg(\tilde{k} a),
	\hspace{0.5cm} \Rightarrow \hspace{0.5cm}
	\delta_0 = - ka + \arctg\left(
		\tfrac{k}{\tilde{k}} \tg(\tilde{k} a)
	\right).
\end{equation*}
Рассмотрим случай $\frac{k}{\tilde{k}} \tg(\tilde{k} a) \ll ka \ll 1$, а тогда $\delta_0 \approx - ka$, а значит $f_0 = \frac{1}{k} e^{i \delta_0} \sin \delta_0 \approx - a$.


Другой случай $\tg(\tilde{k} a) \to \infty$. Тогда $\delta_0 \approx \arctg(\infty) = \frac{\pi}{2}$, что ещё называют резонансным рассеянием, так как $\sin \delta_0 = 1$.

Наконец, посмотрим на $\frac{\tg(\tilde{k} a)}{\tilde{k} a} \approx	1$, тогда $\delta_0 \approx	0$, и получается $\tilde{k} a \ll 1$ и $f_0 \to 0$ -- эффект Рамзаура.

При барьере $\tilde{k} \to i \tilde{k}$, получим уравнения
\begin{equation*}
	\delta_0 = - ka + \arctg\left(
		\frac{ka}{\tilde{k} a} \th(\tilde{k} a)
	\right).
\end{equation*}


