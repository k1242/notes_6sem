\subsection*{Т26}



\textbf{Кремний}. По правилам Хунда конфигурация незаполненой части $\chain{2}{p}{2}$ будет вида: \spin{2p={;,up,up}}, а значит можем найти $J = |L-S| = 0$. 
\begin{equation*}
\text{Si}:\ \chain{1}{s}{2} \chain{2}{s}{2} \chain{2}{p}{6} \chain{3}{s}{2} \chain{3}{p}{2},
\hspace{10 mm} 
\text{основное состояние}:\ 
\term{3}{P}{0}.
\end{equation*}

\textbf{Сера}. Незполненной явлеется оболочка $\chain{2}{p}{4}$, для которой (S: $\chain{1}{s}{2} \chain{2}{s}{2} \chain{2}{p}{6} \chain{3}{s}{2} \chain{3}{p}{4}$) находим основное состояние $\term{3}{P}{0}$ в силу конфигурации \spin{2p={;up, up, pair}}.

\textbf{Все термы}. Найдём все термы для $\chain{}{p}{2}$, $l=1$, тогда $L = \{0,\, 1,\, 2\}$ и $S = \{0,\, 1\}$. 
Для $S=1$ и $L=1$ возможны конфигурации $\term{3}{P}{0, 1, 2}$. Для $S = 0$ и $L=\{0,2\}$ получим $\term{1}{S}{0},\, \term{1}{D}{2}$, аналогичные рассуждения будут верны для $\chain{}{p}{4}$. 


\textbf{Фосфор}. 
Для фосфора $\chain{}{p}{3}$ основным состоянием будет $\term{4}{S}{3/2}$. Состоянию с $M_s = \frac{1}{2}$ соответсвует конфигурация \spin{2p={;,up,pair}}, и $\term{2}{D}{3/2, 5/2}$. Для $M_L = 1$ возможны конфигурации \spin{2p={;,pair,up}} и \spin{2p={;up,,pair}} с обозначениями $\term{2}{P}{1/2,\, 3/2}$. Наконец, для $M_L = 0$ возможны конфигурации \spin{2p={;up,up,down}}, \spin{2p={;up,down,up}}, \spin{2p={;down,up,up}}, не приводящие к новым независимым состояниям.


\textbf{Ванадий}. V: $\ldots \chain{3}{d}{3}$ и конфигурация \spin{1s={;}}\spin{1s={;}}\spin{2p={;up,up,up}} с обозначением $\term{4}{F}{3/2}$. 

\textbf{Кобальт}. Co: $\ldots \chain{3}{d}{7}$  и конфигурация $\term{4}{F}{9/2}$.

\textbf{Церий}. Ce: $\ldots \chain{6}{s}{2} \chain{5}{d}{4}$  в конфигурации $\term{3}{H}{4}$ с $\sub{S}{max} = 1$ и $\sub{L}{max} = 5$, хотя на самом деле $\term{1}{G}{4}$ и конфигурация $\chain{4}{f}{1} \chain{5}{d}{1} \chain{6}{s}{2}$, явлется исключением из правил Хунда.

