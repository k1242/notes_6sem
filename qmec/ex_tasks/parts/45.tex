\Tsec{№4, 5. Спиновые состояния}


\textbf{Два электрона}. Для системы из двух электронов возможны конфигурации
\begin{align*}
	&S = 1, &\ket{1,+1} &= \ket{++}, \\
	&&\ket{1,-1} &= \ket{--},\\
	&&\ket{1,0} &= \tfrac{1}{\sqrt{2}} \left(\ket{+-}+\ket{-+}\right)\\
	&S=0, &\ket{0,0} &= \tfrac{1}{\sqrt{2}}\left(\ket{+-} - \ket{-+}\right).
\end{align*}
Для вывода полезно помнить, что
\begin{equation*}
	\hat{j}_\pm \ket{j, m} = \sqrt{j(j+1) \pm m(m\pm1)} \ket{j,m-1},
\end{equation*}
и заметить, что $S=0$ нечетное по перестановкам, $S=1$ четно по перестановкам. 

\textbf{Гелий}. Для двух протонов (фермионов) в основном состоянии $l=0$, откуда $(-1)^0 = 1$, а значит спиновая часть должна быть антисимметрична
\begin{equation*}
	\ket{p} = \tfrac{1}{\sqrt{2}}\left(\ket{+-}-\ket{-+}\right).
\end{equation*}
Аналогично верно для нейтронов (фермионов) $\ket{n}$.

В обще случае для $l \neq 0$ можем записать 
\begin{equation*}
	\ket{\psi} = \frac{1}{2} \left(
		\ket{+-} + (-1)^{l+1} \ket{-+}
	\right) \otimes \left(
		\ket{+-} + (-1)^{l+1} \ket{-+}
	\right).
\end{equation*}