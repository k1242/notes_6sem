\Tsec{№15. Эффект Рамзауэра}

Найдём сечение рассеяния медленных частиц на глубокой сферической яме радиуса $a_0$. Потеницал
\begin{equation*}
	U(r) = \left\{\begin{aligned}
	    - &U_0, & r \leq a_0, \\
	    &0, & r > a_0. 
	\end{aligned}\right.
\end{equation*}
Теперь
\begin{equation*}
	R_{k0} = \frac{1}{r} u(r),
	\hspace{5 mm} 
	u(0) = 0,
	\hspace{10 mm} 
	- \frac{\hbar^2}{2m} u'' + V u = E u.
\end{equation*}
Для $r > a$ $u'' + k^2 u = 0$, тогда
\begin{equation*}
	\sub{u}{II} = A \sin(kr + \delta_0).
\end{equation*}
Для $r\leq a$ 
\begin{equation*}
	u'' + (k^2 + \kappa^2) u = 0,
	\hspace{5 mm} 
	k^2_u \overset{\mathrm{def}}{=} k^2 + \kappa^2,
	\hspace{5 mm} 
	U_0 = \frac{\hbar^2 \kappa^2}{2m},
	\hspace{0.5cm} \Rightarrow \hspace{0.5cm}
	\sub{u}{I} = B \sin(k_u r).
\end{equation*}
Решение для радиальной волновой функции при $l=0$ запишется в виде
\begin{equation*}
	R_{k0} (r) = \left\{\begin{aligned}
	    &A \tfrac{1}{k_u r} \sin(k_u r),
	    & k_u^2 = \tfrac{2m}{\hbar^2}(E+U_0),
	    && r \leq a_0,
	    \\
	    & B \tfrac{1}{kr} \sin(k r + \delta_0)
	    & k^2 = \tfrac{2m}{\hbar^2} E,
	    && r > a_0,
	\end{aligned}\right.
\end{equation*}
где $A,\, B$ определяются из непрерывности $R_{k0}(r)$. Фазу можем найти из
\begin{equation*}
	\frac{R'_{k0}(r)}{R_{k0}(t)} \bigg|_{r=a_0 - 0} = 
	\frac{R'_{k0}(r)}{R_{k0}(t)} \bigg|_{r=a_0 + 0},
\end{equation*}
% причем для медленных частиц $k a_0 \ll 1$, тогда
% \begin{equation*}
% 	R'_{k0} (r) \sim \frac{1}{kr^2}\left(k r \cos(\ldots) - \sin(\ldots)\right), 
% \end{equation*}
% а при $r = a_0 - 0$ в глубокой яме $k_u a_0 \gg 1$:
% \begin{equation*}
% 	R'_{k0} (r) \sim \frac{1}{k_u r^2}\left(k_u r \cos(...) - \sin(...)\right).
% \end{equation*}
Из непрерывности лоагрифмической производной, находим
\begin{equation*}
	\tg(k a_0 + \delta_0) = \frac{k}{k_u} \tg(k_u a_0),
	\hspace{0.5cm} \Rightarrow \hspace{0.5cm}
	\delta_0 = - k a_0 + \arctg\left(
		\frac{k}{k_u} \tg(k_u a_0)
	\right).
\end{equation*}
% Рассмотрим случай $\frac{k}{\tilde{k}} \tg(\tilde{k} a) \ll ka \ll 1$, а тогда $\delta_0 \approx - ka$, а значит $f_0 = \frac{1}{k} e^{i \delta_0} \sin \delta_0 \approx - a$.
Рассмотрим случай $\tg(k_u a) \to \pm \infty$. Тогда $\delta_0 \approx \arctg(\pm\infty) = \pm\frac{\pi}{2}$, что ещё называют резонансным рассеянием, так как $\sin \delta_0 = \pm 1$.
Также посмотрим на $\frac{\tg(\tilde{k} a)}{\tilde{k} a} \approx	1$, тогда $\delta_0 \approx	0$, и получается $k_u a \ll 1$ и $f_0 \to 0$ -- эффект Рамзаура.
