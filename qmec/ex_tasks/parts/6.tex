\Tsec{№6. Сумма по поляризациям спиноров Дирака}

% киселев, 19.2.4

Для покоящихся спиноров сумму по поляризациям 
\begin{equation*}
	\Pi(\0) = \sum_\lambda u_\lambda(\0) \bar{u}_\lambda(\0)
\end{equation*}
можем вычислить в явном виде, как прямое проивзедение
\begin{equation*}
	\Pi(\0) = mc \left(1,\, 0,\, 1,\, 0\right) \otimes \begin{pmatrix}
		1 \\ 0 \\ 1 \\ 0
	\end{pmatrix} + 
	mc \left(0,\, 1,\, 0,\, 1\right) \otimes \begin{pmatrix}
		0 \\ 1 \\ 0 \\ 1
	\end{pmatrix} = 
	mc \begin{pmatrix}
		1 & 0 & 1 & 0 \\
		0 & 1 & 0 & 1 \\
		1 & 0 & 1 & 0 \\
		0 & 1 & 0 & 1
	\end{pmatrix}.
\end{equation*}
В ковариантном виде
\begin{equation*}
	\Pi(\0) = mc \left(\gamma_0 + \mathbbm{1}\right) = (\not{\hspace{-1.2pt}k} + mc),
	\hspace{5 mm} 
	k_\mu = (mc,\, \0).
\end{equation*}
Можем посчитать  (\red{посчитать}), что $\Lambda \hspace{-1.2pt}\not{\hspace{-1.2pt}k}\hspace{1.2pt} \Lambda^{-1} \to \p$, откуда сразу находим
\begin{equation*}
	\Pi(\vc{p}) = (\p + mc),
	\hspace{10 mm} 
	\Pi^c (\vc{p}) = \sum_\lambda v_\lambda(\vc{p}) \bar{v}_\lambda (\vc{p}) = (\p - mc),
\end{equation*}
где $\p = p_\mu \gamma^\mu = p_0 \gamma_0 - \vc{\gamma} \cdot \vc{p}$.