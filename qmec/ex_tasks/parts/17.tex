\Tsec{№17. Рассеяние на сфере}

Рассмотрим рассеяние медленных частиц на потенциале
\begin{equation*}
	U(r) = \left\{\begin{aligned}
	    &U_0 \to + \infty, & r \leq a_0, \\
	    &0, &r > a_0,
	\end{aligned}\right.
\end{equation*}
где для $ka\ll 1$ существенным будет только $l=0$, тогда
\begin{equation*}
	R_{k0} (r) = A \frac{\sin (kr + \delta_0)}{kr},
	\hspace{5 mm} 
	k^2 = \frac{2 m E}{\hbar^2},
	\hspace{5 mm} 
	A = \const.
\end{equation*}
Из непрерывности волновой функции находим, что $\sin(k a_0 + \delta_0) = 0$, откуда
\begin{equation*}
	\delta_0 = - ka_0,
	\hspace{5 mm} 
	f_0 \approx \frac{\delta_0}{k} - a_0,
\end{equation*}
что и обуславливает определение длины рассеяния. 


