\Tsec{№7, 8, 9. Термы оболочки}

\textbf{Кремний $\chain{2}{p}{2}$}. По правилам Хунда конфигурация незаполненой части $\chain{2}{p}{2}$ будет вида: \spin{2p={;,up,up}}, а значит можем найти $J = |L-S| = 0$. Основное состояние $\term{3}{P}{0}$.

\textbf{Сера  $\chain{2}{p}{4}$}. Незполненной явлеется оболочка $\chain{2}{p}{4}$, для которой находим основное состояние $\term{3}{P}{2}$ в силу конфигурации \spin{2p={;up, up, pair}}.

\textbf{Все термы}. 
 Найдём все термы для $\chain{}{p}{2}$, $l=1$, тогда $L = \{0,\, 1,\, 2\}$ и $S = \{0,\, 1\}$. 
Для $S=1$ и $L=1$ возможны конфигурации $\term{3}{P}{0, 1, 2}$. Для $S = 0$ и $L=\{0,2\}$ получим $\term{1}{S}{0},\, \term{1}{D}{2}$, аналогичные рассуждения будут верны для $\chain{}{p}{4}$. SDP.

\textbf{Фосфор  $\chain{2}{p}{3}$}. 
Для фосфора $\chain{}{p}{3}$ основным состоянием будет $\term{4}{S}{3/2}$. Состоянию с $M_s = \frac{1}{2}$ соответсвует конфигурация \spin{2p={;,up,pair}}, и $\term{2}{D}{3/2, 5/2}$. Для $M_L = 1$ возможны конфигурации \spin{2p={;,pair,up}} и \spin{2p={;up,,pair}} с обозначениями $\term{2}{P}{1/2,\, 3/2}$. Наконец, для $M_L = 0$ возможны конфигурации \spin{2p={;up,up,down}}, \spin{2p={;up,down,up}}, \spin{2p={;down,up,up}}, не приводящие к новым независимым состояниям. SPD.

