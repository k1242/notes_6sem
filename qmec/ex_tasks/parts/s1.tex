\subsection*{Т1}


\textbf{Линейное возмущение}.
Во-первых будем работать в представление операторов $\hat{a}$ и $\hat{a}^\dag$:
\begin{equation*}
        \hat{x} = \frac{x_0}{\sqrt{2}}\left(\hat{a} + \hat{a}^\dag\right),
        \hspace{5 mm} 
        \hat{p} = \frac{p_0}{\sqrt{2}} (\hat{a} - \hat{a}^\dag),
        \hspace{5 mm} 
        x_0 = \sqrt{\frac{\hbar}{m \omega}},
        \hspace{5 mm} 
        p_0 = \frac{\hbar}{x_0}.
\end{equation*}
Рассмотрим возмущение, вида
\begin{equation*}
    \hat{V} = \alpha x,
\end{equation*}
Заметим, что в первом порядке
\begin{equation*}
    V_{nn} =  \frac{\alpha x_0}{\sqrt{2}} \bk{n}[\hat{a} + \hat{a}^\dag]{n} = 0.
\end{equation*}
Тогда для второго порядка рассмотрим
\begin{equation*}
    V_{kn} = \frac{\alpha x_0}{\sqrt{2}}  \left(\bk{k}[\sqrt{n}]{n-1}
        + \sqrt{n+1} \ket{n+1}
    \right) = \frac{\alpha x_0}{\sqrt{2}} \left(
        \sqrt{n} \delta_{k,n-1} + \sqrt{n+1} \delta_{k, n+1}
    \right).
\end{equation*}
Теперь находим $\Delta_2$:
\begin{equation*}
    \Delta_2 = \frac{\alpha x_0^2}{2} \left(
        \frac{n}{\hbar \omega} - \frac{n+1}{\hbar \omega}
    \right) = - \frac{\alpha^2}{2 m \omega^2}.
\end{equation*}
Действительно, при замене переменных в $\hat{H}_0$ можем увидеть, что вторая поправка даёт точный ответ:
\begin{equation*}
    \hat{H} = \frac{m \omega^2}{2} \left(\hat{x} + \frac{\alpha}{m \omega^2}\right)^2 - \frac{\alpha^2}{2 m \omega^2} + \frac{\hat{p}^2}{2m}.
\end{equation*}


\textbf{Нелинейное возмущение}. Рассмотрим возмущение вида
\begin{equation*}
    \hat{V} = A x^3 + B x^4.
\end{equation*}
Тогда первая поправка к энергии:
\begin{equation*}
    \Delta_1^B = V_{nn} = \frac{3 B \hbar^2}{4 m^2 \omega^2} (2 n^2 + 2n +1),
    \hspace{5 mm} 
    \Delta_1^A = 0.
\end{equation*}
Вторую поправку найдём через
\begin{equation*}
    V_{kn}^A = A \left(\frac{x_0}{\sqrt{2}}\right)^3  \left(
        3 \delta_{k,n-1} n \sqrt{n} + 3 \delta_{k,n+1}  (n+1) \sqrt{n+1} + \delta_{k, n+3} \sqrt{(n+1)(n+2)(n+3)} + 
        \delta_{k,n-3} \sqrt{n (n-1) (n-2)}
    \right).
\end{equation*}
Тогда
\begin{equation*}
    \Delta_2^A = - \frac{A^2 \hbar^2}{8 m^3 \omega^4} \left(30 n^2 + 30 n + 1\right),
    \hspace{5 mm} 
    \Delta \approx \Delta_1^B +  \Delta_2^A.
\end{equation*}



\subsection*{Т2}

\textbf{Атом-ион}.
Рассмоотрим возмущение, вида
\begin{equation*}
    \hat{V} = - \sub{\vc{d}}{ат} \cdot \sub{\vc{E}}{ион},
    \hspace{5 mm} 
    \sub{\vc{E}}{ион} = \frac{Q \vc{r}}{r^3},
    \hspace{5 mm} 
    \sub{\vc{d}}{ат} = \sum_{i=1}^{} e_i \vc{r}_i.
\end{equation*}
Живём в парадигме
\begin{equation*}
    \mathbb{\hat{P}}\, \sub{\psi}{at} = \lambda_p \sub{\psi}{at},
    \hspace{5 mm} \lambda_p = \pm 1,
    \hspace{10 mm} 
    \mathbb{\hat{P}}\, \hat{\vc{r}} = - \hat{\vc{r}} \mathbb{\hat{P}},
    \hspace{5 mm} \mathbb{\hat{P}}^2 = \mathbbm{1}.
\end{equation*}
Для начала заметим, что
\begin{equation*}
    \Delta_1 = \bk{\sub{\psi}{at}}[\hat{V}]{\sub{\psi}{at}} = - \sub{\vc{E}}{ион} \cdot \langle \sub{\vc{d}}{ат} \rangle = 0.
\end{equation*}
Для второй поправки
\begin{equation*}
    \Delta_2 \sim - \frac{1}{r^4}.
\end{equation*}


\textbf{Атом-атом}. Возмущение теперь вида
\begin{equation*}
    \hat{V} = - \frac{1}{r^3} \left(
        3 (\vc{d}_1 \cdot \vc{n}) (\vc{d}_2 \cdot \vc{n}) - \vc{d}_1 \cdot \vc{d}_2
    \right),
    \hspace{10 mm} 
    \vc{n} = \frac{\vc{r}}{r}.
\end{equation*}
Первая поправка как обычно
\begin{equation*}
    \Delta_1 = \bk{\psi_1}[d_1^\alpha]{\psi_1} \bk{\psi_2}[d_2^\beta]{\psi_2} \delta_{\alpha \beta} = 0.
\end{equation*}
Зато вторая поправка
\begin{equation*}
    \Delta_2 \sim - \frac{1}{r^6}.
\end{equation*}




\subsection*{Т3}

Рассмотрим процесс, вида
\begin{equation*}
    {}^3_1 \text{H} \longrightarrow {}_2^3 \text{He} + e^- + \bar{\nu}_e.
\end{equation*}
Энкергия в основном состоянии
\begin{equation*}
    U_H = - \frac{e^2}{r},
    \hspace{10 mm} 
    U_{He} = - \frac{2 e^2}{r}.
\end{equation*}
Волновые функции:
\begin{equation*}
    \psi^H_{100} = \frac{1}{\sqrt{\pi a^3}} e^{-r/a},
    \hspace{10 mm} 
    \psi^{He}_{100} = \sqrt{\frac{2^3}{\pi a^3}} e^{-2r/a}.
\end{equation*}
При $n=2$: $l = 0, \pm 1$, тогда
\begin{equation*}
    \psi^{He}_{200} = \frac{1}{\sqrt{\pi a^3}} e^{- r/a} \left(1 - \frac{r}{a}\right).
\end{equation*}
Заметим, что остальные функции можем игнорировать, но для этого на них нужно посмотреть:
\begin{align*}
    \psi_{2,1,-1}^{He} &= \frac{2^{5/2}}{8 a \sqrt{\pi a^3}} e^{-r/a} e^{- i \varphi} r \sin \theta; \\
    \psi_{2,1,0}^{He} &= \frac{2^{5/2}}{4 a \sqrt{2 \pi a^3} } e^{-r/a} r \cos \theta; \\
    \psi_{2,1,1}^{He} &= \bar{\psi}_{2,1,-1}^{He}.
\end{align*}
Тогда искомая вероятность
\begin{align*}
    w_{100} &= |\bk{\psi^{He}_{100}}{\psi^H_{100}}|^2 \approx 0.7, \\ 
    w_{200} &= |\bk{\psi^{He}_{200}}{\psi^H_{100}}|^2 \approx 0.25,
\end{align*}
с их отношением $w_{100} / w_{200} \approx 2.8$.

% \newpage

\subsection*{Т4}

Продолжаем работать с основным состоянием водорода, а значит
\begin{equation*}
    \bk{\vc{r}}{\psi} = \psi_{100} (r) = \frac{1}{\sqrt{\pi a^3}} e^{-r/a}.
\end{equation*}

\textbf{Электростатика}. Вспоминаем, что
\begin{equation*}
    \Delta \varphi = - 4 \pi \rho_0,
    \hspace{5 mm} 
    \frac{4 \pi}{3} \rho_0 r^3 = - e > 0,
    \hspace{5 mm} 
    r_0 \approx 10^{-13}\ \text{см}.
\end{equation*}
Расписываем лапласиан в сферических координатах:
\begin{equation*}
    \Delta \varphi(r) = \nabla^2 \varphi(r) = \varphi'' + \frac{2}{r} \varphi' = \frac{1}{r} ( r\varphi)'',
    \hspace{0.5cm} \Rightarrow \hspace{0.5cm}
    r \varphi = - 4 \pi \rho_0 \iint r,
\end{equation*}
а значит
\begin{equation*}
    \varphi = \frac{e}{r_0^3} \frac{r^2}{2} + C_1 + \frac{C_0}{r}.
\end{equation*}
Считая $\Delta \varphi$ понимаем, что $\delta(\vc{r})$ быть не должно, а значит $C_0 = 0$. По условиям сшивки находим, что
\begin{equation*}
    U = \left\{\begin{aligned}
        - &e^2 /r, &r \geq r_0 \\
        &e^2 r^2 / 2 r_0^3 + C_1 e, & r_1 \leq r_0
    \end{aligned}\right.
    \hspace{0.5cm} \Rightarrow \hspace{0.5cm}
    C_1 = - \frac{3}{2} \frac{e}{r_0}.
\end{equation*}
Итого, искомый потенциал 
\begin{equation*}
    \varphi = \frac{e}{r_0^3} \frac{r^2}{2} - \frac{3}{2} \frac{e}{r_0}.
\end{equation*}

\textbf{Кванты}. Поправку можем найти, считая
\begin{equation*}
    - \frac{e^2}{r} \mapsto U(r),
    \hspace{0.5cm} \Rightarrow \hspace{0.5cm}
    \sub{\varphi}{new} - \varphi = \frac{e^2 r^2}{2 r_0^3}  - \frac{3e}{2r_0} - \left(- \frac{e^2}{r}\right), \  \ \  r \leq r_0.
\end{equation*}
А значит, интегрируя, находим
\begin{equation*}
    \Delta_1 = \bk{\psi}[\hat{V}]{\psi} = \int_{0}^{r_0} r^2 \d r \int_{-1}^{1} \d \cos \theta \int_{0}^{2\pi} \d \varphi \left(
        \frac{e^2 r^2}{2 r_0^3} - \frac{3}{2} \frac{e}{r_0} + \frac{e^2}{r}
    \right) = \frac{2 e^2}{5 a} \left(\frac{r_0}{a}\right)^2.
\end{equation*}

% \newpage

\subsection*{Т5}

Помним, что
\begin{equation*}
    \psi_{100} = \frac{1}{\sqrt{\pi a^3}} e^{-r/a}, \hspace{5 mm} a = \frac{\hbar}{m c \alpha_{em}}.
\end{equation*}
Также помним, что
\begin{equation*}
    \vc{d} = e \vc{r},
    \hspace{5 mm} 
    \hat{V} = - \vc{d} \cdot \vc{E} = - e E r \cos \theta.
\end{equation*}
При этом мы знаем, что
\begin{equation*}
    \Delta = - \frac{1}{2} \alpha_{ij} E^i E^j,
\end{equation*}
где $\alpha_{ij}$ -- тензор поляризуемости.

Замечаем, что всё также
\begin{equation*}
    \Delta_1  = \bk{\psi_{100}}[\hat{V}]{\psi_{100}} = 0.
\end{equation*}
Вторую поправку можем найти, как
\begin{equation*}
    \Delta_2 = \bk{n^{(0)}}[\hat{V}]{n^{(1)}}.
\end{equation*}
\textbf{Поиск возмущения}. Волновую функцию $\psi^{(1)}$ можем найти, как решение уравнения, вида
\begin{equation*}
    \left(
        - \frac{\hbar^2}{2m} \Delta - \frac{e^2}{r}
    \right) \psi^{(1)} = - \frac{e^2}{2 a} \frac{1}{n^2} \psi^{(1)} + \frac{\varepsilon E r \cos \theta}{\sqrt{\pi a^3}} e^{-r/a}.
\end{equation*}
Ищем решение в виде
\begin{equation*}
    \psi^{(1)} (\vc{r}) = \sum_{l,m} R_l (r) Y_{l,m} (\theta, \varphi).
\end{equation*}
Подставляя, находим
\begin{equation*}
    - \frac{\hbar^2}{2 m} \Delta \psi_{l,m} = - \frac{\hbar^2}{2m } \frac{1}{r} (r R_l)'' Y_{l,m} + \frac{\hbar^2}{2m} \frac{l (l+1)}{r^2} R_l Y_{l,m}.
\end{equation*}
Так как $Y_{10} \sim \cos \theta$, то нам подходит только $\psi_{10}$, а значит
\begin{equation*}
    \psi^{(1)} (r) = \frac{e E}{\sqrt{\pi a^3}} e^{-r/a} \cos \theta \cdot f(r).
\end{equation*}
Подставляя это в модифицированное уравнение Шрёдингера, найдём $f(r)$. Так приходим к диффуру
\begin{equation*}
    \frac{f''}{2} + f'\left(\frac{1}{r}-\frac{1}{a}\right) - f \frac{1}{r^2} = - r \frac{1}{a e^2}.
\end{equation*}
Далее будем искать $f$ в виде полинома второй степени: $f(r) = A r + B r^2$. Тогда
\begin{equation*}
    A = \frac{a}{e^2},
    \hspace{5 mm} 
    B = \frac{1}{2 e^2},
    \hspace{0.5cm} \Rightarrow \hspace{0.5cm}
    f(r) = \frac{r a}{e^2} + \frac{r^2}{2 e^2}.
\end{equation*}
А значит искомая функция
\begin{equation*}
    \psi^{(1)} = \frac{e E}{\sqrt{\pi a^3}} e^{-r/a} \cos \theta \cdot \frac{r}{e^2} \left(a + \frac{r}{2}\right).
\end{equation*}

\textbf{Сдвиг по энергии}. Интегрируя $\psi^{(1)}$, находим
\begin{equation*}
    \Delta_2 = \int_{0}^{\infty} r^2 \d r \int_{-1}^{1} \d \cos \theta \int_{0}^{2 \pi} \d \varphi \frac{1}{\pi a^3} e^{-2r/a} \cos \theta \times 
    \left(- e E r \cos \theta\right) \frac{r a}{e^2} \left(1 + \frac{r}{2a}\right) = - \frac{9}{4}  E^2 a^3.
\end{equation*}
Сопостовляя с поляризуемостью, находим
\begin{equation*}
    \alpha = \frac{9}{2} a^3.
\end{equation*}

