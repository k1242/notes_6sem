\subsection*{Т30}


Для потенциала, вида
\begin{equation*}
	V(r) = \frac{\beta}{r^2},
	\hspace{5 mm} 
	\beta > 0,
\end{equation*}
найдём фазы рассеняи $\delta_l$. 

Запишем уравнение Шредингера для парциальной волны $u_l (r) = r R_l (r)$:
\begin{equation*}
	\left(
		\frac{d^2 }{d r^2} + k^2 - \frac{1}{r^2}\left(
			l(l+1) + \frac{2 m \beta}{\hbar^2}
		\right)
	\right) u_l (r) = 0.
\end{equation*}
Рассмотрим замену $u_l (r) = \sqrt{r} \varphi(r)$
\begin{equation*}
	\varphi'' + \frac{1}{r} \varphi' + \left(
		k^2 - \frac{1}{r^2}\left(
			\left(l +\frac{1}{2}\right)^2 + \frac{2 m \beta}{\hbar^2}
		\right)
	\right)\varphi = 0,
\end{equation*}
решения которого знаем в виде функций Бесселя $J_{\pm \nu} (kr)$, где
\begin{equation*}
	\nu = \sqrt{\left(l + \frac{1}{2}\right)^2 + \frac{2 m \beta}{\hbar^2}}.
\end{equation*}
Требуя $u_l (0) = 0$, находим решение в виде
\begin{equation*}
	u_l (r) = c \sqrt{\frac{\pi k r}{2}} J_\nu (kr).
\end{equation*}
Полезно посмотреть асмиптотику на бесконечности, для которой
\begin{equation*}
	u_l(r) \sim c \sin\left(kr - \frac{\pi \nu}{2} + \frac{\pi}{4}\right) = c \sin\left(kr - \frac{\pi l}{2} + \delta_l\right),
\end{equation*}
откуда находим искомые фазы рассеяния
\begin{equation*}
	\delta_l = - \frac{\pi}{2} \left(
		\sqrt{\left(l + \frac{1}{2}\right)^2 + \frac{2 m \beta}{\hbar^2}} - \left(l + \frac{1}{2}\right)
	\right).
\end{equation*}

\textbf{Предельный случай}. В пределе $2 m \beta / \hbar^2 \ll 1$ получаем
\begin{equation*}
	\delta_l \approx - \frac{\pi}{2} \frac{m \beta}{\hbar^2(l + \tfrac{1}{2})},
\end{equation*}
откуда также получаем $|\delta_l| \ll 1$. 

В таком случае можем просуммировать ряд
\begin{equation*}
	f(\theta) = \frac{1}{2 ik} \sum_{l=0}^{\infty} (2l+1) \left(e^{2 i \delta_l}- 1\right) P_l  (\cos \theta) \approx \frac{1}{k} \sum_{l=0}^{\infty} (2l+1) \delta_l P_l(\cos \theta) \approx	
	-\frac{\pi m \beta}{\hbar^2 k} \sum_{l=0}^{\infty} P_l(\cos \theta).
\end{equation*}
Суммируя полиному Лежанда, находим
\begin{equation*}
	f(\theta) \approx	 - \frac{\pi m \beta}{2 k \hbar^2 \sin(\theta/2)},
\end{equation*}
аналогично тому, что получили бы в борновском приближении.



