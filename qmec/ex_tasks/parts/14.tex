\Tsec{14. Время жизни уровня}


Уже считали для атома водорода
\begin{equation*}
	\frac{1}{\tau} = \frac{2}{3c^3} \omega_f^3 |d_{fi}|^2 \frac{1}{\hbar}.
\end{equation*}
Ион гелия -- водородоподобный атом, с отличным радиусом Бора:
\begin{equation*}
	\sub{a}{H} = \frac{\hbar^2}{m e^2},
	\hspace{5 mm} 
	\sub{a}{He} = \frac{\hbar^2}{m e^2 Z} = \frac{\hbar^2}{m e^2 2},
	\hspace{0.5cm} \Rightarrow \hspace{0.5cm}
	d_{fi}^{\text{He}} = \frac{1}{Z} d_{fi}.
\end{equation*}
Аналогично для частоты
\begin{equation*}
	\omega_{f} \sim e^4 \sim \frac{1}{a^2},
	\hspace{0.5cm} \Rightarrow \hspace{0.5cm}
	\omega_f^{\text{He}} \sim Z^2 \omega_f.
\end{equation*}
Таким образом находим, что
\begin{equation*}
	\sub{\tau}{He} = \frac{1}{Z^4} \sub{\tau}{H} = 10^{-10} \text{ с}.
\end{equation*}