% 1 -- Т6
% 2 -- Т11
% 3 -- Т12-
% 4 -- | 
% 5 -- |суммарный спин двух частиц
% 6 -- kiselev: 19.2.4.
% 7 -- |
% 8 -- | 
% 9 -- |термы оболочек
% 10-- kiselev: 24.2
% 11-- |
% 12-- |
% 13-- |правило отбора
% 14-- спонтанное излучение
% 15-- Л27, 16.2
% 16-- Т33
% 17-- Л27, 16.1
% 18-- Т30
% 19-- Т29



% 15, 22, 23, 26, 

% +15 -- посчитать псевдовектор Паули-Любанского
% 22 -- уровни Ландау (атом в магнитном поле)
% +23 -- тождественные частицы в потенциальном ящике
% 26 -- духота с LSJ по правилам Хунда
% +33 -- рассеяние тождественных частиц

% document's head

\begin{center}
    \LARGE \textsc{Заметки по курсу \\ 
    <<Уравнения математической физики>>}
\end{center}

\hrule

\phantom{42}

\begin{flushright}
    \begin{tabular}{rr}
    % written by:
        % \textbf{Источник}: 
        % & \href{__ссылка__}{__название__} \\
        % & \\
        \textbf{Семинарист}: 
        & Александр Сергеевич Осин \\
        & \\
        \textbf{Автор заметок}: 
        & Хоружий Кирилл \\
        % & Примак Евгений \\
        & \\
    % date:
        \textbf{От}: &
        \textit{\today}\\
    \end{tabular}
\end{flushright}

\thispagestyle{empty}
\tableofcontents
\newpage


% добавить условия

% \section*{Заметки с семинара}

\textbf{Вырожденный случай}. Пусть теперь
\begin{equation*}
    \hat{H}_0 \ket{n^{(0)}, a} = E_n \ket{n^{(0)}, a},\hspace{5 mm} 
    a = 1, \ldots, k.
\end{equation*}
Тогда
\begin{equation*}
    \ket{n} = \sum c_a \ket{n^{(0)}, a} + \sum_{p=1}^{\infty} \lambda^p \ket{n^{(p)}},
    \hspace{5 mm} 
    \hat{H} \ket{n} = E_n \ket{n}.
\end{equation*}
Дополнительно накладываем уловие
\begin{equation*}
    \bk{n^{(0)}, a}{n} = c_a,
    \hspace{0.5cm} \Rightarrow \hspace{0.5cm}
    \bk{n^{(0)}, a}{n^{(k)}} = 0,
\end{equation*}
при $k \geq 1$. При этом всё равно выполняется
\begin{equation*}
    \bk{n^{(0)}, a}{n^{(0)}, b} = \delta_{ab}.
\end{equation*}
Подставляя это всё в стацонарное уравнение Шрёдигера, группируя по $\lambda$, находим
\begin{equation*}
    \left(V - \varepsilon_1 \mathbbm{1}\right) \vc{c} = 0.
\end{equation*}
Накладывая дополнительное условие $\vc{c} \neq 0$, находим
\begin{equation*}
    \det(V - \varepsilon_1 \mathbbm{1}) = 0.
    % \hspace{0.5cm} \Rightarrow \hspace{0.5cm}
    % \varepsilon_1 \mathbbm{1}.
\end{equation*}

% 

\textbf{Рассеяние}. Умеем решать задачу для свободной частицы
\begin{equation*}
	- \frac{h^2}{2m} \Delta \psi = \frac{\hbar^2 k^2}{2m}\psi.
\end{equation*}
Делаем подстановку, вида
\begin{equation*}
	\psi_{klm} = Y_{lm} (\theta, \varphi) R_{kl} (r).
\end{equation*}
Тогда получаем решение, вида
\begin{equation*}
	R_{kl} (r) = c_l \frac{r^l}{k^l} \left( \frac{1}{r} \frac{d }{d r} \right)^l \frac{1}{r} \left(
		- e^{- ik r} + S_l e^{ikr}
	\right).
\end{equation*}
Из асимптотики в $0$ можем восстановить $S_l^{(0)} = 1$ для $V \equiv 0$. Таким образом получаем выражение
\begin{equation*}
	R_{kl}^{(0)} = c_l^{(0)} (-1)^l \frac{r^l}{k^l} \left(
		\frac{1}{r} \frac{d }{d r} 
	\right)^l \frac{2i}{r} \sin(kr),
\end{equation*}
что при $r \to \infty$ переходит в 
\begin{equation*}
	R_{kl}^{(0)} \approx c_l^{(0)} \frac{2i}{r} \sin(kr - \tfrac{\pi l}{2}).
\end{equation*}
По физическому смыслу понимаем, что $e^{-ikr}$ отвечает за расходящиеся от центра волны, а $e^{ikr}$ за сходящиеся. Значит, при упругом рассеянии $|S_l| = 1$, тогда можем переписать $S_l = e^{2 i \delta_l}$:
\begin{equation*}
	R_{kl} \approx c_l^{(0)} \frac{2i}{r} e^{i \delta_l} \sin(kr - \tfrac{\pi l}{2} + \delta_l),
	\hspace{10 mm} 
	c_l^{(0)} = i^l \frac{1}{2ik}.
\end{equation*}
Можем найти сечение рассения для каждого значения $l$:
\begin{equation*}
	f_l = \frac{2l+1}{k} e^{i \delta_l} \sin(\delta_l),
	\hspace{10 mm} 
	f(\theta) = \sum_l f_l P_l(\cos \theta).
\end{equation*}
А также сразу находим
\begin{equation*}
	\sigma_l = \frac{4 \pi}{k^2} (2l + 1) \sin^2 \delta_l,
	\hspace{10 mm} 
	\sigma = \sum_l \sigma_l.
\end{equation*}
Для каждого парциального вклада можем записать оптическую теорему:
\begin{equation*}
	\sigma_l^{\text{tot}} = \frac{4 \pi}{k} \Im f_l.
\end{equation*}




% \section*{Первое задание}
% \subsection*{Т1}


\textbf{Линейное возмущение}.
Во-первых будем работать в представление операторов $\hat{a}$ и $\hat{a}^\dag$:
\begin{equation*}
        \hat{x} = \frac{x_0}{\sqrt{2}}\left(\hat{a} + \hat{a}^\dag\right),
        \hspace{5 mm} 
        \hat{p} = \frac{p_0}{\sqrt{2}} (\hat{a} - \hat{a}^\dag),
        \hspace{5 mm} 
        x_0 = \sqrt{\frac{\hbar}{m \omega}},
        \hspace{5 mm} 
        p_0 = \frac{\hbar}{x_0}.
\end{equation*}
Рассмотрим возмущение, вида
\begin{equation*}
    \hat{V} = \alpha x,
\end{equation*}
Заметим, что в первом порядке
\begin{equation*}
    V_{nn} =  \frac{\alpha x_0}{\sqrt{2}} \bk{n}[\hat{a} + \hat{a}^\dag]{n} = 0.
\end{equation*}
Тогда для второго порядка рассмотрим
\begin{equation*}
    V_{kn} = \frac{\alpha x_0}{\sqrt{2}}  \left(\bk{k}[\sqrt{n}]{n-1}
        + \sqrt{n+1} \ket{n+1}
    \right) = \frac{\alpha x_0}{\sqrt{2}} \left(
        \sqrt{n} \delta_{k,n-1} + \sqrt{n+1} \delta_{k, n+1}
    \right).
\end{equation*}
Теперь находим $\Delta_2$:
\begin{equation*}
    \Delta_2 = \frac{\alpha x_0^2}{2} \left(
        \frac{n}{\hbar \omega} - \frac{n+1}{\hbar \omega}
    \right) = - \frac{\alpha^2}{2 m \omega^2}.
\end{equation*}
Действительно, при замене переменных в $\hat{H}_0$ можем увидеть, что вторая поправка даёт точный ответ:
\begin{equation*}
    \hat{H} = \frac{m \omega^2}{2} \left(\hat{x} + \frac{\alpha}{m \omega^2}\right)^2 - \frac{\alpha^2}{2 m \omega^2} + \frac{\hat{p}^2}{2m}.
\end{equation*}


\textbf{Нелинейное возмущение}. Рассмотрим возмущение вида
\begin{equation*}
    \hat{V} = A x^3 + B x^4.
\end{equation*}
Тогда первая поправка к энергии:
\begin{equation*}
    \Delta_1^B = V_{nn} = \frac{3 B \hbar^2}{4 m^2 \omega^2} (2 n^2 + 2n +1),
    \hspace{5 mm} 
    \Delta_1^A = 0.
\end{equation*}
Вторую поправку найдём через
\begin{equation*}
    V_{kn}^A = A \left(\frac{x_0}{\sqrt{2}}\right)^3  \left(
        3 \delta_{k,n-1} n \sqrt{n} + 3 \delta_{k,n+1}  (n+1) \sqrt{n+1} + \delta_{k, n+3} \sqrt{(n+1)(n+2)(n+3)} + 
        \delta_{k,n-3} \sqrt{n (n-1) (n-2)}
    \right).
\end{equation*}
Тогда
\begin{equation*}
    \Delta_2^A = - \frac{A^2 \hbar^2}{8 m^3 \omega^4} \left(30 n^2 + 30 n + 1\right),
    \hspace{5 mm} 
    \Delta \approx \Delta_1^B +  \Delta_2^A.
\end{equation*}



\subsection*{Т2}

\textbf{Атом-ион}.
Рассмоотрим возмущение, вида
\begin{equation*}
    \hat{V} = - \sub{\vc{d}}{ат} \cdot \sub{\vc{E}}{ион},
    \hspace{5 mm} 
    \sub{\vc{E}}{ион} = \frac{Q \vc{r}}{r^3},
    \hspace{5 mm} 
    \sub{\vc{d}}{ат} = \sum_{i=1}^{} e_i \vc{r}_i.
\end{equation*}
Живём в парадигме
\begin{equation*}
    \mathbb{\hat{P}}\, \sub{\psi}{at} = \lambda_p \sub{\psi}{at},
    \hspace{5 mm} \lambda_p = \pm 1,
    \hspace{10 mm} 
    \mathbb{\hat{P}}\, \hat{\vc{r}} = - \hat{\vc{r}} \mathbb{\hat{P}},
    \hspace{5 mm} \mathbb{\hat{P}}^2 = \mathbbm{1}.
\end{equation*}
Для начала заметим, что
\begin{equation*}
    \Delta_1 = \bk{\sub{\psi}{at}}[\hat{V}]{\sub{\psi}{at}} = - \sub{\vc{E}}{ион} \cdot \langle \sub{\vc{d}}{ат} \rangle = 0.
\end{equation*}
Для второй поправки
\begin{equation*}
    \Delta_2 \sim - \frac{1}{r^4}.
\end{equation*}


\textbf{Атом-атом}. Возмущение теперь вида
\begin{equation*}
    \hat{V} = - \frac{1}{r^3} \left(
        3 (\vc{d}_1 \cdot \vc{n}) (\vc{d}_2 \cdot \vc{n}) - \vc{d}_1 \cdot \vc{d}_2
    \right),
    \hspace{10 mm} 
    \vc{n} = \frac{\vc{r}}{r}.
\end{equation*}
Первая поправка как обычно
\begin{equation*}
    \Delta_1 = \bk{\psi_1}[d_1^\alpha]{\psi_1} \bk{\psi_2}[d_2^\beta]{\psi_2} \delta_{\alpha \beta} = 0.
\end{equation*}
Зато вторая поправка
\begin{equation*}
    \Delta_2 \sim - \frac{1}{r^6}.
\end{equation*}




\subsection*{Т3}

Рассмотрим процесс, вида
\begin{equation*}
    {}^3_1 \text{H} \longrightarrow {}_2^3 \text{He} + e^- + \bar{\nu}_e.
\end{equation*}
Энкергия в основном состоянии
\begin{equation*}
    U_H = - \frac{e^2}{r},
    \hspace{10 mm} 
    U_{He} = - \frac{2 e^2}{r}.
\end{equation*}
Волновые функции:
\begin{equation*}
    \psi^H_{100} = \frac{1}{\sqrt{\pi a^3}} e^{-r/a},
    \hspace{10 mm} 
    \psi^{He}_{100} = \sqrt{\frac{2^3}{\pi a^3}} e^{-2r/a}.
\end{equation*}
При $n=2$: $l = 0, \pm 1$, тогда
\begin{equation*}
    \psi^{He}_{200} = \frac{1}{\sqrt{\pi a^3}} e^{- r/a} \left(1 - \frac{r}{a}\right).
\end{equation*}
Заметим, что остальные функции можем игнорировать, но для этого на них нужно посмотреть:
\begin{align*}
    \psi_{2,1,-1}^{He} &= \frac{2^{5/2}}{8 a \sqrt{\pi a^3}} e^{-r/a} e^{- i \varphi} r \sin \theta; \\
    \psi_{2,1,0}^{He} &= \frac{2^{5/2}}{4 a \sqrt{2 \pi a^3} } e^{-r/a} r \cos \theta; \\
    \psi_{2,1,1}^{He} &= \bar{\psi}_{2,1,-1}^{He}.
\end{align*}
Тогда искомая вероятность
\begin{align*}
    w_{100} &= |\bk{\psi^{He}_{100}}{\psi^H_{100}}|^2 \approx 0.7, \\ 
    w_{200} &= |\bk{\psi^{He}_{200}}{\psi^H_{100}}|^2 \approx 0.25,
\end{align*}
с их отношением $w_{100} / w_{200} \approx 2.8$.

\newpage

\subsection*{Т4}

Продолжаем работать с основным состоянием водорода, а значит
\begin{equation*}
    \bk{\vc{r}}{\psi} = \psi_{100} (r) = \frac{1}{\sqrt{\pi a^3}} e^{-r/a}.
\end{equation*}

\textbf{Электростатика}. Вспоминаем, что
\begin{equation*}
    \Delta \varphi = - 4 \pi \rho_0,
    \hspace{5 mm} 
    \frac{4 \pi}{3} \rho_0 r^3 = - e > 0,
    \hspace{5 mm} 
    r_0 \approx 10^{-13}\ \text{см}.
\end{equation*}
Расписываем лапласиан в сферических координатах:
\begin{equation*}
    \Delta \varphi(r) = \nabla^2 \varphi(r) = \varphi'' + \frac{2}{r} \varphi' = \frac{1}{r} ( r\varphi)'',
    \hspace{0.5cm} \Rightarrow \hspace{0.5cm}
    r \varphi = - 4 \pi \rho_0 \iint r,
\end{equation*}
а значит
\begin{equation*}
    \varphi = \frac{e}{r_0^3} \frac{r^2}{2} + C_1 + \frac{C_0}{r}.
\end{equation*}
Считая $\Delta \varphi$ понимаем, что $\delta(\vc{r})$ быть не должно, а значит $C_0 = 0$. По условиям сшивки находим, что
\begin{equation*}
    U = \left\{\begin{aligned}
        - &e^2 /r, &r \geq r_0 \\
        &e^2 r^2 / 2 r_0^3 + C_1 e, & r_1 \leq r_0
    \end{aligned}\right.
    \hspace{0.5cm} \Rightarrow \hspace{0.5cm}
    C_1 = - \frac{3}{2} \frac{e}{r_0}.
\end{equation*}
Итого, искомый потенциал 
\begin{equation*}
    \varphi = \frac{e}{r_0^3} \frac{r^2}{2} - \frac{3}{2} \frac{e}{r_0}.
\end{equation*}

\textbf{Кванты}. Поправку можем найти, считая
\begin{equation*}
    - \frac{e^2}{r} \mapsto U(r),
    \hspace{0.5cm} \Rightarrow \hspace{0.5cm}
    \sub{\varphi}{new} - \varphi = \frac{e^2 r^2}{2 r_0^3}  - \frac{3e}{2r_0} - \left(- \frac{e^2}{r}\right), \  \ \  r \leq r_0.
\end{equation*}
А значит, интегрируя, находим
\begin{equation*}
    \Delta_1 = \bk{\psi}[\hat{V}]{\psi} = \int_{0}^{r_0} r^2 \d r \int_{-1}^{1} \d \cos \theta \int_{0}^{2\pi} \d \varphi \left(
        \frac{e^2 r^2}{2 r_0^3} - \frac{3}{2} \frac{e}{r_0} + \frac{e^2}{r}
    \right) = \frac{2 e^2}{5 a} \left(\frac{r_0}{a}\right)^2.
\end{equation*}

\newpage

\subsection*{Т5}

Помним, что
\begin{equation*}
    \psi_{100} = \frac{1}{\sqrt{\pi a^3}} e^{-r/a}, \hspace{5 mm} a = \frac{\hbar}{m c \alpha_{em}}.
\end{equation*}
Также помним, что
\begin{equation*}
    \vc{d} = e \vc{r},
    \hspace{5 mm} 
    \hat{V} = - \vc{d} \cdot \vc{E} = - e E r \cos \theta.
\end{equation*}
При этом мы знаем, что
\begin{equation*}
    \Delta = - \frac{1}{2} \alpha_{ij} E^i E^j,
\end{equation*}
где $\alpha_{ij}$ -- тензор поляризуемости.

Замечаем, что всё также
\begin{equation*}
    \Delta_1  = \bk{\psi_{100}}[\hat{V}]{\psi_{100}} = 0.
\end{equation*}
Вторую поправку можем найти, как
\begin{equation*}
    \Delta_2 = \bk{n^{(0)}}[\hat{V}]{n^{(1)}}.
\end{equation*}
\textbf{Поиск возмущения}. Волновую функцию $\psi^{(1)}$ можем найти, как решение уравнения, вида
\begin{equation*}
    \left(
        - \frac{\hbar^2}{2m} \Delta - \frac{e^2}{r}
    \right) \psi^{(1)} = - \frac{e^2}{2 a} \frac{1}{n^2} \psi^{(1)} + \frac{\varepsilon E r \cos \theta}{\sqrt{\pi a^3}} e^{-r/a}.
\end{equation*}
Ищем решение в виде
\begin{equation*}
    \psi^{(1)} (\vc{r}) = \sum_{l,m} R_l (r) Y_{l,m} (\theta, \varphi).
\end{equation*}
Подставляя, находим
\begin{equation*}
    - \frac{\hbar^2}{2 m} \Delta \psi_{l,m} = - \frac{\hbar^2}{2m } \frac{1}{r} (r R_l)'' Y_{l,m} + \frac{\hbar^2}{2m} \frac{l (l+1)}{r^2} R_l Y_{l,m}.
\end{equation*}
Так как $Y_{10} \sim \cos \theta$, то нам подходит только $\psi_{10}$, а значит
\begin{equation*}
    \psi^{(1)} (r) = \frac{e E}{\sqrt{\pi a^3}} e^{-r/a} \cos \theta \cdot f(r).
\end{equation*}
Подставляя это в модифицированное уравнение Шрёдингера, найдём $f(r)$. Так приходим к диффуру
\begin{equation*}
    \frac{f''}{2} + f'\left(\frac{1}{r}-\frac{1}{a}\right) - f \frac{1}{r^2} = - r \frac{1}{a e^2}.
\end{equation*}
Далее будем искать $f$ в виде полинома второй степени: $f(r) = A r + B r^2$. Тогда
\begin{equation*}
    A = \frac{a}{e^2},
    \hspace{5 mm} 
    B = \frac{1}{2 e^2},
    \hspace{0.5cm} \Rightarrow \hspace{0.5cm}
    f(r) = \frac{r a}{e^2} + \frac{r^2}{2 e^2}.
\end{equation*}
А значит искомая функция
\begin{equation*}
    \psi^{(1)} = \frac{e E}{\sqrt{\pi a^3}} e^{-r/a} \cos \theta \cdot \frac{r}{e^2} \left(a + \frac{r}{2}\right).
\end{equation*}

\textbf{Сдвиг по энергии}. Интегрируя $\psi^{(1)}$, находим
\begin{equation*}
    \Delta_2 = \int_{0}^{\infty} r^2 \d r \int_{-1}^{1} \d \cos \theta \int_{0}^{2 \pi} \d \varphi \frac{1}{\pi a^3} e^{-2r/a} \cos \theta \times 
    \left(- e E r \cos \theta\right) \frac{r a}{e^2} \left(1 + \frac{r}{2a}\right) = - \frac{9}{4}  E^2 a^3.
\end{equation*}
Сопостовляя с поляризуемостью, находим
\begin{equation*}
    \alpha = \frac{9}{2} a^3.
\end{equation*}


% \section{Семинар №2}



\textbf{Секулярные члены}. Пусть есть уравнение вида
\begin{equation*}
    \ddot{x} + \omega_0^2 x = - \varepsilon \omega_0^2 x,\hspace{5 mm} 
    x(0) = a,
    \hspace{5 mm} \dot{x} (0) = 0.
\end{equation*}
Решение может быть найдено в виде
\begin{equation*}
    x(t) = a \cos \left(\omega_0 \sqrt{1 + \varepsilon} t\right) \approx  
    a \cos\left(\omega_0 \left(1 + \tfrac{\varepsilon}{2}\right)t\right) = a \cos \omega_0 t - \frac{a \varepsilon \omega_0 t}{2} \sin (\omega_0 t) + o(\varepsilon).
\end{equation*}
И вот видна беда, при $\varepsilon \omega_0 t \sim 1$ теория возмущений не работает. В большей части резонансных систем возникают секулярные члены. 

Получим этот результат в терминах теории возмущений. Пусть есть тот же гармонический осциллятор, заданы начальные условия, и знаем решение в виде
\begin{equation*}
    x(t) = x(0) \cos \omega_0 t + \frac{\dot{x}(0)}{\omega_0} \sin \omega_0 t + \int_{0}^{t} \frac{\sin \omega_0 (t-\tau)}{\omega_0} f(\tau) \d \tau.
\end{equation*}
Разложим это всё по $\varepsilon$ и приравняем при степенях $\varepsilon$:
\begin{align*}
    &\varepsilon^0: 
    & \ddot{x}_0 + \omega_0^2 x_0 &= 0 
    &&x_0(0)= a, \ \ \dot{x}_0(0) = 0, \\
    &\varepsilon^1: 
    & \ddot{x}_1 + \omega_0^2 x_1 &= - \omega_0^2 x_0 
    &&x_1(0)= 0, \ \ \dot{x}_1(0) = 0,
\end{align*}
так приходим к
\begin{equation*}
    x_1 (t) = - a \omega_0 \int_{0}^{t} \sin \left(\omega_0 (t-\tau)\right) \cos \left(\omega_0 \tau\right) \d \tau = - \frac{a \omega_0}{2} \sin (\omega_0 t) \cdot t,
\end{equation*}
что получается даёт ответ только на конечном интервале времени. 

\textbf{Медленные переменные}. 
Основная идея решения таких возмущений:
\begin{equation*}
    \ddot{x} + \omega_0^2 x = f(x, \dot{x}, t),
\end{equation*}
где $f$ содержит малость $\sim \varepsilon \ll 1$ -- ввести медленно меняющиеся переменные:
\begin{equation*}
    x(t) = A(t) \sin (\omega_0 t + \varphi(t)).
\end{equation*}
Подставляем это в диффур
\begin{align*}
    \dot{x} &= \dot{A} \sin(\omega_0 t + \varphi) + A \cos (\omega_0 t + \varphi) (\omega_0 + \dot{\varphi}) \\
    \ddot{x} &= \ddot{A} \sin (\omega_0 t + \varphi) + 2 \dot{A} \cos(\omega_0 t + \varphi) (\omega_0 + \dot{\varphi}) + A \ddot{\varphi} \cos(\omega_0  t + \varphi) - A (\omega_0 + \dot{\varphi})^2 \sin(\omega_0 t + \varphi).
\end{align*}
Зафиксируем, что $\dot{A} (t) \ll \omega_0 A(t)$ и $\dot{\varphi} \ll \omega_0$. Оставим здесь только слагаемые до первого порядка малости:
\begin{align*}
    \ddot{x} = 2 \dot{A} \omega_0 \cos(\omega_0 t + \varphi)  - 2 A \omega_0 \dot{\varphi} \sin(\omega_0 t + \varphi) = 
    f\left(
        A \sin(\omega_0 t + \varphi),\, A \omega_0 \cos (\omega_0 t + \varphi),\, t
    \right).
\end{align*}
Домножим это уравнение на $\cos(\omega_0 \tau + \varphi(t))$, также на $\sin \ldots$ и проинтегрируем по периоду:
\begin{equation*}
    \int_{t-T/2}^{t+T/2} \left(
        2 \dot{A}(\tau) \omega_0 \cos^2 \left(\omega_0 \tau + \varphi(t)\right) - 2 A(\tau) \omega_0 \dot{\varphi} (\tau) \sin(\omega_0 \tau + \varphi(t)) \cos(\omega_0 \tau + \varphi(t))
    \right) \d \tau.
\end{equation*}
Так как $A$ и $\varphi$ меняются медленно, то можем считать их на масштабе интегрирования $A(\tau) = A(t)$, $\varphi(\tau) = \varphi(t)$. 
Тогда уравнения перепишется в виде
\begin{align*}
    \dot{A} \omega_0 &= \langle f \cos(\omega_0 \tau + \varphi(t))\rangle_\tau, \\
    A \dot{\varphi} \omega_0 &= - \langle f \sin\left(\omega_0 \tau + \varphi(t)\right)\rangle_\tau.
\end{align*}


\textbf{Пример №1}. Рассмотрим осциллятор с затуханием, пусть $f = - 2 \gamma \dot{x}$:
\begin{equation*}
    \dot{A} \omega_0 = - \frac{2 \gamma}{T} \int_{t-T/2}^{t+T/2} 
        A \omega_0 \cos^2 \left(\omega_0 \tau + \varphi\right)
     \d \tau = - \gamma A \omega_0,
     \hspace{0.5cm} \Rightarrow \hspace{0.5cm}
     A(t) = A(0) e^{- \gamma t}.
\end{equation*}
Для фазы:
\begin{equation*}
    A \dot{\varphi} \omega_0 = \frac{2 \gamma}{T} \int_{t-T/2}^{t+T/2} A \omega_0 \sin (\ldots) \cos (\ldots) \d \tau = 0,
    \hspace{0.5cm} \Rightarrow \hspace{0.5cm}
    \varphi = \const + 0(\gamma).
\end{equation*}


\textbf{Пример №2}. Пусть теперь $f = - \varepsilon x^3$, $\varepsilon \ll 1$:
\begin{equation*}
    \dot{A} \omega_0 = - \frac{\varepsilon}{T} \int_{t-T/2}^{t+T/2} A^3 \sin^3 \xi \cos \xi \d \tau = 0,
\end{equation*}
для фазы:
\begin{equation*}
    A \dot{\varphi} \omega_0 = + \frac{\varepsilon}{T} \int_{t-T/2}^{t+T/2} A^3(t) \sin^4 \xi \d \tau = \frac{3 \varepsilon A^3(t)}{8},
\end{equation*}
но так как $A = \const$, находим
\begin{equation*}
    \dot{\varphi} = \frac{3 \varepsilon \dot{A}}{8 \omega_0},
    \hspace{0.5cm} \Rightarrow \hspace{0.5cm}
    x(t) =  A \sin \left(\omega_0 t + \frac{3 \varepsilon A^2}{8 \omega_0} t\right).
\end{equation*}


\textbf{Пример №3}. Рассмотрим генератор Ван-дер-Поля, $f = \varepsilon \dot{x} (1-x^2)$, $\varepsilon \ll 1$:
\begin{equation*}
    \dot{A} \omega_0 = \frac{\varepsilon}{T} \int_{t+T/2}^{t-T/2} A \omega_0 \cos(\xi) (1- A^2 \sin^2 \xi) \cos \xi \d \tau = 
    \frac{\varepsilon A \omega_0}{2} - \frac{\varepsilon A^3 \omega_0}{8} = \frac{\varepsilon A \omega_0}{2} \left(1 - \frac{A^2}{4}\right).
\end{equation*}
Теперь уравнение на фазу:
\begin{equation*}
    A \dot{\varphi} \omega_0 = - \frac{\varepsilon}{T} \int_{t-T/2}^{t+T/2} A \omega_0 \sin \xi \cos \xi (1 - A^2 \sin^2 \xi) \d \tau = 0,
    \hspace{0.5cm} \Rightarrow \hspace{0.5cm}
    \varphi(t) = \const.
\end{equation*}
Найдём $A$, решая уравнение с разделяющимися переменными:
\begin{equation*}
    \frac{\dot{A}}{A\left(1 - \frac{A^2}{4}\right)} = \frac{\varepsilon}{2},
    \hspace{0.5cm} \overset{A^2 = \alpha}{=} \hspace{0.5cm}
    \frac{\alpha}{4-\alpha} = C e^{\varepsilon t},
    \hspace{0.5cm} \Rightarrow \hspace{0.5cm}
    A = \frac{2 C ^{\varepsilon t/2}}{\sqrt{1 + C^2 e^{\varepsilon t}}},
\end{equation*}
где $A \to 2$ при $t \to \infty$ -- предельный цикл.

\textbf{Пример №4}. Рассмотрим параметрический резонанс:
\begin{equation*}
    \ddot{x} + \omega_0^2 \left(1 + h \cos\left(2(\omega_0 + \delta \omega) t\right)\right) x(t) = 0,
\end{equation*}
что также гордо именуется уравнением Матье. Это аналогично наличию $f = - h \cos(2 (\omega_0 +\delta \omega) t) x$.  Введем параметр $\theta = \omega \omega t - \varphi(t)$, тогда
\begin{equation*}
    \dot{A} \omega_0 = - \frac{h}{T} \int_{t-T/2}^{t+T/2} A \sin \xi \cos \xi \cos(2 \xi + 2 \theta) \d \tau = - \frac{h \omega_0^2}{2T} A(t) \int_{t-/2}^{t+T/2} \sin(2 \xi) \left(
        0 - \sin(2 \xi) \sin(2 \theta)
    \right) \d \tau = \frac{\omega_0^2 h}{4}A \sin(2 \theta).
\end{equation*}
Итого, окончательное уравнение
\begin{equation*}
    \dot{A} = \frac{\omega_0 h}{4} A \sin(2 \theta).
\end{equation*}
Для фазы же
\begin{equation*}
    A \dot{\varphi} \omega_0  = - \frac{h \omega_0^2}{T} \int_{t-T/2}^{t+T/2} A \sin^2 \xi \left(
        \cos 2 \xi \cos 2 \theta - 0
    \right) \d \tau = \frac{h \omega_0^2}{4} A \cos 2 \theta,
    \hspace{0.5cm} \Rightarrow \hspace{0.5cm}
    \dot{\varphi} = \frac{h \omega_0}{4} \cos 2 \theta.
\end{equation*}
Но лучше решать уравнение на $\dot{\varphi} = \delta \omega - \dot{\theta}$:
\begin{equation*}
    \dot{\theta} = \delta \omega - \frac{h \omega_0}{4} \cos 2 \theta,
    \hspace{0.5cm} \Rightarrow \hspace{0.5cm}
    \bigg/ |\delta \omega| < \bigg| \frac{h \omega_0}{4} \bigg| \bigg/
    \hspace{5 mm} 
    \exists \theta_0 \colon  \theta(t) = \theta_0 = \const,
\end{equation*}
а значит
\begin{equation*}
    A(t) = A(0) \exp\left(
        \frac{\omega_0 h \sin 2 \theta_0}{4} t
    \right).
\end{equation*}
Кстати, вроде $A^2 \dot{\theta}$ -- первый интеграл системы.


% последние две задачи можно не решать к следующему разу.  
% \subsection*{Т11}

\textbf{Матричный элемент}.
Найдём матричный элемент опратора эволюции для свободной частицы
\begin{equation*}
    Z[0] = \bk{q_N}[U(t'',t')]{q_0} = \int \mathcal D q\ \mathcal D p\
    e^{\frac{i}{\hbar}S} = \lim_{N\to \infty}
    \int \frac{\d p_N}{2\pi\hbar} \prod_{k=1}^{N-1}
    \frac{d q_k\ d p_k}{2 \pi \hbar} \exp\left(
        \frac{i}{\hbar} \sum_{k=1}^{N} \left(
            p_k \dot{q}_k \d t - \frac{p_k^2}{2m}\d t
        \right)
    \right).
\end{equation*}
Перепишем аргумент экспоненты в виде
\begin{equation*}
    \sum_{k=1}^{N} p_k \dot{q}_k \d t - \frac{p_k^2}{2m}\d t =
    \sum_{k=1}^{N-1} q_k (p_k-p_{k+1}) + q_N p_N - q_0 p_1 - \sum_{k=1}^{N} \frac{p_k^2}{2m}\d t
\end{equation*}
Вспоминая, что
\begin{equation*}
    \int_{\mathbb{R}} dt\ \delta(t) e^{i \omega t} = 1,
    \hspace{0.5cm} \Rightarrow \hspace{0.5cm}
    \int_{\mathbb{R}} \frac{\d \omega}{2\pi}e^{-i \omega t} = \delta(t),
\end{equation*}
можем проинтегрировать по всем координатам и получить
\begin{equation*}
    \int \exp\left(\frac{i}{\hbar} q_k (p_k - p_{k+1})\right) = 2 \pi \hbar \, \delta(p_k - p_{k+1}),
    \hspace{5 mm} 
    k = 1,\ldots, N-1.
\end{equation*}
Теперь интегрирование по импульсу тривиально:
\begin{equation*}
    Z[0] = \lim_{N \to \infty} \int \frac{d p_N}{2 \pi \hbar} \exp\bigg(
        \frac{i}{\hbar} p_N (q_N-q_0) - \frac{p_N^2}{2m} \underbrace{ N \d t}_{t''-t'}
    \bigg) = \sqrt{\frac{-i m}{2 \pi \hbar (t'' - t')}} \exp\left(
        \frac{i}{\hbar} \frac{m}{2} \frac{(q'' - q')^2}{t''-t'}
    \right),
\end{equation*}
где мы воспользовались 
\begin{equation*}
    \int_{\mathbb{R}} e^{-a x^2 + b x} \d x = \sqrt{\frac{\pi}{a}} e^{b^2/4a},
    \hspace{5 mm} 
    \int_{\mathbb{R}} e^{\pm i x^2} \d x = e^{\pm i \pi/4} \sqrt{\pi}.
\end{equation*}


\textbf{Уравнение Шрёдингера}. Убедимся, что $Z[0]=\bk{q}[U(t,t')]{q'}$ удовлетворяет уравнению Шредингера
\begin{equation*}
    i \hbar \partial_t Z = - \frac{\hbar^2}{2m} \partial_q^2 Z.
\end{equation*}
Введем для удобства
\begin{equation*}
    \sqrt{\frac{-i m}{2 \pi \hbar (t'' - t')}} \exp\left(
        \frac{i}{\hbar} \frac{m}{2} \frac{(q'' - q')^2}{t''-t'}
    \right) \overset{\mathrm{def}}{=}  \alpha e^{\beta}.
\end{equation*}
Тогда
\begin{align*}
    \partial_t Z = \alpha \left(- \frac{1}{2}\frac{1}{t-t'} - \frac{i m}{2 \hbar} \frac{(q-q')^2}{(t-t')^2} \right) e^\beta, 
    \hspace{5 mm} 
    \partial_q Z = \alpha \frac{i m}{\hbar} \frac{q-q'}{t-t'} e^\beta, 
    \hspace{5 mm} 
    \partial_q^2 Z = \frac{i \alpha m}{\hbar (t-t')} \left(
        1 + \frac{i m}{\hbar} \frac{(q-q')^2}{t-t'}
    \right)e^\beta,
\end{align*}
что и требовалось доказать.



\subsection*{Т12}

Как обычно
\begin{equation*}
    Z[j] = \mathcal N \int \mathcal D q\ \exp\left(
        \frac{i}{\hbar}S + \frac{i}{\hbar} \int_{t'}^{t''} j(t) q(t) \d t
    \right).
\end{equation*}
Запишем в виде
\begin{equation*}
    q(t) = \tilde{q}(t) + G^{(1)} (t),
    \hspace{10 mm}
    \hat{\Gamma} G^{(1)} (t) = 0,
    \hspace{5 mm} 
    G^{(1)} (t') = q',
    \hspace{5 mm} 
    G^{(1)}(t'') = q''.
\end{equation*}
Тогда верно, что
\begin{equation*}
    \int q \hat{\Gamma} q \d t = \int \tilde{q} \hat{\Gamma} \tilde{q} \d t,
    \hspace{0.5cm} \Rightarrow \hspace{0.5cm}
    Z = \mathcal N e^{\frac{i}{\hbar}\int_{t'}^{t''} j(t) G^{(1)} (t) \d t} \int \mathcal D \tilde{q} \exp\left(
        \frac{i}{2\hbar} \int \tilde{q} \hat{\Gamma} \tilde{q} \d t + \frac{i}{\hbar} \int j \tilde{q} \d t
    \right).
\end{equation*}
Можем записать, что
\begin{equation*}
    \tilde{q}(t) = \bar{q}(t) - \hat{\Gamma}^{-1} j(t),
    \hspace{10 mm} 
    \hat{\Gamma}^{-1} j(t_1) = - \int G^{(2)} (t_1-t_2) j(t_2) \d t_2.
\end{equation*}
Подставляя $\bar{q}$, находим
\begin{equation*}
    \frac{1}{2} \int \tilde{q} \hat{\Gamma} \tilde{q} \d t + \int j \tilde{q} \d t = 
    \frac{1}{2} \int \left(
        \bar{q} \hat{\Gamma} \bar{q} + \hat{\Gamma}^{-1} j \hat{\Gamma} \hat{\Gamma}^{-1} j - 2 j \Gamma^{-1} j
    \right),
\end{equation*}
а значит
\begin{equation*}
    Z = \mathcal N \exp\left(
        \frac{i}{\hbar} \int_{t'}^{t''} j G^{(1)} \d t + \frac{1}{\hbar} 
        \int G^{(2)} (t_1 - t_2) j(t_1) j(t_2) \d t_1 \d t_2
    \right) 
    \int \mathcal D \bar{q} e^{\frac{i}{2\hbar} \int \bar{q} \hat{\Gamma} \bar{q} \d t},
\end{equation*}
где $\int \bar{q} \hat{\Gamma} \bar{q} \d t = e^{\frac{i}{/h} G_0}$, а значит
\begin{equation*}
    Z = \mathcal N e^{\frac{i}{\hbar} G[j]},
    \hspace{5 mm} 
    G[j] = G_0 + \int G^{(1)} j(t) \d t + \frac{1}{2!} \int G^{(2)} (t_1-t_2) j(t_1) j(t_2) \d t_1 \d t_2.
\end{equation*}



% \newpage
% \section*{Второе задание}
% \subsection*{Уравнение Паули для позитрона}

Известно, что для электрона можем записать уравнение Паули:
\begin{equation*}
	\psi(t, \vc{r}) = e^{
		\frac{i}{\hbar} m c^2 t
	} \begin{pmatrix}
		\varphi(t, \vc{r}) \\ x(t, \vc{r})
	\end{pmatrix}.
\end{equation*}
Для позитрона можем воспользоваться анзацем
\begin{equation*}
	\psi(t, \vc{r}) = e^{
		-\frac{i}{\hbar} m c^2 t
	} \begin{pmatrix}
		\varphi(t, \vc{r}) \\ x(t, \vc{r})
	\end{pmatrix}.
\end{equation*}
Подставляем это в уравнение Дирака
\begin{equation*}
	\mathcal{\tilde{P}} = \mathcal P_\mu \gamma^\mu,
	\hspace{10 mm} 
	(\mathcal{\tilde{P}} - mc) \psi(t, \vc{r}),
	\hspace{10 mm} 
	\gamma^0 = \begin{pmatrix}
		\mathbbm{1} & 0 \\ 0 & - \mathbbm{1}
	\end{pmatrix},
	\hspace{10 mm} 
	\vc{\gamma} = \begin{pmatrix}
	    0 & -\vc{\sigma}  \\
	    \vc{\sigma} & 0  \\
	\end{pmatrix}.
\end{equation*}
Где помним про $\mathcal P_\mu = i \hbar \partial_\mu - \frac{e}{c} A_\mu$.
Помним про $\partial_{x_0} = \frac{1}{c}\partial_t$.

Введем оператор $\hat{\varepsilon} = i \hbar \partial_t$, $A_0 = \Phi$ и запишем получившеюся систему вида
\begin{align*}
	(\hat{\varepsilon} - e \Phi - 2 mc^2) \varphi - c (\vc{\sigma} \cdot \vc{\mathcal P}) \chi = 0, \\
	(\vc{\sigma} \cdot \vc{\mathcal P}) \varphi + (e \Phi - \hat{\varepsilon}) \chi = 0.
\end{align*}
Решая, находим
\begin{equation*}
	\varphi = - \frac{1}{1 -\frac{\hat{\varepsilon} - e \Phi}{2 m c^2}}
	\frac{(\vc{\sigma} \cdot \vc{\mathcal P})}{2 m c} \chi = -
	(1 + \hat{W}) \frac{(\vc{\sigma} \cdot \vc{\mathcal P})}{2 m c} \chi.
\end{equation*} 
Теперь, кстати $\varphi$ -- малая компонента, $\chi$ -- большая, при рассмотрении $\frac{v}{c} \ll 1$. 

Вводя $\bar{\psi} = \psi^+ \gamma^0$, получаем
\begin{equation*}
	c \bar{\psi} \left(\mathcal{\tilde{P}} - m c\right) \psi = 
	\chi^+ \left(
		\hat{\varepsilon} - e \Phi + \frac{1}{2m} (\vc{\sigma} \cdot \vc{\mathcal P}) 
 (1 + \hat{W}) (\vc{\sigma} \cdot \vc{\mathcal P})
 \right).
\end{equation*}
Считая $\frac{v}{c} \ll 1$, верно, что
\begin{equation*}
	1 + \hat{W} \approx	1 + \frac{\hat{\varepsilon}-e \Phi}{2 mc^2}.
\end{equation*}


Умея сворачивать $\vc{\sigma}$, находим
\begin{equation*}
	(\vc{\sigma} \cdot \vc{\mathcal P})^2 = \vc{\mathcal P}^2 + i \vc{\sigma} \cdot \left[\vc{\mathcal P} \times  \vc{\mathcal P}\right].
\end{equation*}
Распишем покомпоненто
\begin{equation*}
	\left[\vc{\mathcal P} \times  \vc{\mathcal P}\right]_\alpha = \varepsilon_{\alpha \beta \gamma} \left(
		- i \hbar \partial_\beta - \tfrac{e}{c} A_\beta
	\right)\left(
		- i \hbar \partial_\gamma - \tfrac{e}{c} A_\gamma
	\right) = i \hbar \tfrac{e}{c} \varepsilon_{\alpha \beta \gamma} \left(
		\partial_\beta A_\gamma
	\right),
\end{equation*}
откуда находим
\begin{equation*}
	\left[\vc{\mathcal P} \times  \vc{\mathcal P}\right]_\alpha = i \hbar \tfrac{e}{c} \vc{\mathcal H},
	\hspace{10 mm} 
	(\rot \vc{A})_\alpha = \mathcal H_\alpha.
\end{equation*}

Таким образом можем записать действие:
\begin{equation*}
	S_{NR} = \int d^4 x\ 
	\chi^+ \left(
		\hat{\varepsilon} - e \Phi + \frac{\vc{\mathcal P}^2}{2m} - \frac{e \hbar}{2 m c} \left(\vc{\sigma} \cdot \vc{\mathcal H}\right)
	\right).
\end{equation*}
Варьируя действие, находим
\begin{equation*}
	i \hbar \partial_t \chi = \left(
		e \Phi - \frac{\vc{\mathcal P}^2}{2m} + \frac{e \hbar}{2 m c} \left(\vc{\sigma} \cdot \vc{\mathcal H}\right)
	\right) \chi,
\end{equation*}
которое по идее есть Шрёдингер, вида $i \hbar \partial_t \chi = \hat{H} \chi$. 

Нужно вспомнить, что у нас должа быть правильная спинорная метрика (переходим к другой киральности):
\begin{equation*}
	\chi_{NR} = i \sigma_2 \chi^*.
\end{equation*}
Так что подставляя это наверх, находим
\begin{equation*}
	- i \hbar \partial_t \chi^* = \left(\ldots\right) \chi^*.
\end{equation*}
Воспользуемся свойством $\sigma_2 \vc{\sigma}^* \sigma_2 = - \vc{\sigma}$. Домножая последнее уравнение на $i \sigma_2$, находим
\begin{equation*}
	i \hbar \partial_t \xi_{NR} = \left(
		\frac{\vc{\mathcal P}^2}{2m} - e \Phi + \frac{e \hbar}{2 m c} g 
		\left(
			\vc{s} \cdot \vc{\mathcal H}
		\right)
	\right) \chi_{NR},
	\hspace{20 mm} 
	g=2, \hspace{5 mm} 
	\vc{s} = \frac{1}{2} \vc{\sigma}.
\end{equation*}







\subsection*{Правила Хунда}

\textbf{Определитель Слетера}. Рассмотрим $N$ электронов со спином $s=\frac{1}{2}$, и квантовыми числами $\{k_1,\, \ldots,\, k_N\}$ (обычно $\{n,\, l,\, m\}$). Далее описываем систему в виде $x = \{\vc{r},\, m_s\}$.

Рассмотрим случай отсутсвия спин-орбитального взаимодействия, то есть спин коммутирует с гамильтонианом, тогда будет иметь место факторизация $\ket{k} \otimes \ket{m_s}$.

Вспоминаем связь спина со статистикой, тогда
\begin{equation*}
	\Phi_{k_1,\, \ldots,\, k_N} (x_1,\, \ldots,\, x_N) = 
	\frac{1}{\sqrt{N!}} \det \begin{pmatrix}
		\psi_{k_1} (x_1) & \ldots & \psi_{k_1} (x_N) \\
		\vdots & \ddots & \vdots \\
		\psi_{k_N} (x_1) & \ldots & \psi_{k_N} (x_N)
	\end{pmatrix}.
\end{equation*}

\textbf{Правила Хунда: феноменология}. 
Располагаем нерелятивисткие термы ${}^{2s+1}L$ по I правилу Хунда. 
Во-первых считаем, что $E\to \min$ при $s \to \max$. 
Далее меняем $E \to \min$ при $L \to \max$.


\textbf{Правила Хунда: релятивистские поправки I}. Спин-орбитальное взаимодействие вносит в незаполненные оболочки ($2^{2(2l +1)}$) 
\begin{equation*}
	V_{sl} = \sum_f \frac{e \hbar^2}{2 m^2 c^2} \frac{V'(r_f)}{r_f} \left(
		\vc{s}_f \cdot \vc{l}_f
	\right).
\end{equation*}
Рассмотрим $n_f \leq 2 l +1$, тогда будем считать $\vc{s}_f \approx	\frac{1}{n_f} \vc{S}$, $r_f \approx \langle r\rangle$, тогда
\begin{equation*}
	\langle V_{sl}\rangle_{LS} \approx	
	\sum_{f=1}^{n_f} \frac{e \hbar^2}{2 m^2 c^2} \frac{\langle V'\rangle}{\langle r\rangle} \frac{1}{n_f} \left(\vc{S} \cdot \vc{l}_f\right) = 
	\underbrace{\frac{e \hbar^2}{2 m^2 c^2} \frac{\langle V'\rangle}{\langle r\rangle} \frac{1}{n_f}}_{A_{LS} \sim \alpha^4 m c^2 > 0} \left(\vc{S} \cdot \vc{L}\right).
\end{equation*}
Считая скалярное произведение, находим
\begin{equation*}
	\langle V_{sl}\rangle_{LS} = \frac{1}{2} A_{LS} \left(
		J(J+1) - L(L+1) - S (S+1)
	\right).
\end{equation*}
Так получаем правило интервалов Ланде
\begin{equation*}
	\Delta \langle V_{sl}\rangle_{LS} \approx A_{LS} J.
\end{equation*}




\textbf{Правила Хунда: релятивистские поправки II}. Теперь рассмотрим $n_f > 2 l + 1$, но заполненные оболочки теперь существуют вместе с дырками:
\begin{equation*}
	n_h = 2(2 l + 1) -n f  < 2 l +1.
\end{equation*}
Понятно, что
\begin{equation*}
	\sum_{f=1}^{n_f} \vc{s}_f + \sum_{n=1}^{n_h} \vc{s}_h = 0.
\end{equation*}
Можем рассматривать 
\begin{equation*}
	\vc{S} = \sum_{f=1}^{n_f}  \vc{s}_f  + \sum_{f=1}^{n_h} \tilde{\vc{s}} + \sum_{h=1}^{n_h} \vc{s}_h	= \vc{S}_h.
\end{equation*}

Теперь рассматриваем орбитальный момент
\begin{equation*}
	\vc{L} = \sum_{f=1}^{n_f} \vc{l}_f,
	\hspace{5 mm} 
	\vc{L} = \sum_{f=1}^{n_h} \tilde{\vc{l}}_f	= \sum_{h=1}^{n_h} \vc{l}_h = \vc{L}_h,
\end{equation*}
где теперь выполняется
\begin{equation*}
	\tilde{\vc{L}} + \vc{L}_h = 0,
	\hspace{0.5cm} \Rightarrow \hspace{0.5cm}	
	\langle V_{sl}\rangle_{LS} \approx \frac{1}{2} A_{SL}^h \left(\vc{S}_h cdot \vc{L}_h\right) = - \frac{1}{2} A_{SL} \left(\vc{S} \cdot \vc{L}\right).
\end{equation*}
Таким образом при фиксированных $S, L$ энергия $E\to \min$:
$n_f \leq 2l+1$ при $J = |L-S|$;
$n_f > 2l+1$ при $J = L+S$.


% \subsection*{Т26}

% Для кремния $S = 1$ и $L=0,1,2$ или $S=0$ и $L=0, 2$. основное состояние будет $3 s^2 3 p^2$. 





% \newpage

% релятивизм
% Т14
\subsection*{Т15}

По определению
\begin{equation*}
	W^\mu = - \frac{1}{2} \varepsilon^{\mu \nu \lambda \rho} p_\nu S_{\lambda \rho},
	\hspace{10 mm} 
	S_{ik} = \hbar \varepsilon_{ikl} s^l.
\end{equation*}
Тогда подставляя $\mu = 0$, находим
\begin{equation*}
	W^0 = - \frac{1}{2} \varepsilon^{0 i j k} p_i \hbar \varepsilon_{jkn} s^n = - \hbar p_i s^i = \hbar \left(\vc{p} \cdot \vc{s}\right).
\end{equation*}
Теперь, с учетом $S_{0i} = i \sign(\mathfrak{s}) s^i \hbar$, находим
\begin{align*}
	W^i &= - \tfrac{1}{2}\varepsilon^{i0jk} p_0 S_{jk} - \varepsilon^{i j 0 k} p_j S_{0k} = \tfrac{1}{2} \varepsilon^{0 ijk} p_0 \hbar \varepsilon_{jkn} s^n - i \sign(\mathfrak{s}) \varepsilon^{o i jk} p_j s_k \hbar = \\ &= \hbar \left(
		p_0 s^i - i \sign(\mathfrak s)  \left[\vc{p} \times  \vc{s}\right]^i
	\right).
\end{align*} %S_I
% Т16
% Т17


% 
% Т20
% Т21
\subsection*{Т22}

\textbf{Уровни Ландау}. 
Для частицы в постоянном магнитном поле гамильтониан запишется в виде
\begin{equation*}
	\hat{H} = \frac{\hat{\vc{\mathcal{P}}}^2}{2m}-
	\frac{\mu}{s} \hat{s}_z \H
	+\cancel{e A_0} ,
	\hspace{10 mm} 
	\hat{\mathcal{P}}^\alpha = -i  \hbar \partial_\alpha - \frac{e}{c} A_\alpha.
\end{equation*}
Удобно зафиксировать калибровку в виде
\begin{equation*}
	A_x = - \H y, \hspace{5 mm} A_y = A_z = 0. 
\end{equation*}
Тогда гамильтониан можем записать в виде
\begin{equation*}
	\hat{H} = \frac{1}{2m}\left(
		\hat{p}_x + \frac{e\H}{c}y
	\right)^2 + \frac{\hat{p}^2_y + \hat{p}^2_z}{2m}.
\end{equation*}
Так как $[\hat{s}_z,\, H] = 0$, то может рассмотреть собственные состояния $\hat{s}_z$ и не думать про это:
\begin{equation*}
	\hat{H} \psi = E \psi,
	\hspace{0.5cm} \Rightarrow \hspace{0.5cm}
	\psi = e^{\frac{i}{\hbar}\left(p_x x + p_z z\right)} \chi(y),
\end{equation*}
так как $[\hat{p}_x,\, \hat{H}] = [\hat{p}_z,\, \hat{H}] = 0$. Движение вдоль поля <<не квантуется>>.

Подставляя предполагаемые вид функции в уравнение Шредингера, получаем дифференциальное уравнение на $\chi$
\begin{equation*}
	\chi'' + \frac{2m}{\hbar^2} \bigg(
		\underbrace{\left(E + \tfrac{\mu \sigma}{s}\H - \tfrac{1}{2m} p_z^2\right)}_{\sub{E}{osc} = \hbar \omega_\H (n +1/2)} - \tfrac{m}{2} \omega_\H^2 (y-y_0)^2
	\bigg) \chi = 0,
\end{equation*}
где введены
\begin{equation*}
	y_0 \overset{\mathrm{def}}{=} - \frac{c p_x}{e \H},
	\hspace{10 mm} 
	\omega_\H \overset{\mathrm{def}}{=}  \frac{|e| \H}{m c}.
\end{equation*}
Таким образом для уровней энергии частицы находим
\begin{equation*}
	E = \left(n + \tfrac{1}{2}\right) \hbar \omega_\H + \frac{p_z^2}{2m} - \frac{\mu \sigma}{s}\H,
\end{equation*}
что и называют уровнями Ландау. Подставляя $\mu/s = -|e| \hbar / mc$, можем написать уровни в виде
\begin{equation*}
	E = \left(n + \tfrac{1}{2} + \sigma\right) \hbar \omega_\H + \frac{p_z^2}{2m}.
\end{equation*}
Собственные функции можем написать в терминах полиномов Эрмита:
\begin{equation*}
	\chi_n (y) = \frac{1}{\sqrt{a_H \sqrt{\pi} 2^n n!}} \exp\left(
		- \frac{(y-y_0)^2}{2 a_H^2}
	\right) H_n \left(\frac{y-y_0}{a_H}\right),
	\hspace{5 mm} 
	a_H = \sqrt{\frac{\hbar}{m \omega_\H}}.
\end{equation*}


\textbf{Кратность вырождения уровней}. Пусть движение в плоскости $xy$ ограничено большой, но конечной площадью $S = L_x L_y$. Тогда число различных дискретных значений $p_x$ в интервале $\Delta p_x$  можно найти в виде
\begin{equation*}
	N_{p_x}(\Delta p_x) = \frac{L_x}{2 \pi \hbar} \Delta p_x.
\end{equation*}
Считая $0 < y_0 < L_y$ можем найти связь $\Delta p_x = e H l_y / c$, а значит число состояний для заданных $n$ и $p_z$:
\begin{equation*}
	N_{n,p_z} = \frac{e \H S}{2\pi \hbar c}.
\end{equation*}
Добавляя ограничение по $z$ в размере $L_z$, получаем число состояний в интервале $\Delta p_z$:
\begin{equation*}
	N_n = \frac{e \H V}{4 \pi^2 \hbar^2 x} \Delta p_z.
\end{equation*} %ЛЛ3, S113
\subsection*{Т23}

Найдём уровни энергии и волновые функции стационарных состояний двух невзаимодействующих тождественных частиц в потенциальном ящике
\begin{equation*}
	V(x) = \left\{\begin{aligned}
	    &0,\, &0 < x < a,\\
	    &\infty,\, &x<0,\ x > a.
	\end{aligned}\right.
\end{equation*}
Для одной частицы знаем, что
\begin{equation*}
	\psi_k (x) \sim \sin(k_n x), \hspace{5 mm} 
	k_n = \frac{\pi n}{a},
\end{equation*}
с характерной энергией $E_0 = \frac{\pi^2 \hbar^2}{2 m a^2}$.

\textbf{Фермионы}. Рассмотрим $s=\frac{1}{2}$, тогда суммарный спин $S = \{0,\, 1\}$. Полная волновая функция антисимметрична:
\begin{equation}
	\Psi_{n_1 n_2} = \psi_{\pm} \times  \chi_{\mp}(2S=1 \pm 1),
	\label{sympsi}
\end{equation}
где $\pm$ соответсвует симметричной и антисимметричной функции. 

Энергию при $n_1 \neq n_2$ можем найти в виде
\begin{equation*}
	E_{n_1 n_2} = E_0 \left(n_1^2 + n_2^2\right).
\end{equation*}
При $n_1 = n_2$ невозможно состояние с $S=1$, поэтому энергия запищется в виде
\begin{equation*}
	E_{nn} = E_0 n^2.
\end{equation*}
Для поиска энергии основного состояния $N$-частиц, задача сводится к сумме квадратов
\begin{equation*}
	\sum_{n=1}^{m} n^2 = \frac{1}{6} m (m+1) (2 m+1),
	\hspace{0.5cm} \Rightarrow \hspace{0.5cm}
	E_N = \frac{E_0}{12} (N+1)(N^2 + 2 N + 3 \cdot  (N\ \text{mod}\, 2) ),
\end{equation*}

С учетом \eqref{sympsi}, волновую функцию можем записать в виде
\begin{equation*}
	\psi_F^S(x_1,\, x_2) = \frac{1}{\sqrt{2}}\left(
		\psi_{n_1}(x_1) \psi_{n_2} (x_2) + (-1)^{S} \psi_{n_1} (x_2) \psi_{n_2} (x_1)
	\right),
\end{equation*}
где $\psi(x_1,\, x_2)$ обращается в $\equiv 0$ при $n_1 = n_2$ и $S=1$.

\textbf{Бозоны}.  Энергия представима в виде
\begin{equation*}
	E_{n_1 n_2} = E_0 (n_1^2 + n^2).
\end{equation*}
Энергия основного состояния для $N$ бозонов не зависит от спина и равна
\begin{equation*}
	E_N = E_0 \ N.
\end{equation*}

Для частиц с нулевым спином полная волновая функция может быть только симметричной, значит представима в виде $\psi_F^0$.  Для частиц с единчиным спином $\Psi$ симметрична, поэтому
\begin{equation*}
	\Psi_{n_1 n_2} = \psi_{\pm} \times  \chi_{\pm}(S).
	\label{sympsi2}
\end{equation*}
а значит $S=1$ соотвествует $\psi_+$ и $S=\{0,\, 2\}$ соответсвует $\psi_-$. 




 %(2.10.6.) 
% Т24
% Т25
\subsection*{Т26}



\textbf{Кремний}. По правилам Хунда конфигурация незаполненой части $\chain{2}{p}{2}$ будет вида: \spin{2p={;,up,up}}, а значит можем найти $J = |L-S| = 0$. 
\begin{equation*}
\text{Si}:\ \chain{1}{s}{2} \chain{2}{s}{2} \chain{2}{p}{6} \chain{3}{s}{2} \chain{3}{p}{2},
\hspace{10 mm} 
\text{основное состояние}:\ 
\term{3}{P}{0}.
\end{equation*}

\textbf{Сера}. Незполненной явлеется оболочка $\chain{2}{p}{4}$, для которой (S: $\chain{1}{s}{2} \chain{2}{s}{2} \chain{2}{p}{6} \chain{3}{s}{2} \chain{3}{p}{4}$) находим основное состояние $\term{3}{P}{0}$ в силу конфигурации \spin{2p={;up, up, pair}}.

\textbf{Все термы}. Найдём все термы для $\chain{}{p}{2}$, $l=1$, тогда $L = \{0,\, 1,\, 2\}$ и $S = \{0,\, 1\}$. 
Для $S=1$ и $L=1$ возможны конфигурации $\term{3}{P}{0, 1, 2}$. Для $S = 0$ и $L=\{0,2\}$ получим $\term{1}{S}{0},\, \term{1}{D}{2}$, аналогичные рассуждения будут верны для $\chain{}{p}{4}$. 


\textbf{Фосфор}. 
Для фосфора $\chain{}{p}{3}$ основным состоянием будет $\term{4}{S}{3/2}$. Состоянию с $M_s = \frac{1}{2}$ соответсвует конфигурация \spin{2p={;,up,pair}}, и $\term{2}{D}{3/2, 5/2}$. Для $M_L = 1$ возможны конфигурации \spin{2p={;,pair,up}} и \spin{2p={;up,,pair}} с обозначениями $\term{2}{P}{1/2,\, 3/2}$. Наконец, для $M_L = 0$ возможны конфигурации \spin{2p={;up,up,down}}, \spin{2p={;up,down,up}}, \spin{2p={;down,up,up}}, не приводящие к новым независимым состояниям.


\textbf{Ванадий}. V: $\ldots \chain{3}{d}{3}$ и конфигурация \spin{1s={;}}\spin{1s={;}}\spin{2p={;up,up,up}} с обозначением $\term{4}{F}{3/2}$. 

\textbf{Кобальт}. Co: $\ldots \chain{3}{d}{7}$  и конфигурация $\term{4}{F}{9/2}$.

\textbf{Церий}. Ce: $\ldots \chain{6}{s}{2} \chain{5}{d}{4}$  в конфигурации $\term{3}{H}{4}$ с $\sub{S}{max} = 1$ и $\sub{L}{max} = 5$, хотя на самом деле $\term{1}{G}{4}$ и конфигурация $\chain{4}{f}{1} \chain{5}{d}{1} \chain{6}{s}{2}$, явлется исключением из правил Хунда.

 %(2.11.18)
% Т27
% Т28
% Т29

% рассеяние
\subsection*{Т29}

Рассмотрим в борновском приближении два короткодейтсвующих потенциала. Амплитуда рассения может быть найдена в виде
\begin{equation*}
	f(\theta) = - \frac{m}{2 \pi \hbar^2} \int V(r) e^{-i \smallvc{q} \smallvc{r}} \d^3 \vc{r} = - \frac{2m}{\hbar^2 q} \int_{0}^{\infty} V(r) \sin(qr) r \d r,
	\hspace{10 mm} 
	\vc{q} \overset{\mathrm{def}}{=} \vc{k}' - \vc{k},
	\hspace{5 mm} 
	q = 2 k \sin(\theta/2).
\end{equation*}
Полное сечение рассения находим интегрируя амплитуду рассеяния:
\begin{equation*}
	\sigma = \int |f(\theta)|^2 \d \Omega = 2 \pi \int_{0}^{\pi} |f(\theta)|^2 \sin \theta \d \theta.
\end{equation*}
Условие применимости запишется в виде
\begin{equation*}
	\frac{m}{2\pi \hbar^2} \bigg| \int \frac{e^{i k r}}{r} V(r) e^{i k z} \d^3 r \bigg| \ll 1.
\end{equation*}


\textbf{Потенциал Юкавы}. Подставляя $V(r) = \frac{\alpha}{r} e^{- \kappa r}$, находим
\begin{equation*}
	f = - \frac{2 m \alpha}{\hbar^2 (\kappa^2 + q^2)},
	\hspace{0.5cm} \Rightarrow \hspace{0.5cm}
	\sigma = \left(\frac{m \alpha}{\hbar^2 \kappa}\right)^2 \frac{4\pi}{4 k^2 + \kappa^2},
	\hspace{5 mm} 
	k^2 = \frac{2 mE}{\hbar^2}.
\end{equation*}
Условие применимости для любых энергий: $\alpha m / \kappa \\ll \hbar^2$. Для быстрых частиц можем ослабить условие до $\alpha \ll \hbar \times \hbar k/m$.

\textbf{Прямоугольная яма}. Аналогично вычисляем
\begin{equation*}
	f = \frac{2m V_0 a}{\hbar^2 q^2} \left(\cos qa - \frac{\sin q a}{qa}\right),
	\hspace{0.25cm} \Rightarrow \hspace{0.25cm}
	\sigma = \frac{2\pi}{k^2}\left(
		\frac{m V_0 a^2}{\hbar}
	\right)^2 \left(
		1 - \frac{1}{(2ka)^2} + \frac{\sin 4 ka}{(2 ka)^3} - \frac{\sin^2 2ka}{(2ka)^4}
	\right),
\end{equation*}
с условием применимости $\sqrt{2 m V_0} a \ll \hbar$ и для быстрых частиц $\sqrt{2 m V_0} a \ll  \hbar \sqrt{ka}$.
% \subsection*{Т30}


Для потенциала, вида
\begin{equation*}
	V(r) = \frac{\beta}{r^2},
	\hspace{5 mm} 
	\beta > 0,
\end{equation*}
найдём фазы рассеняи $\delta_l$. 

Запишем уравнение Шредингера для парциальной волны $u_l (r) = r R_l (r)$:
\begin{equation*}
	\left(
		\frac{d^2 }{d r^2} + k^2 - \frac{1}{r^2}\left(
			l(l+1) + \frac{2 m \beta}{\hbar^2}
		\right)
	\right) u_l (r) = 0.
\end{equation*}
Рассмотрим замену $u_l (r) = \sqrt{r} \varphi(r)$
\begin{equation*}
	\varphi'' + \frac{1}{r} \varphi' + \left(
		k^2 - \frac{1}{r^2}\left(
			\left(l +\frac{1}{2}\right)^2 + \frac{2 m \beta}{\hbar^2}
		\right)
	\right)\varphi = 0,
\end{equation*}
решения которого знаем в виде функций Бесселя $J_{\pm \nu} (kr)$, где
\begin{equation*}
	\nu = \sqrt{\left(l + \frac{1}{2}\right)^2 + \frac{2 m \beta}{\hbar^2}}.
\end{equation*}
Требуя $u_l (0) = 0$, находим решение в виде
\begin{equation*}
	u_l (r) = c \sqrt{\frac{\pi k r}{2}} J_\nu (kr).
\end{equation*}
Полезно посмотреть асмиптотику на бесконечности, для которой
\begin{equation*}
	u_l(r) \sim c \sin\left(kr - \frac{\pi \nu}{2} + \frac{\pi}{4}\right) = c \sin\left(kr - \frac{\pi l}{2} + \delta_l\right),
\end{equation*}
откуда находим искомые фазы рассеяния
\begin{equation*}
	\delta_l = - \frac{\pi}{2} \left(
		\sqrt{\left(l + \frac{1}{2}\right)^2 + \frac{2 m \beta}{\hbar^2}} - \left(l + \frac{1}{2}\right)
	\right).
\end{equation*}

\textbf{Предельный случай}. В пределе $2 m \beta / \hbar^2 \ll 1$ получаем
\begin{equation*}
	\delta_l \approx - \frac{\pi}{2} \frac{m \beta}{\hbar^2(l + \tfrac{1}{2})},
\end{equation*}
откуда также получаем $|\delta_l| \ll 1$. 

В таком случае можем просуммировать ряд
\begin{equation*}
	f(\theta) = \frac{1}{2 ik} \sum_{l=0}^{\infty} (2l+1) \left(e^{2 i \delta_l}- 1\right) P_l  (\cos \theta) \approx \frac{1}{k} \sum_{l=0}^{\infty} (2l+1) \delta_l P_l(\cos \theta) \approx	
	-\frac{\pi m \beta}{\hbar^2 k} \sum_{l=0}^{\infty} P_l(\cos \theta).
\end{equation*}
Суммируя полиному Лежанда, находим
\begin{equation*}
	f(\theta) \approx	 - \frac{\pi m \beta}{2 k \hbar^2 \sin(\theta/2)},
\end{equation*}
аналогично тому, что получили бы в борновском приближении.




% \subsection*{Т31}

Найдём сечение рассеяния для $k a \ll 1$, а значит доминирует $s$-рассеяние и $p$-рассеяние. 
Для потенциала
\begin{equation*}
	U(r) = \left\{\begin{aligned}
	    -&U_0, &r \leq a, \\
	    &0,  &r>a.
	\end{aligned}\right.
\end{equation*}
Теперь
\begin{equation*}
	R_{k0} = \frac{1}{r} u(r),
	\hspace{5 mm} 
	u(0) = 0,
	\hspace{10 mm} 
	- \frac{\hbar^2}{2m} u'' + V u = E u.
\end{equation*}
Для $r > a$ $u'' + k^2 u = 0$, тогда
\begin{equation*}
	\sub{u}{II} = A \sin(kr + \delta_0).
\end{equation*}
Для $r\leq a$ 
\begin{equation*}
	u'' + (k^2 + \kappa^2) u = 0,
	\hspace{5 mm} 
	\tilde{k}^2 \overset{\mathrm{def}}{=} k^2 + \kappa^2,
	\hspace{5 mm} 
	U_0 = \frac{\hbar^2 \kappa^2}{2m},
	\hspace{0.5cm} \Rightarrow \hspace{0.5cm}
	\sub{U}{I} = B \sin(\tilde{k} r).
\end{equation*}
Сшиваем на границах:
\begin{equation*}
	\frac{\sub{U}{I}'}{\sub{U}{I}} = \frac{\sub{U}{II}'}{\sub{U}{II}},
	\hspace{0.5cm} \Rightarrow \hspace{0.5cm}
	\tg(ka + \delta_0) = \frac{k}{\tilde{k}} \tg(\tilde{k} a),
	\hspace{0.5cm} \Rightarrow \hspace{0.5cm}
	\delta_0 = - ka + \arctg\left(
		\tfrac{k}{\tilde{k}} \tg(\tilde{k} a)
	\right).
\end{equation*}
Рассмотрим случай $\frac{k}{\tilde{k}} \tg(\tilde{k} a) \ll ka \ll 1$, а тогда $\delta_0 \approx - ka$, а значит $f_0 = \frac{1}{k} e^{i \delta_0} \sin \delta_0 \approx - a$.


Другой случай $\tg(\tilde{k} a) \to \infty$. Тогда $\delta_0 \approx \arctg(\infty) = \frac{\pi}{2}$, что ещё называют резонансным рассеянием, так как $\sin \delta_0 = 1$.

Наконец, посмотрим на $\frac{\tg(\tilde{k} a)}{\tilde{k} a} \approx	1$, тогда $\delta_0 \approx	0$, и получается $\tilde{k} a \ll 1$ и $f_0 \to 0$ -- эффект Рамзаура.

При барьере $\tilde{k} \to i \tilde{k}$, получим уравнения
\begin{equation*}
	\delta_0 = - ka + \arctg\left(
		\frac{ka}{\tilde{k} a} \th(\tilde{k} a)
	\right).
\end{equation*}



% 

\subsection*{Т32}


\textbf{II}.
Для случая быстрых частиц $ka \gg 1$ рассмотрим <<черную дыру>>, тогда для $l < ka$ получаем\footnote{
	Вообще верно, что $\hbar k \cdot b = \hbar l$, где $b$ -- прицельный параметр.	
}  $S_l = 0$ и для $l > ka$  будет $S_l = 1$.

Записываем оптическую теорему
\begin{equation*}
	\sub{\sigma}{tot} = \frac{4\pi}{k} \sum_l \Im f_l.
\end{equation*}
Сохраняются $f_l$ сохраняются
\begin{equation*}
	f_l = \frac{2l+1}{2 ik } (S_l -1).
\end{equation*}
Также помним, что нужно суммировать до $l = ka$, при больших $l$ сечение обращается в $0$. Итого получаем
\begin{equation*}
	\sub{\sigma}{tot} = \frac{4 \pi}{k} \sum_{l=0}^{ka} (2l+1) \frac{1}{2k} = \frac{2\pi}{k^2} (ka+1)^2 = 2 \pi a^2.
\end{equation*}


\textbf{I}. Рассмотрим непроницаемую сферу
\begin{equation*}
	U(r) = \left\{\begin{aligned}
	    &0, &r \leq a, \\
	    &\infty, &r > a.
	\end{aligned}\right.
\end{equation*}
Вспоминаем
\begin{equation*}
	R_{kl} \approx	\frac{c_l}{r} \sin\left(kr - \tfrac{\pi l}{2} + \delta_l\right).
\end{equation*}
Верно, что $R_{kl} |_{r=a} = 0$:
\begin{equation*}
	ka - \frac{\pi l}{2} + \delta_l = \pi n \to 0,
	\hspace{0.5cm} \Rightarrow \hspace{0.5cm}
	\delta_l = \frac{\pi l}{2} - ka.
\end{equation*}
Находим сечение рассеяния:
\begin{equation*}
	f_l = \frac{2l+1}{k} e^{i \delta_l} \sin \delta_l.
\end{equation*}
Пользуемся оптической теоремой, находим
\begin{equation*}
	\Im f_l = \frac{2l+1}{2k} - \frac{2l+1}{2k}\cos\left(
		\pi l - 2 ka
	\right).
\end{equation*}
Вклад от первого слагаемого дает половину $\sub{\sigma}{geom} = 4 \pi a^2$. Для расчёта второго слагаемого рассмотрим четные/нечетные значения $l$:
\begin{equation*}
	\sub{\sigma}{чёт} = \frac{4\pi}{k} \sum_{m=0}^{ka/2} (2l+1) \frac{\cos(2ka)}{2k} = \frac{2 \pi}{k^2} \cos(2 ka)(ka + 1) \frac{ka + 1}{2},
	\hspace{5 mm} 
	l = 2m.
\end{equation*}
Теперь нечётный вклад $l = 2m+1$:
\begin{equation*}
	\sub{\sigma}{нечет} = \frac{2p}{k} \sum_{m=0}^{ka/2-1} (4m + 3) \frac{\cos(2 ka)}{2k} = \frac{2 \pi}{k^2} \cos(2ka) (ka - 1) \frac{ka}{2}.
\end{equation*}
Таким образом находим
\begin{equation*}
	\sub{\sigma}{чёт} - \sub{\sigma}{неч} = \frac{\pi}{k^2} \cos(2 ka) \left(
		\frac{5}{2} ka + 2
	\right).
\end{equation*}
Однако в финальное выражение входит только первое слагаемое
\begin{equation*}
	\sub{\sigma}{tot}|_{ka \gg 1} = 2 \pi a^2.
\end{equation*}



% \subsection*{Т33}

Для двух тождественных частиц можем написать $\Psi$ в виде
\begin{equation*}
	\Psi(\vc{r}_1,\, \vc{r}_2,\, s_1,\, s_2) = \Phi(\vc{R}) \psi(\vc{r}) \chi(s_1,\, s_2),
\end{equation*}
для приведенной массы $\mu = m/2$, $\vc{r} = \vc{r}_1 - \vc{r}_2$, $\vc{R}$ -- коордианты центра масс. 

\textbf{$\vc{\alpha}$-частицы}. Спин $\alpha$-частицы равен нулю, так что говорим про $\Psi$ для бозонов, симметричную по перестановкам. Тогда асмиптотика на бесконечности имеет вид
\begin{equation*}
	\psi(\vc{r})|_{r\to \infty} \sim e^{i \smallvc{k} \smallvc{r}} + e^{- i \smallvc{k} \smallvc{r}} + \left(
		f(\theta) + f(\pi-\theta)
	\right) \frac{e^{i \smallvc{k} r
	sv}}{r}.
\end{equation*}
Тогда сечение рассеяние может быть записано в виде
\begin{equation*}
	d \sigma = |f(\theta) + f(\pi-\theta)|^2 \d \Omega.
\end{equation*}

\textbf{Протоны}. Рассмотрим теперь случай фермионов с антисимметричной по перестановке $\Psi$. Для состояния с $s_1 + s_2 = S = 0$ $\chi$ антисимметрично, а значит $\psi$ симметрична, то есть совпадает с рассмотренным случаем для $\alpha$-частиц. 

Для $S = 1$ спиновая функция $\chi$ симметрично, тогда $\psi$ антисимметрична:
\begin{equation*}
	d \sigma_{S=1} = |f(\theta) - f(\pi-\theta)|^2 \d \Omega.
\end{equation*}
Считая состояния равновероятными $\ket{\uparrow}$ и $\ket{\downarrow}$, находим, что
\begin{equation*}
	\langle d \sigma\rangle_S = \frac{1}{4} |f(\theta) +  f(\pi-\theta)|^2 + \frac{3}{4} |f(\theta) - f(\pi-\theta)|^2.
\end{equation*} %? почему 1/4 и 3/4
% \subsection*{Т34}

Найдём сечение фотоэффекта для атома водорода. Рассмотрим реакцию
\begin{equation*}
	\gamma + H \longrightarrow p + e^-,
\end{equation*}
где считаем электрон свободной нерелятивистской частицей. По условию  энергия $\gamma$-кванта $\hbar \omega \gg \Ry$. Рассматриваем основное состояние атома водорода
\begin{equation*}
	\psi_{100} = \frac{1}{\sqrt{\pi a^3}} e^{-r/a}.
\end{equation*}

По определению
\begin{equation*}
	d \sigma = \frac{d w_{fi}}{\sub{j}{in}}.
\end{equation*}
С учётом нормировки
\begin{equation*}
	\bk{\lambda', \vc{k}'}{\lambda, \vc{k}} = (2 \pi)^3 \delta(\vc{k} - \vc{k}') \delta_{\lambda \lambda'} 2 \hbar \omega,
	\hspace{0.5cm} \Rightarrow \hspace{0.5cm}
	\sub{j}{in} = 2 \hbar \omega c.
\end{equation*}
Из правила Ферми:
\begin{equation*}
	d w_{fi} = \frac{2\pi}{\hbar} \delta\left(\textstyle \sum E\right) |V_{fi}|^2 \d \nu_\text{f},
	\hspace{10 mm} 
	d \nu_\text{f} = \frac{d^3 p_\text{f}}{(2 \pi \hbar)^3} = \frac{d^3 k_\text{f}}{(2 \pi)^3}.
\end{equation*}
Рассматриваем переход из $\ket{i} = \ket{\psi_{100}} \ket{\vc{k}\sbin,\, \lambda\sbin}$ в $\ket{f} = \ket{\vc{p}_\text{f}} \ket{0}$ (фотон поглотился), где $\lambda\sbin = \{1, 2\}$ -- возможные полярицации, по которым впоследствие усредним.

\textbf{Квантованное поле}.
Будем решать задачу в дипольном приближение:
\begin{equation*}
	\hat{V} = - \vc{d} \cdot \vc{E},
	\hspace{5 mm} 
	\vc{d} = - e \vc{r}.
\end{equation*}
Так как энергия поглощается из ЭМ поля, то рассматриваем
\begin{equation*}
	\hat{\vc{\mathcal A}}(t, \vc{r}) = \int \frac{\d^3 k}{(2\pi)^3} \frac{1}{2 k_0} \sum_{\lambda= 1, 2} \left(
		\hat{a}_{\lambda, \smallvc{k}} \vc{\epsilon}_\lambda e^{- i \omega t + i \smallvc{k} \smallvc{r}} + \hat{a}\D_{\lambda, \smallvc{k}} \vc{\epsilon}^*_\lambda e^{i \omega t - i \smallvc{k} \smallvc{r}}
	\right).
\end{equation*}
Для свободных полей
\begin{equation*}
	\hat{\vc{E}}(t, \vc{r}) = - \frac{1}{c} \frac{\partial }{\partial t} \hat{\vc{\mathcal A}} = \int \frac{d^3 k}{(2\pi)^3} \frac{i}{2} \left(
		\hat{a} \vc{\epsilon} e^{i \omega t + i \smallvc{k} \smallvc{r}} - \hat{a}\D \vc{\epsilon}^* e^{i \omega t - i \smallvc{k} \smallvc{r}}
	\right).
\end{equation*}

\textbf{Матричный элемент}.
Таким образом можем найти матричный элемент
\begin{equation*}
	V_{fi} = \bk{f}[-\hat{\vc{d}} \cdot \hat{\vc{E}}]{i} = \int d^3 r\ e^{-i \smallvc{k}_\text{f} \smallvc{r}} (-e \vc{r}) \frac{1}{\sqrt{\pi a^3}} e^{-r/a} \cdot \bk{0}[\hat{\vc{E}}]{\vc{k}\sbin,\, \lambda\sbin}.
\end{equation*}
Так как для фотона итоговое состояние вакуум, то вклад будет только от $\hat{a}$:
\begin{equation*}
	\bk{0}[\hat{\vc{E}}]{\vc{k}\sbin,\, \lambda\sbin} = \int \frac{d^3 k}{(2\pi)^3} \frac{i}{2} \left(
		\vc{\epsilon}_\lambda e^{-i \omega t  + i \smallvc{k} \smallvc{r}} \bk{0}[\hat{a}]{\vc{k}\sbin,\, \lambda\sbin} + 0
	\right),
\end{equation*}
где подставляя условие нормировки
\begin{equation*}
		\bk{0}[\hat{a}]{\vc{k}\sbin,\, \lambda\sbin} = 
	\bk{\vc{k},\lambda}{\vc{k}\sbin,\, \lambda\sbin} =
	(2 \pi)^3 \delta_{\lambda \lambda\sbin} \delta(\vc{k} - \vc{k}\sbin) 2 \hbar \omega,
\end{equation*}
находим выражение для матричного элемента поля
\begin{equation*}
	\bk{0}[\hat{\vc{E}}]{\vc{k}\sbin,\, \lambda\sbin} = i \hbar \omega\sbin \vc{\epsilon}_{\lambda\sbin} e^{- i \omega\sbin t + i \smallvc{k}\sbin \smallvc{r}}.
\end{equation*}
Подставляя это в матричный элемент $V_{fi}$, наконец приходим к выражению, вида
\begin{equation*}
	V_{fi} = - i \hbar \omega\sbin e \frac{1}{\sqrt{\pi a^3}} \int d^3 r \ e^{i \smallvc{q} \smallvc{r} - r/a} (\vc{\epsilon}_{\lambda\sbin} \cdot \vc{r}) = - \frac{i e \hbar \omega\sbin}{\sqrt{\pi a^3}} \left(\vc{\epsilon}_{\lambda\sbin} \cdot \vc{q}\right) \frac{32 \pi a^5}{\left(
		(qa)^2 + 1
	\right)^3} 	\approx	
	\frac{i e \hbar \omega\sbin}{\sqrt{\pi a^3}} \left(\vc{\epsilon}_{\lambda\sbin} \cdot \vc{k}_\text{f}\right) \frac{32 \pi a^5}{(k_\text{f} a)^6},
\end{equation*}
где ввели $\vc{q} \overset{\mathrm{def}}{=} \vc{k}\sbin - \vc{k}_\text{f}$ и воспользовались приближением
\begin{equation*}
	\frac{(\hbar k_\text{f})^2}{2m} = \hbar \omega + (-\sub{W}{ион}) \approx \hbar \omega,
	\hspace{10 mm} 
	k\sbin \ll k_\text{f},
	\hspace{0.5cm} \Rightarrow \hspace{0.5cm}
	\vc{q} \approx -\vc{k}_\text{f}.
\end{equation*}

\textbf{Усреднение}. Вычислим усредненное по поляризациям значение
\begin{equation*}
	|(\vc{\epsilon}_{\lambda\sbin} \cdot \vc{k}_\text{f})|^2 = \frac{1}{2} \sum_{\lambda=1}^{2} \left(\vc{\epsilon}_{\lambda\sbin} \cdot \vc{k}_\text{f}\right) \left(\vc{\epsilon}^*_{\lambda\sbin} \cdot \vc{k}_\text{f}\right) = \frac{1}{2} k_\text{f}^\alpha k_\text{f}^\beta \sum \epsilon^\alpha_\lambda \bar{\epsilon}^\beta_\lambda = \frac{1}{2} k_\text{f}^\alpha k_\text{f}^\beta \left(
		\delta^{\alpha \beta} - \frac{k^\alpha\sbin k^\beta\sbin}{k^2\sbin}
	\right) = \frac{1}{2} (k_\text{f}^2 - \frac{(\vc{k}_\text{f} \cdot \vc{k}\sbin)}{k^2\sbin}),
\end{equation*}
то есть просто часть, ортогональная $\vc{k}\sbin$, что можно было сказать с самого начала. Здесь воспользовались
\begin{equation*}
	\vc{\epsilon}_\lambda \vc{\epsilon}^*_\lambda  = 1,
	\ \Rightarrow \ 
	\epsilon^\alpha_\lambda \bar{\epsilon}^\alpha_\lambda = 2,
	\hspace{10 mm} 
	\vc{\epsilon}_\lambda \bot \vc{k}\sbin.
\end{equation*}
Вводя сферические координаты с осью $Oz$ вдоль $\vc{k}\sbin$, приходим к выражению
\begin{equation*}
	|(\vc{\epsilon}_{\lambda\sbin} \cdot \vc{k}_\text{f})|^2 = \frac{1}{2} k^2_\text{f} (1 - \cos^2 \theta).
\end{equation*}


\textbf{Сечение рассеяния}.  Теперь подставляем вычисленный выражения в формулу для полного сечения:
\begin{equation*}
	\int d\sigma = \int \frac{d w_{fi}}{2 \hbar \omega c} = \frac{1}{2 \pi \hbar \omega c} \frac{2 \pi}{\hbar} \int_{0}^{\infty} \frac{k_\text{f}^2 \d k_\text{f}}{(2\pi)^3} 2 \pi \int_{-1}^{1} d \cos \theta\  
	\delta\left(\hbar \omega -\tfrac{\hbar^2 k_\text{f}^2}{2m}\right) \frac{k_\text{f}^2}{2} \left(1 - \cos^2 \theta\right) \left(
		\frac{32 \pi a^5}{(k_\text{f} a)^6}
	\right)^2 \frac{(e \hbar \omega)^2}{\pi a^3},
\end{equation*}
откуда получаем выражения для $\sub{\sigma}{tot}$:
\begin{equation*}
	\sub{\sigma}{tot} = \int d\sigma = \frac{2^8}{3} 4 \pi a^2 \left(\frac{\sub{W}{ион}}{\hbar \omega}\right)^{7/2},
	\hspace{5 mm} 
	\sub{W}{ион} = \Ry = \frac{e^2}{2a}.
\end{equation*}











% Т14+, Т15+, Т16(О?), Т17+

% Т20+, Т21+, Т22+
% Т23+, Т24+, Т25+, Т26+
%-Т27?, Т28(2.11.10.), Т29+(2.12.1.), Т30(2.12.4.)

% Т31+, Т32+, Т33(2.12.12.), Т34+

