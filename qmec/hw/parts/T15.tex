\subsection*{Т15}

По определению
\begin{equation*}
	W^\mu = - \frac{1}{2} \varepsilon^{\mu \nu \lambda \rho} p_\nu S_{\lambda \rho},
	\hspace{10 mm} 
	S_{ik} = \hbar \varepsilon_{ikl} s^l.
\end{equation*}
Тогда подставляя $\mu = 0$, находим
\begin{equation*}
	W^0 = - \frac{1}{2} \varepsilon^{0 i j k} p_i \hbar \varepsilon_{jkn} s^n = - \hbar p_i s^i = \hbar \left(\vc{p} \cdot \vc{s}\right).
\end{equation*}
Теперь, с учетом $S_{0i} = i \sign(\mathfrak{s}) s^i \hbar$, находим
\begin{align*}
	W^i &= - \tfrac{1}{2}\varepsilon^{i0jk} p_0 S_{jk} - \varepsilon^{i j 0 k} p_j S_{0k} = \tfrac{1}{2} \varepsilon^{0 ijk} p_0 \hbar \varepsilon_{jkn} s^n - i \sign(\mathfrak{s}) \varepsilon^{o i jk} p_j s_k \hbar = \\ &= \hbar \left(
		p_0 s^i - i \sign(\mathfrak s)  \left[\vc{p} \times  \vc{s}\right]^i
	\right).
\end{align*}