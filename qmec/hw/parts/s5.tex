\subsection*{Уравнение Паули для позитрона}

Известно, что для электрона можем записать уравнение Паули:
\begin{equation*}
	\psi(t, \vc{r}) = e^{
		\frac{i}{\hbar} m c^2 t
	} \begin{pmatrix}
		\varphi(t, \vc{r}) \\ x(t, \vc{r})
	\end{pmatrix}.
\end{equation*}
Для позитрона можем воспользоваться анзацем
\begin{equation*}
	\psi(t, \vc{r}) = e^{
		-\frac{i}{\hbar} m c^2 t
	} \begin{pmatrix}
		\varphi(t, \vc{r}) \\ x(t, \vc{r})
	\end{pmatrix}.
\end{equation*}
Подставляем это в уравнение Дирака
\begin{equation*}
	\mathcal{\tilde{P}} = \mathcal P_\mu \gamma^\mu,
	\hspace{10 mm} 
	(\mathcal{\tilde{P}} - mc) \psi(t, \vc{r}),
	\hspace{10 mm} 
	\gamma^0 = \begin{pmatrix}
		\mathbbm{1} & 0 \\ 0 & - \mathbbm{1}
	\end{pmatrix},
	\hspace{10 mm} 
	\vc{\gamma} = \begin{pmatrix}
	    0 & -\vc{\sigma}  \\
	    \vc{\sigma} & 0  \\
	\end{pmatrix}.
\end{equation*}
Где помним про $\mathcal P_\mu = i \hbar \partial_\mu - \frac{e}{c} A_\mu$.
Помним про $\partial_{x_0} = \frac{1}{c}\partial_t$.

Введем оператор $\hat{\varepsilon} = i \hbar \partial_t$, $A_0 = \Phi$ и запишем получившеюся систему вида
\begin{align*}
	(\hat{\varepsilon} - e \Phi - 2 mc^2) \varphi - c (\vc{\sigma} \cdot \vc{\mathcal P}) \chi = 0, \\
	(\vc{\sigma} \cdot \vc{\mathcal P}) \varphi + (e \Phi - \hat{\varepsilon}) \chi = 0.
\end{align*}
Решая, находим
\begin{equation*}
	\varphi = - \frac{1}{1 -\frac{\hat{\varepsilon} - e \Phi}{2 m c^2}}
	\frac{(\vc{\sigma} \cdot \vc{\mathcal P})}{2 m c} \chi = -
	(1 + \hat{W}) \frac{(\vc{\sigma} \cdot \vc{\mathcal P})}{2 m c} \chi.
\end{equation*} 
Теперь, кстати $\varphi$ -- малая компонента, $\chi$ -- большая, при рассмотрении $\frac{v}{c} \ll 1$. 

Вводя $\bar{\psi} = \psi^+ \gamma^0$, получаем
\begin{equation*}
	c \bar{\psi} \left(\mathcal{\tilde{P}} - m c\right) \psi = 
	\chi^+ \left(
		\hat{\varepsilon} - e \Phi + \frac{1}{2m} (\vc{\sigma} \cdot \vc{\mathcal P}) 
 (1 + \hat{W}) (\vc{\sigma} \cdot \vc{\mathcal P})
 \right).
\end{equation*}
Считая $\frac{v}{c} \ll 1$, верно, что
\begin{equation*}
	1 + \hat{W} \approx	1 + \frac{\hat{\varepsilon}-e \Phi}{2 mc^2}.
\end{equation*}


Умея сворачивать $\vc{\sigma}$, находим
\begin{equation*}
	(\vc{\sigma} \cdot \vc{\mathcal P})^2 = \vc{\mathcal P}^2 + i \vc{\sigma} \cdot \left[\vc{\mathcal P} \times  \vc{\mathcal P}\right].
\end{equation*}
Распишем покомпоненто
\begin{equation*}
	\left[\vc{\mathcal P} \times  \vc{\mathcal P}\right]_\alpha = \varepsilon_{\alpha \beta \gamma} \left(
		- i \hbar \partial_\beta - \tfrac{e}{c} A_\beta
	\right)\left(
		- i \hbar \partial_\gamma - \tfrac{e}{c} A_\gamma
	\right) = i \hbar \tfrac{e}{c} \varepsilon_{\alpha \beta \gamma} \left(
		\partial_\beta A_\gamma
	\right),
\end{equation*}
откуда находим
\begin{equation*}
	\left[\vc{\mathcal P} \times  \vc{\mathcal P}\right]_\alpha = i \hbar \tfrac{e}{c} \vc{\mathcal H},
	\hspace{10 mm} 
	(\rot \vc{A})_\alpha = \mathcal H_\alpha.
\end{equation*}

Таким образом можем записать действие:
\begin{equation*}
	S_{NR} = \int d^4 x\ 
	\chi^+ \left(
		\hat{\varepsilon} - e \Phi + \frac{\vc{\mathcal P}^2}{2m} - \frac{e \hbar}{2 m c} \left(\vc{\sigma} \cdot \vc{\mathcal H}\right)
	\right).
\end{equation*}
Варьируя действие, находим
\begin{equation*}
	i \hbar \partial_t \chi = \left(
		e \Phi - \frac{\vc{\mathcal P}^2}{2m} + \frac{e \hbar}{2 m c} \left(\vc{\sigma} \cdot \vc{\mathcal H}\right)
	\right) \chi,
\end{equation*}
которое по идее есть Шрёдингер, вида $i \hbar \partial_t \chi = \hat{H} \chi$. 

Нужно вспомнить, что у нас должа быть правильная спинорная метрика (переходим к другой киральности):
\begin{equation*}
	\chi_{NR} = i \sigma_2 \chi^*.
\end{equation*}
Так что подставляя это наверх, находим
\begin{equation*}
	- i \hbar \partial_t \chi^* = \left(\ldots\right) \chi^*.
\end{equation*}
Воспользуемся свойством $\sigma_2 \vc{\sigma}^* \sigma_2 = - \vc{\sigma}$. Домножая последнее уравнение на $i \sigma_2$, находим
\begin{equation*}
	i \hbar \partial_t \xi_{NR} = \left(
		\frac{\vc{\mathcal P}^2}{2m} - e \Phi + \frac{e \hbar}{2 m c} g 
		\left(
			\vc{s} \cdot \vc{\mathcal H}
		\right)
	\right) \chi_{NR},
	\hspace{20 mm} 
	g=2, \hspace{5 mm} 
	\vc{s} = \frac{1}{2} \vc{\sigma}.
\end{equation*}







\subsection*{Правила Хунда}

\textbf{Определитель Слетера}. Рассмотрим $N$ электронов со спином $s=\frac{1}{2}$, и квантовыми числами $\{k_1,\, \ldots,\, k_N\}$ (обычно $\{n,\, l,\, m\}$). Далее описываем систему в виде $x = \{\vc{r},\, m_s\}$.

Рассмотрим случай отсутсвия спин-орбитального взаимодействия, то есть спин коммутирует с гамильтонианом, тогда будет иметь место факторизация $\ket{k} \otimes \ket{m_s}$.

Вспоминаем связь спина со статистикой, тогда
\begin{equation*}
	\Phi_{k_1,\, \ldots,\, k_N} (x_1,\, \ldots,\, x_N) = 
	\frac{1}{\sqrt{N!}} \det \begin{pmatrix}
		\psi_{k_1} (x_1) & \ldots & \psi_{k_1} (x_N) \\
		\vdots & \ddots & \vdots \\
		\psi_{k_N} (x_1) & \ldots & \psi_{k_N} (x_N)
	\end{pmatrix}.
\end{equation*}

\textbf{Правила Хунда: феноменология}. 
Располагаем нерелятивисткие термы ${}^{2s+1}L$ по I правилу Хунда. 
Во-первых считаем, что $E\to \min$ при $s \to \max$. 
Далее меняем $E \to \min$ при $L \to \max$.


\textbf{Правила Хунда: релятивистские поправки I}. Спин-орбитальное взаимодействие вносит в незаполненные оболочки ($2^{2(2l +1)}$) 
\begin{equation*}
	V_{sl} = \sum_f \frac{e \hbar^2}{2 m^2 c^2} \frac{V'(r_f)}{r_f} \left(
		\vc{s}_f \cdot \vc{l}_f
	\right).
\end{equation*}
Рассмотрим $n_f \leq 2 l +1$, тогда будем считать $\vc{s}_f \approx	\frac{1}{n_f} \vc{S}$, $r_f \approx \langle r\rangle$, тогда
\begin{equation*}
	\langle V_{sl}\rangle_{LS} \approx	
	\sum_{f=1}^{n_f} \frac{e \hbar^2}{2 m^2 c^2} \frac{\langle V'\rangle}{\langle r\rangle} \frac{1}{n_f} \left(\vc{S} \cdot \vc{l}_f\right) = 
	\underbrace{\frac{e \hbar^2}{2 m^2 c^2} \frac{\langle V'\rangle}{\langle r\rangle} \frac{1}{n_f}}_{A_{LS} \sim \alpha^4 m c^2 > 0} \left(\vc{S} \cdot \vc{L}\right).
\end{equation*}
Считая скалярное произведение, находим
\begin{equation*}
	\langle V_{sl}\rangle_{LS} = \frac{1}{2} A_{LS} \left(
		J(J+1) - L(L+1) - S (S+1)
	\right).
\end{equation*}
Так получаем правило интервалов Ланде
\begin{equation*}
	\Delta \langle V_{sl}\rangle_{LS} \approx A_{LS} J.
\end{equation*}




\textbf{Правила Хунда: релятивистские поправки II}. Теперь рассмотрим $n_f > 2 l + 1$, но заполненные оболочки теперь существуют вместе с дырками:
\begin{equation*}
	n_h = 2(2 l + 1) -n f  < 2 l +1.
\end{equation*}
Понятно, что
\begin{equation*}
	\sum_{f=1}^{n_f} \vc{s}_f + \sum_{n=1}^{n_h} \vc{s}_h = 0.
\end{equation*}
Можем рассматривать 
\begin{equation*}
	\vc{S} = \sum_{f=1}^{n_f}  \vc{s}_f  + \sum_{f=1}^{n_h} \tilde{\vc{s}} + \sum_{h=1}^{n_h} \vc{s}_h	= \vc{S}_h.
\end{equation*}

Теперь рассматриваем орбитальный момент
\begin{equation*}
	\vc{L} = \sum_{f=1}^{n_f} \vc{l}_f,
	\hspace{5 mm} 
	\vc{L} = \sum_{f=1}^{n_h} \tilde{\vc{l}}_f	= \sum_{h=1}^{n_h} \vc{l}_h = \vc{L}_h,
\end{equation*}
где теперь выполняется
\begin{equation*}
	\tilde{\vc{L}} + \vc{L}_h = 0,
	\hspace{0.5cm} \Rightarrow \hspace{0.5cm}	
	\langle V_{sl}\rangle_{LS} \approx \frac{1}{2} A_{SL}^h \left(\vc{S}_h cdot \vc{L}_h\right) = - \frac{1}{2} A_{SL} \left(\vc{S} \cdot \vc{L}\right).
\end{equation*}
Таким образом при фиксированных $S, L$ энергия $E\to \min$:
$n_f \leq 2l+1$ при $J = |L-S|$;
$n_f > 2l+1$ при $J = L+S$.


% \subsection*{Т26}

% Для кремния $S = 1$ и $L=0,1,2$ или $S=0$ и $L=0, 2$. основное состояние будет $3 s^2 3 p^2$. 


