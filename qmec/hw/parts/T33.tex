\subsection*{Т33}

Для двух тождественных частиц можем написать $\Psi$ в виде
\begin{equation*}
	\Psi(\vc{r}_1,\, \vc{r}_2,\, s_1,\, s_2) = \Phi(\vc{R}) \psi(\vc{r}) \chi(s_1,\, s_2),
\end{equation*}
для приведенной массы $\mu = m/2$, $\vc{r} = \vc{r}_1 - \vc{r}_2$, $\vc{R}$ -- коордианты центра масс. 

\textbf{$\vc{\alpha}$-частицы}. Спин $\alpha$-частицы равен нулю, так что говорим про $\Psi$ для бозонов, симметричную по перестановкам. Тогда асмиптотика на бесконечности имеет вид
\begin{equation*}
	\psi(\vc{r})|_{r\to \infty} \sim e^{i \smallvc{k} \smallvc{r}} + e^{- i \smallvc{k} \smallvc{r}} + \left(
		f(\theta) + f(\pi-\theta)
	\right) \frac{e^{i \smallvc{k} r
	sv}}{r}.
\end{equation*}
Тогда сечение рассеяние может быть записано в виде
\begin{equation*}
	d \sigma = |f(\theta) + f(\pi-\theta)|^2 \d \Omega.
\end{equation*}

\textbf{Протоны}. Рассмотрим теперь случай фермионов с антисимметричной по перестановке $\Psi$. Для состояния с $s_1 + s_2 = S = 0$ $\chi$ антисимметрично, а значит $\psi$ симметрична, то есть совпадает с рассмотренным случаем для $\alpha$-частиц. 

Для $S = 1$ спиновая функция $\chi$ симметрично, тогда $\psi$ антисимметрична:
\begin{equation*}
	d \sigma_{S=1} = |f(\theta) - f(\pi-\theta)|^2 \d \Omega.
\end{equation*}
Считая состояния равновероятными $\ket{\uparrow}$ и $\ket{\downarrow}$, находим, что
\begin{equation*}
	\langle d \sigma\rangle_S = \frac{1}{4} |f(\theta) +  f(\pi-\theta)|^2 + \frac{3}{4} |f(\theta) - f(\pi-\theta)|^2.
\end{equation*}