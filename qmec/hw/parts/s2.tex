\subsection*{Т6}


Теперь
\begin{equation*}
    \hat{H} = \frac{\hat{p}^2}{2m} - \frac{e^2}{r} + \hat{V},
\end{equation*}
где $\hat{V} = - e E \hat{z}$. Известно, что $n=2$, тогда вырождение $n^2 = 4$. Можем явно выписать несколько функций
\begin{align*}
    \ket{200} &= \frac{2}{\sqrt{4 \pi}} \left(\frac{z}{2a}\right)^{3/2} e^{-r/2a} \left(
        1 - \frac{r}{2a}
    \right), \\
    \ket{210} &= \sqrt{\frac{3}{4\pi}} \cos \theta \left(\frac{1}{2a}\right)^{3/2} e^{-r/2a} \frac{r}{\sqrt{3} a}, 
\end{align*}
а для $\ket{211}$ и $\ket{21-1}$ важно только что есть фактор $e^{im\varphi}$.

Действительно,
\begin{equation*}
    \bk{21m}[\hat{V}]{21m'} = 0,
    \hspace{5 mm} 
    m,\, m' = \pm 1.
\end{equation*}
Осталось посчитать
\begin{equation*}
    \kappa \overset{\mathrm{def}}{=}  \bk{200}[\hat{V}]{210} = \int_{\mathbb{R}^3}
        \ldots
    d^3 \vc{r} = 3 e E \frac{a}{z}.
\end{equation*}
Получилось матрица  ненулевыми коэффициентами только в первом блоке 2 на 2:
\begin{equation*}
    \hat{V} = \begin{pmatrix}
        0 & \kappa  \\
        \kappa & 0  \\
    \end{pmatrix},
    \hspace{10 mm} 
    \lambda_1 = \kappa, \hspace{5 mm} 
    \lambda_2 = - \kappa,
    \hspace{5 mm}   
    \lambda_3 =  \lambda_4 = 0.
\end{equation*}
Решая секулярное уравнение, находим
\begin{equation*}
    E_2 =  - \frac{\text{Ry}}{2^2},
    \hspace{5 mm} 
    \left[\hat{H} + \hat{V} - (E_2 \pm \kappa) \mathbbm{1}\right] \ket{\psi} = 0,
    \hspace{0.5cm} \Rightarrow \hspace{0.5cm}
    \vc{c}_+ = \frac{1}{\sqrt{2}} \left(1,\, 1,\, 0,\, 0\right),
    \hspace{5 mm} 
    \vc{c}_- = \frac{1}{\sqrt{2}} \left(1,\, -1,\, 0,\, 0\right).
\end{equation*}
Энергии расщепления
\begin{equation*}
    E^+ = E_2^{(0)} + \kappa,
    \hspace{5 mm} 
    E^- = E_{2}^{(0)} -\kappa.
\end{equation*}




% \subsection*{Т7}

\subsection*{Т9+Т10}

И снова задача на решение нестационарного уравнения Шрёдингера. Пусть в невозмущенном варианте всё $\parallel z$, возмущением будет $\sigma_+$ поляризованная волна, падающая по $Oz$. 

Гамильтониан систеы:
\begin{equation*}
    \hat{H} = - \vc{\mu} \cdot \vc{H},
    \hspace{5 mm} 
    \vc{\mu} = \frac{e \hbar}{2 m c} \vc{s} g,
\end{equation*}
где $g = 2$. Магнитное поле
\begin{equation*}
    \vc{H} = \vc{H}_0 + \vc{e}_x h \cos (\omega t) + \vc{e}_y h \sin \omega t.
\end{equation*}
Тогда $\hat{H}$ перепишется в виде
\begin{equation*}
    \hat{H} = \frac{|e| H_0}{2 m c} \hbar \sigma_z + \frac{|e| h}{2 m c} \hbar \left(\sigma_x \cos (\omega t) + \sigma_y \sin(\omega t) \right).
\end{equation*}
введем обозначения
\begin{equation*}
    \Omega_0 \overset{\mathrm{def}}{=} \frac{|e| H_0}{m c},
    \hspace{5 mm} 
    \Omega' = \frac{|e| h}{mc}.
\end{equation*}
Вводя $\sigma_\pm$ переходим к
\begin{equation*}
    \hat{H} = \frac{1}{2} \Omega_0 \sigma_z + \frac{1}{2} \hbar \Omega' \left(
        \sigma_+ e^{- i \omega t} + \sigma_{-} e^{i \omega t}
    \right).
\end{equation*}
Далее будем решать нестационаное уравнение Шрёдингера
\begin{equation*}
    i \hbar \partial_t \chi = \hat{H} \xi,
    \hspace{10 mm} 
    \chi(t) = \exp\left(
        - \frac{i}{2} \tilde{\Omega} t \sigma_z \tilde{\chi}(t)
    \right).
\end{equation*}
Подставляем и находим
\begin{equation*}
    i \hbar \partial_t \chi = \exp\left(
        - \frac{i}{2} t \tilde{\Omega} \sigma_z
    \right) \left(
        \frac{1}{2} \hbar \tilde{\Omega} \sigma_z + i \hbar \partial_t
    \right) \tilde{\chi},
\end{equation*}
которое в свою очередь равно
\begin{equation*}
    i \hbar \partial_t \chi = \hat{H} \exp\left(
        - \frac{i}{2} t \tilde{\Omega} \sigma_z
    \right)  \tilde{\chi}.
\end{equation*}
Домножим это всё слева на $\exp\left(\frac{i}{2} \tilde{\Omega} t \sigma_z\right)$, так приходим к
\begin{equation*}
    \left(\frac{1}{2} \hbar \tilde{\Omega} \sigma_z + i \hbar \partial_t\right)\tilde{\xi} = \left(
        \frac{1}{2} \hbar \Omega_0 \sigma_z + \frac{1}{2} \hbar \Omega' \left(
            \tilde{U}^+ \sigma_+ \tilde{U} e^{- i \omega t} + \tilde{U}^+ \sigma_- \tilde{U} e^{i \omega t}
        \right)
    \right) \tilde{\chi}.
\end{equation*}
Введем обозначения
\begin{equation*}
    \sigma_{\pm} = e^{\frac{i}{2} \tilde{\Omega} t \sigma_z} \sigma_{\pm} e^{- \frac{i}{2} \tilde{\Omega} t \sigma_z}.
\end{equation*}
Помним коммутаторы для $\sigma$,  получаем
\begin{equation*}
    \frac{d }{d t} \sigma_{\pm} (t) = \pm i \tilde{\Omega} \sigma_{\pm} (t),
\end{equation*}
где
\begin{equation*}
    \sigma_{\pm} (0) = \sigma_{\pm},
    \hspace{0.5cm} \Rightarrow \hspace{0.5cm}
    \sigma_{\pm} (t) = \sigma_{\pm} e^{\pm i \tilde{\Omega} t}.
\end{equation*}
Замечаем, что наша жизнь становится лучше, если $\tilde{\Omega} = \omega$, а значит
\begin{equation*}
    i \hbar \partial_t \tilde{\chi} = \left(
        \frac{1}{2} \hbar \left(\Omega_0 - \omega\right) \sigma_z + \frac{1}{2} \hbar' \left(\sigma_+ + \sigma_-\right)
    \right) \tilde{\chi}.
\end{equation*}
Новый $\hat{\tilde{H}}$ можем переписать в виде
\begin{equation*}
    \hat{\tilde{H}} = \frac{1}{2} \hbar \begin{pmatrix}
        \Omega_0 - \omega & \Omega'  \\
        \Omega' & -(\Omega_0 - \omega)  \\
    \end{pmatrix} \overset{\mathrm{def}}{=} \begin{pmatrix}
        E_1 & V  \\
        V & -E_1  \\
    \end{pmatrix}.
\end{equation*}
Перехоим к базису, диагонализирующем $\hat{\tilde{H}}$. Находим его собственные числа:
\begin{equation*}
    E_\pm = \pm \sqrt{V^2 + E_1^2}.
\end{equation*}
Считая, что $\Omega_0, \Omega' \gg \omega$ и $\Omega^2 = \Omega_0^2 + \Omega'{}^2$, можем получить
\begin{equation*}
    E_\pm \approx  \pm \frac{\hbar}{2} \Omega \left(1 - \omega \frac{\Omega_0}{\Omega^2}\right).
\end{equation*}
Вспоминаем, что
\begin{equation*}
    \Omega_0 = \frac{|e| H_0}{mc},
    \hspace{5 mm} 
    \Omega' = \frac{|e| h}{mc},
    \hspace{5 mm} 
    \Omega = \frac{|e| H}{mc},
    \hspace{0.5cm} \Rightarrow \hspace{0.5cm}
    \frac{\Omega_0}{\Omega} = \frac{H_0}{H} = \cos \theta.
\end{equation*}
Тогда
\begin{equation*}
    E_\pm = \pm \frac{\hbar}{2} \left(\Omega + \omega \cos \theta\right).
\end{equation*}


Теперь вводим собственные вектора
\begin{equation*}
    \ket{\uparrow (t)} = e^{- \frac{i}{\hbar} E_+ t} \ket{\uparrow},
    \hspace{10 mm} 
    \ket{\downarrow (t)} = e^{\frac{i}{\hbar} E_- t} \ket{\downarrow}.
\end{equation*}
Собственно, сами собственные векторы
\begin{equation*}
    \begin{pmatrix}
        E_1 - E_\pm & V  \\
        V & -E_1 - E_\pm  \\
    \end{pmatrix} \vc{v} = 0,
    \hspace{0.5cm} \Rightarrow \hspace{0.5cm}   
    \vc{v} = \frac{2}{\hbar}\begin{pmatrix}
        E_1 + E_\pm \\ V
    \end{pmatrix} \approx   
    \begin{pmatrix}
        \Omega_0 - \omega \pm (\Omega - \omega \cos \theta) \\
        \Omega'
    \end{pmatrix}.
\end{equation*}
Тогда матрица перехода
\begin{equation*}
    S = \begin{pmatrix}
        \Omega_0 - \omega + \Omega - \omega \cos \theta &  \Omega_0 - \omega - \Omega + \omega \cos \theta  \\
        \Omega' & \Omega'  \\
    \end{pmatrix}.
\end{equation*}
Находим к ней обратную
\begin{equation*}
    S^{-1} = \frac{1}{2 \Omega' (\Omega - \omega \cos \theta)} \begin{pmatrix}
        \Omega' & -\Omega_0 + \omega + \Omega - \omega \cos \omega t  \\
        - \Omega' & \Omega_0 - \omega + \Omega - \omega \cos \theta  \\
    \end{pmatrix}.
\end{equation*}
Тогда
\begin{equation*}
    \tilde{\xi} (t) = S \begin{pmatrix}
        e^{\frac{i}{2} E_+ t} & 0  \\
        0 & e^{- \frac{i}{2} E_- t}  \\
    \end{pmatrix} S^{-1} \ket{\chi(0)}.
\end{equation*}
Перемножая, находим
\begin{align*}
    \ket{\tilde{\chi}(t)}_1 &= \cos \left(\frac{E_+ t}{\hbar}\right) + i  (\omega - \Omega_0) \sin \left(\frac{E_+ t}{\hbar}\right),
    \ket{\tilde{\chi}(t)}_2 &=  \frac{i \Omega' \sin \left(\frac{E_+ t}{\hbar}\right)}{\Omega - \omega \cos \theta},
\end{align*}
а искомая величина будет 
\begin{equation*}
    \ket{\chi(t)} = e^{- \frac{i}{2} \omega t \sigma_z} \ket{\tilde{\chi}(t)}.
\end{equation*}


\textbf{Поляризация}. Осталось найти
\begin{equation*}
    \vc{P} = \bk{\xi(t)}[\vc{\sigma}]{\xi(t)},
\end{equation*}
которое считать и считать, а получится
\begin{align*}
    P_x &= \sin \varphi \left(
        \cos \varphi (1 - \cos \Omega t) \cos \omega t - \sin \Omega t \sin \omega t
    \right), \\
    P_y &= \sin \varphi \left(
        \cos \varphi (1 - \cos \Omega t) \sin \omega t + \sin \Omega t \cos \omega t
    \right), \\
    P_z &= \cos^2 \varphi + \sin^2 \varphi \cos \Omega t,
\end{align*}
где было введено обозначение
\begin{equation*}
    \sin \varphi = \frac{\Omega}{\Omega - \omega \cos \theta},
    \hspace{5 mm} 
    \cos \varphi = \frac{\omega - \Omega_0}{\Omega - \omega \cos \theta}.
\end{equation*}

\textbf{Next}. Вообще
\begin{equation*}
    \begin{pmatrix}
        P_x \\ P_y
    \end{pmatrix} = \begin{pmatrix}
        \cos \omega t & - \sin \omega t  \\
        \sin \omega t & \cos \omega t  \\
    \end{pmatrix} \begin{pmatrix}
        \tilde{P}_x \\ \tilde{P}_y
    \end{pmatrix},
\end{equation*}
поэтому поляризаця <<следует>> за $\vc{H}$.


\textbf{Фаза Берри}. Её можно посчитать, как
\begin{equation*}
    \Delta_c \gamma = \oint A_\mu \d a^\mu = \oint_0^{2 \pi} A_\varphi \d \varphi,
    \hspace{10 mm} 
    A_\varphi = \bk{\psi}[\partial_\mu]{\psi}= i \bk{\uparrow}[\partial_\varphi]{\uparrow}.
\end{equation*}
где $a$ -- адиабатически меняющийся параметр гамильтониана. 

Знаем, что
\begin{equation*}
    \ket{\uparrow} = \cos\left(\frac{\theta}{2}\right) \ket{+} + e^{i \varphi} \sin\left(\frac{\theta}{2}\right) \ket{-}.
\end{equation*}
Тогда
\begin{equation*}
    A_\varphi = -\sin^2 \frac{\theta}{2} = \frac{1}{2} \left(\cos \theta -1\right),
\end{equation*}
тогда искомая фаза Берри
\begin{equation*}
    \Delta_c \gamma = \pi \left(\cos \theta - 1\right).
\end{equation*}

Связь с телесным углом можны найти, посчитав
\begin{equation*}
    \Omega = \int_{0}^{1} \d \cos \theta' \int_{0}^{2\pi} \d \varphi = 2 \pi \left(1 - \cos \theta\right),
\end{equation*}
действительно пропорциональны.