\subsection*{Т23}

Найдём уровни энергии и волновые функции стационарных состояний двух невзаимодействующих тождественных частиц в потенциальном ящике
\begin{equation*}
	V(x) = \left\{\begin{aligned}
	    &0,\, &0 < x < a,\\
	    &\infty,\, &x<0,\ x > a.
	\end{aligned}\right.
\end{equation*}
Для одной частицы знаем, что
\begin{equation*}
	\psi_k (x) \sim \sin(k_n x), \hspace{5 mm} 
	k_n = \frac{\pi n}{a},
\end{equation*}
с характерной энергией $E_0 = \frac{\pi^2 \hbar^2}{2 m a^2}$.

\textbf{Фермионы}. Рассмотрим $s=\frac{1}{2}$, тогда суммарный спин $S = \{0,\, 1\}$. Полная волновая функция антисимметрична:
\begin{equation}
	\Psi_{n_1 n_2} = \psi_{\pm} \times  \chi_{\mp}(2S=1 \pm 1),
	\label{sympsi}
\end{equation}
где $\pm$ соответсвует симметричной и антисимметричной функции. 

Энергию при $n_1 \neq n_2$ можем найти в виде
\begin{equation*}
	E_{n_1 n_2} = E_0 \left(n_1^2 + n_2^2\right).
\end{equation*}
При $n_1 = n_2$ невозможно состояние с $S=1$, поэтому энергия запищется в виде
\begin{equation*}
	E_{nn} = E_0 n^2.
\end{equation*}
Для поиска энергии основного состояния $N$-частиц, задача сводится к сумме квадратов
\begin{equation*}
	\sum_{n=1}^{m} n^2 = \frac{1}{6} m (m+1) (2 m+1),
	\hspace{0.5cm} \Rightarrow \hspace{0.5cm}
	E_N = \frac{E_0}{12} (N+1)(N^2 + 2 N + 3 \cdot  (N\ \text{mod}\, 2) ),
\end{equation*}

С учетом \eqref{sympsi}, волновую функцию можем записать в виде
\begin{equation*}
	\psi_F^S(x_1,\, x_2) = \frac{1}{\sqrt{2}}\left(
		\psi_{n_1}(x_1) \psi_{n_2} (x_2) + (-1)^{S} \psi_{n_1} (x_2) \psi_{n_2} (x_1)
	\right),
\end{equation*}
где $\psi(x_1,\, x_2)$ обращается в $\equiv 0$ при $n_1 = n_2$ и $S=1$.

\textbf{Бозоны}.  Энергия представима в виде
\begin{equation*}
	E_{n_1 n_2} = E_0 (n_1^2 + n^2).
\end{equation*}
Энергия основного состояния для $N$ бозонов не зависит от спина и равна
\begin{equation*}
	E_N = E_0 \ N.
\end{equation*}

Для частиц с нулевым спином полная волновая функция может быть только симметричной, значит представима в виде $\psi_F^0$.  Для частиц с единчиным спином $\Psi$ симметрична, поэтому
\begin{equation*}
	\Psi_{n_1 n_2} = \psi_{\pm} \times  \chi_{\pm}(S).
	\label{sympsi2}
\end{equation*}
а значит $S=1$ соотвествует $\psi_+$ и $S=\{0,\, 2\}$ соответсвует $\psi_-$. 




