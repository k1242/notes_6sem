

\subsection*{Т32}


\textbf{II}.
Для случая быстрых частиц $ka \gg 1$ рассмотрим <<черную дыру>>, тогда для $l < ka$ получаем\footnote{
	Вообще верно, что $\hbar k \cdot b = \hbar l$, где $b$ -- прицельный параметр.	
}  $S_l = 0$ и для $l > ka$  будет $S_l = 1$.

Записываем оптическую теорему
\begin{equation*}
	\sub{\sigma}{tot} = \frac{4\pi}{k} \sum_l \Im f_l.
\end{equation*}
Сохраняются $f_l$ сохраняются
\begin{equation*}
	f_l = \frac{2l+1}{2 ik } (S_l -1).
\end{equation*}
Также помним, что нужно суммировать до $l = ka$, при больших $l$ сечение обращается в $0$. Итого получаем
\begin{equation*}
	\sub{\sigma}{tot} = \frac{4 \pi}{k} \sum_{l=0}^{ka} (2l+1) \frac{1}{2k} = \frac{2\pi}{k^2} (ka+1)^2 = 2 \pi a^2.
\end{equation*}


\textbf{I}. Рассмотрим непроницаемую сферу
\begin{equation*}
	U(r) = \left\{\begin{aligned}
	    &0, &r \leq a, \\
	    &\infty, &r > a.
	\end{aligned}\right.
\end{equation*}
Вспоминаем
\begin{equation*}
	R_{kl} \approx	\frac{c_l}{r} \sin\left(kr - \tfrac{\pi l}{2} + \delta_l\right).
\end{equation*}
Верно, что $R_{kl} |_{r=a} = 0$:
\begin{equation*}
	ka - \frac{\pi l}{2} + \delta_l = \pi n \to 0,
	\hspace{0.5cm} \Rightarrow \hspace{0.5cm}
	\delta_l = \frac{\pi l}{2} - ka.
\end{equation*}
Находим сечение рассеяния:
\begin{equation*}
	f_l = \frac{2l+1}{k} e^{i \delta_l} \sin \delta_l.
\end{equation*}
Пользуемся оптической теоремой, находим
\begin{equation*}
	\Im f_l = \frac{2l+1}{2k} - \frac{2l+1}{2k}\cos\left(
		\pi l - 2 ka
	\right).
\end{equation*}
Вклад от первого слагаемого дает половину $\sub{\sigma}{geom} = 4 \pi a^2$. Для расчёта второго слагаемого рассмотрим четные/нечетные значения $l$:
\begin{equation*}
	\sub{\sigma}{чёт} = \frac{4\pi}{k} \sum_{m=0}^{ka/2} (2l+1) \frac{\cos(2ka)}{2k} = \frac{2 \pi}{k^2} \cos(2 ka)(ka + 1) \frac{ka + 1}{2},
	\hspace{5 mm} 
	l = 2m.
\end{equation*}
Теперь нечётный вклад $l = 2m+1$:
\begin{equation*}
	\sub{\sigma}{нечет} = \frac{2p}{k} \sum_{m=0}^{ka/2-1} (4m + 3) \frac{\cos(2 ka)}{2k} = \frac{2 \pi}{k^2} \cos(2ka) (ka - 1) \frac{ka}{2}.
\end{equation*}
Таким образом находим
\begin{equation*}
	\sub{\sigma}{чёт} - \sub{\sigma}{неч} = \frac{\pi}{k^2} \cos(2 ka) \left(
		\frac{5}{2} ka + 2
	\right).
\end{equation*}
Однако в финальное выражение входит только первое слагаемое
\begin{equation*}
	\sub{\sigma}{tot}|_{ka \gg 1} = 2 \pi a^2.
\end{equation*}


