% вводные слова
% структура документа
% операторы существующие
% операторы к добавлению -> команды

% \lstinline{begin}
% \codeword{begin}
% \addtocounter{equation}{-1}

\section*{Первые шаги в документ}
\addcontentsline{toc}{section}{Первые шаги в документ}


Допустим у вас уже есть готовая преамбула, которая компилируется без ошибок и предупредений (например эта), в которой вы разобрались или обязательно как-нибудь потом разберетесь. Можем начинать писать!

\textbf{Структура документа}. 
Для начала замечу, что писать сплошной простынкой это варварство. Избежать файла на 100500 строчек вам поможет
\begin{lstlisting}
\input{folder_name/file_name}
\end{lstlisting}
где, кстати, допустим и \codeword{file_name}, и \codeword{file_name.tex}.


\textbf{Виды окружений для формул}. Документ структурирован, можем писать текст с формулами, которые оформляются окружениями, среди которых вам будут полезны:



\begin{itemize}
    \item <<внутристрочное>> окружение, используется для записи формул внутри текста: \\
\begin{minipage}{0.45\textwidth}
    \begin{lstlisting}
Supposing $\alpha > 0$ we can ...
\end{lstlisting}
\end{minipage}
\hfill
\vline
\hfill
\begin{minipage}{0.45\textwidth}
    Supposing $\alpha > 0$ we can ...
\end{minipage}
    \item <<equation>>, базовое окружение, удобно использовать для формулы в одну строчку \\
\begin{minipage}{0.45\textwidth}
    \begin{lstlisting}
\begin{equation}
    ...    
\end{equation}
\end{lstlisting}
\end{minipage}
\hfill
\vline
\hfill
\begin{minipage}{0.45\textwidth}
    \begin{equation}
        i \hbar \partial_t \Psi = \hat{H} \Psi
    \end{equation}
\end{minipage}
\\ Можем быть удобно заменить \codeword{\begin{equation*}...\end{equation*}} на более короткую запись: \codeword{$$...$$} или \codeword{\[...\]}, где последнее чуть более вариативно.

\addtocounter{equation}{-1}

\item <<align>>, удобно использовать для формул друг под другом: \\
\begin{minipage}{0.45\textwidth}
    \begin{lstlisting}
\begin{align}
    ... \\
    ...
\end{align}
\end{lstlisting}
\end{minipage}
\hfill
\vline
\hfill
\begin{minipage}{0.45\textwidth}
    \begin{align}
        \vc{E} = \frac{3 (\vc{d} \cdot \vc{n}) \vc{n} - \vc{d}}{r^3} - \frac{4\pi}{3} \delta(\vc{r}) \vc{d}, \\
        \vc{H} = \frac{3 (\vc{\mu} \cdot \vc{n}) \vc{n} - \vc{\mu}}{r^3} + \frac{8\pi}{3} \delta(\vc{r}) \vc{\mu}.
    \end{align}
\end{minipage}
\\ или для длинных формул, обозначая нужные для выравнивания места через \codeword{&}, \\
\begin{minipage}{0.45\textwidth}
    \begin{lstlisting}
\begin{align*}
    ... = ... &= ... \\
              &= ...
\end{align*}
\end{lstlisting}
\end{minipage}
\hfill
\vline
\hfill
\begin{minipage}{0.45\textwidth}
    \begin{align*}
        S = \int_{t_1}^{t_2} L \d t &= \int_{a}^{b} \left(-mc \d s - \frac{e}{c} A_i \d x^i\right) = \\
        &= \int_{t_1}^{t_2} \left(
            - mc \d s + \frac{e}{c} \vc{A} \d \vc{r} - e \varphi \d t
            % - mc^2 \sqrt{1 - \frac{v^2}{c^2}} + \frac{e}{c} \vc{A} \cdot \vc{v} - e \varphi
        \right) \d t.
    \end{align*}
\end{minipage}
\\ здесь использовано общее правило в \LaTeX e, используя звёздочку после объявления среды \codeword{align*} вы отказываетесь от нумерации этой формулы.


\end{itemize}

Ещё есть среда \codeword{gather}, которая повторяет \codeword{align}, но с немного другими правилами центрирования. 




\textbf{Операторы}. Если вы пишите математическую функцию, пишите её как оператор: \\
\begin{minipage}{0.45\textwidth}
    \begin{lstlisting}
\begin{align*}
    \cos a \\
     cos a \\
\end{align*}
\end{lstlisting}
\end{minipage}
\hfill
\vline
\hfill
\begin{minipage}{0.45\textwidth}
    \begin{align*}
        \cos a \\
         cos a
    \end{align*}
\end{minipage}
\\И вообще старайтесь придерживаться правила, что переменные выделяются \textit{курсивом}, соответственно всё что не переменная (обычно это и есть какой-нибудь оператор) -- не курсив.

Если нужного вам оператора в системе нет, его можно задать, прописав в преамбуле: \\
\begin{minipage}{0.65\textwidth}
\begin{lstlisting}
\newcommand{\spec}{\mathop{\mathrm{spec}}\nolimits}
\end{lstlisting}
\end{minipage}
\hfill
\vline
\hfill
\begin{minipage}{0.25\textwidth}
    \begin{align*}
        \spec A
    \end{align*}
\end{minipage}
\\Не выделять операторы прямым текстом -- очень дурной тон. 


\textbf{Команды}. В предыдущем пункте использовалось задание новой команды через \codeword{newcommand}, но бывает, что нужное вам имя занято и его удобно переименовать, тогда пользуемся \codeword{renewcommand}: 
\begin{lstlisting}
\renewcommand{\Im}{\mathop{\mathrm{Im}}\nolimits}
\renewcommand{\Re}{\mathop{\mathrm{Re}}\nolimits}
\end{lstlisting}
Если вам нужна команда с обязательными аргументами, то просто дописываем \codeword{[...]}, указывая в квадратных скобках число аргументов \\
\begin{minipage}{0.45\textwidth}
\begin{lstlisting}
\newcommand{\sub}[2]{#1_{\textnormal{#2}}}
\end{lstlisting}
\end{minipage}
\hfill
\vline
\hfill
\begin{minipage}{0.22\textwidth}
\begin{lstlisting}
$\sub{A}{in}$
\end{lstlisting}
\end{minipage}
\hfill
\vline
\hfill
\begin{minipage}{0.22\textwidth}
$\sub{A}{in}$
\end{minipage}

