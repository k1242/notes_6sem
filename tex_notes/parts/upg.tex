\section*{Для любознательных}
\addcontentsline{toc}{section}{Для любознательных} 


\textbf{Необязательные аргументы}.
Иногда очень хочется, чтобы у функции были необязательные аргументы, например, как у \codeword{\sqrt[3]{8}}: $\sqrt[3]{8}$,
тогда вам поможет чуть более сложная конструкция: вы объявляете команду через \codeword{DeclareDocumentCommand}, потом указываете имя команды, далее идут аргументы, пока\footnote{
    Там чуть более богатая история, которую рекомендую на досуге почитать.
}  можете считать, что \codeword{m} -- обязательные, они укзаваются через \codeword{{...}}, \codeword{o} -- необязательные, которые указываются через \codeword{[...]}. Потом, например, через \codeword{IfNoValueTF}, которая проверяет наличие/отсутствие указания определенного аргумента (№ 2 в примере) распиваете обычный if. Звучит громоздко, так что давайте разберем на примере:
\begin{lstlisting}
\DeclareDocumentCommand{\bk}{m o m}{
    \IfNoValueTF{#2}{\langle #1 | #3 \rangle}{\langle #1 | #2 | #3 \rangle}
}
\end{lstlisting}
Что у нас получилось? Что-то в духе \\
\begin{minipage}{0.45\textwidth}
    \begin{lstlisting}
\begin{align*}
    \bk{m}{n} \\
    \bk{m}[\hat{a}]{n}
\end{align*}
\end{lstlisting}
\end{minipage}
\hfill
\vline
\hfill
\begin{minipage}{0.45\textwidth}
\begin{align*}
    \bk{m}{n} \\
    \bk{m}[\hat{a}]{n}
\end{align*}
\end{minipage}


% \begin{equation*}
%     \boldsymbol v + e^{\boldsymbol k \cdot \boldsymbol r} + e^{\smallvc{k}}
% \end{equation*}