% \subsection*{Дополнение}

% \textbf{Уравнения движения}.
% Подставляя вид силовой функции в уравнения движения, находим:
% \begin{align*}
%     \ddot{x} &=2   \dot{y} \omega + \alpha\frac{ R-x}{\left((R-x)^2 + y^2\right)^{3/2}} - \frac{ x + \alpha R}{\left((\alpha R + x)^2 + y^2\right)^{3/2}} + \frac{x}{R^3} \frac{1}{(1+\alpha)^2}, \\
%     \ddot{y} &= -2 \dot{x} \omega - \frac{\alpha y}{\left((R-x)^2 + y^2\right)^{3/2}} - \frac{ y}{\left((\alpha R + x)^2 + y^2\right)^{3/2}}  +  \frac{y}{R^3} \frac{1}{(1+\alpha)^2}.
% \end{align*}





\subsection*{Motivation letter}

+
Я хочу заниматься наукой в принципе, квантовой механикой в частности, более того хотел бы продожить мою исследовательскую деятельность в Германии, поэтому вижу MCQST, как прекрасную возможность познакомиться с research groups в Германии.

-
My interest in experimental physics began around three years ago when I 
was in school, it was в лаборатории академического университете, где Я проходил научную практику. Там была готовая устанвка по росту просветляющих покрытий. Мы с коллегой подобрали оптимальные параметры для быстрорастущего покрытия. По итогом практики научный руководитель помог на выступить на конференции в СПб и Сингапуре, и опубликоваться по ее итогам.

+
Однако меня всегда интересовало устройство мира начиная с его основ, так что поступил на факультет теоретической физики в МФТИ. Здесь, только коснувшись квантовой механики, понял что этим и хотел бы заниматься. Построенная на данный момент теория квантовой механики/квантовой оптики/квантовой теории поля позволяет реализовать без малого фантастические проекты. 
Квантовые вычисления, квантовые симуляторы, квантовая оптика в принципе -- все это кажется мне невероятно красивым. Более того, в будущем вижу себя определенно в экспериментальной физике. 

+
Поэтому начинаю со второго курса начал работать на проектом с генератором квантовых случайных чисел, где успешно применил нейросетевые методы для работы с данными и принял участие в подготовке патента. 

+
На втором курсе начал работать в лаборатории ультрахолодных фермионов в Российском Квантовом Центре (РКЦ). Построенная теоретическая модель взаимодействия света с атомом позволила оптимизировать параметры насыщенной спектроскопии и улучшить стабилизацию лазера. Летом работал над созданием колимиированнго пучка атомов 6Li, где приобрел навыки работы с высоковакуумными системами. Самым важным в этих проектах считаю опыт преодоления технических трудностей при реализации проектов. Концептуально эксперимент может быть очень протым, но дьявол в деталях. По итогам этих проектов могу подвести итог, что Я могу и хочу преодолевать препятствия при реализации, потому что каждый шаг исследовательской деятельности приводит меня в восторг.

+
Мне кажется в процессе MCQST будут очень полезны мои навыки работы на оптическом столе (разработка и юстировка оптических схем, настройка АОМа, лазера), а также навыки работы с высоковакуумным оборудованием. Также считаю, что поможет в постановке эксперимента умение построить теоретическую модель и изучить ее поведение. 

+
На третьем курсе мне довелось вести курс лабораторных работ по реализации квантовой механике для студентов МФТИ. Также на третьем курсе начал вести квантовую механику в кружке при RQC. Это был очень полезный опыт, углубивший мое понимание квантовой механики и способов про нее рассказать, что мне кажется очень важным в научной среде. Нельзя заниматься наукой в изоляции, важно чтобы тебя понимали, и ты понимал других. В этом плане MCQST  также кажется мне замечательной возможностью научиться у других, и, надеюсь, чему-то их научить.



I hope that I have been able to convince you of my passionate motivation and  capabilities to become part of your research team. I really look forward to your reply and to continue my scientific activities in Germany.




% ---------------------------------------------------------------------



Помимо этого остаются вопросы на которые теоретически очень трудно (или невозможно вовсе), поэтому меня очень заинтересовала сфера квантовых симуляторов. Эта сфера прекрасно объединяет стремление к практической реализации систем с наблюдаемыми квантовыми эффектами и глобий и безумно красивый мат аппарат необходимый для описания и контролирования этих систем. 




% ---------------------------------------------------------------------

I want to do science in general, quantum mechanics in particular, moreover, I would like to continue my research activities in Germany, so I see MCQST as a great opportunity to get acquainted with research groups in Germany.


My interest in experimental physics began around three years ago when I
was in school, it was in the laboratory of the academic university where I had my scientific practice. There was a ready-made installation for the growth of antireflection coatings. My colleague and I selected the optimal parameters for a fast-growing coverage. As a result of the practice, the supervisor helped to speak at a conference in St. Petersburg and Singapore, and publish its results.


However, I have always been interested in the structure of the world, starting with its foundations, so I entered the Faculty of Theoretical Physics at the Moscow Institute of Physics and Technology. Here, only touching on quantum mechanics, I realized that this is what I would like to do. The theory of quantum mechanics/quantum optics/quantum field theory that has been constructed so far makes it possible to implement almost fantastic projects.
Quantum computing, quantum simulators, quantum optics in general, all of this seems incredibly beautiful to me. Moreover, in the future I see myself definitely in experimental physics.


Therefore, starting from the second year, I started working on a project with a quantum random number generator, where I successfully applied neural network methods to work with data and took part in the preparation of a patent.


In his second year, he began working in the laboratory of ultracold fermions at the Russian Quantum Center (RKC). The constructed theoretical model of the interaction of light with an atom made it possible to optimize the parameters of saturated spectroscopy and improve laser stabilization. In the summer he worked on the creation of a collimated beam of 6Li atoms, where he acquired skills in working with high-vacuum systems. I think the most important thing in these projects is the experience of overcoming technical difficulties in the implementation of projects. Conceptually, the experiment may be very simple, but the devil is in the details. Based on the results of these projects, I can sum up that I can and want to overcome obstacles in the implementation, because every step of the research activity excites me.


It seems to me that during the MCQST my skills of working on an optical table (development and adjustment of optical circuits, tuning of AOM, laser), as well as skills of working with high-vacuum equipment, will be very useful. I also think that the ability to build a theoretical model and study its behavior will help in setting up an experiment.


In my third year, I had the opportunity to conduct a course of laboratory work on the implementation of quantum mechanics for students of the Moscow Institute of Physics and Technology. Also in his third year, he began to teach quantum mechanics in a circle at the RQC. It was a very rewarding experience that deepened my understanding of quantum mechanics and how to talk about it, which I think is very important in the scientific community. You cannot do science in isolation, it is important that you are understood, and that you understand others. In this regard, MCQST also seems to me a great opportunity to learn from others, and hopefully teach them something.



I hope that I have been able to convince you of my passionate motivation and capabilities to become part of your research team. I really look forward to your reply and to continue my scientific activities in Germany.

% --------------------------------------------------------------------------


I want to do science in general, quantum mechanics in particular, moreover, I would like to continue my research activities in Germany, so I see MCQST as a great opportunity to get acquainted with research groups in Germany and acquire relevant skills.

My interest in experimental physics began around three years ago when I
was in school, it was in the laboratory of the Academic University in St. Petersburg, where I had my scientific practice. My colleague and I selected the optimal parameters for a fast-growing antireflective coating. As a result of the practice, the supervisor helped to speak at a conference in St. Petersburg and Singapore. The article was published as a result of the conferences.

However, I have always been interested in the structure of the world, starting with its foundations, so I entered the Faculty of Theoretical Physics at the Moscow Institute of Physics and Technology. Here, only touching on quantum mechanics, I realized that this is what I would like to do. The theory of quantum mechanics/quantum optics/quantum field theory that has been constructed so far makes it possible to implement almost fantastic projects.
Quantum computing, quantum simulators, quantum optics in general, all of this seems incredibly beautiful to me. Moreover, in the future, I see myself definitely in experimental physics.

Therefore, starting from the second year, I started working on a project with a quantum random number generator, where I successfully applied neural network methods to work with data and took part in the preparation of a patent.

In his second year, he began working in the laboratory of ultracold fermions at the Russian Quantum Center (RKC). The constructed theoretical model of the interaction of light with an atom made it possible to optimize the parameters of saturated spectroscopy and improve laser stabilization. In the summer he worked on the creation of a collimated beam of 6Li atoms, where he acquired skills in working with high-vacuum systems. I think the most important thing in these projects is the experience of overcoming technical difficulties in the implementation of projects. Based on the results of these projects, I can sum up that I can and want to overcome obstacles in the implementation because every step of the research activity excites me.

It seems to me that during the MCQST my skills of working on an optical table (development and adjustment of optical circuits, tuning of AOM, laser), as well as skills of working with high-vacuum equipment, will be very useful. I also think that the ability to build a theoretical model and study its behavior will help in setting up an experiment.

In my third year, I had the opportunity to conduct a course of laboratory work on the implementation of quantum mechanics for students of the Moscow Institute of Physics and Technology. Also in his third year, he began to teach quantum mechanics in a circle at the RQC. It was a very rewarding experience that deepened my understanding of quantum mechanics and how to talk about it, which I think is very important in the scientific community. You cannot do science in isolation, it is important that you are understood, and that you understand others. In this regard, MCQST also seems to be a great opportunity to learn from others, and hopefully teach them something.

I hope that I have been able to convince you of my passionate motivation and capabilities to become part of your research team. I really look forward to your reply and to continuing my scientific activities in Germany.
