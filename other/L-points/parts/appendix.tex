% \subsection*{Дополнение}

% \textbf{Уравнения движения}.
% Подставляя вид силовой функции в уравнения движения, находим:
% \begin{align*}
%     \ddot{x} &=2   \dot{y} \omega + \alpha\frac{ R-x}{\left((R-x)^2 + y^2\right)^{3/2}} - \frac{ x + \alpha R}{\left((\alpha R + x)^2 + y^2\right)^{3/2}} + \frac{x}{R^3} \frac{1}{(1+\alpha)^2}, \\
%     \ddot{y} &= -2 \dot{x} \omega - \frac{\alpha y}{\left((R-x)^2 + y^2\right)^{3/2}} - \frac{ y}{\left((\alpha R + x)^2 + y^2\right)^{3/2}}  +  \frac{y}{R^3} \frac{1}{(1+\alpha)^2}.
% \end{align*}





\subsection*{Motivation letter}


My interest in experimental physics began around three years ago when I 
was in school, it was в лаборатории академического университете, где Я проходил научную практику. Там была готовая устанвка по росту просветляющих покрытий. Мы с коллегой научились на ней работать и подобрали оптимальные параметры для быстрорастущего покрытия. По итогом практики научный руководитель помог на выступить на конференции и опубликоваться по ее итогам .
% (статью процитировали, что меня очень порадовала, наша деятельность оказалась полезна и за пределами ау)


Однако меня всегда интересовало устройство мира начиная с его основ, так что поступил на факультет теоретической физики в МФТИ. Здесь, только коснувшись квантов, понял что ими и хотел бы заниматься. Построенная на данный момент теория квантовой механики/квантовой оптики/квантовой теории поля позволяет реализовать без малого фантастические проекты. 
Квантовые вычисления, квантовые симуляторы, квантовая оптика в принципе -- все это кажется мне невероятно красивым. Более того, в будущем вижу себя определенно в экспериментальной квантовой физики. 







Помимо этого остаются вопросы на которые теоретически очень трудно (или невозможно вовсе), поэтому меня очень заинтересовала сфера квантовых симуляторов. Эта сфера прекрасно объединяет стремление к практической реализации систем с наблюдаемыми квантовыми эффектами и глобий и безумно красивый мат аппарат необходимый для описания и контролирования этих систем. 





