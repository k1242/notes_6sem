\section{С2. Интерполяция}

\subsection*{Интерполяция полиномом}


\textbf{Постановка задачи}. В 1D есть множество пар точек ${x_i, y_i}$, нужно построить непрерывную гладкую $u(x)$. 
Вообще можем говорить, что у нас есть сеточная проекция функции $u(x) \colon  \{u_i\}_{i=0}^{N} = \{u(x_i)\}$.  


\textbf{Полиномы Ланранжа}. 
Можем по $N+1$ точке провести полином степени $N$, например написав полином в форме Лагранжа
\begin{equation}
	L_N (x) = \sum_{k=0}^{N} c_k (x) u_k,
	\hspace{10 mm} 
	c_k (x) = \prod_{i=0, i\neq k}^{N} \frac{x-x_i}{x_k-x_i},
	% = \frac{(x-x_0)(x-x_1)\ldots (x-x_{k-1})(x-x_{k+1})\ldots(x-x_N)}{(x_k-x_0)(x_k-x_1)\ldots(x_k -x_{k-1})(x_k-x_{k+1})\ldots(x_k - x_N)}
\end{equation}
где $c_k (x_i) = \delta_{ki}$.

\textbf{Полиномы Ньютона}. 
Можем написать полином в форме Ньютона через разностный аналог формулы Тейлора:
\begin{equation}
	U_n (x) = u (x_0) + (x-x_0) u(x_0, x_1) + (x-x_0) (x-x_1) u(x_0,x_1, x_2) + \ldots
\end{equation}
где ввели функцию вида
\begin{equation}
	u(x_i, x_{i+1}) = \frac{u(x_{i+1}) - u(x_i)}{x_{i+1}-x_i},
	\hspace{5 mm} 
	u(x_i, x_{i+1}, x_{i+2}) = \frac{u(x_{i+1}, x_{i+2}) - u(x_i, x_{i+1})}{x_{i+2}-x_i}, \hspace{5 mm} \ldots
\end{equation}
Таким образом и формируется разностная схема, позволяющая строить интерполяционные полиномы. 

Полиномы Лагранжа удобно использовать для фиксированного числа узлов. Полиномы Ньютона больше подходят для фиксированной функции с переменным числом узлов.

\textbf{Ошибка интерполяции}. Введем ошибку, как
\begin{equation*}
	R_N (x)= u(x) - P_N (x).
\end{equation*}
Предположим, что на отрезке $[a, b]$ функция $u(x)$ $N+1$ раз непрерывно дифференцируема, тогда
\begin{equation*}
	R_N (x) = \frac{u^{(N+1) (\xi)}}{(N+1)!} \prod_{j=0}^{N} (x-x_j),
	\hspace{5 mm} 
	\xi \in [a, b].
\end{equation*}
Считая сетку регулярной с шагом $h \colon  b-a = N h$:
\begin{equation*}
	R_N (x) = \frac{h^{N+1}}{(N+1)!} \max_{\xi \in [a, b]} |u^{(N+1)} (\xi)|.
\end{equation*}
Вообще полезно ввести константу Лебега, порядка $2^N$ для полиномов, характеризующую размер осцилляций интерполяционных полиномов между узлами.
Решить проблему можно выбрав оптимальную сетку, для котрой константа Лебега порядка $\ln N$. Заметим, что всё это подходит только для локальной интерполяции. 




\subsection*{Сплайн интепроляция}


\textbf{Постановка задачи}. В 1D есть множество пар точек ${x_i, y_i}$, нужно построить непрерывную гладкую $u(x)$.  Можем на каждом отрезке $[x_{i-1}, x_i]$ заменить функцию $u(x)$ некоторым многочленом $s_i (x)$. 


Обычно используется полином третьей степени, для которого требуем
\begin{equation*}
	s_{i-1} (x_{i-1}) = s_{i} (x_{i-1}),
	\hspace{5 mm} 
	s_{i} (x_i) = s_{i+1} (x_i),
\end{equation*}
и аналогично для $s'$ и $s''$.

Степенью сплайна называется максимальная степень $s_i$, гладкостью -- количество непрерывных производных, деффектом -- разность между степенью и гладкостью. 


\textbf{Коэффициенты}.
Сплайн можем найти в виде
\begin{equation*}
	s_i (x) = a_i + b_i (x-x_i) + \frac{c_i}{2} (x-x_i)^2 + \frac{d_i}{6} (x-x_i)^3,
\end{equation*}
откуда сразу знаем смысл коэффициентов
\begin{equation*}
	a_i = s_i(x_i),
	\hspace{5 mm} 
	b_i = s_i' (x_i), 
	\hspace{5 mm} 
	c_i = s_i''(x_i),
	\hspace{5 mm} 
	d_i = s_i'''(x_i).
\end{equation*}
Всего у нас 4 неизвестных, 4 условия на их значения: $u_i = s_i=s_{i-1}|_{x=x_i}$, $s_i'=s_{i-1}'|_{x=x_i}$, $s_i''=s_{i-1}''|_{x=x_i}$, откуда однозначно достаются коэффициенты:
\begin{equation*}
	\left\{\begin{aligned}
	    a_{i-1} &= a_i - b_i h  + \tfrac{c_i}{2} h^2 - \tfrac{d_i}{6} h^3, \\
	    b_{i-1} &= b_i - c_i h + \tfrac{d_i}{2} h^2, \\
		c_{i-1}	&= c_i - d_i h,
	\end{aligned}\right.
\end{equation*}
где первое условие для $i=1,\ldots,N$ и остальные для $i=2, \ldots, N$. 
Всего получается на $3 N-2$ условия на $3 N$ неизвестных. 

Есть некоторая свобода в выборе краевых условий на $s'$ и $s''$. Выбор $u''(x_0) = u''(x_N) = 0$ называют естественным сплайном. Допустим также $u'''(x_0) = u'''(x_N) = 0$. Ещё бывает периодический сплайн, когда требуем равенство производных на краях. 

\textbf{Свойства сплайна}. Если $u(x)$ -- непрерывна, то последовательность кубических сплайнов $u_N (x)$ будет сходиться к $u(x)$ \textit{равномерно}. 


\textbf{Многомерный случай}. Для 2D функции $u(x, y)$ можем построить 2D полиномы Лагранжа
\begin{equation*}
	L_{NM} (x, y) = \sum_{n=0}^{N} \sum_{m=0}^{M} u_{nm} \prod_{i \neq n} \prod_{j \neq m} \frac{(x-x_i) (y-x_j)}{(x_n -x_i)(y_m-y_j)}.
\end{equation*}
Аналогично можем строить многомерные сплайны. 
