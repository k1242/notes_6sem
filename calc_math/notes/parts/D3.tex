\section{D3. Численное решение уравнений Лапласа и Пуассона}

\subsection*{Уравнение Лапласа}


\textbf{Интерполяция}. Вспомним, что в 2D
 для $u(x, y) \in \mathbb{R}^2$ 
$\forall (x, y)$
можем написать значение функции в терминах неопределенных коэффициентов:
\begin{equation*}
	u(x, y) = \sum_{i=1}^{I} \alpha_i u(x + \Delta x_i,\, y + \Delta y_i),
	\hspace{10 mm} 
	\Delta x_i = x_i-x,
	\hspace{5 mm} 
	\Delta y_i = y_i-y.
\end{equation*}
Коэффициенты находили рассматривая оператор
\begin{equation*}
	\left(dx \partial_x + dy \partial_y\right)^k = \left(\Delta x_i \partial_x |_x + \Delta y_i \partial_y |_y\right)^k.
\end{equation*}
Вспоминаем разложение в ряд Тейлора:
\begin{align*}
	u(x + \Delta x_i, y + \Delta y_i) = u(x, y) + &\Delta x_i \cdot u_x  + \Delta y_i u_y + \frac{1}{2}\left(
		\Delta x_i^2 u_{xx} + 2 \Delta x_i \Delta y_i u_{xy} + \Delta y_i^2 u_{yy}
	\right) + \\ &\frac{1}{6} \left(
		\Delta x_i^3 u_{xxx} + 3 \Delta x_i^2 \Delta y_i u_{xxy} + 3 \Delta x_i \Delta y_i^2 u_{xyy} + \Delta y_i^3 u_{yyy}
	\right) + \ldots
\end{align*}
Откуда и находим условие на наши коэффициенты:
\begin{align*}
	\sum_{i=1}^{I} \alpha_i = 1,
	\hspace{10 mm} 
	&\sum_{i=1}^{I} \Delta x_i \, \alpha_i = 0, 
	\hspace{2.5 mm} 
	\sum_{i=1}^{I} \Delta y_i \, \alpha_i = 0, \\
	&\sum_{i=1}^{I} \Delta x_i^2 \, \alpha_i = 0,
	\hspace{2.5 mm} 
	\sum_{i=1}^{I} \Delta x_i \Delta y_i \, \alpha_i = 0,
	\hspace{2.5 mm} 
	\sum_{i=1}^{I} \Delta y_i^2 \, \alpha_i = 0, \\
	&\sum_{i=1}^{I} \Delta x_i^3 \, \alpha_i = 0,
	\hspace{2.5 mm} 
	\sum_{i=1}^{I} \Delta x_i^2 \Delta y_i \, \alpha_i = 0,
	\hspace{2.5 mm} 
	\sum_{i=1}^{I} \Delta x_i \Delta y_i^2 \, \alpha_i = 0,
	\hspace{2.5 mm} 
	\sum_{i=1}^{I} \Delta y_i^3 \, \alpha_i = 0,
\end{align*}
воот. Коэффициентов получается $I = 10$. 


\textbf{Решение уравнений}. Запишем уравнение Лапласа:
\begin{equation*}
	u_{xx} + u_{yy} = 0.
\end{equation*}
У него есть несколько удобных нам свойств:
\begin{equation*}
	u_{xx} = - u_{yy},
	\hspace{5 mm} 
	u_{xxx} = - u_{xyy},
	\hspace{5 mm} 
	u_{xxy} = - u_{yyy}.
\end{equation*}
То есть модифицируются уравнения интерполяции до $I = 7$ соседей:
\begin{align*}
	\sum_{i=1}^{I} \alpha_i = 1,
	\hspace{10 mm} 
	&\sum_{i=1}^{I} \Delta x_i \, \alpha_i = 0, 
	\hspace{2.5 mm} 
	\sum_{i=1}^{I} \Delta y_i \, \alpha_i = 0, \\
	&\sum_{i=1}^{I} (\Delta x_i^2-\Delta y_i^2) \, \alpha_i = 0,
	\hspace{2.5 mm} 
	\sum_{i=1}^{I} \Delta x_i \Delta y_i \, \alpha_i = 0, \\
	&\sum_{i=1}^{I} (\Delta x_i^3-\Delta x_i \Delta y_i^2) \, \alpha_i = 0,
	\hspace{2.5 mm} 
	\sum_{i=1}^{I} (\Delta y_i^3-\Delta x_i^2 \Delta y_i) \, \alpha_i = 0.
\end{align*}


\textbf{Крест}. 
Допустим теперь будем рассматривать сетку с постоянным шагом с шаблном крест: $(x \zpm \Delta x_i, y \zpm \Delta y_i)$. Также перейдём к двойной индексации $\alpha_{i, j}$ стоящей перед $(x, y) + (i\, \Delta x,\, j\, \Delta y)$.

Тогда система перепишется в виде
\begin{equation*}
	\alpha_{1,0} = \alpha_{-1,0},
	\hspace{5 mm} 
	\alpha_{0,1} = \alpha_{0, -1},
	\hspace{5 mm} 
	\alpha_{1,0} + \alpha_{0, 1} = \frac{1}{2},
	\hspace{5 mm} 
	\Delta x^2 \alpha_{1,0} = \Delta y^2 \alpha_{0,1},
\end{equation*}
а остальные уравнения оказываются вырожденными, так что можем остановиться. 


Решая систему, находим
\begin{align*}
	\alpha_{1,0} = \alpha_{-1,0} = \frac{\Delta y^2}{2 (\Delta x^2 + \Delta y^2)} \overset{*}{=}  \frac{1}{4}, \\
	\alpha_{0,1} = \alpha_{0,-1} = \frac{\Delta x^2}{2 (\Delta x^2 + \Delta y^2)}\overset{*}{=} \frac{1}{4},
\end{align*}
где $\overset{*}{=}$ верны при $\Delta x = \Delta y$. 

Подставляя это в разностную схему, находим
\begin{equation*}
	u_{0,0} = u(x,y) = \frac{\Delta y^2}{2 (\Delta x^2 + \Delta y^2)}\left(u_{-1, 0} + u_{1,0}\right) + \frac{\Delta x^2}{2 (\Delta x^2 + \Delta y^2)} \left(u_{0,-1} + u_{0, 1}\right).
\end{equation*}
Или можем переписать в виде
\begin{equation*}
	 \frac{u_{-1,0} - 2 u_{0,0} + u_{1,0}}{\Delta x^2} + \frac{u_{0, -1} - 2 u_{0,0} + u_{0, 1}}{\Delta y^2} = 0,
	 \hspace{0.5cm} \overset{*}{\Leftrightarrow}  \hspace{0.5cm}
	 u_{0,0} = u(x,y) = \frac{1}{4}\left(
	 	u_{-1, 0} + u_{1,0} + u_{0,-1} + u_{0, 1}
	 \right),
\end{equation*}
который даёт решение уравнения Лапласа $\Delta u = 0$ со вторым порядком аппроксимации. 



Вообще, говоря об устойчивости, верно, что
\begin{equation}
	|\alpha_{ij}| \leq 1 
	\hspace{5 mm} 
	\Leftrightarrow
	\hspace{5 mm} 
	\text{схема устойчива},
\end{equation}
так что наша схема устойчива, более того монотонна по Фридрихсу. 

\begin{to_def}
    \textit{Монотонной по Фридрихсу} схемой называется разностная схема с $0 \leq \alpha_{ij} \leq 1$. 
\end{to_def}

% теорема Годунова Рябенькова


\textbf{Неравномерная сетка}. Пусть задана на некоторой плохой области уравнение Лапласа $\Delta u = 0$ с граничными условиями вида $u|_{\partial G} = \varphi(x, y)$ и неравномерной сеткой. 
Тогда нужно для системы вида $u_{i} = \sum_{j=1}^{7} \alpha_{j,i} u_j$ найти 7 соседей. Верно, что $\Delta x_j = x_j - x_i$, $\Delta y_j = y_j - y_i$, $j = 1,\ldots,7$ для второго порядка аппроксимации. 

\begin{to_lem}
    Устойчивость схемы можно обеспечить выпуклостью выбранного многоугольника. 
\end{to_lem}

 
Индекс $i$ пробегает точки которые не попали на границу.  На каждый шаг можем взять 7 точек $\vc{u} = (u_1,\,  \ldots,\, u_I)$,  и получить систему линейных уравнений
\begin{equation*}
	A \vc{u} = \vc{f},
	\hspace{10 mm} 
	\vc{f} = (f_1,\, \ldots, f_I).
\end{equation*}
В принципе мы знаем матрицу $A$:
\begin{equation*}
	u_i - \sum_{j=1}^{7} \alpha_{i,j} u_j = 0,
	\hspace{5 mm} i = 1,\ldots,I,
\end{equation*}
а значит матрица $A$ будем с единицами на диагонали, а дальше как-то раскидаются $\alpha_{i,j}$. Она неплотная с высокой разреженности, так что лучше вводить равномерную сетку, тогда $A$ будет иметь ленточный вид. 


\subsection*{Уравнение Пуассона}

Теперь рассмотрим уравнение, вида
\begin{equation*}
	\Delta u(x, y) = f(x, y),
	\hspace{5 mm} 
	u|_G = \varphi(x, y).
\end{equation*}
Не конкретезируя разностную схему, будем обознать её за $\Delta_h$, тогда
\begin{equation*}
	\Delta_h u_{m,n} = f_{m,n}.
\end{equation*}
Решение здорово искать через разложение в ряд Фурье в аналитическом случае. 

При использовании схемы крест матрица $A$ для систем $A \vc{u} = \vc{f}$, верно что матрица $A$ симметрична. Значит существует ОНБ базис. 

Тогда раскладывая $\vc{u}$ по этому базису, нахожим
\begin{equation*}
	\vc{u} = \sum_{i=1}^{(M-1)^2}  c_i \vc{q}_i,
	\hspace{5 mm} 
	C_i = \bk{\vc{u}}{\vc{q}_i},
	\hspace{10 mm} 
	\vc{f} = \sum_{i=1}^{(M-1)^2},
	\hspace{5 mm} 
	f_i = \bk{\vc{f}}{\vc{q}_i}.
\end{equation*}
Таким образом получаем систему, вида
\begin{equation*}
	\sum_i c_i \lambda_i \vc{q}_i = \sum_i f_i \vc{q}_i,
	\hspace{0.5cm} \Rightarrow \hspace{0.5cm}
	c_i = \frac{f_i}{\lambda_i}.
\end{equation*}



\textbf{Проверка}. В терминах схемы крест решим уравнение Пуассона
\begin{equation*}
	\frac{u_{m-1,n} - 2 u_{m,n} + u_{m+1, n}}{h^2} + \frac{u_{m,n-1} - 2 u_{m,n} + u_{m, n+1}}{h^2} = f_{m,n},
\end{equation*}
где $h = \frac{1}{M}$ и $0 \leq x_m = m h \leq 1$, $0 \leq y_n = nh \leq 1$. Введем скалярное умножение в дискретном пространстве:
\begin{equation*}
	\bk{u}{v} = h^2 \sum_{m, n=0}^{M} u_{m,n} v_{m,n}.
\end{equation*}
Введем базисные функции
\begin{equation*}
	\psi^{(k,l)} = 2 \sin(\pi k x_m) \sin(l \pi y_n),
\end{equation*}
которые образуют ортонормированный базис. 

Подставяя в разностную схему наши базисные функции, находим
\begin{align*}
	&\frac{\psi^{(k, l)}_{m-1,n} - 2 \psi^{(k, l)}_{m,n} + \psi^{(k, l)}_{m+1, n}}{h^2} + \frac{\psi^{(k, l)}_{m,n-1} - 2 \psi^{(k, l)}_{m,n} + \psi^{(k, l)}_{m, n+1}}{h^2} 
	=
	%  \\ =  &
	% \frac{2}{h^2}\bigg[
	% \left(
	% 	\sin\left( k \pi \frac{m-1}{M}\right) + 
	% 	\sin\left(k \pi \frac{m+1}{M}\right)
	% \right) \sin\left( \frac{l \pi n}{M}\right) + \\ &+
	% \left(
	% 	\sin\left(\pi l \frac{n-1}{M}\right) + \sin\left(
	% 		l \pi \frac{n+1}{M}
	% 	\right)
	% \right) \sin \left(\frac{k \pi m}{M}\right)  + \\ &+
	% 4 \sin \left(\frac{k \pi m}{M}\right) \sin\left(\frac{l \pi n}{M}\right)
	% \bigg] =
	\lambda_{k,l}\,  \psi_{m,n}^{(k,l)},
\end{align*}
где собственные значения
\begin{equation}
	\lambda_{k, l} = - \frac{4}{h^2}\left[
		\sin^2\left(\frac{k \pi}{2M}\right) + \sin^2 \left(\frac{l \pi}{2 M}\right)
	\right].
\end{equation}
Таким образом теперь мы знаем решением
\begin{equation}
	u_{m,n} = 2 \sum \left(\frac{f_{k,l}}{\lambda_{k,l}}\right) 
	\sin\left(k \pi x_m\right) \sin\left(l \pi y_n\right),
\end{equation}
где $m,n = 1, \ldots, M_1$, а также
\begin{equation}
	f_{k,l} = 2 h^2 \sum_{m,n=0}^{M} f(x_m, y_n) 
	\sin(k \pi x_m) \sin(l \pi y_n).
\end{equation}
Кстати, если мы выбрали $M = 2^p$, то все $u_{m,n}$ вычисляются за $O(M^2 \ln M)$ операций. 

