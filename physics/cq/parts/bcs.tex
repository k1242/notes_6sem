Рассмотрим основное состояние <<спаренных>> электронов в терминах вторичного квантования:
\begin{equation*}
	\ket{\psi_G} = \prod_{\vc{k}} \left(
		u_{\vc{k}} + v_{\vc{k}} \hat{c}\D_{\vc{k} \up} \hat{c}\D_{-\vc{k}\down}
	\right) \ket{\psi_0},
\end{equation*}
где $\psi_0$ -- вакуумное состояние. 

Количество спаренных частиц может быть найдено через оператор полного числа частиц
\begin{equation*}
	\hat{N} = \sum_{\vc{k}}  \hat{c}\D_{\vc{k} \up} \hat{c}_{\vc{k} \up} + \hat{c}\D_{\vc{k} \down} \hat{c}_{\vc{k} \down}.
\end{equation*}
Прямым вычислением, находим
\begin{equation*}
	\langle N\rangle = \bk{\psi_G}[\hat{N}]{\psi_G} = 2 \langle \textstyle\sum_{\vc{k}} \hat{N}_{\vc{k} \up} \rangle 
	= 2 \sum_{\vc{k}} \bra{\psi_0}
	 \left(
		u_\vc{k} + v_{\vc{k}}^* \hat{c}_{-\vc{k}\down} \hat{c}_{\vc{k} \up}
	\right)
	\hat{c}_{\vc{k} \up}\D \hat{c}_{\vc{k} \up} 
	(
		u_k + v_k \hat{c}\D_{\vc{k} \up} \hat{c}_{-\vc{k}\down}\D
	) \ket{\psi_0} = 2 \sum_{\vc{k}} |v_{\vc{k}}|^2,
\end{equation*}
как и ожидалось. 

% Набор 


\begin{equation*}
	\hat{H} = \sum_{\vc{k}} \varepsilon_{\vc{k}} \left(
		\hat{c}\D_{\vc{k} \up} \hat{c}_{\vc{k} \up} + \hat{c}\D_{\vc{k} \down} \hat{c}_{\vc{k} \down}
	\right) + \sum_{\vc{k},\, \vc{k}'} V_{\vc{k} \vc{k}'} c\D_{\vc{k} \up} \hat{c}\D_{-\vc{k}' \down} \hat{c}_{-\vc{k}' \down} \hat{c}_{\vc{k}' \up}.
\end{equation*}

\begin{equation*}
	\hat{N} = \sum_{\vc{k}}  \hat{c}\D_{\vc{k} \up} \hat{c}_{\vc{k} \up} + \hat{c}\D_{\vc{k} \down} \hat{c}_{\vc{k} \down}.
\end{equation*}


\begin{equation*}
	\BCS = \prod_{\vc{k}} \left(
		u_{\vc{k}} + v_{\vc{k}} \hat{c}\D_{\vc{k} \up} \hat{c}\D_{-\vc{k}\down}
	\right) \ket{0},
\end{equation*}


\begin{equation*}
	|u_{\vc{k}}|^2 + |v_{\vc{k}}|^2 = 1. 
\end{equation*}

\begin{equation*}
	\mathbb{E} = \bra{\textnormal{БКШ}} \hat{H} - \mu \hat{N} \BCS,
\end{equation*}