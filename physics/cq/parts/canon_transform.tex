В отличии от обычного металла, из-за наличия когерентности
\begin{equation*}
	\langle \hat{c}_{-\vc{k}\down}\hat{c}_{\vc{k}\up} \rangle  = b_k \neq 0,
	\hspace{1 cm}
	\hat{c}_{-\vc{k}\down}\hat{c}_{\vc{k}\up} = b_{\vc{k}} + (\hat{c}_{-\vc{k}\down}\hat{c}_{\vc{k}\up}  - b_k).
\end{equation*}
Подставив замену с точностью до флуктуаций второго порядка получим модельный гамильтониан
\begin{equation*}
	\hat{H}_M = \sum_{\vc{k} \sigma} \xi_{\vc{k}} \hat{c}_{\vc{k} \sigma}^\dagger \hat{c}_{\vc{k} \sigma}
	+
	\sum_{\vc{k} \vc{k'}} V_{\vc{k} \vc{k'}} (\hat{c}_{\vc{k}\up}^\dagger \hat{c}_{-\vc{k}\down}^\dagger b_{\vc{k'}} + b_{\vc{k}}^* \hat{c}_{-\vc{k'}\down} \hat{c}_{\vc{k'}\up} - b_{\vc{k}}^* b_{\vc{k'}}).
\end{equation*}
Определив щель $\Delta_{\vc{k}} = - \sum_{\vc{k}'} V_{\vc{k} \vc{k'}} b_{k'}$ получим
\begin{equation*}
	\hat{H}_M = \sum_{\vc{k} \sigma} \xi_{\vc{k}} \hat{c}_{\vc{k} \sigma}^\dagger \hat{c}_{\vc{k} \sigma}
	+
	\sum_{\vc{k}} \Delta_{\vc{k}}(\hat{c}_{\vc{k}\up}^\dagger \hat{c}_{-\vc{k}\down}^\dagger + \Delta_{\vc{k}}^* \hat{c}_{-\vc{k}\down} \hat{c}_{\vc{k}\up} - b_{\vc{k}}^* b_{\vc{k'}}).
\end{equation*}
Теперь произведём линейную замену c $|v_k|^2 + |u_k|^2 = 1$ на переменные Н.Н. Боголюбова
\begin{equation*}
\left\{
	\begin{aligned}
		&\hat{c}_{\vc{k} \up} &= \phantom{-} u_{\vc{k}}^* \hat{\gamma}_{\vc{k} 0} + v_{\vc{k}} \hat{\gamma}_{\vc{k} 1}^\dagger \\
		&\hat{c}_{-\vc{k} \down}^\dagger &= -v_{\vc{k}}^* \hat{\gamma}_{\vc{k} 0} + u_{\vc{k}} \hat{\gamma}_{\vc{k} 1}^\dagger	
	\end{aligned}
\right.
\hspace{0.5 cm}
\leadsto
\hspace{0.5 cm}
\left\{
	\begin{aligned}
		\hat{\gamma}_{\vc{k} 0} = u_{\vc{k}} \hat{c}_{\vc{k} \up}- v_{\vc{k}} \hat{c}_{-\vc{k}\down}^\dagger \\
		\hat{\gamma}_{\vc{k} 1} = v_{\vc{k}} \hat{c}_{\vc{k} \up}^\dagger - u_{\vc{k}} \hat{c}_{-\vc{k}\down}
	\end{aligned}
\right.
\end{equation*}
Получим большой гамильтониан, в котором подбором параметров $u_{\vc{k}}$, $v_{\vc{k}}$ добьёмся обнуления коэффициентов перед недиагональными слагаемыми $\hat{\gamma}_{\vc{k} 0}\hat{\gamma}_{\vc{k} 1}$
\begin{equation*}
	u_{\vc{k}} v_{\vc{k}} = \frac{1}{2} \frac{\Delta_{\vc{k}}}{E_{\vc{k}}},
	\hspace{1.0 cm}
	|v_{\vc{k}}|^2 = 1 - |u_{\vc{k}}|^2 = \frac{1}{2} \left(1 - \frac{\xi_{\vc{k}}}{E_{\vc{k}}}\right).
\end{equation*}
Получим тогда
\begin{equation*}
	\hat{H}_M = \sum_{\vc{k}} (\xi_{\vc{k}} - E_{\vc{k}} + \Delta_{\vc{k}} b_{\vc{k}}^*) + \sum_{\vc{k}} E_{\vc{k}} (\hat{\gamma}_{\vc{k}0}^\dagger \hat{\gamma}_{\vc{k} 0} + \hat{\gamma}_{\vc{k} 1}^\dagger \hat{\gamma}_{\vc{k} 1})
\end{equation*}
И тогда ширина щели задаётся
\begin{equation*}
	\Delta_{\vc{k}} = - \sum_{\vc{k'}} V_{\vc{k} \vc{k'}} \langle \hat{c}_{-\vc{k}'\down} \hat{c}_{\vc{k}'\down} \rangle
	=
	- \sum_{\vc{k'}} V_{\vc{k} \vc{k'}} u_{\vc{k'}}^* v_{\vc{k'}} \langle 1 - \hat{\gamma}_{\vc{k}0}^\dagger \hat{\gamma}_{\vc{k} 0} - \hat{\gamma}_{\vc{k} 1}^\dagger \hat{\gamma}_{\vc{k} 1}\rangle.
\end{equation*} 
При $T=0$ формула переходит в \eqref{delta0} в виду отсутствия квазичастиц.