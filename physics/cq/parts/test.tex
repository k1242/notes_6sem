\subsection*{Гамильтониан системы}


Рассмотрим систему фермионов в двух равнонаселенных спиновых состояниях $\up,\, \down$.
Считаем возможным только парное взаимодествие в центрально-симметричном потенциале с частицами с разными спинами
\begin{equation*}
	\Vc(\hat{r}) = \Vc (\hat{r}) \kb{\up}{\down} + \cc 
\end{equation*}
Помещаем частицы в онородный резонатор объема $\Omega$ с периодическими граничными условиями. 
В силу однородности систему локальный и глобальный химические потенциалы совпадают $\sub{\mu}{лок} = \mu$ \red{уточнить}. Оператор $\hat{c}\D_{\vc{k} \up}$ рождает частицу в состоянии с волновой функцией $\frac{1}{\sqrt{\Omega}} e^{i \vc{k} \cdot \vc{r}}$. Гамильтониан системы в наиболее общем случае имеет вид (\red{ЛЛ9, \S6}):
\begin{equation*}
	\hat{H} = \sum_{\vc{k}} \frac{\hbar^2 \vc{k}^2}{2m} \left(
		\hat{c}\D_{\vc{k} \up} \hat{c}_{\vc{k} \up} + \hat{c}\D_{\vc{k} \down} \hat{c}_{\vc{k} \down}
	\right) + 
	\frac{1}{\Omega} \sum_{\vc{k}, \vc{k}', \vc{K}} g(\vc{k}-\vc{k}') \hat{c}\D_{\vc{k}+\vc{K}/2, \up} \hat{c}\D_{-\vc{k}+\vc{K}/2, \down} 
	\hat{c}_{-\vc{k}' + \vc{K}/2, \down} \hat{c}_{\vc{k}' + \vc{K}/2, \up},
\end{equation*}
где $g(\vc{q})$ -- Фурье образ потенциала взаимодействия
\begin{equation*}
	g(\vc{q}) = \int_\Omega e^{i \vc{q} \cdot \vc{r}} \Vc (r) \d^3 \vc{r}. 
\end{equation*}
Модель уже содержит два упрощения. Во-первых, рассматриваются только парные взаимодействия. Во-вторых, в гамильтониане не учтено наличие 
% \red{синглетных молекул}. 


В качестве дальнейшего упрощения положим $\vc{K}=0$, то есть взаимодействуют лишь частицы с равными противоположными импульсами. Это приближение обосновано при наличии сферы Ферми, то есть в пределе БКШ $- \pF a \ll 1$. Состояния с импульсом $< \pF$ в основном заняты, и поэтому частицы в них рассеиваться из-за запрета Паули. Добавив к запрету Паули законы сохранения энергии и импульса, видим, что наиболее вероятно рассеяние, при котором и в начальном и в конечном состоянии импульсы двух взаимодействующих частиц противоположны и лежат на поверхности Ферми. 

Воспользуемся приближением $g(\vc{q}) = g_0$. Тогда упрщенный гамильтониан принимает вид
\begin{equation*}
	\hat{H} = \sum_{\vc{k}} \frac{\hbar^2 \vc{k}^2}{2m} \left(
		\hat{c}\D_{\vc{k} \up} \hat{c}_{\vc{k} \up} + \hat{c}\D_{\vc{k} \down} \hat{c}_{\vc{k} \down}
	\right) + g_0 \sum_{\vc{k},\, \vc{k}'} c\D_{\vc{k} \up} \hat{c}\D_{-\vc{k}' \down} \hat{c}_{-\vc{k}' \down} \hat{c}_{\vc{k}' \up}.
\end{equation*}


\subsection*{Состояние}

Предпложим, что основное состояние имеет вид
\begin{equation*}
	\BCS = \prod_{\vc{k}} \left(
		u_{\vc{k}} + v_{\vc{k}} \hat{c}\D_{\vc{k} \up} \hat{c}\D_{-\vc{k}\down}
	\right) \ket{0},
\end{equation*}
где $\ket{0}$ -- вакуум. \red{В этом приближении частицы появляются только в виде пар}. Коэффициенты $u_{\vc{k}}$, $v_{\vc{k}}$ выбраны вещественными и связаны нормировоным соотношением
\begin{equation*}
	u_{\vc{k}}^2 + v_{\vc{k}}^2 = 1. 
\end{equation*}
Из множества состояния $\BCS$ выберем состояние с наименьшей энергие вариационным методом. Минимизируем ожидаемое значение
\begin{equation*}
	\bra{\textnormal{БКШ}} \hat{H} - \mu \hat{N}_{\Sigma} \BCS,
\end{equation*}
варьируя коэффициенты $u_{\vc{k}}$, $v_{\vc{k}}$ с учетом нормировки. Оператор полного числа частиц $\hat{N}_{\Sigma}$
\begin{equation*}
	\hat{N}_{\Sigma} = \sum_{\vc{k}}  \hat{c}\D_{\vc{k} \up} \hat{c}_{\vc{k} \up} + \hat{c}\D_{\vc{k} \down} \hat{c}_{\vc{k} \down}.
\end{equation*}