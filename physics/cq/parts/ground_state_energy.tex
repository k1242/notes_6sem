Энергия задаётся выражением:
\begin{equation*}
	E = \bk{\psi_G}[\hat{H} - \mu \hat{N}]{\psi_G}
	  =  \sum_k \left( \xi_k - \frac{\xi_k^2}{E_k}\right) - \frac{\Delta^2}{V}
\end{equation*}
И отличие энергии в сверхпроводящем и нормальном состояниях при нулевой температуре, то есть $\Delta = 0$ даётся выражением
\begin{equation*}
	\langle E\rangle_S - \langle E\rangle_n = 2 \sum_{|\vc{k}| > k_F} \left(\xi_{\vc{k}} - \frac{\xi_{\vc{k}}^2}{E_{\vc{k}}}\right) - \frac{\Delta^2}{V}
	=
	\left( \int \ldots \right)
	= \underbrace{\left[\frac{\Delta^2}{V} - \frac{1}{2} N(0) \Delta^2\right]}_{\text{кинетическая}} - \underbrace{\frac{\Delta^2}{V}}_{\text{пот.}}
\end{equation*}
% или после интегрирования
% \begin{equation*}
% 	\langle E\rangle_S - \langle E\rangle_n = \underbrace{\left[\frac{\Delta^2}{V} - \frac{1}{2} N(0) \Delta^2\right]}_{\text{кинетическая}} - \underbrace{\frac{\Delta^2}{V}}_{\text{птц-ая}}.
% \end{equation*}
и для внутренной энергии
\begin{equation*}
	U_s(0) - U_n(0) = - \frac{1}{2} N(0) \Delta^2(0).
\end{equation*}

