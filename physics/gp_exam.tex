1. Кристаллические структуры твёрдых тел, трансляционная симметрия кристаллов, решётка Бравэ, элементарная и примитивная ячейки (на примере ГЦК--решётки), базис.
2. Рентгеновские и нейтронные методы исследования кристаллических структур, дифракция Вульфа-Брэгга, обратная решётка, зона Бриллюэна.
3. Неупругое рассеяние рентгеновских лучей и нейтронов: законы сохранения и геометрия опыта с трёхосным дифрактометром.
4. Энергетические диаграммы для квазичастичного туннельного тока и воль-амперные характеристики туннельного контакта металл-металл, металл-сверхпроводник и сверхпроводник-сверхпроводник.
5. Колебания моноатомной цепочки, понятие о квазиимпульсе. Роль первой зоны Бриллюэна. Дискретность квазиимпульса как следствие периодических граничных условий.
6. Колебания двухатомной цепочки, акустическая и оптическая ветви колебаний. Роль первой зоны Бриллюэна.
7. Нормальные моды колебаний решётки, число мод. Понятие о фононах. Фононы как квазичастицы. Ангармонизм.
8. Решёточная теплоёмкость. Модель Эйнштейна и Дебая, температура Дебая.
9. СКВИД: связь сверхпроводящего тока с потоком магнитной индукции через контур СКВИДа. Применение СКВИД-сенсора в качестве чувствительного датчика нуля.
10. Формирование двумерного электронного газа в гетероструктурах. Энергетические диаграммы и «изгиб зон», характерные поверхностные плотности заряда.
11. Решёточная теплопроводность при высоких температурах. Процессы переброса.
12. Решёточная теплопроводность фононного газа при низких температурах.
13. Модель желе для электронного газа. Импульс, скорость и энергия Ферми, плотность состояний на поверхности Ферми, температура вырождения.
14. Термодинамическая неустойчивость одномерных и двумерных кристаллов.
15. Вклад электронов в теплоёмкость металлов, температурная зависимость, соотношение с решёточной теплоёмкостью.
16. Электроны в периодическом потенциале ионной решетки. Теорема Блоха.
17. Физическая причина появления зон разрешённых и запрещённых значений энергии, модели слабой и сильной связи. Проводники, изоляторы и полупроводники.
18. Понятие о ферми-жидкости, электроны и дырки как квазичастицы.
19. Квантование проводимости одномерного проводника в баллистическом режиме.
20. Формула Друде-Лоренца. Температурная зависимость электропроводности при высоких температурах. Правило Матиссена.
21. Формула Друде-Лоренца. Электропроводность металлов при низких температурах. Закон Блоха--Грюнайзена.
22. Электронная теплопроводность. Качественное различие механизмов релаксации энергии и импульса электронов в процессах теплопроводности и электропроводности, закон Видемана-Франца.
23. Электрон-дырочные возбуждения в собственном полупроводнике. Распределение электронов и дырок в зоне проводимости и в валентной зоне. Понятие об эффективной массе.
24. Зависимость концентрации электронов и дырок в невырожденном собственном полупроводнике от температуры. Статфакторы зон. Положение уровня Ферми (химпотенциала) в собственном полупроводнике.
25. Донорные и акцепторные примеси в полупроводниках. Оценка энергии мелкого донорного уровня.
26. Температурная зависимость концентрации носителей в примесных полупроводниках и правило ``рычага''.
27. Экситоны Ванье-Мотта в полупроводниках.
28. Энергетическая диаграмма туннельного диода. Вольт-амперная характеристика туннельного диода, объяснение участка с отрицательным дифференциальным сопротивлением.
29. Энергетическая диаграмма (p-n) — перехода. Распределение зарядов в (p-n) — переходе, (p-n) — переход во внешнем электрическом поле и его вольт-амперная характеристика.
30. Сверхтекучесть гелия-4, основные экспериментальные факты: теплопроводность, вязкость, спектр возбуждений. Критерий Ландау, критическая скорость Ландау.
31. Явление сверхпроводимости, критическая температура, эффект Мейснера, лондоновская глубина проникновения.
32. Термодинамика сверхпроводника: термодинамическое критическое поле, скачок теплоёмкости.
33. Модель БКШ. Роль кристаллической решётки в явлении сверхпроводимости, изотоп-эффект, куперовское спаривание. Суммарный импульс, орбитальный момент и спин куперовской пары электронов. Длина когерентности, её связь с величиной сверхпроводящей щели.
34. Критический ток в сверхпроводниках, связь его величины с критерием Ландау.
35. Квантование магнитного потока в сверхпроводниках.
37. Сверхпроводники I рода и II рода, верхнее и нижнее критические поля, понятие о вихрях магнитного потока (вихрях Абрикосова), вихревая решётка. Пиннинг.
38. Квазичастичные элементарные возбуждения сверхроводника. Связь спектра элементарных возбуждений сверхпроводника с его теплоёмкостью, теплопроводностью и высокочастотными свойствами.
39. Парамагнетизм Паули.
40. Уровни Ландау в двумерном электронном газе в квантующем магнитном поле: энергии уровней, ёмкость уровней Ландау.
41. Целочисленный квантовый эффект Холла: основные экспериментальные факты, метрологическое значение, связь условий наблюдения квантового эффекта Холла с заполнением уровней Ландау.
42. Виды магнитного упорядочения: ферромагнетики, антиферромагнетики и ферримагнетики. Роль обменного взаимодействия. Закон Кюри-Вейса.
43. Туннелирование куперовских пар, стационарный и нестационарный эффект Джозефсона.
44. Элементарные возбуждения в насыщенной фазе цепочки спинов 1/2 с антиферромагнитным взаимодействием, поле насыщения.
45. Спиновые волны. Закон дисперсии спиновых волн в ферромагнетике. Магноны. Закон 3/2 Блоха.
46. Энергия анизотропии. Модель Изинга.
47. Магнетизм электронов проводимости. Критерий Стонера.	